\documentclass[main]{subfiles}
\onlyinsubfile{
  \usepackage{xr-hyper}
  \usepackage{hyperref}
  % \externaldocument{nameHere}
}


\makeatletter
\def\blx@maxline{77}
\makeatother
\begin{document}

\section{Introduction}\label{sec:introduction}

\ali{
\begin{enumerate}
  \item What problems are really holding crowd work back?
  \item Complexity. Workers are afraid
        to do creative work and get shot down for sticking their necks out.
  \item Okay, well \cite{haas2015argonaut} addresses that.
        Is that paper not solving that problem?
        Or is something else the problem?
  \item It addresses that problem to an extent.
        Tasks are also (prohibitively) poorly designed.
  \item There's some work on that as well [task iteration paper].
        So, again, what's missing?
  \item Fundamentally, nothing. We have all the pieces to this puzzle.
        But between iterating on task design,
        setting appropriate pay rates, etc\dots
        the emergent complexity in managing crowd workers has become a full time job.
  \item So what's the fix? Is there one?
  \item The fix is to take \citeauthor{haas2015argonaut}'s work and
        extend it to the logical conclusion
        --- \textit{foremen}, middle managers who are responsible for guiding a number of workers
        (or in this case, tasks) to satisfactory completion.
  \item While the working of these broken--down tasks is called ``micro--work'',
        we call the broken--down management of workers ``micro--management''. :P
\end{enumerate}
}


On--demand workers have expressed frustration with the algorithmic management that largely dictates their work
\cite{uberAlgorithm}. \msb{blah blah blah}
The fear of deviating from expected results causes online information workers 
--- such as workers on Amazon Mechanical Turk (AMT), or ``Turkers'' ---
to avoid creative or open--ended tasks due to the perceived riskiness of
investing a significant amount of time into a task only to be rejected.
\ali{[citation needed]}
\citeauthor{haas2015argonaut} address some of these problems by bootstrapping
human reviewers from vetted crowd workers
\cite{haas2015argonaut}.

We propose an extension of this relationship assigning intermediary management tasks to crowd workers,
building on a body of literature reflecting on the myriad roles and
ultimate importance of assembly line managers known as ``foremen''.
We propose to explore other ways that crowd workers can
guide \& inform workers as they do creative tasks,
as well as consolidate \& propagate feedback to requesters when it becomes apparent something is wrong.
More than reviewing tasks,
the administration of tasks itself becomes an eligible candidate for crowdsourcing.

\onlyinsubfile{
  % \balance{}
  \printbibliography
}

\end{document}
