\documentclass[trackingWork]{subfiles}
\makeatletter
\def\blx@maxline{77}
\makeatother
% \onlyinsubfile{
% \usepackage{xr--hyper}
% \usepackage{hyperref}
% \externaldocument{complexity}
% \externaldocument{relationships}
% % \externaldocument{decomposition}
% }
\begin{document}

\subsection[How far can work be decomposed into smaller microtasks]{Decomposing Work}\label{sec:decomposition}

At its core,
on--demand work has been enabled by decomposition of large goals into many small tasks.
As such,
one of the central questions in the literature is how to design these microtasks,
and which kinds of tasks are amenable to decomposition.
In this section,
we place these questions in the context of piecework's Taylorist evolution.

\begin{comment}
Outline:
Crowd work
	- How do we take work and split it up into smaller work?
	- How small can we go?
	- Once we split up, what happens? Marketplace choosing
Piecework
	- How do we take work and split it up?
	[ - how small can we go? ]
	[- Nothing about how people choose. ]
Same/different
	[- task data science ]
	[- Cognitive barriers, task switching, etc, dominates]
	[ - Switching is different, marketplace model ]
\end{comment}

% \onlyinsubfile{\clearpage}
\subsubsection{\crowdworkpers}
\topic{Many contributions to the design and engineering of crowd work
consist of creative methods for decomposing goals.}
Even when tasks such as writing and editing cannot be reliably performed by individual workers,
researchers demonstrated that
decompositions of these tasks into workflows can succeed~\cite{crowdForgeKittur,
                                                               bernsteinSoylent,
                                                               writingMicroTasks,
                                                               Nebeling:2016:WCW:2858036.2858169}. 
These decompositions typically take the form of workflows,
which are algorithmic sequences of tasks that manage interdependencies~\cite{Bigham2014}. 
Workflows often utilize a first sequence of tasks to identify an area of focus (e.g.,
a paragraph topic~\cite{crowdForgeKittur},
an error~\cite{bernsteinSoylent},
or a concept~\cite{Yu2016a,Yu2016b} and a second sequence of tasks to execute work on that area. 
This decomposition style has been successfully applied across many areas,
including food labeling~\cite{noronha2011platemate},
brainstorming~\cite{siangliulue2015toward,yu2014distributed},
and accessibility~\cite{lasecki2013chorus,lasecki2012real,Lasecki2011}.

\topic{If decomposition is key to success in on--demand work,
the question arises: what can,
and can't,
be decomposed?} 
Moreover,
how thinly can work be sliced and subdivided into smaller and smaller tasks? 
The general trend has been that smaller is better,
and the microtask paradigm has emerged as the overwhelming favorite~\cite{selfsourcingTeevan2014,selfsourcingTeevan2016}.
This work illustrates a broader sentiment in both the study and practice
of crowd work,
that microtasks should be designed resiliently against the variability of workers,
preventing a single errant submission from impacting the agenda of the work as a whole %,
fully exploiting the abstracted nature of each piece of work
~\cite{interruptionIqbal,delayAndOrderLasecki,vaish2014low}.
In this sense,
finer decompositions are seen as more robust
--- both to interruptions and errors~\cite{cheng2015break} ---
even if they incur a fixed time cost.
At the extreme,
recent work has attempted demonstrated
microtasks that take seconds~\cite{Vaish:2014:TCC:2611222.2556996,Cai:2015:WLW:2702123.2702267}
or even fractions of a second~\cite{embracingErrorKrishna}.
However,
workers perform better when similar tasks are strung together~\cite{delayAndOrderLasecki},
or
chained and arranged to maximize the attention threshold of workers~\cite{Cai:2016:CRI:2858036.2858237}.
Despite this,
we as a community have leaned \textit{into} the peril of
low--context work, ``embracing error'' in crowdsourcing~\cite{embracingErrorKrishna}.


\topic{The general lesson has been that the more micro the task,
and the more fine the decomposition,
the greater the risk that workers lose context necessary to perform the work well.}
For example,
workers edit adjacent paragraphs in inconsistent ways~\cite{bernsteinSoylent,Kim2017},
interpret tasks in different ways~\cite{kairam2016parting},
and exhibit lower motivation~\cite{Kinnaird:2012:WTM:2389176.2389219} without sufficient context.
Research has sought to ameliorate this issue by
designing workflows help workers ``act with global understanding when
each contributor only has access to local views''~\cite{verroios2014context},
typically by automatically or manually generating higher--level representations
for the workers to reflect on~\cite{chilton2013cascade,verroios2014context,Kim2017}.

\topic{As the additional context necessary to complete a task diminishes,
the invisible labor of \textit{finding} tasks~\cite{martin2014being} has arisen as a major issue.}
\citeauthor{taskSearch} illustrate the task search challenges on AMT.
Workers seek out good requesters~\cite{martin2014being} and then 
``streak'' to perform many tasks of that same type~\cite{taskSearch}.
and some work has gone into ameliorating the problems specific to this work site (\textit{ReLauncher}),

Researchers have reacted by designing task recommendation systems~\cite[e.g.][]{Cosley:2007:SUI:1216295.1216309}
and others focused on
minimizing the amount of time that people need to spend doing anything other than
the work for which they are paid~\cite{callison2014crowd}.



\onlyinsubfile{\clearpage}
\subsubsection{\pieceworkpers}

\begin{comment}
notes: what info do i assume the reader has seen already?
- Brown: Task variability matters
- Airy and his human computers were great:
  - quickly verifiable
  - independent tasks (could be checked without the whole product)
  - narrowly trainable
\end{comment}

\topic{The research community relating to
piecework and labor
has been wrestling with the decomposition of work for the better part of a century.}
Decomposition was arguably the main innovation of Airy's human computers.
Rather than hire expert computers, Airy identified ways to break down astrological calculations into steps that could be completed with only a basic knowledge of mathematics.
Future piecework efforts followed a similar model.

\topic{Decomposability is core to piecework: one cannot pay by the piece unless the pieces are small and discretizable.}
Focusing on our case study of railway engineers, \citeauthor{Brown01041990} asked 
what enabled and limited the adoption of piecework.
He argued that the main determinants were task homogeneity and the fixed costs of the machinery and training~\cite{Brown01041990}.
In other words, if tasks could be clearly decomposed and defined, and if the firm could afford the training and machinery for the pieceworkers, piecework tended to be adopted.

Decomposition ultimately led to quantification and scientific management.
With \citeauthor{taylor1914principles}'s formalization of scientific management in \textit{Taylorism}
(and Henry Ford's eponymously named \textit{Fordism}),
piecework in the early and mid--\nth{20} century surged, especially in industrial work.
Scientific management promised that the careful measurement of workers would yield
higher efficiency and output~\cite{taylor1914principles,towardsGlobalFordism}.
While \citeauthor{Brown01041990} points out that
piecework dramatically advanced the instrumented measurement of workers,
in \citeauthor{taylor1914principles}'s time highly instrumented,
automatic measurement of workers was all but impossible~\cite{Brown01041990}.
Instead, managers conducted ``stop watch time studies''~\cite{nadworny1955scientific},
using completion times to inform per--task compensation,
similarly to the efficiency experts hired in the Santa Fe Railway, but
substantially more precise.
As a result, the distillation of work into smaller units ultimately bottomed out with tasks as small as could be usefully measured~\cite{10.2307/23702539}.

\topic{Piecework researchers enumerate a number of problems with the decomposition of work,
and the conflicting pressures managers and workers put forth.}
\citeauthor{bewley1999wages} in particular points out that
the approach of paying workers by the piece is
``\dots~not practical for workers doing many tasks,
because of
the cost of establishing the rates and because
piecework does not compensate workers for time spent switching tasks''.
%Ultimately, \citeauthor{bewley1999wages} argues that
%``[piecework is] infeasible,
%because~\dots~total output is the joint product of varying groups of people''~\cite{bewley1999wages}.
To \citeauthor{bewley1999wages}, the main limiter was the ability to define and measure the work.



\onlyinsubfile{\clearpage}
\subsubsection{\whatchanged}
\begin{comment}
outline
	- measurement is more precise,
  so decomposition is deeper
	- not a single position,
  but a marketplace
\end{comment}

\topic{Where measurement and instrumentation were limiting factors for historical piecework,
computation has changed the situation so that a dream of scientific management and Taylorism
--- to measure every motion at every point throughout the workday and beyond ---
is not only doable,
but trivial~\cite{waltz2012quantified}.}
Where \citeauthor{10.2307/23702539} directly implicates measurement as
preventing scientific management from being fully utilized,
no longer exists
modern crowd work is measuring and modeling every click,
scroll,
and keyboard event~\cite{rzeszotarski2011instrumenting,rzeszotarski2012crowdscape}.
The result is that on--demand work can articulate and track far more carefully than piecework historically could.

\topic{A second shift is  
the relative ease with which the metaphorical ``assembly line'' can be changed.}
Historical manufacturing equipment could not quickly be assembled,
edited,
and redeployed~\cite{hu1961parallel}.
In contrast,
today system--designers can share,
modify,
and instantiate environments
like sites of labor in a few lines of code~\cite{lessig2006code,turkitLittle}.
This opportunity has spurred an entire body of work investigating the effects of
ordering,
pacing,
interruptions,
and
other factors in piecework that would have been
all but impossible to manipulate as few as 20 years ago~\cite{dai2015and,Cai:2016:CRI:2858036.2858237,cheng2015break,measuringCrowdsourcingCheng,embracingErrorKrishna}.
 
Third,
modern crowd work has sliced work to such small scales that the marginal activities
--- things like finding work and cognitive task switching ---
have become large relative to the tasks themselves~\cite{taskSearch}.
In the historical case of piecework,
moving metallurgical tools,
mining equipment,
or
other industry materials would have been prohibitively difficult and slow;
workers were encouraged to specialize in a single set of tasks,
allowing pieceworkers to sequence their tasks optimally on their own~\cite{hart2013rise}.
The result is that on--demand workers are more free agents than historically was the case.
However,
because they spend significant time searching for tasks,
the piece rate is less a good estimate of take--home earnings than before.


\onlyinsubfile{\clearpage}
\subsubsection{\implication}
If measurement precision limited the depth of decomposition for piecework historically,
as \citeauthor{10.2307/23702539} argues,
then modern on--demand work stands to become far more finely--sliced and highly decomposed than ever before.
Online tools make measurement and validation so easy~\cite{rzeszotarski2011instrumenting} that these aspects of piecework are solved,
or near enough that they no longer limit task decomposition.
Now,
not just tasks,
but entire workers' histories~\cite{hata2017glimpse},
can be collected and analyzed in detail.

However,
decomposition has hit a second bottleneck: cognition. 
Task switching costs and other cognitive costs make it difficult
to work on tasks so far decontextualized from their original intention~\cite{delayAndOrderLasecki}.
There will of course be tasks that can be decomposed without much context,
and these will form the most fine--grained of microtasks.
However, other tasks cannot be freed from context
--- for example,
logo design requires a deep understanding of the client and their goals.
In part due to this limitation,
99designs workers often recycle old designs rather than make new ones for each client~\cite{araujo201399designs}.

So,
ultimately,
the levels of decomposition are likely to follow the contours of context required.
Low--context work will be extremely highly decomposed.
High--context work will continue to be limited.


% \onlyinsubfile{
%   \balance{}
%   \printbibliography
% }

\end{document}