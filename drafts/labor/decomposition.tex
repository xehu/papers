\documentclass[trackingWork]{subfiles}
\makeatletter
\def\blx@maxline{77}
\makeatother
\begin{document}

\subsubsection[the decomposition of work]{The Decomposition of Work}\label{sec:decomposition}

\subsubsubsection{\crowdworkpers}
\topic{The crowdsourcing research into work decomposition
has largely focused on minimizing the additional context necessary to do tasks%
, and making it easier to do tasks with less time.}
This first thread is
perhaps best described by \citeauthor{verroios2014context} as
making crowd workers ``\dots able to act with
global understanding when each contributor only has access to local views''
\cite{verroios2014context}.
With the exception of a few cases
(specifically, \citeauthor{Kinnaird:2012:WTM:2389176.2389219}'s work
which finds that greater work context fosters more reliably high--quality work)%
, the micro task paradigm has emerged as the overwhelming favorite
\cite{selfsourcingTeevan2014,selfsourcingTeevan2016%
,       cheng2015break,Kinnaird:2012:WTM:2389176.2389219}.


\topic{As the additional context necessary to complete a task diminishes%
, the marginal cost of finding and
\textit{doing} tasks has increasingly become the focus of research.}
\citeauthor{taskSearch} illustrate the challenges on AMT%
, and some work has gone into ameliorating the problems specific to this work site
(\textit{ReLauncher}), % does anyone care for project names?
while other work designs tasks around gap time
(\textit{Twitch Crowdsourcing} \& \textit{Wait--Learning})
\cite{taskSearch,KucherbaevReLauncher,Vaish:2014:TCC:2611222.2556996%
,       Cai:2015:WLW:2702123.2702267}.
% \topic{One of the emergent properties of micro--tasks has been the relative cost of
%     \textit{finding} worthwhile tasks.}
%     The research community has documented and to some extent attempted to intervene in
%     the discovery of worthwhile tasks
%     \cite{taskSearch}.
\citeauthor{Cosley:2007:SUI:1216295.1216309}
attempts to address this by
directing workers to tasks through
``intelligent task routing''
\cite{Cosley:2007:SUI:1216295.1216309}.
Much of this work and the work at the periphery of this space, then%
, has focused on
minimizing the amount of time that people need to spend doing
anything other than the work for which they are paid.

Earlier we discussed \citeauthor{cheng2015break}'s work
measuring the impact that interruption has on worker performance
\cite{cheng2015break}.
This work illustrates a broader sentiment in
both the study and practice
of crowd work, that microtasks should be designed resiliently against the variability of workers%
, fully exploiting the abstracted nature of each piece of work
\cite{interruptionIqbal,delayAndOrderLasecki,vaish2014low}.
That is to say, micro--tasks should be designed such that a single worker's poor performance%
, or a good worker's sudden departure%
, does not significantly impact the agenda of the work as a whole.
While \citeauthor{cheng2015break} found costs with breaking tasks into smaller components
in the form of higher cumulative time to complete
(albeit much shorter real time to complete, owing to parallelization)%
, \citeauthor{delayAndOrderLasecki} found that at least \textit{some} performance can be recouped by stringing 
similar tasks together
\cite[respectively]{cheng2015break,delayAndOrderLasecki}.


Yet more work looks at the general framing of tasks%
, chaining and arranging them to maximally exploit
the attention and stress threshold % ???
of workers
\cite{Cai:2016:CRI:2858036.2858237}.
Rather than attempt to minimize the error rates in micro--tasks%
, as \citeauthor{Kinnaird:2012:WTM:2389176.2389219} suggested%
, we as a community have leaned \textit{into} the peril of
low--context work%
, ``embracing error'' in crowdsourcing
\cite{embracingErrorKrishna}.


\subsubsubsection{\pieceworkpers}
\topic{The research community relating to
piecework and labor
has been wrestling with the decomposition of work for centuries.}
The beginnings of
systematic task decomposition
stretch back as far as the \nth{17} century%
, when Airy employed young boys at the Greenwich Observatory who
``possessed the basic skills of mathematics, including
`Arithmetic, the use of Logarithms, and Elementary Algebra'~''
to compute, by hand, astronomical phenomena
\cite{grier2013computers}.
These workers became the first \textit{computers}.

The work Airy solicited was interesting for several reasons.
First, work output was quickly verifiable;
Airy could assign variably skilled workers to compute values%
, and have other workers check their work.
Second, tasks were discrete --- that is, independent from one another.
Finally, knowledge of the full scope of the project
--- indeed, knowledge of anything more than the problem set at hand ---
was unnecessary.

The insight of breaking tasks down into smaller components didn't find its audience until
the early \nth{20} century%
, with the rise of Fordism and scientific management (or Taylorism).
From scientific management, we found that
we could measure work at unprecedented resolution and precision.
As \citeauthor{Brown01041990} points out%
, piecework most greatly benefits the instrumented measurement of workers, but certainly
in Ford and Taylor's time --- and certainly in Airy's time ---
highly instrumented, automatic measurement of workers was all but impossible.
As a result%
, the distillation of work into smaller chunks
ultimately reached a limit of usefulness.


\subsubsubsection{\whatchanged}
 \topic{A number of factors in crowd work are different from piecework,
 chief among them being the relative ease with which
 the metaphorical ``assembly line'' can be changed.}
 Computers make it possible to switch from one task to another
unlike any arbitrary manufacturing factory possibly could;
a worker could do any number of
different \textit{types} of tasks in the span of just a few minutes,
driven in particular by the power \citeauthor{lessig2006code} points to ---
that system--designers can share, modify, and instantiate environments
like sites of labor in a few lines of code
\cite{delayAndOrderLasecki,lessig2006code}.
This has spurred an entire body of work investigating the effects of
ordering,
pacing,
interruptions, and
other factors in piecework that would have been
all but impossible to measure consistently as few as 20 years ago
\cite{cheng2015break,measuringCrowdsourcingCheng,embracingErrorKrishna}.
 
Further, we've sliced work to such small scales that the marginal activities
--- things like finding work, cognitive task switching, etc\dots ---
have become relatively large compared to the tasks themselves
\cite{taskSearch}.
In the historical case of piecework,
moving metallurgical tools, mining equipment, or
other industry materials would have been prohibitively difficult and slow;
workers were encouraged to specialize in a single set of tasks,
allowing pieceworkers to sequence their tasks optimally on their own
\cite{hart2013rise}.

Rather than fall into the trap that \citeauthor{irani2015cultural} warns of,
--- one which where crowd workers are rendered as
``modular, protocol--defined computational services'' ---
we may yield better results from crowd work if we think of workers as similar to
specialized, repurposable tools
\cite{irani2015cultural}.
\ali{feeling meh about this argument\dots}
 
Finally, instrumentation has reached a sufficiently advanced and ubiquitous point that
the dream of scientific management and Taylorism
--- to measure every motion at every point throughout the workday and beyond ---
is not only doable, but trivial
\cite{waltz2012quantified}.
One of the major challenges \citeauthor{10.2307/23702539} cites as
preventing scientific management from being fully utilized,
the difficulty of tracking work \& workers, no longer exists
\cite{10.2307/23702539}.


\subsubsubsection{\implication}
crowd work research today is on the right track to investigate pipelining and meta--task design.
That is, investigating better work discovery methods, producing tools for workers to make more informed decisions
\cite[see, for example,][]{turkopticon}.
It's not clear how much benefit there is in the further decomposition of work,
given that we've hit bottlenecks with the cognitive stresses of switching between tasks
as \citeauthor{delayAndOrderLasecki} highlight
\cite{delayAndOrderLasecki}.


\end{document}