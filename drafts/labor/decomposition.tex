\documentclass[trackingWork]{subfiles}
\makeatletter
\def\blx@maxline{77}
\makeatother
% \onlyinsubfile{
% \usepackage{xr--hyper}
% \usepackage{hyperref}
% \externaldocument{complexity}
% \externaldocument{relationships}
% % \externaldocument{decomposition}
% }
\begin{document}

\subsection[How far can work be decomposed into smaller microtasks]{Decomposing Work}\label{sec:decomposition}

At its core, on--demand work has been enabled by decomposition of large goals into many small tasks. As such, one of the central questions in the literature is how to design these microtasks, and which kinds of tasks are amenable to decomposition. In this section, we place these questions in the context of piecework's Tayloristic evolution.

\begin{comment}
Outline:
Crowd work
	- How do we take work and split it up into smaller work?
	- How small can we go?
	- Once we split up, what happens? Marketplace choosing
Piecework
	- How do we take work and split it up?
	[ - how small can we go? ]
	[- Nothing about how people choose. ]
Same/different
	[- task data science ]
	[- Cognitive barriers, task switching, etc, dominates]
	[ - Switching is different, marketplace model ]
\end{comment}

\subsubsection{\crowdworkpers}
\topic{Many contributions to the design and engineering of crowd work consist of creative methods for decomposing goals.} Even when tasks such as writing and editing cannot be reliably performed by individual workers, researchers demonstrated that decompositions of these tasks into workflows can succeed~\cite{crowdForgeKittur,bernsteinSoylent,writingMicroTasks,Nebeling:2016:WCW:2858036.2858169}. 
These decompositions typically take the form of workflows, which are algorithmic sequences of tasks that manage interdependencies~\cite{Bigham2014}. 
Workflows often utilize a first sequence of tasks to identify an area of focus (e.g., a paragraph topic~\cite{crowdForgeKittur}, an error~\cite{bernsteinSoylent}, or a concept~\cite{Yu2016a,Yu2016b} and a second sequence of tasks to execute work on that area. 
This decomposition style has been successfully applied across many areas, including food labeling~\cite{noronha2011platemate}, brainstorming~\cite{siangliulue2015toward,yu2014distributed}, and accessibility~\cite{lasecki2013chorus,lasecki2012real,Lasecki2011}.

\topic{If decomposition is key to success in crowd work, the question arises: what can, and can't, be decomposed?} 
Moreover, how thinly can work be sliced and subdivided into smaller and smaller tasks? 
The general trend has been that smaller is better, and the microtask paradigm has emerged as the overwhelming favorite
\cite{selfsourcingTeevan2014,selfsourcingTeevan2016}.
This work illustrates a broader sentiment in both the study and practice
of crowd work, that microtasks should be designed resiliently against the variability of workers, preventing a single errant submission from impacting the agenda of the work as a whole %, fully exploiting the abstracted nature of each piece of work
~\cite{interruptionIqbal,delayAndOrderLasecki,vaish2014low}.
In this sense, finer decompositions are seen as more robust --- both to interruptions and errors~\cite{cheng2015break} --- even if they incur a fixed time cost.
At the extreme, recent work has attempted demonstrated microtasks that take seconds~\cite{Vaish:2014:TCC:2611222.2556996,Cai:2015:WLW:2702123.2702267} or even tenths of a second~\cite{embracingErrorKrishna}.
%Earlier we discussed \citeauthor{cheng2015break}'s work measuring the impact that interruption has on worker performance \cite{cheng2015break}.%While \citeauthor{cheng2015break} found costs with breaking tasks into smaller components in the form of higher cumulative time to complete (albeit much shorter real time to complete, owing to parallelization)%
However, workers perform better when similar tasks are strung together 
\cite{delayAndOrderLasecki}, or chained and arranged to maximize  the attention threshold of workers~\cite{Cai:2016:CRI:2858036.2858237}.
Despite this, 
%rather than attempt to minimize the error rates in micro--tasks
%, as \citeauthor{Kinnaird:2012:WTM:2389176.2389219} suggested%
we as a community have leaned \textit{into} the peril of
low--context work, ``embracing error'' in crowdsourcing~\cite{embracingErrorKrishna}.


\topic{The general lesson has been that the more micro the task, and the more fine the decomposition, the greater the risk that workers lose context necessary to perform the work well.}
For example, workers edit adjacent paragraphs in inconsistent ways~\cite{bernsteinSoylent,Kim2017}, interpret tasks in different ways~\cite{kairam2016parting}, and exhibit lower motivation~\cite{Kinnaird:2012:WTM:2389176.2389219} without sufficient context.
Research has sought to ameliorate this issue by designing workflows help workers ``act with global understanding when each contributor only has access to local views''~\cite{verroios2014context}, typically by automatically or manually generating higher--level representations for the workers to reflect on~\cite{chilton2013cascade,verroios2014context,Kim2017}.

% \topic{The crowdsourcing research into work decomposition
% has largely focused on minimizing the additional context necessary to do tasks%
% , and making it easier to do tasks with less time.}
% This first thread is
% perhaps best described by \citeauthor{verroios2014context} as
% making crowd workers ``\dots able to act with
% global understanding when each contributor only has access to local views''
% \cite{verroios2014context}.
% With the exception of a few cases
% (specifically, \citeauthor{Kinnaird:2012:WTM:2389176.2389219}'s work
% which finds that greater work context fosters more reliably high--quality work)%
% , the micro task paradigm has emerged as the overwhelming favorite
% \cite{selfsourcingTeevan2014,selfsourcingTeevan2016%
% ,       cheng2015break,Kinnaird:2012:WTM:2389176.2389219}.


\topic{As the additional context necessary to complete a task diminishes, the invisible labor of \textit{finding} tasks~\cite{martin2014being} has arisen as a major issue.}
\citeauthor{taskSearch} illustrate the task search challenges on AMT.%
Workers seek out good requesters~\cite{martin2014being} and then  ``streak'' to perform many tasks of that same type~\cite{taskSearch}.
%, and some work has gone into ameliorating the problems specific to this work site (\textit{ReLauncher}), % does anyone care for project names?
%while other work designs tasks around gap time (\textit{Twitch Crowdsourcing} \& \textit{Wait--Learning}) \cite{taskSearch,KucherbaevReLauncher,Vaish:2014:TCC:2611222.2556996,       Cai:2015:WLW:2702123.2702267}.

% \topic{One of the emergent properties of micro--tasks has been the relative cost of
%     \textit{finding} worthwhile tasks.}
%     The research community has documented and to some extent attempted to intervene in
%     the discovery of worthwhile tasks
%     \cite{taskSearch}.
Researchers have reacted by designing task recommendation systems (e.g.,~\cite{Cosley:2007:SUI:1216295.1216309})
%attempts to address this by directing workers to tasks through ``intelligent task routing'' \cite{Cosley:2007:SUI:1216295.1216309}.
and others focused on minimizing the amount of time that people need to spend doing anything other than the work for which they are paid~\cite{callison2014crowd}.



\subsubsection{\pieceworkpers}
\citeauthor{Brown01041990} inquired from another direction, asking
what limited the adoption of piecework in industries that otherwise gravitated toward it
(in the case studies he examined, this mostly focused on railway engineers),
ultimately arguing that factors such as the nature of the work design
(specifically, the homogeneity of tasks) and the costs associated with adopting a piecework model
were the major contributing factors that determined the use of piecework
\cite{Brown01041990}.

\topic{Piecework became an important factor in the war effort for the Second World War,
cementing its role not only in American factories, but in industrial work around the world.}
The 1930s represented a boom for piecework on an unprecedented scale,
especially among engineering and metalworking industries.
As discussed earlier, \citeauthor{hart2013rise} characterize the 1930s
--- and more broadly the first half of the \nth{20} century ---
as the ``heyday'' of the use of piecework.
He attributes this to the shortage of male workers,
who would have gone through a conventional apprenticeship process
affording them more comprehensive knowledge of the total scope of work.
One might reflect on the observation that ``Rosie the Riveter'',
an icon of \nth{20} century America who represented empowerment and opportunity for women \cite{honey1985creating},
was a pieceworker
\cite{davies2014origins}.



\topic{The research community relating to
piecework and labor
has been wrestling with the decomposition of work for centuries.}
The beginnings of
systematic task decomposition
stretch back as far as the \nth{19} century,
when Airy employed young boys at the Greenwich Observatory who
``possessed the basic skills of mathematics, including
`Arithmetic, the use of Logarithms, and Elementary Algebra'~''
to compute astronomical phenomena
\cite{grier2013computers}.
The work that Airy solicited resonates with modern crowd work for several reasons.
First, work output was quickly verifiable;
Airy could assign variably skilled workers to compute values,
and have other workers check their work.
Second, tasks were discrete --- that is, independent from one another.
Finally, workers could be trained on a very narrow subset of mathematical skills to be
sufficiently qualified to do this work.

This approach found its audience in the early \nth{20} century with the rise of Fordism and scientific management (or Taylorism).
Scientific management suggested that it was possible to measure work  at unprecedented resolution and precision.
As \citeauthor{Brown01041990} points out, piecework most greatly benefits the instrumented measurement of workers, but certainly in Ford and Taylor's time, highly instrumented, automatic measurement of workers was all but impossible~\cite{Brown01041990}.
As a result, the distillation of work into smaller units ultimately bottomed out with tasks as small as could be usefully measured~\cite{10.2307/23702539}.
\msb{This subsection is too shallow and needs a bit more. For example, can you give examples of the last point?}


\topic{Piecework researchers enumerate a number of problems with the decomposition of work, and the conflicting pressures managers and workers put forth.}
\citeauthor{bewley1999wages} in particular points out that
the approach of paying workers by the piece is
``\dots~not practical for workers doing many tasks, because of
the cost of establishing the rates and because
piecework does not compensate workers for time spent switching tasks''.
Ultimately, \citeauthor{bewley1999wages} argues that
``[piecework is] infeasible, because \dots
total output is the joint product of varying groups of people''
\cite{bewley1999wages}.


\subsubsection{\whatchanged}
\begin{comment}
outline
	- measurement is more precise, so decomposition is deeper
	- not a single position, but a marketplace
\end{comment}

\topic{Where measurement and instrumentation were limiting factors for historical piecework, computation has changed the situation so that a dream of scientific management and Taylorism
--- to measure every motion at every point throughout the workday and beyond ---
is not only doable, but trivial
\cite{waltz2012quantified}.}
Where \citeauthor{10.2307/23702539} directly implicates measurement as
% cites as
preventing scientific management from being fully utilized,
%the difficulty of tracking work \& workers, no longer exists
%\cite{10.2307/23702539}.}
modern crowd work is measuring and modeling every click, scroll, and keyboard event
\cite{rzeszotarski2011instrumenting,rzeszotarski2012crowdscape}.
The result is that on--demand work can articulate and track far more carefully than piecework historically could.

\topic{A second shift is  
the relative ease with which the metaphorical ``assembly line'' can be changed.}
Historical manufacturing equipment could not quickly be assembled, edited, and redeployed
\cite{hu1961parallel}.
In contrast, today system--designers can share, modify, and instantiate environments
like sites of labor in a few lines of code
\cite{lessig2006code,turkitLittle}.
This opportunity has spurred an entire body of work investigating the effects of
ordering,
pacing,
interruptions, and
other factors in piecework that would have been
all but impossible to manipulate as few as 20 years ago
\cite{dai2015and,Cai:2016:CRI:2858036.2858237,cheng2015break,measuringCrowdsourcingCheng,embracingErrorKrishna}.
 
Third, modern crowd work has sliced work to such small scales that the marginal activities
--- things like finding work and cognitive task switching ---
have become large relative to the tasks themselves
\cite{taskSearch}.
In the historical case of piecework,
moving metallurgical tools, mining equipment, or
other industry materials would have been prohibitively difficult and slow;
workers were encouraged to specialize in a single set of tasks,
allowing pieceworkers to sequence their tasks optimally on their own
\cite{hart2013rise}.
The result is that crowd workers are more free agents than historically was the case.
However, because they spend significant time searching for tasks, the piece rate is less a good estimate of take--home earnings than before.


\subsubsection{\implication}
If measurement precision limited the depth of decomposition for piecework historically, as \citeauthor{10.2307/23702539} argues, then modern on--demand work stands to become far more finely--sliced and highly decomposed than ever before.
Online tools make measurement and validation so easy~\cite{rzeszotarski2011instrumenting} that these aspects of piecework are solved, or near enough that they no longer limit task decomposition.
Now, not just tasks, but entire workers' histories~\cite{hata2017glimpse}, can be collected and analyzed in detail.

However, decomposition has hit a second bottleneck: cognition. 
Task switching costs and other cognitive costs make it difficult to work tasks so far decontextualized from their original intention \cite{delayAndOrderLasecki}.
There will of course be tasks that can be decomposed without much context, and these will form the most fine--grained of microtasks.
However, other tasks cannot be freed from context --- for example, logo design requires a deep understanding of the client and their goals.
In part due to this limitation, 99designs workers often recycle old designs rather than make new ones for each client~\cite{araujo201399designs}.

So, ultimately, the levels of decomposition are likely to follow the contours of context required. Low--context work will be extremely highly decomposed. High--context work will continue to be limited.


% \onlyinsubfile{
%   \balance{}
%   \printbibliography
% }

\end{document}