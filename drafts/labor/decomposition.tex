\documentclass[trackingWork]{subfiles}
\makeatletter
\def\blx@maxline{77}
\makeatother
\begin{document}

\subsubsection[the decomposition of work]{The Decomposition of Work}\label{sec:decomposition}

\subsubsubsection{\crowdworkpers}
% \ali{Does this belong here at all? The complexity argument just dives in.}
\topic{The crowdsourcing research into work decomposition
has largely focused on minimizing the additional context necessary to do tasks%
, and making it easier to do tasks with less time.}
This first thread is
perhaps best described by \citeauthor{verroios2014context} as
making crowd workers ``\dots able to act with
global understanding when each contributor only has access to local views''
\cite{verroios2014context}.
With the exception of a few cases
(specifically, \citeauthor{Kinnaird:2012:WTM:2389176.2389219}'s work
which finds that greater work context fosters more reliably high--quality work)%
, the micro task paradigm has emerged as the overwhelming favorite
\cite{selfsourcingTeevan2014,selfsourcingTeevan2016%
,       cheng2015break,Kinnaird:2012:WTM:2389176.2389219}.


\topic{As the additional context necessary to complete a task diminishes%
, the marginal cost of finding and
\textit{doing} tasks has increasingly become the focus of research.}
\citeauthor{taskSearch} illustrate the challenges on AMT%
, and some work has gone into ameliorating the problems specific to this work site
(\textit{ReLauncher}), % does anyone care for project names?
while other work designs tasks around gap time
(\textit{Twitch Crowdsourcing} \& \textit{Wait--Learning})
\cite{taskSearch,KucherbaevReLauncher,Vaish:2014:TCC:2611222.2556996%
,       Cai:2015:WLW:2702123.2702267}.
Yet more work looks at the general framing of tasks%
, chaining and arranging them to maximally exploit
the attention and stress threshold % ???
of workers
\cite{Cai:2016:CRI:2858036.2858237}.
Rather than attempt to minimize the error rates in micro--tasks%
, as \citeauthor{Kinnaird:2012:WTM:2389176.2389219} suggested%
, we as a community have leaned \textit{into} the peril of
low--context work%
, ``embracing error'' in crowdsourcing
\cite{embracingErrorKrishna}.

Not all of the work toward optimizing crowd work--flows has gone toward
minimizing the creative input of crowd workers;
a thriving body of literature adopts
practices such as pipelining to allow experts to participate in crowd work
\cite{foundry}.


\subsubsubsection{\pieceworkpers}
\topic{The research community relating to
piecework and labor
has been wrestling with the decomposition of work for centuries.}
The beginnings of
systematic task decomposition
stretch back as far as the \nth{17} century%
, when Airy employed young boys at the Greenwich Observatory who
``possessed the basic skills of mathematics, including
`Arithmetic, the use of Logarithms, and Elementary Algebra'~''
to compute, by hand, astronomical phenomena
\cite{grier2013computers}.
These workers became the first \textit{computers}.

The work Airy solicited was interesting for several reasons.
First, work output was quickly verifiable;
Airy could assign variably skilled workers to compute values%
, and have other workers check their work.
\ali{I could point out that the opportunity to check work and
repeat the task is a little like find--fix--verify, but is that jumping the gun?}
Second, tasks were discrete --- that is, independent from one another.
Finally, knowledge of the full scope of the project
--- indeed, knowledge of anything more than the problem set at hand ---
was unnecessary.

The insight of breaking tasks down into smaller components didn't find its audience until
the early \nth{20} century%
, with the rise of Fordism and scientific management (or Taylorism).
From scientific management, we found that
we could measure work at unprecedented resolution and precision.
As \citeauthor{Brown01041990} points out%
, piecework most greatly benefits the instrumented measurement of workers, but certainly
in Ford and Taylor's time --- and certainly in Airy's time ---
highly instrumented, automatic measurement of workers was all but impossible.
As a result%
, the distillation of work into smaller chunks
ultimately reached a limit of usefulness.


\ali{\citeauthor{marx2012economic} in here? Alienated from the context of the work?
Critiques about piecework marginalizing workers?
\citeauthor{american1921problem} wrote about this as well
(albeit they were advocating from workers' perspectives, but who isn't?)}

\subsubsubsection{\whatchanged}
\begin{inlinelist}
  \item Computers make it possible to switch from one task to another
  unlike any arbitrary manufacturing factory possibly could;
  \item we've sliced works to such small sizes that the marginal costs
  --- things like task--finding, cognitive load switching, etc\dots ---
  have become relatively large;
  \item instrumentation has become so advanced that
  the curve of diminishing value on measuring and tracking workers has shifted significantly
  (but not been obliterated);
  \ali{some companies have suggested self--tracking through
  programs that give workers fitbits and whatnot --- I could make the argument that
  this is just an illustration of that, but
  it's not really about work \textit{per se} unless
  you think about it as the general management of workers. Thoughts?}
\end{inlinelist}




\onlyinsubfile{
  \balance{}
  \printbibliography
  
  % \clearpage
  % \nobalance{}
  }
  \endnotetext{
    \section{Slicing notes}
    The finding here is that
    crowdwork can be more carefully micro--managed than piecework could be%
,     and that this is a double--edged sword:
    we can effectively give feedback to workers on everything they do%
,     but this is emboldening us to try to
    over--manage workers just as piecework tried to do.

    My goal for this section is to make two points:
    \begin{enumerate}
      \item show how this is related to the assembly line and scientific management%
,             and how piecework literature tried to measure everything%
,             but found it untenable given the extra equipment that was necessary
            (but generally which didn't exist)
            to track every movement and action that workers took.
      \item show how this work was enabled by the ``verifiability'' of work output(?)
    \end{enumerate}

    As a result, the prevalent mindset of designing work for crowd workers
    --- one which treats micro task workers as
    ``modular, protocol--defined computational services'' ---
    has inexorably alienated workers from the greater context of their work
    \cite{irani2015cultural}.


    \ali{i don't like this section \textit{here}, but i like it in general. what do?}
    \topic{One of the emergent properties of micro--tasks has been the relative cost of
    \textit{finding} worthwhile tasks.}
    The research community has documented and to some extent attempted to intervene in
    the discovery of worthwhile tasks
    \cite{taskSearch}.
    \citeauthor{Cosley:2007:SUI:1216295.1216309}
    attempts to address this by
    directing workers to tasks through
    ``intelligent task routing''
    \cite{Cosley:2007:SUI:1216295.1216309}.
    Much of this work and the work at the periphery of this space, then%
,     has focused on
    minimizing the amount of time that people need to spend doing
    anything other than the work for which they are paid.

    What we take away from this and the previous set of work is that
    the value of adopting crowdsourcing for any particular task
    seems to be mediated by two questions:
    \begin{inlinelist}
      \item How long does it take to train workers to do the work in question? and
      \item How long does it take for the worker to do the work?
    \end{inlinelist}
    Minimizing these criteria has become
    the overarching motivation of the crowdsourcing work design community
    \cite{cheng2015break,Newell:2016:OMA:2858036.2858490}


  }

\end{document}