\documentclass[trackingWork]{subfiles}
\onlyinsubfile{
  \usepackage{xr-hyper}
  \usepackage{hyperref}
  \externaldocument{complexity}
  \externaldocument{relationships}
  \externaldocument{decomposition}
}
\begin{document}




\section{A Review of Piecework}

\topic{The HCI community has used the term ``piecework'' to describe
myriad instantiations of on--demand labor,
but this reference has generally been offered in passing.}
As this paper principally traces a relationship between
the historical piecework and the contemporary crowd work
(or on--demand labor more generally),
this casual familiarity with piecework may prove insufficient.
We'll more carefully discuss piecework in this section in order
to inform the the rest of the argument.
Specifically, we will
\begin{inlinelist}
  \item define ``piecework'' as researchers in the topic understood it; and
  \item trace the rise and fall of piecework at a very high level,
        identifying key figures and ideas during this time;
\end{inlinelist}
This section is not intended to be comprehensive: instead, it sets up the scaffolding necessary for our later investigations of crowd work's three questions: complexity limits, task decomposition, and worker relationships.


\subsection{What is piecework?}
\topic{Aligning on--demand work with piecework requires an understanding of what piecework is.}
While ``piecework'' has had multiple definitions over time,
we can trace a constellation of characteristics that recur throughout the literature.
We will follow this history of research, collecting
descriptions,
examples, and
provided definitions of piecework, trying
to trace the outline of a working understanding of
\textit{what piecework is}.

\citeauthor{hughRaynbirdTaskWork} offers
a concise definition of piecework
--- which he variously also calls ``measure work'', ``grate work'', and ``task work'' ---
by contrasting the ``task--labourer'' with the ``day--labourer'':
``the chief difference lies between the day--labourer,
who receives a certain some of money\dots~for his day's work,
and the task--labourer, whose earnings depend on the \textit{quantity} of work done [emphasis added]''~\cite{hughRaynbirdTaskWork}.
\citeauthor{10.2307/2338394} gives a more illustrative definition of piecework,
offering examples:
``payment is made for each hectare which is pronounced to be well ploughed~[\dots]~
for each living foal got from a mare;~[\dots]~
for each living calf got''~\cite{10.2307/2338394}.
This framing perhaps makes the most intuitive sense;
``payment for results,'' as \citeauthor{10.2307/2338394} calls it,
is not only common in practice, but well--studied in labor economics as well~\cite{Figlio2007901,weitzman1976new,10.2307/3003414,BJIR:BJIR038}.

\topic{It's worth acknowledging that
``this distinction [between piece--rates and time--rates] was not completely clear--cut''~\cite{hart2013rise}.}
Indeed, work adopted piece--rate compensation in some aspects and
time--rate compensation in others.
The ``Rowan premium system'',
which essentially paid workers
a base rate for time plus
(the potential for) an additional pay dependent on output,
was just one of several alternatives to stricter time-- and piece--rate renumeration paradigms, which
muddies the waters for us later as we attempt to categorize cases of piecework~\cite{rowan1901premium}.
As \citeauthor{rowan1901premium}'s premium system guaranteed an hourly rate
regardless of the worker's productive output
\textit{as well as} an additional compensation tied to performance,
workers under this regime were
in some senses ``task--labourers'', and
in other senses (more conventional) ``day--labourers''.

\topic{It may be worth thinking about piecework through the lens of its \textit{emergent} properties to help understand it.}
Returning to
\citeauthor{hughRaynbirdTaskWork}, several arguments for the merits of piecework
crop up; he points out that 
``piece work holds out to the labourer an increase of wages as a reward for his skill and exertion~[\dots]~
he knows that all depends on his own diligence and perseverance~[\dots and]~
so long as he performs his work to the satisfaction of his master,
he is not under that control to which the day--labourer is always subject.''
\citeauthor{hughRaynbirdTaskWork} highlight the freedom from control that ``task--labourers'' enjoy~\cite{hughRaynbirdTaskWork,rowan1901premium}.

\topic{We see this sense of independence regardless of the time, locale, and industry.}
\citeauthor{10.2307/3827491} offers a look into the lives and culture of ``match--girls''
--- young women paid by piecework to assemble matchsticks generally in the late \nth{19} century.
Of particular interest was their independent nature, via their reputation ``\dots~for generosity, independence, and protectiveness,
but also for brashness, irregularity, low morality, and little education''~\cite{10.2307/3827491}.
\citeauthor{10.2307/27508091} documents piecework from 1850--1930 in Australia,
finding similar assertions of the freedom compositors of newspapers experienced as pieceworkers:
``If a piece--work compositor who held a `frame' decided that he did not want to work on a particular day or night,
the management recognised his right to put a `substitute' or `grass' compositor in his place''~\cite{10.2307/27508091}.
From these accounts we identify a sense of independence and autonomy that resonates across decades, industries, and locales where piecework is found.

\topic{Piecework opened the door for people who previously couldn't participate in the labor market --- for example due to lack of training --- to do so, and to acquire job skills incrementally.}
For example, women could receive training in narrow subsets of the general body of skills, enabling them to act in capacities similar to what conventional (male) apprentices would undertake~\cite{hart2013rise}.
In addition, workers with specific slices of skills could be matched to suitable tasks.
Workers without conventional training
--- like women, who had no such opportunities
to engage in engineering and metalworking apprenticeships as men did ---
could be trained very narrowly on a very tightly constrained task,
demonstrate proficiency, and become experts in their own ways.

In summary, piecework:
\begin{inlinelist}
  \item paid workers for quantity of work done, rather than time done,
        but occasionally mixed the two payment models;
  \item afforded workers freedom in when and how much to work; and
  \item structured tasks such that people who didn't have the training
        to engage in the traditional labor force could still participate.
\end{inlinelist}

Viewing crowd work as a modern instantiation of piecework is relatively straightforward by this definition.
\begin{Numberlist}
\item platforms such as Mechanical Turk, Uber and TaskRabbit pay by the task, though some such as Upwork do offer hourly rates as well.
\item workers are attracted to these platforms by the freedom they offer to pick the time and place of work~\cite{martin2014being,whyWouldAnyoneBrewer}.
\item system developers as on Mechanical Turk typically assume no professional skills in transcription or other areas, and attempt to build that expertise into the workflow~\cite{noronha2011platemate,bernsteinSoylent}.
\end{Numberlist}
Given this alignment, many of the same properties of piecework historically will apply to on--demand work as well. In the next section, we perform this application to three of the major questions in crowd work and gig work, identifying similarities and differences between historical piecework and modern on--demand work.




\subsection{A Piecework Primer}\label{sec:pieceworkPrimer} % ali adores alliteration
\topic{In this section we will offer a brief overview of the history of piecework;
this should not be mistaken for a comprehensive background.}
Instead, this section will attempt to provide a sense of orientation when thinking about piecework.
In other words, it will frame piecework in the contexts of
the early days of the Industrial Revolution,
through the political and economic turmoil of the early and mid--\nth{20} century,
and into the \nth{21} century.
While the previous section provided a \textit{definition} of piecework,
this section attempts to shine a light on the \textit{zeitgeist} of piecework.


\topic{Piecework's history traces back further perhaps than most would expect.}
\citeauthor{grier2013computers} describes the process astronomers adopted of hiring young boys
to calculate equations in order
to better--predict the trajectories of various celestial bodies in the \nth{19} century~\cite{grier2013computers}.
George Airy was perhaps the first to rigorously apply piecework--style decomposition of tasks to work;
by breaking complex calculations into constituent parts, and
training young men to solve simple algebraic problems,
Airy could distribute work to many more people than could otherwise complete the full calculations.


\topic{Piecework may have started in the intellectual domain of astronomical calculations and projections,
but it found its foothold in manual labor.}
Piecework took off on in farm work~\cite{hughRaynbirdTaskWork},
in textiles~\cite{restructuringPieceworkBaker,riisOtherSideLives},
on railroads~\cite{Brown01041990}, and 
elsewhere in manufacturing~\cite{10.2307/3827491}.
Fordism and scientific management thrust piecework into higher gear, especially as
mass manufacturing and
a depleted wartime workforce forced industry to find new ways to eke out more production capacity.
\citeauthor{hart2013rise} point out that the Second World War,
which called millions of Americans to military service,
necessitated the rapid training and employment of
a labor pool that hadn't historically been utilized in industrial labor: women~\cite{hart2013rise}.

The early growth of piecework led to discussion surrounding how best to manage pieceworkers~\cite{norton1900textile,clark1908cotton}.
Despite this, workers' needs were mostly ignored,
leading to frustration over poor working and living conditions
(famously documented by
\citeauthor{riisOtherSideLives})~\cite{riisOtherSideLives}.
Discontent reached a crescendo when industry organizations representing
railway workers, mechanical engineers, and other industries began to speak out on pieceworkers' behalves~\cite{american1921problem,richards1904anything}.

\topic{Piecework's popularity in the United States and Europe plummeted almost as quickly as it had climbed.}
\citeauthor{hart2013rise}'s work substantively explores the precipitous decline of piecework in the last third of the \nth{20} century.
In their work, \citeauthor{hart2013rise} offer a number of explanations for the sudden vanishing of piecework.
The salient suggestions include:
\begin{inlinelist}
\item the emergence of more effective, more nuanced incentive models
--- rewarding teams for complex achievements, for instance;
\item the shifting of these industries (manufacturing, clothing, etc\dots)
to other countries; and
\item the quality of ``multidimensional'' work becoming too difficult to evaluate
\end{inlinelist}~\cite{hart2013rise}.


\subsection{Case studies in piecework}
We'll discuss the context and factual details of three instructive cases of piecework to inform the three questions we investigate later ---
       % (\namerefl{sec:complexity}, \namerefl{sec:decomposition}, and \namerefl{sec:relationships}).}
\begin{numberlist}
  \item Airy's employment of \textit{human computers},
  \item domestic and farm work (in particular, the ``match--girls''), and
  \item railroad workers across the United States.
\end{numberlist}


\ali{I'm coming out of this with a sort of structure that \textit{loosely} describes piecework chronologically.
Strictly speaking, I'm talking about each of these topics
(let's call them
\textbf{human computation},
\textbf{domestic and farm work}, and
\textbf{industrial work}),
and some of these things overlap with each other to some extent, but
for the most part
Airy and the human computers came in the mid--\nth{19} century,
farm work and the ``domestic'' work (like making matches) came around at the turn of the \nth{20} century, and
the industrial work (railroad workers, the WWII war effort, and the rise of the labor union movement) unfolded from the early \nth{20} century onward.}





\subsubsection{Airy's Computers}
\topic{Some of the first systematic cases of what we would recognize as crowd work
can be found in the study of astronomy, in the form of George Airy's ``human computers''.}
% {\citeauthor{grier2013computers}~\cite{grier2013computers} gives one of the earliest accounts of
% piecework, found in George Airy's creation of
% the British Nautical Almanac.}
Airy needed to compute tables that would
allow sailors to locate themselves by starlight from sea.
This work ostensibly called for educated people who comprehensively understood mathematics.
Airy realized that he could break the tasks down and delegate the constituent parts
to ``human computers'' who
``\dots~possessed the basic skills of mathematics,
including `Arithmetic, the use of Logarithms, and Elementary Algebra'~''~\cite{grier2013computers}.
As a result, many of Airy's computers had relatively rudimentary educations
compared to the background of education that typically worked in the calculation of solar tables.
Airy distributed tasks by mail,
allowing work to be completed by a somewhat geographically distributed workforce,
and paid for each piece of work completed.

\topic{Airy's method of distributing work was unorthodox, but his workforce management style was especially unconventional.}
For one thing,
Airy had set a policy of firing his computers once they reached age 23.
This practice ensured two outcomes that arguably disfavored workers.
\begin{Numberlist}
\item it eliminated any potential to advance professionally, as workers' careers in this area ended relatively early in their careers,
      and without formal education in mathematics they struggled to find work for which their experience was meaningful.
\item it limited workers' ability to organize
      by ensuring that workers hardly spent sufficient time to successfully rally their peers.
\end{Numberlist}
% \ali{I'm looking into \citeauthor{grier2013computers}'s book to see if there's more about the relationships workers had with Airy.
% Maybe I can/should expand this to discuss other implementations of human computation around this time (if I find any good anecdotes).}





\subsubsection{Domestic and Farmhand Labor}
\topic{When piecework entered the American economy, it was not used for complex work.}
One reason for low complexity was workers' skills:
it was infeasible to provide new pieceworkers with the comprehensive education
that apprenticeships imparted~\cite{hart2013rise}.
So, initially piecework in the United States arose for farm work, and as
\citeauthor{hughRaynbirdTaskWork} and others discuss,
the practice remained relatively obscure until
it blossomed in the textile industry~\cite{hughRaynbirdTaskWork}.
The complexity of the work remained low at the turn of
the \nth{20} century as piecework saturated New York City~\cite{riisOtherSideLives}.
However, writers of the time focused their attention on
wage~\cite{burton1899commercial} and
management regimes~\cite{norton1900textile}
rather than training.
Mass manufacturing,
such as sewing garments~\cite{riisOtherSideLives} and making matchsticks~\cite{10.2307/3827491},
flourished under piecework systems in densely populated cities.

\topic{Workers' relationships with employers quickly soured.}
The match--girls strike of 1888 was one of the earliest and most famous successful worker strikes,
and perhaps the beginning of ``militant trade unionism''~\cite{10.2307/3827491}.
As \citeauthor{weyer1894history} described,
``the match--girls' victory turned a new leaf in Trade Union annals''~\cite{weyer1894history}: in the 30 years after the match--girls strike,
the Trade Union Movement enrollment grew from 20\% of eligible workers to over 60\%.

\topic{The match--girls strike foreshadowed both collective action victories and an emerging paradigm regarding worker management.}
In 1912, national coal miners effected an individual minimum wage after The Great Coal Strike of 1912~\cite{10.2307/2221944}.
Women in the garment industry in Philadelphia secured collective bargaining rights in 1915 after a prolonged strike and threat of a second~\cite{10.2307/41829256}.
It was in the midst of this time that \citeauthor{taylor1914principles} published
the work for which he would later be called ``the father of scientific management''~\cite{RePEc:mtp:titles:0262612062}.
% and it was this framing on work that
It was this framing on work \& worker management that gave workers a concrete adversary
--- if not Taylor himself, then \textit{Taylorism}  ---
against which to rally~\cite{jacoby1983union,parker1920casual}.





\subsubsection{Railroad and other Industrial Workers}
\topic{With \citeauthor{taylor1914principles}'s formalization of scientific management in \textit{Taylorism}
(and Henry Ford's parallel, but coincidental, eponymously named \textit{Fordism}),
piecework found a new audience in the early and mid--\nth{20} century in industrial work.}
Scientific management argued that it was possible to measure work at
unprecedented resolution and precision~\cite{taylor1914principles,towardsGlobalFordism}.
As \citeauthor{Brown01041990} points out,
piecework most greatly benefits the instrumented measurement of workers,
% but certainly in Ford and Taylor's time,
but in \citeauthor{taylor1914principles}'s time highly instrumented,
automatic measurement of workers was all but impossible~\cite{Brown01041990}.
Instead, managers conducted ``stop watch time studies'' \cite{nadworny1955scientific}
as rudimentary guides on per--task compensation.
This approach led to the adoption of piecework systems in myriad industrial labor verticals, such as
railroad maintenance, automobile manufacturing, and later weapons manufacturing.
\ali{???? better word for ``weapons manufacturing''?}
% Instead, as a result,
% the distillation of work into smaller units ultimately
% bottomed out with tasks as small as could be usefully measured~\cite{10.2307/23702539}.


% \topic{With the formalization of scientific management,
% piecework became an important component of the mass manufacturing
% in American factories.}
% \topic{}


\topic{The 1930s represented a boom for piecework on an unprecedented scale,
especially among engineering and metalworking industries.}
\citeauthor{hart2013rise} characterize the 1930s
--- and more broadly the first half of the \nth{20} century ---
as the ``heyday'' of the use of piecework.
They attribute this to the shortage of male workers,
who would have gone through a conventional apprenticeship process
affording them more comprehensive knowledge of the total scope of work.
One might reflect on the observation that ``Rosie the Riveter'',
an icon of \nth{20} century America who
represented empowerment and opportunity for women~\cite{honey1985creating},
was a pieceworker~\cite{davies2014origins}.







\onlyinsubfile{
  \printbibliography
  }

\end{document}