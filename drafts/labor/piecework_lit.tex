\documentclass[trackingWork]{subfiles}
\onlyinsubfile{
  \usepackage{xr-hyper}
  \usepackage{hyperref}
  \externaldocument{complexity}
  \externaldocument{relationships}
  \externaldocument{decomposition}
}
\begin{document}




\section{A Review of Piecework}

\topic{The HCI community has used the term ``piecework'' to describe
myriad instantiations of on--demand labor,
but this reference has generally been offered in passing.}
As this paper principally traces a relationship between
the historical piecework and the contemporary crowd work
(or on--demand labor more generally),
this casual familiarity with piecework may prove insufficient.
We'll more carefully discuss piecework in this section in order
to inform the the rest of the argument.
Specifically, we will
\begin{inlinelist}
  \item define ``piecework'' as researchers in the topic understood it; and
  \item trace the rise and fall of piecework at a very high level,
        identifying key figures and ideas during this time;
  %        and finally
  % \item look at the fall of piecework, such as it was,
  %       considering in particular
  %       the factors that may have led to piecework's
  %       eventual demise in the American and European labor markets.
        % \msb{did it demise? there are sweatshops around the world today}.
        % \ali{is it okay to corner it off like this (by just being about the US/EU)?}
\end{inlinelist}
This section is not intended to be comprehensive: instead, it sets up the scaffolding necessary for our later investigations of crowd work's three questions: complexity limits, task decomposition, and worker relationships.

% While piecework has been studied from myriad perspectives through the lenses of
% political science,
% economics (both micro-- and macro--), and
% a plethora of other fields
% \ali{should I cite examples of all of these?}, for the sake of brevity
% we will discuss the history of piecework in the contexts of the three questions we brought up with crowd work:
% ``\namerefl{sec:complexity}'',
% ``\namerefl{sec:decomposition}'', and 
% ``\namerefl{sec:relationships}''.


\subsection{What is piecework?}
\topic{Aligning on--demand work with piecework requires an understanding of what piecework is.}
While ``piecework'' has had multiple definitions over time,
we can trace a constellation of characteristics that recur throughout the literature.
We will follow this history of research, collecting
descriptions,
examples, and
provided definitions of piecework, trying
to trace the outline of a working understanding of
\textit{what piecework is}.

\citeauthor{hughRaynbirdTaskWork} offers
a concise definition of piecework
--- which he variously also calls ``measure work'', ``grate work'', and ``task work'' ---
by contrasting the ``task--labourer'' with the ``day--labourer'':
``the chief difference lies between the day--labourer,
who receives a certain some of money\dots~for his day's work,
and the task--labourer, whose earnings depend on the \textit{quantity} of work done [emphasis added]''~\cite{hughRaynbirdTaskWork}.
\citeauthor{10.2307/2338394} gives a more illustrative definition of piecework,
offering examples:
``payment is made for each hectare which is pronounced to be well ploughed~[\dots]~
for each living foal got from a mare;~[\dots]~
for each living calf got''~\cite{10.2307/2338394}.
This framing perhaps makes the most intuitive sense;
``payment for results,'' as \citeauthor{10.2307/2338394} calls it,
is not only common in practice, but well--studied in labor economics as well~\cite{Figlio2007901,weitzman1976new,10.2307/3003414,BJIR:BJIR038}.

\topic{It's worth acknowledging that
``this distinction [between piece--rates and time--rates] was not completely clear--cut''~\cite{hart2013rise}.}
Indeed, work adopted piece--rate compensation in some aspects and
time--rate compensation in others.
The ``Rowan premium system'',
which essentially paid workers
a base rate for time plus
(the potential for) an additional pay dependent on output,
was just one of several alternatives to stricter time-- and piece--rate renumeration paradigms, which
muddies the waters for us later as we attempt to categorize cases of piecework~\cite{rowan1901premium}.
As \citeauthor{rowan1901premium}'s premium system guaranteed an hourly rate
regardless of the worker's productive output
\textit{as well as} an additional compensation tied to performance,
workers under this regime were in some senses ``task--labourers'',
and in other senses
(more conventional)
``day--labourers''.

\topic{It may be worth thinking about piecework through the lens of its \textit{emergent} properties to help understand it.}
Returning to
\citeauthor{hughRaynbirdTaskWork}, several arguments for the merits of piecework
crop up; he points out that 
``piece work holds out to the labourer an increase of wages as a reward for his skill and exertion~[\dots]~
he knows that all depends on his own diligence and perseverance~[\dots and]~
so long as he performs his work to the satisfaction of his master,
he is not under that control to which the day--labourer is always subject.''
\citeauthor{hughRaynbirdTaskWork} highlight the freedom from control that ``task--labourers'' enjoy~\cite{hughRaynbirdTaskWork,rowan1901premium}.

\topic{We see this sense of independence regardless of the time, locale, and industry.}
\citeauthor{10.2307/3827491} offers a look into the lives and culture of ``match girls''
--- young women paid by piecework to assemble matchsticks generally in the late \nth{19} century.
Of particular interest was their independent nature, via their reputation ``\dots~for generosity, independence, and protectiveness,
but also for brashness, irregularity, low morality, and little education''~\cite{10.2307/3827491}.
\citeauthor{10.2307/27508091} documents piecework from 1850--1930 in Australia,
finding similar assertions of the freedom compositors of newspapers experienced as pieceworkers:
``If a piece--work compositor who held a `frame' decided that he did not want to work on a particular day or night,
the management recognised his right to put a `substitute' or `grass' compositor in his place''~\cite{10.2307/27508091}.
From these accounts we identify a sense of independence and autonomy that resonates across decades, industries, and locales where piecework is found.
% We'll problematize this supposed advantage as we trace the history of piecework,
% but for now we can say that piecework affords
% independence and some sense of autonomy
% new to people in the working class.

\topic{Piecework opened the door for people who previously couldn't participate in the labor market --- for example due to lack of training --- to do so, and to acquire job skills incrementally.}
For example, women could receive training in narrow subsets of the general body of skills, enabling them to act in capacities similar to what conventional (male) apprentices would undertake~\cite{hart2013rise}. 
% offer another series of compelling insights toward
% the question of the features that sprout from piecework.}
% In their reflection on the features endemic to piecework in the 1930s,
% which they describe as the ``heyday'' of piecework's prominence;
% among them were the following:
% \begin{inlinelist}
%   \item ``female workers who generally had less training'' had to be trained in
%         narrower subsets of the general body of skills that
%         conventional (male) apprentices would undertake, and
In addition, workers with specific slices of skills could be matched to suitable tasks.
Workers without conventional training
--- like women, who had no such opportunities
to engage in engineering and metalworking apprenticeships as men did ---
could be trained very narrowly on a very tightly constrained task,
demonstrate proficiency, and become experts in their own ways.

In summary, piecework:
\begin{enumerate}
  \item paid workers for quantity of work done, rather than time done,
        but occasionally mixed the two payment models;
  \item afforded workers freedom in when and how much to work; and
  \item structured tasks such that people who didn't have the training
        to engage in the traditional labor force could still participate.
\end{enumerate}
% \ali{woah i thought this was a comment; is it okay as is?}

Viewing crowd work as a modern instantiation of piecework is relatively straightforward by this definition. 
First, platforms such as Mechanical Turk, Uber and TaskRabbit pay by the task, though some such as Upwork do offer hourly rates as well. 
% Second, workers on these platforms have significant freedom in choosing which tasks to work on, and when to log on and perform work~\cite{martin2014being,whyWouldAnyoneBrewer}. 
% Finally, computational workflows have successfully scaffolded many complex tasks into a series of small microtasks that can be performed without expertise.
%  \subsection{Piecework and crowd work}
% \msb{I assume this is forthcoming?}
% Using the definition of piecework that we came up with earlier, we argue that
% crowd work is fundamentally an instantiation of piecework, and
% that we can more precisely anticipate the answers to the open research questions we discussed earlier.
% We'll show that the dimensions of crowd work that the broader HCI community has been studying
% align with the history of piecework, and that this can greatly inform predictions about the future of crowd work.
% \subsection{From piecework to on--demand work}
% Crowd work and gig work are fundamentally an instantion of piecework.
% First, workers on platforms such as Mechanical Turk and Uber are generally incentivized by unit of work, even if some may be offered an hourly base salary as well.
Second, workers are attracted to these platforms by the freedom they offer to pick the time and place of work~\cite{martin2014being,whyWouldAnyoneBrewer}.
Third, system developers as on Mechanical Turk typically assume no professional skills in transcription or other areas, and attempt to build that expertise into the workflow~\cite{noronha2011platemate,bernsteinSoylent}.
Given this alignment, many of the same properties of piecework historically will apply to on--demand work as well. In the next section, we perform this application to three of the major questions in crowd work and gig work, identifying similarities and differences between historical piecework and modern on--demand work.

% \ali{REMOVED A BUNCH OF STUFF FROM HERE ONWARD}




\subsection{A Piecework Primer}\label{sec:pieceworkPrimer} % yay rhyming?
\topic{In this section we will offer a brief overview of the history of piecework;
this should not be mistaken for a comprehensive background.}
Instead, this section will attempt to provide a sense of orientation when thinking about piecework.
In other words, it will frame piecework in the contexts of
the early days of the Industrial Revolution,
through the political and economic turmoil of the early and mid--\nth{20} century,
and into the \nth{21} century.
While the previous section provided a \textit{definition} of piecework,
this section attempts to shine a light on the \textit{zeitgeist} of piecework.


\topic{Piecework's history traces back further perhaps than most would expect.}
\citeauthor{grier2013computers} describes the process astronomers adopted of hiring young boys
to calculate equations in order
to better--predict the trajectories of various celestial bodies in the \nth{19} century~\cite{grier2013computers}.
George Airy was perhaps the first to rigorously apply piecework--style decomposition of tasks to work;
by breaking complex calculations into constituent parts, and
training young men to solve simple algebraic problems,
Airy could distribute work to many more people than could otherwise complete the full calculations.

% While this approach didn't become the same economic powerhouse as later examples would,
% Airy and others arguably found the kernel of insight that we pursue throughout this discussion:
% determining the extent to which work can be decomposed, and
% finding the limits of complexity of that decomposed work.
% Airy found that he could train youths in elementary mathematics
% to complete the majority of the calculations he would otherwise have had to solve on his own,
% and that the greater body of work could ultimately be completed sooner
% if he arranged his work appropriately.
% \msb{After reading this paragraph, I don't know what it's supposed to be teaching me.
% What I got out of it is that a bunch of people did piecework, but
% I don't know why or in fact why these are different than the sources we cite earlier.
% Can you hone it?}
% \ali{I wanted that paragraph to be about the rising popularity and application of piecework
% (especially as it approached its ``heyday'' \cite{hart2013rise}),
% coming from humble beginnings as it found its footing.
% Given that intent\dots\\
% Should I \textbf{refactor} or \textbf{rewrite}?}

% \ali{\sout{I'm thinking of
% dropping most of the last paragraph and merging it with the next one. Thoughts?}
% I moved it to \nameref{sec:complexity}}

% \ali{Where do I want to take this? Where are we now?
% \begin{enumerate}
%   \item Piecework started in the \nth{19} century (above). [DONE]
%   \item Piecework went on to be a pretty cool thing in textiles and farming at the turn of the \nth{20} century. [DONE?]
%   \item Piecework went gangbusters through the first half of the century, especially through the Second World War. []
%   \item Piecework died pretty quickly (below). [DONE]
% \end{enumerate}
% Sound good?}

\topic{Piecework may have started in the intellectual domain of astronomical calculations and projections,
but it found its foothold in manual labor.}
Piecework took off on in farm work \cite{hughRaynbirdTaskWork},
in textiles \cite{restructuringPieceworkBaker,riisOtherSideLives},
on railroads \cite{Brown01041990}, and 
elsewhere in manufacturing \cite{10.2307/3827491}.
Fordism and scientific management thrust piecework into higher gear, especially as
mass manufacturing and
a depleted wartime workforce forced industry to find new ways to eke out more production capacity.
\citeauthor{hart2013rise} point out that the Second World War,
which called millions of Americans to military service,
necessitated the rapid training and employment of
a labor pool that hadn't historically been utilized in industrial labor: women~\cite{hart2013rise}.
% With little time and virtually none of the training that was conventionally expected of metalworkers and other industrial work
% (e.g. years--long apprenticeships),
% tasks had to be redesigned and workers trained for these dramatically unorthodox task designs
\ali{I'd like to bring the Rosie the Riveter anecdote in here but it's not feeling natural\dots}
% \cite{honey1985creating,davies2014origins}.

The early proliferation of manual piecework led to discussion surrounding how best to manage pieceworkers~\cite{norton1900textile,clark1908cotton}.
Despite this, workers' means were mostly ignored,
leading to frustration over poor working and living conditions
(famously documented by
% \ali{??? Why is this here? Can it be moved?}
% {As increasing attention revealed problems in piecework as it related to workers,
% workers themselves began to speak out about their frustration with this new regime.}
\citeauthor{riisOtherSideLives})~\cite{riisOtherSideLives}.
This led to industry organizations representing
railway workers, mechanical engineers, and others beginning to speak out on pieceworkers' behalf~\cite{american1921problem,richards1904anything}.
% Nevertheless, piecework continued to permeate low--skilled labor.
% \ali{How does this tie in with the rest of the narrative? Can I move this around? Does it break everything to take it out of chronological order?}

\topic{Piecework's popularity in the United States and Europe plummeted almost as quickly as it had climbed.}
\citeauthor{hart2013rise}'s work substantively explores the precipitous decline of piecework in the last third of the \nth{20} century.
In their work, \citeauthor{hart2013rise} offer a number of explanations for the sudden vanishing of piecework.
The salient suggestions include:
\begin{inlinelist}
\item the emergence of more effective, more nuanced incentive models
--- rewarding teams for complex achievements, for instance;
\item the shifting of these industries (manufacturing, clothing, etc\dots)
to other countries; and
\item the quality of ``multidimensional'' work becoming too difficult to evaluate
\end{inlinelist}~\cite{hart2013rise}.

\onlyinsubfile{
  \printbibliography
  % \clearpage
  }

\end{document}