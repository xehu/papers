\documentclass[trackingWork]{subfiles}



\makeatletter
\makeatother
\begin{document}
\section{A Review of Piecework}
\topic{The HCI community has used the term ``piecework'' to describe
myriad instantiations of on--demand labor,
but this reference has generally been offered in passing.}
As this paper principally traces a relationship between
the historical piecework and the contemporary crowdwork
(or on--demand labor more generally),
this casual familiarity with piecework may prove insufficient.
We'll more carefully discuss piecework in this section in order
to inform the subsequent sections --- and indeed, the entire argument.
Specifically, we will
\begin{inlinelist}
  \item define ``piecework'' as researchers in the topic understood it;
  \item trace the rise of piecework at a very high level,
        identifying key figures and ideas during this time; and finally
  \item look at the fall of piecework, such as it was,
        considering in particular
        the factors that may have led to piecework's eventual demise
\end{inlinelist}.

\topic{While ``piecework'' has proven difficult to concretize from the literature,
we can trace a constellation of characteristics of piece work that recur throughout the literature.}
We'll follow the history of research, collecting
descriptions,
examples, and
provided definitions of piecework, trying
to trace the outline of a working understanding of
\textit{what piecework is}.

\topic{One of the earliest definitions of piecework,
in \citeyear{hughRaynbirdTaskWork}, also proves to be the most circumspect in its wording.}
\citeauthor{hughRaynbirdTaskWork} offers what appears to be
one of the first concise definitions of piecework,
which he variously also calls ``measure work'', ``grate work'', and ``task work'':
``\dots the chief difference lies between the day--labourer,
who receives a certain some of money\dots~for his day's work,
and the task--labourer, whose earnings depend on the \textit{quantity} of work done [emphasis added]''
\cite{hughRaynbirdTaskWork}.

\citeauthor{hughRaynbirdTaskWork} makes several arguments for the merits of piece work,
pointing out that\dots
``piece work holds out to the labourer an increase of wages as a reward for his skill and exertion\dots
he knows that all depends on his own diligence and perseverance\dots~[and]
so long as he performs his work to the satisfaction of his master,
he is not under that control to which the day--labourer is always subject.''

\citeauthor{10.2307/2338394} gives a more concise definition of piecework
--- ``payments for results'' --- and further illustrates the concept with examples:
``\dots~payment is made for each hectare which is pronounced to be well ploughed~\dots~
for each living foal got from a mare;~\dots~
for each living calf got~\dots'' etc\dots
\cite{10.2307/2338394}.



\ali{
todo:
\begin{enumerate}
  \item piecework section
  \item what is it
  \item historical arc
  \item why do i care? how is it relevant
\end{enumerate}
}

\onlyinsubfile{
  % \balance{}
  \printbibliography
  \clearpage
  % \nobalance{}
  }

\end{document}