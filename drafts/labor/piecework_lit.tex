\documentclass[trackingWork]{subfiles}

% \renewcommand{\topic}[1]{#1}

\makeatletter
\makeatother
\begin{document}
\section{A Review of Piecework}
\topic{The HCI community has used the term ``piecework'' to describe
myriad instantiations of on--demand labor,
but this reference has generally been offered in passing.}
As this paper principally traces a relationship between
the historical piecework and the contemporary crowd work
(or on--demand labor more generally),
this casual familiarity with piecework may prove insufficient.
We'll more carefully discuss piecework in this section in order
to inform the the rest of the argument.
Specifically, we will
\begin{inlinelist}
  \item define ``piecework'' as researchers in the topic understood it;
  \item trace the rise of piecework at a very high level,
        identifying key figures and ideas during this time; and finally
  \item look at the fall of piecework, such as it was,
        considering in particular
        the factors that may have led to piecework's
        eventual demise \msb{did it demise? there are sweatshops around the world today}.
\end{inlinelist}



\subsection{What was piecework?}
\topic{Aligning on--demand work with piecework requires an understanding of what piecework is. While ``piecework'' has had multiple definitions over time,
we can trace a constellation of characteristics that recur throughout the literature.}
We will follow this history of research, collecting
descriptions,
examples, and
provided definitions of piecework, trying
to trace the outline of a working understanding of
\textit{what piecework is}.

\citeauthor{hughRaynbirdTaskWork} offers
a concise definition of piecework
--- which he variously also calls ``measure work'', ``grate work'', and ``task work'' ---
by contrasting the ``task--labourer'' with the ``day--labourer'':
``\dots the chief difference lies between the day--labourer,
who receives a certain some of money\dots~for his day's work,
and the task--labourer, whose earnings depend on the \textit{quantity} of work done [emphasis added]''
\cite{hughRaynbirdTaskWork}.
\citeauthor{10.2307/2338394} gives a more illustrative definition of piecework,
offering examples:
``\dots~payment is made for each hectare which is pronounced to be well ploughed~\dots~
for each living foal got from a mare;~\dots~
for each living calf got~\dots'' etc\dots
\cite{10.2307/2338394}.
This framing perhaps makes the most intuitive sense;
``payment for results'', as \citeauthor{10.2307/2338394} calls it,
is not only common in practice, but well--studied in labor economics as well
\cite{Figlio2007901,weitzman1976new,10.2307/3003414,BJIR:BJIR038}.

\topic{It's worth acknowledging that
``this distinction [between piece--rates and time--rates] was not completely clear--cut'' \msb{cite? who said that? Hart?};
indeed, we see work that adopts
piece--rate compensation in some aspects and
 time--rate compensation in others
\cite{hart2013rise}.}
The ``Rowan premium system'',
which essentially paid workers
a base rate for time plus
(the potential for) an additional pay dependent on output,
was just one of several alternatives to stricter time-- and piece--rate renumeration paradigms, which
muddies the waters for us later as we attempt to categorize cases of piecework
\cite{rowan1901premium}.
As \citeauthor{rowan1901premium}'s premium system guaranteed an hourly rate
regardless of the worker's productive output
\textit{as well as} an additional compensation tied to performance,
workers under this regime were in some senses ``task--labourers'',
and in other senses
(more \ali{(familiar|conventional)})
``day--labourers''.
\ali{I want to come back to this later when we talk about Uber \& Lyft, and
     how they increasingly offer to guarantee drivers an hourly rate
     \cite[see][]{uberHourly,lyftHourly}.
     That's why I'm nitpicking about this muddied system --- to avoid someone saying
     ``it doesn't seem like piecework to me anymore if there's an hourly wage floor''}

\topic{It may be worth thinking about piecework through the lens of its \textit{emergent} properties to help understand it.}
Returning to
\citeauthor{hughRaynbirdTaskWork}, several arguments for the merits of piece work \msb{inconsistent use of ``piece work'' vs. ``piecework'' in the non-quoted text. Which do you want?}%
crop up; he points out that\dots
``piece work holds out to the labourer an increase of wages as a reward for his skill and exertion\dots
he knows that all depends on his own diligence and perseverance\dots~[and]
so long as he performs his work to the satisfaction of his master,
he is not under that control to which the day--labourer is always subject.''
\citeauthor{hughRaynbirdTaskWork} (and others, as we will see)
highlight the freedom from control that ``task--labourers'' enjoy
\cite{hughRaynbirdTaskWork,rowan1901premium}.

\topic{We see this sense of independence regardless of the time, locale, and industry.}
\citeauthor{10.2307/3827491} offers a look into the lives and culture of ``match girls''
--- young women paid by piecework to assemble matchsticks generally in the late \nth{19} century.
Of particular interest was their independent nature, via their reputation ``\dots~for generosity, independence, and protectiveness,
but also for brashness, irregularity, low morality, and little education''
\cite{10.2307/3827491}.
\citeauthor{10.2307/27508091} documents piecework from 1850--1930 in Australia,
finding similar assertions of the freedom compositors of newspapers experienced as piece workers:
``If a piece--work compositor who held a `frame' decided that he did not want to work on a particular day or night,
the management recognised his right to put a `substitute' or `grass' compositor in his place''
\cite{10.2307/27508091}.
From these accounts we should be able to identify
a sense of independence that
resonates across decades, industries, and locales where piecework is found.
We'll problematize this supposed advantage as we trace the history of piecework,
but for now we can say that piecework affords
independence and some sense of autonomy
new to people in the working class.


\topic{\citeauthor{hart2013rise} offer another series of compelling insights toward the question of the features that sprout from piecework.}
In their reflection on the features endemic to piecework in the 1930s,
which they describe as the ``heyday'' of piecework's prominence;
among them were the following:
\begin{inlinelist}
\item ``female workers who generally had less training'' had to be trained in narrower subsets of the general body of skills that conventional (male) apprentices would undertake, and
\item workers with specific slices of skills could be more appropriately matched to suitable tasks
\end{inlinelist}
\cite{hart2013rise}.
Piecework thus opened the door for people who previously couldn't participate in the labor market
--- either for lack of training or for other reasons ---
to do so, and to acquire job skills incrementally.
Workers without conventional training
--- like women, who had no such opportunities to engage in engineering and metalworking apprenticeships as men did ---
could be trained very narrowly on a very tightly constrained task,
demonstrate proficiency, and become experts in their own ways.
% \msb{explain that in your own words now --- that the piecework infrastructure allows lower-skilled people to be used for skilled positions, by building the expertise into the workflow}

% \topic{Consolidating what we've learned from these sources about piecework, we might be able to arrive at a working definition that will
% suffice for our needs.} \msb{Again, I don't think the goal for the reader should be to arrive at a definition through a narrative, they instead need to know how to think about piecework. This  paragraph could probably be cut}
% Those needs being that the definition be
% \begin{inlinelist}
% \item faithful to the historical cases of piecework that we see in the scholarship; and
% \item relevant and informative to potential cases of piecework today.
% \end{inlinelist}
% We offer the following:
% that \pieceworkdefinition.

In summary, piecework:
\begin{enumerate}
  \item paid workers for quantity of work done, rather than time done, but occasionally mixed the two payment models;
  \item afforded workers freedom in when and how much to work; and
  \item structured tasks such that people who didn't have the training to engage in the traditional labor force could still participate.
\end{enumerate}

\subsection{The Historical Arc of Piecework}\label{sec:pieceworkArc}

\topic{Piecework's history traces back further perhaps than most would expect.}
\citeauthor{grier2013computers} describes the process astronomers adopted of hiring young boys
to calculate equations in order
to better--predict the trajectories of various celestial bodies in the \nth{19} century
\cite{grier2013computers}.
While this approach didn't become the same economic powerhouse as later examples would,
Airy \msb{This is the first time Airy comes up. Who are they?} and others arguably found the kernel of insight that we pursue throughout this discussion:
determining the extent to which work can be decomposed, and
finding the limits of complexity of that decomposed work.
That is, Airy found that he could train youths in elementary mathematics
to complete the majority of the calculations he would otherwise have had to solve on his own,
and that the greater body of work could ultimately be completed sooner
if he arranged his work appropriately.

% \topic{Piecework took a circuitous path in its rise to the mainstream%
% each time finding additional ways to leverage the advantages of piecework.}\msb{that sentence has some circular logic going on, it took a path to leveraging the advantages of itself?} \ali{Yeah, this is really weird; commenting out for now.}
Piecework then began to grow: first applied to farm work, as
\citeauthor{hughRaynbirdTaskWork} and others illustrate,
the practice remained relatively obscure until
it blossomed in the textile industry
\cite{hughRaynbirdTaskWork}.
This growth was so marked that by the turn of the \nth{20} century,
\citeauthor{riisOtherSideLives} was documenting abhorrent working \& living conditions of pieceworkers in New York City,
and \citeauthor{norton1900textile} was providing substantive guidance on various wage regimes,
offering guidance on how best to manage pieceworkers
\cite{riisOtherSideLives,norton1900textile}.
\citeauthor{clark1908cotton}, for instance,
relays his observations of textile mill pieceworkers and his realization that
``When he works by the day the Italian operative wishes to leave before the whistle blows,
but if he works by the piece he will work as many hours as it is possible for him to stand''
\cite{clark1908cotton}.
During this period, best practices regarding the measurement and management of
piecework rates, and of workers in the engineering industry,
were beginning to take shape
\cite{burton1899commercial}.

\topic{Researchers sought to understand the characteristics of piecework
that fueled its rise to popularity.}
\citeauthor{10.2307/23702539} argued that the first sparks of scientific management
could be found in piecework;
the approach of paying workers for each piece of output necessitated
the rigorous tracking, measurement, and training of workers
for which scientific management became famous
\cite{10.2307/23702539}.
This argument is certainly compelling;
it would seem to make the concurrent upswing of
scientific management and Fordism
through the first two--thirds of the \nth{20} century
alongside piecework not only understandable, but predictable
\cite{hart2013rise}.
\citeauthor{Brown01041990} inquired from another direction, asking
what limited the adoption of piecework in industries that otherwise gravitated toward it
(in the case studies he examined, this mostly focused on railway engineers),
ultimately arguing that factors such as the nature of the work design
(specifically, the homogeneity of tasks) and the costs associated with adopting a piecework model
were the major contributing factors that determined the use of piecework
\cite{Brown01041990}.


\topic{At this point, piecework became an important factor in the war effort for the Second World War,
cementing its role not only in American factories, but in industrial work around the world.}
The 1930s represented a boom for piecework on an unprecedented scale,
especially among engineering and metalworking industries.
As discussed earlier, \citeauthor{hart2013rise} characterize the 1930s
--- and more broadly the first half of the \nth{20} century ---
as the ``heyday'' of the use of piecework.
He attributes this to the shortage of male workers,
who would have gone through a conventional apprenticeship process
affording them more comprehensive knowledge of the total scope of work.

\topic{As increasing attention revealed problems in piecework as it related to workers,
workers themselves began to speak out about their frustration with this new regime.}
It began, arguably, with \citeauthor{riisOtherSideLives}'s photo--documentary work,
but this led to industry organizations representing
railway workers, mechanical engineers, and others contributing their perspectives
\cite{american1921problem,richards1904anything,riisOtherSideLives}.
%Nevertheless, piecework continued to permeate low--skilled labor.
% \topic{Necessity drove the widespread adoption of piecework and work decomposition%
% , perhaps more even than Ford's assembly line.}
%\topic{Despite the intense growth of the piecework approach to renumeration,
%this time was not without turmoil.}
%These worker organizations weighed in on
%(or, more precisely, against) piecework and the myriad oversights it made in valuing workers' time
%\cite{american1921problem,richards1904anything}.
For example, \citeauthor{10.2307/3827491} describes worker resistance among a largely disempowered community --- young women employed by piecework
\cite{10.2307/3827491}.

\topic{While many workers participated in piecework, worker sentiment toward the practice was --- by all accounts --- mostly negative.}
The match girls strikes which \citeauthor{10.2307/3827491} describes were just one early
--- albeit critical ---
case study in this space;
the national coal strike of 1912 led to an overwhelming vote among federated coal miner pieceworkers
to strike for
an individual minimum wage, among other demands
\cite{10.2307/2221944}.
\citeauthor{10.2307/41829256} documents a series of efforts among women in the garment industries in Philadelphia to negotiate collective bargaining rights and recognition of their own labor union
\cite{10.2307/41829256}.
The adoption of piecework's time--studies and other Taylorist and scientific management approaches reliably precipitated strikes and more generally gave workers a clear enemy against which to rally
\cite{jacoby1983union}.

\topic{However, piecework's popularity in the United States and Europe plummeted almost as quickly as it had climbed.}
\citeauthor{hart2013rise}'s work substantively explores the precipitous decline of piecework in the last third of the \nth{20} century.
In their work, \citeauthor{hart2013rise} offer a number of explanations for the sudden vanishing of piecework.
The salient suggestions include:
\begin{inlinelist}
\item the emergence of more effective, more nuanced incentive models
--- rewarding teams for complex achievements, for instance;
\item the shifting of these industries (manufacturing, clothing, etc\dots)
to other countries;
\item the quality of ``multidimensional'' work becoming too difficult to evaluate.
\end{inlinelist}
\cite{hart2013rise}.
% the gendered and stratified dynamics at play in the matchgirls strike 

% \topic{Piecework declined in popularity in the United States, but
% its spirit thrived through the mid--\nth{20} century in
% scientific management and Fordism.}
% Concurrent with the fall of piecework was arguably the rise of factories as centers of work.


\subsection{From piecework to on--demand work}
Crowd work and gig work are fundamentally an instantion of piecework.
First, workers on platforms such as Mechanical Turk and Uber are generally incentivized by unit of work, even if some may be offered an hourly base salary as well.
Second, workers are attracted to these platforms by the freedom they offer to pick the time and place of work~\cite{martin2014being,whyWouldAnyoneBrewer}.
Third, system developers as on Mechanical Turk typically assume no professional skills in transcription or other areas, and attempt to build that expertise into the workflow~\cite{noronha2011platemate,bernsteinSoylent}.

Given this alignment, many of the same properties of piecework historically will apply to on--demand work as well. In the next section, we perform this application to three of the major questions in crowd work and gig work, identifying similarities and differences between historical piecework and modern on--demand work.

\onlyinsubfile{
  \printbibliography
  % \clearpage
  }

\end{document}