\documentclass[trackingWork]{subfiles}

% \renewcommand{\topic}[1]{#1}

\makeatletter
\makeatother
\begin{document}
\section{A Review of Piecework}
\topic{The HCI community has used the term ``piecework'' to describe
myriad instantiations of on--demand labor%
, but this reference has generally been offered in passing.}
As this paper principally traces a relationship between
the historical piecework and the contemporary crowdwork
(or on--demand labor more generally)%
, this casual familiarity with piecework may prove insufficient.
We'll more carefully discuss piecework in this section in order
to inform the subsequent sections --- and indeed, the entire argument.
Specifically, we will
\begin{inlinelist}
  \item define ``piecework'' as researchers in the topic understood it;
  \item trace the rise of piecework at a very high level%
,         identifying key figures and ideas during this time; and finally
  \item look at the fall of piecework, such as it was%
,         considering in particular
        the factors that may have led to piecework's eventual demise
\end{inlinelist}.




\subsection{What was piecework?}
\topic{While ``piecework'' has proven difficult to concretize from the literature%
, we can trace a constellation of characteristics of piece work that recur throughout the literature.}
We'll follow the history of research, collecting
descriptions%
, examples, and
provided definitions of piecework, trying
to trace the outline of a working understanding of
\textit{what piecework is}.

\topic{One of the earliest definitions of piecework%
, in \citeyear{hughRaynbirdTaskWork}, also proves to be the most circumspect in its wording.}
\citeauthor{hughRaynbirdTaskWork} offers
a concise definition of piecework
--- which he variously also calls ``measure work'', ``grate work'', and ``task work'' ---
by contrasting the ``task--labourer'' with the ``day--labourer'':
``\dots the chief difference lies between the day--labourer,
who receives a certain some of money\dots~for his day's work,
and the task--labourer, whose earnings depend on the \textit{quantity} of work done [emphasis added]''
\cite{hughRaynbirdTaskWork}.
This description offers the first rudimentary definition of piecework
from which the practice will grow for more than a century;
piece work, as \citeauthor{hughRaynbirdTaskWork} offers,
``depend[s] on the quantity of work done''.

\citeauthor{10.2307/2338394} gives a more illustrative definition of piecework%
, offering examples:
``\dots~payment is made for each hectare which is pronounced to be well ploughed~\dots~
for each living foal got from a mare;~\dots~
for each living calf got~\dots'' etc\dots
\cite{10.2307/2338394}.
This framing perhaps makes the most intuitive sense;
``payment for results'', as \citeauthor{10.2307/2338394} calls it%
, is not only common in practice, but well--studied in labor economics as well
\cite{Figlio2007901,weitzman1976new,10.2307/3003414,BJIR:BJIR038}.

\topic{It's worth acknowledging that --- as \citeauthor{hart2013rise} point out ---
``this distinction [between piece--rates and time--rates] was not completely clear--cut''
\cite{hart2013rise}.}
The ``Rowan premium system''%
, which essentially paid workers a base rate for time with the opportunity for additional pay associated with output,
was just one of several alternatives to stricter time-- and piece--rate renumeration paradigms, which
muddies the waters for us later as we attempt to categorize cases of piecework
\cite{rowan1901premium}.
Nevertheless,
this work offers us an intuition for what piecework is, if not a bright--line rule.

\topic{It may be worth thinking about piecework through the lens of its \textit{emergent} properties to help understand it.}
Returning to
\citeauthor{hughRaynbirdTaskWork}, several arguments for the merits of piece work%
crop up; he points out that\dots
``piece work holds out to the labourer an increase of wages as a reward for his skill and exertion\dots
he knows that all depends on his own diligence and perseverance\dots~[and]
so long as he performs his work to the satisfaction of his master%
, he is not under that control to which the day--labourer is always subject.''
\citeauthor{hughRaynbirdTaskWork} (and others, as we will see)
highlight the freedom from control that ``task--labourers'' enjoy
\cite{hughRaynbirdTaskWork,rowan1901premium}.

\topic{We see this sense of independence regardless of the time, locale, and industry.}
\citeauthor{10.2307/3827491} offers us a look into the lives and culture of ``match girls''
--- young women paid by piecework to assemble matchsticks generally in the late \nth{19} century.
Of particular interest was their reputation ``\dots~for generosity, independence, and protectiveness,
but also for brashness, irregularity, low morality, and little education''
\cite{10.2307/3827491}.
\citeauthor{10.2307/27508091} documents piecework from 1850--1930 in Australia%
, finding similar assertions of the freedom compositors of newspapers experienced as piece workers:
``If a piece--work compositor who held a `frame' decided that he did not want to work on a particular day or night,
the management recognised his right to put a `substitute' or `grass' compositor in his place''
\cite{10.2307/27508091}.
From these accounts we should be able to identify
a sense of independence that
resonates across decades, industries, and locales where piecework is found.
We'll problematize this supposed advantage as we trace the history of piecework%
, but for now we can say that piecework affords
independence and some sense of locus
otherwise unknown to workers.


\topic{\citeauthor{hart2013rise} offer another series of compelling insights toward the question of the features that sprout from piecework.}
In their reflection on the features endemic to piecework in the 1930s,
which they describe as the ``heyday'' of piecework's prominence;
among them were the following:
\begin{inlinelist}
\item ``female workers who generally had less training'' had to be trained in narrower subsets of the general body of skills that conventional (male) apprentices would undertake, and
\item workers with specific slices of skills could be more appropriately matched to suitable tasks
\end{inlinelist}
\cite{hart2013rise}.

\topic{Consolidating what we've learned from these sources about piecework, we might be able to arrive at a working definition that will
suffice for our needs.}
Those needs being that the definition be
\begin{inlinelist}
\item faithful to the historical cases of piecework that we see in the scholarship; and
\item relevant and informative to potential cases of piecework today.
\end{inlinelist}
We offer the following:
that \pieceworkdefinition.


\subsection{What was piecework's historical arc?}

\topic{Piecework's history traces back further perhaps than most would expect.}
\citeauthor{grier2013computers} describes the process astronomers adopted of hiring young boys
to calculate equations in order to better--predict the trajectories of various celestial bodies
\cite{grier2013computers}.
While this approach didn't become an economic powerhouse as later examples would prove%
, Airy and others arguably found the kernel of insight that we pursue throughout this discussion:
determining the extent to which work can be decomposed, and
finding the limits of complexity of that decomposed work.
That is, Airy found that he could train youths in elementary mathematics
to complete the majority of the calculations he would otherwise have had to solve on his own%
, and that the greater body of work could ultimately be completed sooner
if he arranged his work appropriately.

\topic{Piecework took a circuitous path in its rise to the mainstream%
% from the field into the home and finally into factories%
, each time finding additional ways to leverage the advantages of piecework.}
First applied to farm work, as
\citeauthor{hughRaynbirdTaskWork} and others illustrate%
, the practice remained relatively obscure until
it was brought to the textile industry
\cite{hughRaynbirdTaskWork}.
At the turn of the \nth{20} century%
, when \citeauthor{riisOtherSideLives} was documenting abhorrent working \& living conditions of pieceworkers in New York City%
, \citeauthor{norton1900textile} was providing substantive guidance on various wage regimes%
, describing piecework comprehensively
% ``The timeworkers' wages should be verified by the timekeeper's book.''
\cite{riisOtherSideLives,norton1900textile}.
Soon after,
we saw the application of piecework systems in textile mills on the realization that
``[pieceworkers in Italy] will work as many hours as it is possible for him to stand''
\cite{clark1908cotton}.
% By the turn of the \nth{20} century%
% , \citeauthor{riisOtherSideLives} 
% the poor working and living conditions that were pervasive in
% an increasingly industrial United States
% \cite{riisOtherSideLives}.
% By the turn of the \nth{20} century,
Best practices regarding the measurement and management of
piecework rates, and of workers in the engineering industry,
were beginning to take shape
\cite{burton1899commercial}.

\topic{Researchers have since struggled
to understand the mechanisms and characteristics in piecework
which fueled its rise to popularity during this time.}
\citeauthor{10.2307/23702539} argued that the first sparks of scientific management
could be found in piecework;
the approach of paying workers for each piece of output necessitated
the rigorous tracking, measurement, and training of workers
for which scientific management became famous
\cite{10.2307/23702539}.
This argument is certainly compelling;
it would seem to make the concurrent upswing of
scientific management and Fordism
through the first two--thirds of the \nth{20} century
alongside piecework not only understandable, but predictable
\cite{hart2013rise}.
\citeauthor{Brown01041990} inquired from another direction, asking
what limited the adoption of piecework in industries that otherwise gravitated toward it
(in the case studies he examined, this mostly focused on railway engineers)
\cite{Brown01041990}.


\topic{As increasing attention revealed problems in piecework as it related to workers%
, workers themselves began to speak out about their frustration with this new regime.}
It began, arguably, with \citeauthor{riisOtherSideLives}'s photo--documentary work%
, but this led to industry organizations representing
railway workers, mechanical engineers, and others contributing their myriad perspectives
\cite{american1921problem,richards1904anything,riisOtherSideLives}.
Nevertheless, piecework continued to permeate low--skilled labor.

\topic{Piecework became an important contributor to the war effort in the Second World War%
, cementing its role not only in American factories, but in industrial work around the world.}
While piecework began to catch on at the turn of the \nth{20} century%
, the 1930s represented a boom for piecework on an unprecedented scale%
, especially among engineering and metalworking industries.
As discussed earlier, \citeauthor{hart2013rise} characterize the 1930s
--- and more broadly the first half of the \nth{20} century ---
as the ``heyday'' of the use of piecework.
He attributes this to the shortage of male workers%
, who would have gone through a conventional apprenticeship process
affording them more comprehensive knowledge of the total scope of work.

% \topic{Necessity drove the widespread adoption of piecework and work decomposition%
% , perhaps more even than Ford's assembly line.}

\topic{Despite the intense growth of the piecework approach to renumeration%
, this time was not without turmoil.}
As previously discussed, a number of worker organizations weighed in on
(or, more precisely, against) piecework and the myriad oversights it made in valuing workers' time
\cite{american1921problem,richards1904anything}.
\citeauthor{10.2307/3827491} describes worker resistance among a largely disempowered community --- young women employed by piecework
\cite{10.2307/3827491}.

\topic{While many workers participated in piecework, worker sentiment toward the practice was --- by all accounts --- mostly negative.}
The match girls strikes which \citeauthor{10.2307/3827491} describes were just one early
--- albeit critical ---
case study in this space;
the national coal strike of 1912 led to an overwhelming vote among federated coal miner pieceworkers
to strike for
an individual minimum wage, among other demands
\cite{10.2307/2221944}.
\citeauthor{10.2307/41829256} documents a series of efforts among women in the garment industries in Philadelphia to negotiate collective bargaining rights and recognition of their own labor union
\cite{10.2307/41829256}.
The adoption of piecework time--study and other principles associated with Taylor and scientific management
itself reliably precipitated strikes and more generally gave workers a clear enemy
against which to rally
\cite{jacoby1983union}.

\topic{Piecework's popularity in the United States and Europe plummeted almost as quickly as it had climbed just a few decades earlier.}
\citeauthor{hart2013rise}'s work substantively explores the precipitous decline of piecework in the last third of the \nth{20} century.
In their work, \citeauthor{hart2013rise} offer a number of explanations for the sudden vanishing of piecework.
We summarize some of the salient suggestions here:
\begin{inlinelist}
\item the emergence of more effective, more nuanced incentive models
--- rewarding teams for complex achievements, for instance;
\item the shifting of these industries (manufacturing, clothing, etc\dots)
to other countries;
\item the quality of ``multidimensional'' work becoming too difficult to evaluate.
\end{inlinelist}
\cite{hart2013rise}.
% the gendered and stratified dynamics at play in the matchgirls strike 

% \topic{Piecework declined in popularity in the United States, but
% its spirit thrived through the mid--\nth{20} century in
% scientific management and Fordism.}
% Concurrent with the fall of piecework was arguably the rise of factories as centers of work.


\subsection{Why is piecework relevant to crowdwork?}
Using the definition of piecework that we came up with earlier, we argue that
crowdwork is fundamentally an instantiation of piecework, and
that we can more precisely anticipate the answers to the open research questions we discussed earlier.
We'll show that the dimensions of crowdwork that the broader HCI community has been studying
align with the history of piecework, and that this can greatly inform predictions about the future of crowd work.


\onlyinsubfile{
  \printbibliography
  % \clearpage
  }

\end{document}