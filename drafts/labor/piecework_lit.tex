\documentclass[trackingWork]{subfiles}
\onlyinsubfile{
  \usepackage{xr-hyper}
  \usepackage{hyperref}
  \externaldocument{complexity}
  \externaldocument{relationships}
  \externaldocument{decomposition}
}
\begin{document}



% \onlyinsubfile{\clearpage}
\section{A Review of Piecework}

\topic{The HCI community has used the term ``piecework'' to describe
myriad instantiations of on--demand labor,
but this reference has generally been offered in passing.}
As this paper principally traces a relationship between
the historical piecework and the contemporary crowd work
(or on--demand labor more generally),
this casual familiarity with piecework may prove insufficient.
We will more carefully discuss piecework in this section in order
to inform the the rest of the argument.
Specifically, we will
\begin{numberlist}[itemjoin*={, and },label={\nthwords*}]
  \item define ``piecework'' as researchers in its field understand it
  \item trace the rise and fall of piecework at a high level,
        identifying key figures and ideas during this time.
\end{numberlist}
This section is not intended to be comprehensive: instead, it sets up the scaffolding necessary for our later investigations of crowd work's three questions: complexity limits, task decomposition, and worker relationships.

% \onlyinsubfile{\clearpage}
\subsection{What is piecework?}\label{sec:whatIsPiecework}
\topic{Aligning on--demand work with piecework requires an understanding of what piecework is.}
While it has had several definitions over time,
we can trace a constellation of characteristics that recur throughout the literature.
We'll follow this research, collecting
descriptions,
examples, and
definitions of piecework,
tracing an outline of % a working understanding of
\textit{what piecework is}.

\citeauthor{hughRaynbirdTaskWork} provides
a concise definition of piecework
(which he also calls ``measure work'', ``grate work'', and ``task work'')
by contrasting with the ``day--labourer'':
``the chief difference lies between the day--labourer,
who receives a certain some of money~\dots~for his day's work,
and the task--labourer, whose earnings depend on the quantity of work done''~\cite{hughRaynbirdTaskWork}.
\citeauthor{10.2307/2338394} defines it through examples:
``payment is made for each hectare which is pronounced to be well ploughed~\dots~for each living foal got from a mare;~\dots~for each living calf got''~\cite{10.2307/2338394}.
This framing offers a more intuitive sense;
``payment for results,'' as he calls it,
is not only common in practice, but
well--studied in labor economics as well~\cite{Figlio2007901,weitzman1976new,10.2307/3003414,BJIR:BJIR038}.

\topic{It's worth acknowledging that
``this distinction [between piece--rates and time--rates] was not completely clear--cut''~\cite{hart2013rise}.}
Indeed, employers adopted piece--rate compensation in some aspects and
time--rate compensation in others.
The Rowan premium system,
which essentially paid workers
a base rate for time plus, potentially,
an additional pay dependent on output~\cite{rowan1901premium};
this was just one of several alternatives to categorical time-- and piece--rate renumeration paradigms.
% which will complicate our categorization of cases of piecework.
As \citeauthor{rowan1901premium}'s premium system guaranteed an hourly rate
regardless of the worker's productive output
\textit{as well as} additional compensation tied to performance,
workers under this regime were
in some senses ``task--labourers'', but
in other senses ``day--labourers''.

\topic{It may be worth thinking about piecework through the lens of its \textit{emergent} properties to help understand it.}
\citeauthor{hughRaynbirdTaskWork} argues for the merits of piecework, % do you like what i did there?
pointing out that
``piece work holds out to the labourer an increase of wages as a reward for his skill and exertion~\dots~he knows that all depends on his own diligence and perseverance~\dots~[and] so long as he performs his work to the satisfaction of his master, he is not under that control to which the day--labourer is always subject''.
The argument that ``task--labourers'' enjoy freedom from control crops up in \citeauthor{hughRaynbirdTaskWork}'s and later \citeauthor{rowan1901premium}'s works~\cite{hughRaynbirdTaskWork,rowan1901premium}.

\topic{We see this sense of independence in myriad times, locales, and industries.}
\citeauthor{10.2307/3827491} offers a look into the lives and culture of ``match--girls''
--- young pieceworkers, mostly women, who assembled matchsticks in the late \nth{19} century.
Of interest was their reputation ``\dots~for generosity, independence, and protectiveness,
but also for brashness, irregularity, low morality, and little education''~\cite{10.2307/3827491}.
\citeauthor{10.2307/27508091} document piecework from 1850 through 1930 in Australia,
finding similar assertions of the freedom compositors of newspapers experienced as pieceworkers:
``If a piece--work compositor~\dots~decided that he did not want to work on a particular day or night,
the management recognised his right to put a `substitute' or `grass' compositor in his place''~\cite{10.2307/27508091}.
We therefore draw a \textit{sense} of independence and autonomy inherent to piecework.
%that resonates across decades, industries, and locales where piecework is found.

\topic{Piecework opened the door for people who previously couldn't participate in the labor market to do so, and to acquire job skills incrementally.}
During World War II, women received training in narrow subsets of more comprehensive jobs, enabling work in capacities similar to conventional (i.e. male) workers~\cite{hart2013rise}.
Workers with specific skill subsets could be matched to suitable tasks.
Women previously had virtually no opportunities
to engage in engineering and metalworking apprenticeships as men did;
now, they
could be trained quickly on narrowly scoped tasks,
demonstrate proficiency, and become experts.

In summary, piecework:
\begin{inlinelist}[itemjoin*={;~and~},itemjoin={;~}]
  \item paid workers for quantity of work done, rather than time done,
        but occasionally mixed the two payment models
  \item afforded workers freedom in when and how much to work
  \item structured tasks such that people who didn't have the training
        to engage in the traditional labor force could still participate.
\end{inlinelist}

Viewing crowd work as a modern instantiation of piecework is relatively straightforward by this definition.
\begin{Numberlist}
\item platforms such as Mechanical Turk, Uber and TaskRabbit pay by the task, though some such as Upwork do offer hourly rates as well.
\item workers are attracted to these platforms by the freedom they offer to pick the time and place of work~\cite{martin2014being,whyWouldAnyoneBrewer}.
\item system developers as on Mechanical Turk typically assume no professional skills in transcription or other areas, and attempt to build that expertise into the work flow~\cite{noronha2011platemate,bernsteinSoylent}.
\end{Numberlist}
Given this alignment, many of the same properties of piecework historically will apply to on--demand work as well. 
\msb{This transition seems misplaced with the Piecework primer and Case Studies subsections up next:}
In the next section, we perform this application to three of the major questions in crowd work and gig work, identifying similarities and differences between historical piecework and modern on--demand work.



% \onlyinsubfile{\clearpage}
\subsection{A Piecework Primer}\label{sec:pieceworkPrimer} % ali adores alliteration
\msb{This section is so brief that it feels like a bit of a cheat. I think we need to frame it differently. Maybe that it should survey the different ``eras'' of piecework? And each paragraph is laying out a major historical era that we will return to later? Otherwise this entire thing feels so brief that I don't feel that I get a ton of detail out of it: it rose, people were unhappy, it fell again.}
\ali{Do you mean that we'd have (sub)subsections under this? I'm tempted to merge this subsection into the ``\namerefl{sec:whatIsPiecework}'' subsection instead. Not exactly sure how that would work (i.e. would I interleave the history or just append it, basically removing the header).}
\topic{In this section, we will discuss the history of piecework at a high level.} Our goal is to lay the groundwork necessary to address our three major questions about crowd work, rather than provide a comprehensive historical account.
% Instead, this section will attempt to provide a sense of orientation when thinking about piecework.
This section will frame piecework,
in the early days of the Industrial Revolution \&
through the political and economic turmoil of the early and mid--\nth{20} century,
and into the \nth{21} century.
% While the previous section provided a \textit{definition} of piecework,
% this section attempts to shine a light on the \textit{zeitgeist} of piecework;
% as close to a \textit{thick description} of piecework
% as we can afford~\cite{geertz1973interpretation}.
% \msb{is this really thick? as a reader it flies by really quickly and super high level...}


\topic{Piecework's history traces back further perhaps than most would expect.}
\citeauthor{grier2013computers} describes the process astronomers adopted of hiring young boys
to calculate equations in order
to better--predict the trajectories of various celestial bodies in the \nth{19} century~\cite{grier2013computers}.
George Airy was perhaps the first to rigorously apply piecework--style decomposition of tasks to work;
by breaking complex calculations into constituent parts, and
training young men to solve simple algebraic problems,
Airy could distribute work to many more people than could otherwise complete the full calculations.


\topic{Piecework may have started in the intellectual domain of astronomical calculations and projections,
but it found its foothold in manual labor.}
Piecework took hold in farm work~\cite{hughRaynbirdTaskWork},
in textiles~\cite{restructuringPieceworkBaker,riisOtherSideLives},
on railroads~\cite{Brown01041990}, and 
elsewhere in manufacturing~\cite{10.2307/3827491}.
Fordism and scientific management thrust piecework into higher gear, especially as
mass manufacturing and
a depleted wartime workforce forced industry to find new ways to eke out more production capacity.
\msb{I think a sentence here to characterize the height of piecework would be helpful --- how big a deal was it at its peak?}
\ali{I'm not sure; I'll look for some data.}
% \msb{This last sentence seems disconnected from the broader argument; cut it or draw it back to the topic sentence of the paragraph}
% \citeauthor{hart2013rise} point out that the Second World War,
% which called millions of Americans to military service,
% necessitated the rapid training and employment of
% a labor pool that hadn't historically been utilized in industrial labor: women~\cite{hart2013rise}.

% \msb{The topic sentence is about management, but the content is about discontent. Which is it?}
\topic{The early growth of piecework led to discussion surrounding how best to manage pieceworkers,
generally in an adversarial approach, consequently
frustrating workers~\cite{norton1900textile,clark1908cotton}.}
This frustrations of poor working and living conditions
(famously documented by
\citeauthor{riisOtherSideLives}~\cite{riisOtherSideLives}).
Discontent reached a crescendo when industry organizations representing
railway workers, mechanical engineers, and other industries began to speak out on pieceworkers' behalves~\cite{american1921problem,richards1904anything}.

\msb{Was the plummet related to the discontent in the prior paragraph? Placing these paragraphs next to each other seems to indicate you think so. If so, make the transition obvious.}
\topic{Piecework's popularity in the United States and Europe plummeted almost as quickly as it had climbed.}
\msb{How quickly did it plummet? By when was it gone? Where is it now?}
\msb{This sentence is empty calorie:}
\citeauthor{hart2013rise}'s work substantively explores the precipitous decline of piecework in the last third of the \nth{20} century.
\msb{Very few calorie sentence:}
In their work, \citeauthor{hart2013rise} offer a number of explanations for the sudden vanishing of piecework.
The salient suggestions include:
\begin{inlinelist}[itemjoin*={;~and~},itemjoin={;~}]
\item the emergence of more effective, more nuanced incentive models
--- rewarding teams for complex achievements, for instance
\item the shifting of these industries (e.g., manufacturing, clothing)
to other countries
\item the quality of ``multidimensional'' work becoming too difficult to evaluate
\end{inlinelist}~\cite{hart2013rise}.







% \onlyinsubfile{\clearpage}
\subsection{Case studies in piecework}
Throughout the rest of the paper, we will return to three major case studies to frame our analyses:
       % (\namerefl{sec:complexity}, \namerefl{sec:decomposition}, and \namerefl{sec:relationships}).}
first, railroad and other industrial workers;
second, Airy's employment of \textit{human computers};
and third, domestic and farm work (in particular, the ``match--girls'').
Here, we introduce the basic historical details of these three cases.
We begin with the most familiar case (industrial workers),
working backwards through the \nth{20} and then \nth{19} century,
tracing the origins of the ideas that \dots
% \msb{incomplete sentence}


\ali{I'm coming out of this with a sort of structure that \textit{loosely} describes piecework chronologically.
Strictly speaking, I'm talking about each of these topics
(let's call them
\textbf{human computation},
\textbf{domestic and farm work}, and
\textbf{industrial work}),
and some of these things overlap with each other to some extent, but
for the most part
Airy and the human computers came in the mid--\nth{19} century,
farm work and the ``domestic'' work (like making matches) came around at the turn of the \nth{20} century, and
the industrial work (railroad workers, the WWII war effort, and the rise of the labor union movement) unfolded from the early \nth{20} century onward.}


% \ali{maybe go backwards on this}



% \ali{I'm looking into \citeauthor{grier2013computers}'s book to see if there's more about the relationships workers had with Airy.
% Maybe I can/should expand this to discuss other implementations of human computation around this time (if I find any good anecdotes).}
% \onlyinsubfile{\clearpage}
\subsubsection{Railroad and other Industrial Workers}
{Piecework might be most familiar to the HCI researcher in the context of the assembly line, which largely defined manufacturing through the \nth{20} century.}
It was here that
scientific management, Fordism, and Taylorism
dramatically influenced how workers were managed and
the ways in which workers were perceived and envisioned paradigmatically;
it's here we'll start our overview of piecework's history through case studies. \msb{we're not starting an overview of history; aren't we midway through it?}


\msb{I'm feeling very disoriented in this paragraph. The subsection title is about railroads, but we're all fast-forwarded to auto manufacturing?}
\topic{Piecework through the \nth{20} century centered around auto and other mass manufacturing, but
found its way into the war effort during World War II.}
With the vast majority of men drafted into service,
factories found themselves turning to
a predominantly female workforce that had neither
the formal training nor
the years of apprenticeship experience
that conventional workers would have had.
Rather than attempting to train this new labor force in every aspect of industrial work,
these women were trained for individual tasks
% --- such as riveting component to another ---
and assigned to that task. %, decomposed from a more comprehensive task.
One might reflect on the observation that ``Rosie the Riveter'',
an icon of \nth{20} century America who
represented empowerment and opportunity for women~\cite{honey1985creating},
was a pieceworker~\cite{davies2014origins}.

% This approach would turn out to yield two benefits:
% \begin{numberlist}[itemjoin*={;~and~},itemjoin={;~}]
% \item these workers could be productive with significantly less ``on--boarding'' cost
% \item workers' performance could be tracked \& evaluated, and better methods could be disseminated to workers.
% \end{numberlist}

\msb{confusing timing; now we're going backwards?}
{With \citeauthor{taylor1914principles}'s formalization of scientific management in \textit{Taylorism}
(and Henry Ford's eponymously named \textit{Fordism}),
piecework in the early and mid--\nth{20} century surged, especially in industrial work.}
Scientific management promised that the careful measurement of workers would yield
higher efficiency and output~\cite{taylor1914principles,towardsGlobalFordism}.
While \citeauthor{Brown01041990} points out that
piecework dramatically advanced the instrumented measurement of workers,
in \citeauthor{taylor1914principles}'s time highly instrumented,
automatic measurement of workers was all but impossible~\cite{Brown01041990}.
Instead, managers conducted ``stop watch time studies''~\cite{nadworny1955scientific},
using completion times to inform per--task compensation.
% Under this regime, piecework became commonplace in railroad maintenance and a number 
% This approach led to the adoption of piecework systems in myriad industrial labor verticals, such as
% railroad maintenance, automobile manufacturing, and later weapons manufacturing.
% \ali{???? better word for ``weapons manufacturing''?}
% Instead, as a result,
% the distillation of work into smaller units ultimately
% bottomed out with tasks as small as could be usefully measured~\cite{10.2307/23702539}.


% \topic{With the formalization of scientific management,
% piecework became an important component of the mass manufacturing
% in American factories.}
% \topic{}

\msb{even further backwards??? I would have expected this section to take a time slice and tell me about one group of people in detail}
\msb{this content might go above in the paragraph about the heydey of piecework in the piecework review}
\topic{The 1930s represented a boom for piecework on an unprecedented scale,
especially among engineering and metalworking industries.}
\citeauthor{hart2013rise} characterize the 1930s
--- and more broadly the first half of the \nth{20} century ---
as the ``heyday'' of the use of piecework.
They attribute this to the shortage of male workers,
who would have gone through a conventional apprenticeship process
affording them more comprehensive knowledge of the total scope of work.

\msb{This subsection feels like it never delivered on what it promised: a case study of one group of pieceworkers. I thought the goal was to introduce the railroad workers --- who they were, what they did, what their conditions were like.}




% \onlyinsubfile{\clearpage}
\subsubsection{Domestic and Farmhand Labor}
\topic{When piecework entered the American economy, it was not used for complex work.}
One reason for low complexity was workers' skills:
it was infeasible to provide new pieceworkers with the comprehensive education
that apprenticeships imparted~\cite{hart2013rise}.
So, initially piecework in the United States arose for farm work, and as
\citeauthor{hughRaynbirdTaskWork} and others discuss,
the practice remained relatively obscure until
it blossomed in the textile industry~\cite{hughRaynbirdTaskWork}.
\msb{this is not about domestic/farmhand labor...might need to move up or be cut...or is it? clarify that this is domestic labor moving into NYC?}
The complexity of the work remained low at the turn of
the \nth{20} century as piecework saturated New York City~\cite{riisOtherSideLives}.
\msb{how is this sentence relevant to the case study?:}
However, writers of the time focused their attention on
wage~\cite{burton1899commercial} and
management regimes~\cite{norton1900textile}
rather than training.
Mass manufacturing,
such as sewing garments~\cite{riisOtherSideLives} and making matchsticks~\cite{10.2307/3827491},
flourished under piecework systems in densely populated cities.

\topic{Workers' relationships with employers quickly soured.}
\msb{We haven't met the match--girls yet. Is this case study supposed to be about them? They come out of nowhere, and we're talking about their strike rather than them.}
The match--girls strike of 1888 was one of the earliest and most famous successful worker strikes,
and perhaps the beginning of ``militant trade unionism''~\cite{10.2307/3827491}.
As \citeauthor{weyer1894history} described,
``the match--girls' victory turned a new leaf in Trade Union annals''~\cite{weyer1894history}: in the 30 years after the match--girls strike,
the Trade Union Movement enrollment grew from 20\% of eligible workers to over 60\%.

\topic{The match--girls strike foreshadowed both collective action victories and an emerging paradigm regarding worker management.}
\msb{this paragraph seems to be about collective action rather than domestic/farmhand labor. also, coal miners aren't domestic or farmhand.}
Coal miners won a nationwide individual minimum wage after The Great Coal Strike of 1912~\cite{10.2307/2221944}.
Garment workers in Philadelphia secured collective bargaining rights in 1915 after a prolonged strike and threat of a second~\cite{10.2307/41829256}.
\msb{Why is Taylor relevant to this case study?}
It was in the midst of this time that \citeauthor{taylor1914principles} published
the work for which he would later be called ``the father of scientific management''~\cite{RePEc:mtp:titles:0262612062}.
It was this framing on work \& worker management that gave workers a concrete adversary
--- if not Taylor, then \textit{Taylorism}  ---
against which to rally~\cite{jacoby1983union,parker1920casual}.



% \onlyinsubfile{\clearpage}
\subsubsection{Airy's Computers}
\msb{This one seems way thinner than the others? Maybe it should be merged with the earlier description of Airy.}
\topic{Some of the first systematic cases of what we would recognize as crowd work
can be found in the study of astronomy.}
In the \nth{19} century, the calculation of celestial bodies had become
a competitive field with 
Airy needed to compute tables that would
allow sailors to locate themselves by starlight from sea.
This work ostensibly called for educated people who comprehensively understood mathematics.
Airy realized that he could break the tasks down and delegate the constituent parts
to ``human computers'' who
``\dots~possessed the basic skills of mathematics,
including `Arithmetic, the use of Logarithms, and Elementary Algebra'~''~\cite{grier2013computers}.
As a result, many of Airy's computers had relatively rudimentary educations
compared to the background of education that typically worked in the calculation of solar tables.
Airy distributed tasks by mail,
allowing work to be completed by a somewhat geographically distributed workforce,
and paid for each piece of work completed.
% \ali{also talk about the challenge/competition Airy was in with other astronomers,
% and the approach of verifying worker results}
% Airy's method of distributing work was unorthodox, but his workforce management style was especially unconventional.
% For one thing,
Airy also instituted a policy of firing his computers once they reached age 23.


\ali{THIS GOES LATER: This practice ensured two outcomes that arguably disfavored workers.
\begin{Numberlist}
\item it eliminated any potential to advance professionally, as workers' careers in this area ended relatively early in their careers,
      and without formal education in mathematics they struggled to find work for which their experience was meaningful.
\item it limited workers' ability to organize
      by ensuring that workers hardly spent sufficient time to successfully rally their peers.
\end{Numberlist}}








\onlyinsubfile{
  \printbibliography
  }

\end{document}