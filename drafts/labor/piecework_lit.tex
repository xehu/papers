\documentclass[trackingWork]{subfiles}
\onlyinsubfile{
  \usepackage{xr-hyper}
  \usepackage{hyperref}
  \externaldocument{complexity}
  \externaldocument{relationships}
  \externaldocument{decomposition}
}
\begin{document}



% \onlyinsubfile{\clearpage}
\section{A Review of Piecework}

\topic{The HCI community has used the term ``piecework'' to describe
myriad instantiations of on--demand labor, but
researchers have generally made this allusion in passing.}
Since this paper traces a much stronger parallel between
(historical) piecework and (contemporary) on--demand work,
a more comprehensive background on piecework will be useful.
%We will more carefully discuss piecework in this section
%to help make our observations and arguments with better familiarity with the topic.
Specifically,
first, we'll define ``piecework'' as researchers in its field understand it;
and second, we'll trace the rise and fall of piecework at a high level,
identifying key figures and ideas during this time.
This section is not intended to be comprehensive:
instead, it sets up the scaffolding necessary for
our later investigations of on--demand work's three questions:
complexity limits,
task decomposition, and
worker relationships.




\subsection{What is piecework?: A primer and timeline}\label{sec:whatIsPiecework}

\topic{Aligning on--demand work with piecework requires an understanding of what piecework is.}
While it has had several definitions over the years,
we can trace a constellation of characteristics that recur throughout the literature.
We'll follow this research, collecting
descriptions,
examples, and
definitions,
to develop a working understanding.

\topic{Piecework's history traces back further perhaps than most would expect.}
\citeauthor{grier2013computers} describes the process astronomers adopted of hiring young boys
to calculate equations in order
to better--predict the trajectories of various celestial bodies in the 1830s~\cite{grier2013computers}.
George Airy was perhaps the first to rigorously apply piecework--style decomposition of tasks to work;
by breaking complex calculations into constituent parts, and
training young men to solve simple algebraic problems,
Airy could distribute work to many more people than could otherwise complete the full calculations.


\topic{Piecework may have started in the intellectual domain of astronomical calculations and projections,
but it found its foothold in manual labor.}
Piecework took hold in farm work~\cite{hughRaynbirdTaskWork},
in textiles~\cite{restructuringPieceworkBaker,riisOtherSideLives},
on railroads~\cite{Brown01041990}, and 
elsewhere in manufacturing~\cite{10.2307/3827491}.
Fordism and scientific management thrust piecework into higher gear, especially as
mass manufacturing and
a depleted wartime workforce forced industry to find new ways to eke out more production capacity.
\msb{I think a sentence here to characterize the height of piecework would be helpful --- how big a deal was it at its peak?}
\ali{I'm not sure; I'll look for some data.

update: hard to find anything saying ``piecework represented eleventy billion dollars of the U.S. economy'', which is what I'd like to say,
but I can say something like ``at its height, $k\%$ of garment workers were pieceworkers'' or something.
that seems a bit wishy--washy though; I'd like to say something like
``$k\%$ of the \textit{entire U.S. workforce} was working under some sort of piecework regime''.}


\topic{By \citeyear{hughRaynbirdTaskWork} we find
a concise definition of piecework
in \citeauthor{hughRaynbirdTaskWork}'s essay on piecework,
% (where he also calls it ``measure work'', ``grate work'', and ``task work''),
particularly driven toward encapsulating the manual labor of farmwork.}
\citeauthor{hughRaynbirdTaskWork} does this by contrasting with the ``day--labourer'' ---
``the chief difference lies between the day--labourer,
who receives a certain some of money~\dots~for his day's work,
and the task--labourer, whose earnings depend on the quantity of work done''~\cite{hughRaynbirdTaskWork}.
\citeauthor{10.2307/2338394} defines it through examples:
``payment is made for each hectare which is pronounced to be well ploughed~\dots~for each living foal got from a mare;~\dots~for each living calf got''~\cite{10.2307/2338394}.
This framing offers an intuitive sense of piecework;
``payment for results,'' as he calls it,
is not only common in practice, but
well--studied in labor economics as well~\cite{Figlio2007901,weitzman1976new,10.2307/3003414,BJIR:BJIR038}.

\topic{It's worth acknowledging that
``this distinction [between piece--rates and time--rates] was not completely clear--cut''~\cite{hart2013rise}.}
Employers implemented piece--rates in some aspects and
time--rates in others.
The Rowan premium system, for example,
essentially paid workers
a base rate for time plus
additional pay dependent on output~\cite{rowan1901premium}.
As \citeauthor{rowan1901premium}'s premium system guaranteed an hourly rate
regardless of the worker's productive output
\textit{as well as} additional compensation tied to performance,
workers were
in some senses ``task--labourers'', but
in other senses ``day--labourers''.
This was just one of several alternatives to strict time-- and piece--rate renumeration paradigms.

\topic{It may be worth thinking about piecework through the lens of its \textit{emergent} properties to help understand it.}
\citeauthor{hughRaynbirdTaskWork} argues for the merits of piecework,
pointing out that
``piece work holds out to the labourer an increase of wages as a reward for his skill and exertion~\dots~he knows that all depends on his own diligence and perseverance~\dots~[and] so long as he performs his work to the satisfaction of his master, he is not under that control to which the day--labourer is always subject''.
The argument that ``task--labourers'' enjoy freedom from control crops up in \citeauthor{hughRaynbirdTaskWork}'s and later \citeauthor{rowan1901premium}'s works~\cite{hughRaynbirdTaskWork,rowan1901premium}.

\topic{We see this sense of independence in myriad times, locales, and industries.}
\citeauthor{10.2307/3827491} offers a look into the lives and culture of ``match--girls''
--- young pieceworkers, mostly women, who assembled matchsticks in the late \nth{19} century.
Of interest was their reputation ``\dots~for generosity, independence, and protectiveness,
but also for brashness, irregularity, low morality, and little education''~\cite{10.2307/3827491}.
\citeauthor{10.2307/27508091} document piecework from 1850 through 1930 in Australia,
finding similar notions of independence and autonomy among piecework newspaper compositors:
``If a piece--work compositor~\dots~decided that he did not want to work on a particular day or night,
the management recognised his right to put a `substitute' or `grass' compositor in his place''~\cite{10.2307/27508091}.
This sense of independence and autonomy appears to be a common component of piecework.
%that resonates across decades, industries, and locales where piecework is found.

\ali{I think now that this is moved to this area there's a good opportunity to frame this more as
``now that workers could choose their own schedule, style, etc\dots,
newfound interest in how to manage these workers emerged''.
Thoughts?}
\topic{Since workers could now choose their own schedule and style, there arose a discussion on how best to manage pieceworkers.}
This conversation generally regarded workers antagonistically~\cite{roy1954efficiency}, a far cry from the earlier rhetoric on piecework, which promised that
piece workers would gladly work as diligently and as hard as possible because incentive--based pay rewards hard work~\cite{clark1908cotton}.


\topic{Piecework opened the door for people who previously couldn't participate in the labor market to do so, and to acquire job skills incrementally.}
During World War II, women received training in narrow subsets of more comprehensive jobs, enabling work in capacities similar to conventional (male) workers~\cite{hart2013rise}.
%Workers with specific skill subsets could be matched to suitable tasks.
Women previously had virtually no opportunities
to engage in engineering and metalworking apprenticeships as men did;
now, they
could be trained quickly on narrowly scoped tasks,
demonstrate proficiency, and become experts.

\topic{Piecework's popularity in the United States and Europe fell almost as quickly as it had climbed.}
Between 1938 and 1942, the proportion of metal workers under piecework systems had climbed steeply from 11\% to 60\%~\cite{hart2005piecework}.
By 1961, that proportion dropped to 8\%~\cite{carlson1982time}.
\citeauthor{carlson1982time} details that, from 1973 to 1980, the holdouts of piecework
--- where more than 50\% worked under incentive wage plans ---
were principally in clothes--making (e.g. hosiery, footwear, and garments).
\citeauthor{hart2013rise} offer a number of explanations for the sudden vanishing of piecework.
The salient suggestions include:
\begin{inlinelist}
\item the emergence of more effective, more nuanced incentive models
--- rewarding teams for complex achievements, for instance;
\item the shifting of piecework industries such as manufacturing and textiles to other countries; and
\item the quality of ``multidimensional'' work becoming too difficult to evaluate
\end{inlinelist}~\cite{hart2013rise}.



In summary, piecework:
\begin{inlinelist}
  \item paid workers for \textit{quantity} of work done, rather than \textit{time} done,
        but occasionally mixed the two payment models;
  \item afforded workers a sense of freedom and independence; and
  \item structured tasks in such a way as to facilitate more narrowly scoped training and education.
\end{inlinelist}

\msb{If you want to make the argument in the last paragraph of this subsection that piecework assumed no professional skills, it's worth calling that out slightly more visibly throughout the subsection.}\ali{I don't want to suggest that, actually, at least not ultimately; in the beginning, sure, but by WWII it definitely required certain specialized training (arguably in Airy's case too)}

Viewing on--demand work as a modern instantiation of piecework is relatively straightforward by this definition.
First, platforms such as Mechanical Turk, Uber and TaskRabbit pay by the task, though some such as Upwork do offer hourly rates as well.
Second, workers are attracted to these platforms by the freedom they offer to pick the time and place of work~\cite{martin2014being,whyWouldAnyoneBrewer}.
Third, system developers as on Mechanical Turk typically assume no professional skills in transcription or other areas, and attempt to build that expertise into the work flow~\cite{noronha2011platemate,bernsteinSoylent}.
Given this alignment, many of the same properties of piecework historically will apply to on--demand work as well. 





\subsection{Case studies in piecework}
\topic{Throughout the rest of the paper, we will return to four case studies to frame our analyses:
Airy's employment of human computers;
domestic and farm workers;
the ``match--girls'' strike;
and industrial and assembly-line workers.}
In introducing these cases at a high level,
we'll trace the history of piecework
while also framing the later analysis of the major research threads we named earlier: complexity, decomposition, and relationships.

% \onlyinsubfile{\clearpage}
\subsubsection{Airy's computers}

\begin{comment}
What did I pull from the threads that are related to industrial and railroad workers (i.e. 1920 onward?)

- Airy and his human computers were great:
  - quickly verifiable
  - independent tasks (could be checked without the whole product)
  - narrowly trainable

\end{comment}

\msb{This one seems way thinner than the others? Maybe it should be merged with the earlier description of Airy.}
\topic{In the \nth{19} century, the calculation of celestial bodies had become a competitive field, and Airy needed to compute tables that would allow sailors to locate themselves by starlight from sea.}
This work ostensibly called for educated people who comprehensively understood mathematics.
Airy realized that he could break the tasks down and delegate the constituent parts
to human computers, or people who could compute basic functions.
These human computers ``\dots~possessed the basic skills of mathematics,
including `Arithmetic, the use of Logarithms, and Elementary Algebra'~''~\cite{grier2013computers}.
As a result, many of Airy's computers had relatively rudimentary educations
compared to the background of education that typically worked in the calculation of solar tables.
Airy distributed tasks by mail,
allowing work to be completed by a somewhat geographically distributed workforce,
and paid for each piece of work completed.
Airy also instituted a policy of firing his computers once they reached age 23.
%\ali{this has a ``gosh isn't that interesting?'' feeling to it.
%should i just leave it out until later when there's something to say about it
%(namely, that the mode of work was initially designed to stymie professional growth)?}%

The human computers captured several aspects of task decomposition that would become common. 
First, the work was designed such that it could be done independently and without collaboration. 
%This enabled geographically-distributed workers to complete the tasks. 
Second, the work was designed so that intermediate results could be quickly verified: Airy would have two workers each do the calculation, and another person compare their answers.
Third, Airy identified ways to decompose the large task into narrowly-trainable subtasks.
%\ali{This should go later someplace, but not clear where:

This practice ensured two outcomes that disfavored workers.
First, it eliminated any potential to advance professionally, as
workers' careers ended relatively early in their careers,
and without formal education in mathematics they struggled to find work for which their experience was meaningful.
Second, it limited workers' ability to organize
by ensuring that workers were barely in communication with each other.% hardly spent sufficient time to successfully rally their peers.


% \onlyinsubfile{\clearpage}
\subsubsection{Domestic and farmhand labor}

\begin{comment}
What did I pull from the threads that are related to domestic/farmwork?

- Graves: sparks of Scientific Management in Piecework, especially starting here
- 19th century: piecework was mostly cottage industry with untrained or informally trained workers
  (unlike industrial metal workers during WWII)
- Brown: Task variability matters
- Clark: pieceworkers work harder, more diligently, etc...
- Riis saw terrible conditions, documented and communicated it to the world

\citeauthor{clark1908cotton} observed textile mill pieceworkers and reported,
``When he works by the day the Italian operative wishes to leave before the whistle blows,
but if he works by the piece he will work as many hours as it is possible for him to stand''~\cite{clark1908cotton}.


\end{comment}

\topic{The application of piecework to farm work in the late \nth{19} century and
later to manufacturing of small goods, such as garments and matches, at the turn of the \nth{20} century
proved to be a formative period for piecework as we would come to know it.}
Piecework regimes in farms and in homes engaged workers in assembling clothing. %, and
%finally in the early sweatshops where women made matchsticks under dangerous and even hazardous conditions.
%\topic{Farm and field work emerged early on as a rich field for the use of piecework,
%perhaps in part due to the relatively low skillset necessary to complete work.}
Textile manufacturers found that they could deliver fabric to people at their homes, asking them to sew together clothing.
The manufacturers would later return to retrieve the finished garments,
paying these workers for each piece of clothing completed. 
Farm work applied the idea of piecework by
paying workers for tasks like picking bushels of fruit or bringing to birth animals~\cite{10.2307/2338394}.
\msb{cutting this next part, it seems irrelevant to the paragraph:}
%\citeauthor{hughRaynbirdTaskWork} argues from another perspective, but along a similar thread, that
%the only factor important in piecework was
%whether the worker's final product was to the ``satisfaction of his master''~\cite{hughRaynbirdTaskWork}.
%Piecework's advantage,
%\citeauthor{hughRaynbirdTaskWork} argues,
%was this hands--off management of workers.

Workers could, in principle, assemble as much or as little clothing as they wanted;
the reality was more grim, as
\citeauthor{riisOtherSideLives} documented in ``\citetitle{riisOtherSideLives}'' in \citeyear{riisOtherSideLives}~\cite{riisOtherSideLives}.
He found that
workers endured bleak living conditions and
worked long hours attempting to scrape together a living.


\subsubsection{The match--girls' strike}

%\topic{Piecework played a role in early mass manufacturing during the Industrial Revolution,
%as well as the early years of worker collective organization and action.}
\topic{Match--makers were some of the first workers in mass manufacturing
to successfully rally for political causes.}
At the end of the \nth{19} century,
manufacturers employed women to assemble matchsticks in factories.
These women rallied first in the form of a march on parliament in 1871 to protest a proposed tax, and 
later (more famously) in what was later called ``the match--girls strike of 1888''~\cite{10.2307/3827491}.

\topic{The match--girls strike of 1888 was sparked by a worker's arbitrary docking of pay, but
much deeper resentment had been simmering for years.}
Match--girls were already frustrated with
the arbitrariness of management,
poor working conditions, and
having to work with hazardous phosphoric materials.
\ali{I might need to add some info here, and certainly citations, but there'll be overlap with the other citations here}

\topic{Regardless of the reasons, the lasting impact of the match--girls strike of 1888 was profound.}
This was one of the earliest and most famous successful worker strikes,
and perhaps the beginning of ``militant trade unionism''~\cite{10.2307/3827491}.
As \citeauthor{weyer1894history} described,
``the match--girls' victory turned a new leaf in Trade Union annals''~\cite{weyer1894history}: in the 30 years after the match--girls strike,
the Trade Union Movement enrollment grew from 20\% of eligible workers to over 60\%.

%\topic{To understand how these women became such influential, if underreported, figures in labor advocacy's history, we should shine a light onto the reputation that had formed for these workers.}
Match--girls were the only group in \citeyear{booth1903life} to have formed a trade union,
according to \citeauthor{booth1903life}'s account at the time~\cite{booth1903life}.
\citeauthor{10.2307/3827491} noted that match--girls
``\dots~pooled their resources to purchase their plumes and clothes~\dots~and expressed their solidarity through small [and major] strikes''~\cite{booth1903life}.
But they were also, as \citeauthor{10.2307/3827491} confesses, known for ``brashness, irregularity, low morality, and little education''~\cite{10.2307/3827491}.
These were workers who treasured their independence, but also fiercely protected one another, contributing to the common good.
Their ``brashness'' for instance may have detracted from their public image, but almost undoubtedly contributed to their sense of solidarity,
making their propensity to act against such unfair treatment and poor conditions understandable and maybe predictable.



\ali{I want to discuss a few things here:
\begin{itemize}
  \item there were protests about something sort of like surge pricing in crop--picking, but i don't know how to make that a paragraph (maybe merge it with match--girls?)
\end{itemize}}


\subsubsection{Industrial workers}
\topic{Piecework might be most familiar in the context of industrial and factory work, which largely defined manufacturing through the \nth{20} century.}
%In the railroad industry and especially on the assembly line we see many of the mechanisms of work and worker management that later made their way into crowd work.
%Furthermore, it's from here the piecework literature draws most upon to find limits speaking to complexity and task granularity.
%We'll expose these facets of piecework very superficially in the context of industrial work to better speak to these research questions in crowd work later.
\topic{Before the factory assembly line arose, however, railway companies adopted piecework regimes in the early \nth{20} century. \msb{What was pieceworky about this? They were paid per meter of railroad that they laid down?}
What followed was a flourishing of piecework management practices,
as railway companies worked to find effective ways
to motivate and evaluate this skilled workforce of engineers.}
\citeauthor{10.2307/23702539} takes up a case study of the Santa Fe Railway,
finding that they employed ``efficiency experts'' to develop a ``standard time''
to determine pay for each task at the company informed by
``thousands of individual operations''; % ~\cite{10.2307/23702539}.
\citeauthor{10.2307/23702539} goes on to list
some of the roles required to facilitate piecework
in the early \nth{20} century
--- among them, ``piecework clerks, inspectors, and `experts'~''~\cite{10.2307/23702539}. \topic{This oversight, while controversial
(especially among workers~\cite{american1921problem}),
paved the way for piecework to grow substantially.}

\topic{The 1930s represented a boom for piecework on an unprecedented scale,
especially among engineering and metalworking industries.}
\citeauthor{hart2013rise} characterize the 1930s
--- and more broadly the first half of the \nth{20} century ---
as the ``heyday'' of the use of piecework.
They attribute this to the shortage of male workers,
who would have gone through a conventional apprenticeship process
affording them more comprehensive knowledge of the total scope of work.

\topic{Piecework %through the \nth{20} century centered around automobile and other mass manufacturing, but
found its way into the war effort during World War~II.}
With the vast majority of men drafted into service,
factories found themselves turning to
a predominantly female workforce that had neither
the formal training nor
the years of apprenticeship experience
that conventional workers would have had.
Rather than attempting to train this new labor force in every aspect of industrial work,
these women were trained for individual tasks
% --- such as riveting component to another ---
and assigned to that task. %, decomposed from a more comprehensive task.
``Rosie the Riveter'',
an icon of \nth{20} century America who
represented empowerment and opportunity for women~\cite{honey1985creating},
was a pieceworker~\cite{davies2014origins}.


% \msb{confusing timing; now we're going backwards?}

% \msb{even further backwards??? I would have expected this section to take a time slice and tell me about one group of people in detail}
% \msb{this content might go above in the paragraph about the heydey of piecework in the piecework review}



\ali{some things I didn't cover:
\begin{itemize}
  \item research finding that task variability is bad for piecework, but worker variability is good
  \item talk about the foreman
  \item worker advocacy groups
\end{itemize}
should I be making a serious effort to get that stuff in here? this section is kind of long already
(and it'll definitely be much longer than the other case studies).}






\begin{comment}
What did I pull from the threads that are related to industrial and railroad workers (i.e. 1920 onward?)

- Graves: railway companies used ``efficiency experts'' to study how long tasks should take
- Hart: evaluation limits complexity (we can affect that with peer evaluation!)
- Graves: sparks of Scientific Management in Piecework
- organization types are important determinants of piecework viability: lots of types of tasks? bad
  - Hart (I think?): variability in *worker* quality is fine
- Foreman is important
- Worker advocacy groups arose to speak out against piecework

\end{comment}









\onlyinsubfile{
  \printbibliography
  }

\end{document}