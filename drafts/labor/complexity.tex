\documentclass[trackingWork]{subfiles}
\makeatletter
\def\blx@maxline{77}
\makeatother
\begin{document}

% \renewcommand{\topic}[1]{#1}
\subsubsection[finding crowd work's limits]{Achieving greater complexity}\label{sec:complexity}
\subsubsubsection{\crowdworkpers}
\topic{Crowdsourcing research has spent the better part of a decade attempting
to prove the viability of crowdsourcing in increasingly complex work.}  % this is my topic sentence.
\citeauthor{crowdForgeKittur}
map the discussion toward this goal in their work on
crowdsourcing complex work
\cite{crowdForgeKittur}.
The broader body of work has varied significantly in type
--- providing conversational assistants%
, interpreting medical data, and
telling coherent and compelling stories%
, to name a few examples
\cite{Lasecki:2013:CCC:2501988.2502057,mavandadi2012distributed,KimStoria}.

\topic{This body of research has involved similar approaches to problems%
, often involving insights made in Computer Science and applied to human work--flows.}
The crowd work literature typically identifies target milestones in CS
that have presented significant challenges for researchers%
, leverages some of the approaches and insights that Computer Science researchers have already made
(for example, MapReduce in the case of \citeauthor{crowdForgeKittur}'s \textit{CrowdForge})%
, and arranges humans as computational black boxes within those approaches and processes
\cite[][and others]{crowdForgeKittur,foundry}.
This approach has proven a compelling one because
it leverages the in--built advantages that technology and digital media afford.
\textit{Foundry}'s tools for managing and arranging expert groups into a cohort
allow researchers to convincingly argue that expert teams can be rapidly formed%
, just like non--expert teams
\cite{foundry}.



\subsubsubsection{\pieceworkpers}
\topic{The research into piecework makes the case that
piecework has been limited principally by the challenges of human management and oversight.}
\citeauthor{10.2307/23702539} describes a case study in Santa Fe Railway, which
deployed scientific management and a piecework regime in an attempt to stymie rising repair costs
\cite{10.2307/23702539}.
Returning to \citeauthor{hart2016rise}'s reflections on piecework's limitations,
we recall the multidimensional problem
--- tasks comprising of numerous, sometimes conflicting, goals
\cite{hart2016rise}.
It would be reasonable, then, to infer that work like this
--- reasonably highly skilled work where quality is difficult to assess ---
would be unsuitable for piecework.

\topic{\citeauthor{hart2016rise} and \citeauthor{10.2307/23702539}%
, without acknowledging one another%
, seemingly corroborate one another's conclusions at different levels of observation.}
\citeauthor{10.2307/23702539} enumerates some of the roles required
to facilitate piecework in the early \nth{20} century:
    ``\dots~piecework clerks, inspectors, and `experts'\dots''
\cite{hart2016rise,10.2307/23702539}.
\citeauthor{10.2307/23702539} and \citeauthor{hart2016rise} may seem
to be making differing claims about the limitations of piecework, but
we argue that \citeauthor{10.2307/23702539}
is simply making a more concrete observation illustrating the insight that
\citeauthor{hart2016rise} later makes.
\citeauthor{10.2307/23702539} recognizes that
it's necessary for a successful piecework shop to employ
clerks,
inspectors, and
other experts to properly design and evaluate complex work.
\citeauthor{hart2016rise} argues an ultimate limit to how far this can go;
at some point, evaluating multidimensional work output for quality
(rather than for quantity) becomes infeasible.

\topic{This isn't to say that complex work is outside of the realm of piecework;
indeed, we've discussed complex applications of crowd work already.}
As \citeauthor{hart2013rise} described,
the 1930s saw a flourishing of clever piecework job design out of necessity
due to the fact that it was infeasible to provide new workers with
the comprehensive education that was familiar to men
\cite{hart2013rise}.
This constraint led to much more tightly scoped work, and
(perhaps surprisingly at the time) more efficient allocations of workers,
who could now specialize in extremely narrowly defined roles.
The same could be said of Airy and his \textit{computers}
--- young boys whose preparations consisted principally of
a relatively specific mathematics curriculum
\cite{grier2013computers}.



\topic{Piecework researchers also make claims regarding
the organizations that benefit from piecework in the first place.}
\citeauthor{Brown01041990} discusses the factors necessary for piecework to thrive:
    ``\dots~incentive pay is less likely in jobs with
    a variety of duties than in jobs with a narrow set of routinized duties''
\cite{Brown01041990}.
\citeauthor{10.2307/23702539} adds further, that
successful cases of piecework owed themselves in part to the fact that
    ``\dots~only [the largest and most wealthy railroads] had the resources to~\dots
    pay the overhead involved in installing work reorganization''
\cite{10.2307/23702539}.
Together, \citeauthor{10.2307/23702539} and \citeauthor{Brown01041990}
make a persuasive argument that piecework is limited in complexity by
managerial overhead and the fixed cost of adopting a piecework payment regime.

\topic{There are other characteristics to effective complex piecework institutions, such as
appropriately designed management practices.}
\citeauthor{10.2307/2118435} describe the role of the foreman in West Virginia coal mines under the piecework model:
``The foreman had the power to hire and fire workers and allocate workplaces,
but then left the face--worker largely free to his own efforts so that
often he went all day without seeing the foreman''
\cite{10.2307/2118435}.
The general approach adopted by these West Virginia mines was,
as in other factories with active foremen,
to let the foreman be the intermediary between management and the worker.
Specifically, foremen were responsible for allocating resources and
understanding when and how to modify work as necessary
\cite{wray1949marginal}.


\subsubsubsection{\whatchanged}
\topic{Digital media have expanded the scope of
viable piecework by pushing drastically on the limits
cited by piecework researchers.}
The research on piecework tells us that
we should expect piecework to thrive in industries where
the nature of the work is limited in complexity
\cite{Brown01041990}.
Given the flourishing of on--demand labor platforms such as
Uber, AMT, and others, we ask ourselves
what --- if anything --- has changed.
We argue that
the Internet has trivialized
the costs and challenges of the earlier limiting factors for two reasons:
\begin{inlinelist}
  \item Technology make it much easier to do complex work aided by computers; and % is there a source for this? Is this necessary to cite?
  \item The Internet allows us to leverage the benefits of
        ``economies of scale'' at very little cost
        to the system--designer \cite{lessig2006code,miller2011understanding}
\end{inlinelist}.

\topic{Technology has made it possible
for non--experts to do work that was once considered
within the domain of experts.}
\citeauthor{yuanAlmost} builds on the work of others
(\textit{Voyant} and, more relevantly, \textit{CrowdCrit})
to design workflows that yield ``expert--level feedback''
\cite{yuanAlmost,Xu:2014:VGS:2531602.2531604,Luther:2014:CCA:2556420.2556788}.
This body of work identifies ways to transform a variety of duties comprising complex tasks
and distills them into ``a narrow set of routinized duties''%
, informed in part by researchers --- acting as inspectors --- and experts
\cite[quotations from][]{10.2307/23702539}
Where \citeauthor{10.2307/23702539} would call additionally for the identification of
crowdsourcing's version of ``piecework clerks'', we point out that
today algorithms manage workers as pieceworkers once did
\cite{uberAlgorithm,10.2307/23702539}.

\topic{Furthermore, technology more directly facilitates the subversion of expertise requirements
by giving non--experts access to information that would otherwise be unavailable.}
Taxi drivers in London endure rigorous training to pass a test known as ``The Knowledge''
--- a demonstration of the driver's comprehensive familiarity.
Researchers have identified significant growth of
the hippocampal regions of the brains in veteran drivers%
, generally understood to be responsible for spatial functions such as navigation
\cite{Maguire11042000,Maguire2894,Skok:1999:KML:299513.299625%
,       skok2000managing,Woollett1407,woollett2011acquiring}.
Services such as Google Maps \& Waze make it possible for
people entirely unfamiliar with a city
to know more about a city even than experts through
the collective data generated by other users
ranging topics such as police activity, congestion, construction, etc\dots
\cite{silva2013traffic,hind2014outsmarting}.

\subsubsubsection{\implication}
\topic{The piecework literature gives us a template for pushing the boundaries of complexity in piecework, but
it also signals some of the ultimate limitations of crowd work and piecework in general.}
While the threshold preventing task requesters from utilizing piecework
has dropped thanks to affordances of the Internet, the ceiling on task complexity hasn't moved significantly.
If we're to make use of \citeauthor{Brown01041990}'s prescriptions,
we would benefit from finding ways to decompose varied tasks into homogeneous microtasks.

We should also consider exploring the limitations that algorithmic management bring along more carefully.
While research has touched on this subject, we've yet to make out the bigger picture of this theme
\cite{uberAlgorithm}.
If we can resolve the tension between workers and perilously antagonistic managers, as \citeauthor{10.2307/2118435} suggest,
then we may be able to break a toxic cycle of mistrustful requesters \cite[for example][]{MaliciouscrowdworkersGadiraju}
and develop more considerate platforms as \citeauthor{takingAHITMcInnis} advocate
\cite{takingAHITMcInnis}.

Finally, and perhaps most importantly,
we need to replicate the success of narrowly slicing education and training for expert work
as \citeauthor{hart2013rise} and \citeauthor{grier2013computers} described in their piecework examples
\cite{hart2013rise,grier2013computers}.
That is, we need to identify new ways to train crowd workers for uniquely narrowly defined work.
To some extent, an argument can be made that
MOOCs and other online education resources
provide crowd workers with the resources that they need, but 
it remains to be seen whether that work will be appropriately valued, let alone
properly interpreted by task solicitors
\cite{aguaded2013mooc}.
If we can overcome this obstacle,
we might be able to empower crowd workers to do complex work such as engineering and metalworking,
rather than doom them to match girl reputations:
``brash, irregular, immoral, and uneducated''
\cite{10.2307/3827491}.




\onlyinsubfile{
  \printbibliography
  }

\end{document}