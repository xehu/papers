\documentclass[trackingWork]{subfiles}
\makeatletter
\def\blx@maxline{77}
\makeatother
\begin{document}

\subsubsection[finding crowdwork's limits]{Achieving greater complexity}\label{sec:complexity}
\ali{maybe a subssubsubsusbsubsection on the result/implication?
more people can make crowd tasks, but the algiorithmic manag}

\subsubsubsection{\crowdworkpers}
\topic{Crowdsourcing research has spent the better part of a decade attempting
to prove the viability of crowdsourcing in increasingly complex work.}  % this is my topic sentence.
\citeauthor{crowdForgeKittur}
map the discussion toward this goal in their work on
crowdsourcing complex work
\cite{crowdForgeKittur}.
The broader body of work has varied significantly in type
--- providing conversational assistants%
, interpreting medical data, and
telling coherent and compelling stories%
, to name a few examples
\cite{Lasecki:2013:CCC:2501988.2502057,mavandadi2012distributed,KimStoria}.

\topic{This body of research has involved similar approaches to problems%
, often involving insights made in Computer Science and applied to human work--flows.}
The crowdwork literature typically identifies target milestones in CS
that have presented significant challenges for researchers%
, leverages some of the approaches and insights that Computer Science researchers have already made
(for example, MapReduce in the case of \citeauthor{crowdForgeKittur}'s \textit{CrowdForge})%
, and arranges humans as computational black boxes within those approaches and processes
\cite[][and others]{crowdForgeKittur,foundry}.
This approach has proven a compelling one because
it leverages the in--built advantages that technology and digital media afford.
\textit{Foundry}'s tools for managing and arranging expert groups into a cohort
allow researchers to convincingly argue that expert teams can be rapidly formed%
, just like non--expert teams
\cite{foundry}.


\subsubsubsection{\pieceworkpers}
\topic{The research into piecework makes the case that
piecework has been limited principally by the challenges of human management and oversight.}
\citeauthor{10.2307/23702539}%
, who describes piecework as
    ``\dots~based on examination of various shop jobs,
      which included calculation of the standard time and compensation for each task''%
, argues that piecework must be rigorously evaluated at a time that demands
\textit{other people} perform the evaluation.
% \cite{10.2307/23702539}.
\citeauthor{10.2307/23702539} later enumerates some of the roles required
to facilitate piecework in the early \nth{20} century:
    ``\dots~piecework clerks, inspectors, and ``experts''\dots''
\cite{10.2307/23702539}.
This criterion strictly limits the extent to which piecework can grow in complexity;
it must, for instance, be quickly evaluable by another person.


\topic{Piecework researchers also make claims regarding
the organizations that benefit from piecework in the first place.}
\citeauthor{Brown01041990} discusses the factors necessary for piecework to thrive:
    ``\dots~incentive pay is less likely in jobs with
    a variety of duties than in jobs with a narrow set of routinized duties''
\cite{Brown01041990}.
\citeauthor{10.2307/23702539} adds further, that
successful cases of piecework owed themselves in part to the fact that
    ``\dots~only [the largest and most wealthy railroads] had the resources to~\dots
    pay the overhead involved in installing work reorganization''
\cite{10.2307/23702539}.
Together, \citeauthor{10.2307/23702539} and \citeauthor{Brown01041990}
make a persuasive argument that piecework is limited in complexity by
managerial overhead and the fixed cost of adopting a piecework payment regime.


\subsubsubsection{\whatchanged}
\topic{Digital media have expanded the scope of
viable piecework by pushing drastically on the limits
cited by piecework researchers.}
The research on piecework tells us that
we should expect piecework to thrive in industries where
the nature of the work is limited in complexity
\cite{Brown01041990}.
Given the flourishing of on--demand labor platforms such as
Uber, AMT, and others, we ask ourselves
what --- if anything --- has changed.
We argue that
the Internet has trivialized
the costs and challenges of the earlier limiting factors for two reasons:
\begin{inlinelist}
  \item Technology make it much easier to do complex work aided by computers; and % is there a source for this? Is this necessary to cite?
  \item The Internet allows us to leverage the benefits of
        ``economies of scale'' at very little cost
        to the system--designer \cite{lessig2006code,miller2011understanding}
\end{inlinelist}.

\topic{Technology has made it possible
for non--experts to do work that was once considered
within the domain of experts.}
\citeauthor{yuanAlmost} builds on the work of others
(\textit{Voyant} and, more relevantly, \textit{CrowdCrit})
to design workflows that yield ``expert--level feedback''
\cite{yuanAlmost,Xu:2014:VGS:2531602.2531604,Luther:2014:CCA:2556420.2556788}.
This body of work identifies ways to transform a variety of duties comprising complex tasks
and distills them into ``a narrow set of routinized duties''%
, informed in part by researchers --- acting as inspectors --- and experts
\cite[quotations from][]{10.2307/23702539}
Where \citeauthor{10.2307/23702539} would call additionally for the identification of
crowdsourcing's version of ``piecework clerks'', we point out that
today algorithms manage workers as pieceworkers once did
\cite{uberAlgorithm,10.2307/23702539}.

\topic{Technology more directly facilitates the subversion of expertise requirements
by giving non--experts access to information that would otherwise be unavailable.}
Taxi drivers in London endure rigorous training to pass a test known as ``The Knowledge''
--- a demonstration of the driver's comprehensive familiarity.
Researchers have identified significant growth of
the hippocampal regions of the brains in veteran drivers%
, generally understood to be responsible for spatial functions such as navigation
\cite{Maguire11042000,Maguire2894,Skok:1999:KML:299513.299625%
,       skok2000managing,Woollett1407,woollett2011acquiring}.
Services such as Google Maps \& Waze make it possible for
people entirely unfamiliar with a city
to know more about a city even than experts through
the collective data generated by other users
ranging topics such as police activity, congestion, construction, etc\dots
\cite{silva2013traffic,hind2014outsmarting}.

\topic{Crowdsourcing falters when the routinization of complex work proves difficult.}
\ali{Something about complexity being difficult to make routine when
there are lots of little, varied, unpredictable tasks (nod to \citeauthor{Brown01041990})}



\onlyinsubfile{
  \balance{}
  \printbibliography
  % \clearpage
  \nobalance{}
  }

  \endnotetext{
  \section{Complexity notes}
    \topic{\joke{great} topic sentence here.}
    \citeauthor{selfsourcingTeevan2014} push the boundaries of decomposed work%
,     exploring ``selfsourcing'', and further this work with \citeauthor{selfsourcingTeevan2016}
    \cite{selfsourcingTeevan2014,selfsourcingTeevan2016}.
    While some of this work doesn't strictly fall under ``crowdsourcing''%
,     the [scientific] management of the self as a worker
    (of sorts)
    will prove relevant as we trace the literature surrounding piecework.

    This is also kind of off-topic, isn't it? where to take it?

    \topic{A smaller but growing body of work has reflected on this
    and brought to our attention a number of ethical questions surrounding the
    increasing complexity of crowdwork and the hazards that increasingly arise.}
    \citeauthor{professionalCrowdworkEthics} bring some of these issues at stake
    --- working for increasing amounts of time on tasks of growing complexity%
,     only to discover that requesters are not willing to pay%
,     for instance ---
    but these and other dangers range an enormous landscape
    \cite{crowdworkFuture,professionalCrowdworkEthics,nickerson2013crowd,dynamo}.

    \topic{We summarize the work discussed so far in this way:
    researchers of crowdsourcing and on--demand labor in general
    have attempted to find the characteristics which
    enable and stymie successful applications of this form of work%
,     and how to get past or around these boundaries.}
    Whether this work has focused on the self (as in the case of ``selfsourcing'')
    or on others (as in \textit{CrowdForge} and \textit{Foundry})%
,     crowdsourcing literature has generally studied the constituent work
    and attempted either to work around
    or indeed with
    the limitations of crowdwork to accomplish a difficult task
    \cite{selfsourcingTeevan2014,selfsourcingTeevan2016,crowdForgeKittur,foundry}.

    Crowdforge, turkomatic, others talk about limiting factors; bring them up


      Talked to someone about this section and got some feedback\dots
      \begin{enumerate}
        \item Explain what ``clever arrangement of work'' in the first paragraph means
              (should be simple; Find--Fix--Verify is the \textit{sort} of thing I mean);
        \item I should make the ``selfsourcing'' stuff clearer for people unfamiliar with it
              (maybe even to justify its placement here); and
        \item I should break up \citeauthor{10.2307/23702539} \& \citeauthor{Brown01041990}
              to make the historical stuff on piecework
              directly speak to the three previous paragraphs.
        \item (Maybe?) summarize the piecework stuff
              (the last paragraph approaches that, but more could be done).
      \end{enumerate}
      Thoughts on this?

  \citeauthor{10.2307/23702539} also argues that
  a significant obstacle to the introduction of piecework in railroad shops was
  the resistance of workers
  (the other being the resistance of management (for different and varying reasons)).
  In the cases of online platforms like AMT and digitally mediated platforms like Uber%
,   where in both cases workers rarely
  if ever
  interact face--to--face and
  opportunities for coordination and collective action are severely limited%
,   we should expect to see
  the challenges stymieing collective action
  that we see in \citeauthor{dynamo} and elsewhere
  \cite{10.2307/23702539,dynamo}.
}

\end{document}