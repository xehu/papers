\documentclass[trackingWork]{subfiles}
\onlyinsubfile{
  \usepackage{xr-hyper}
  \usepackage{hyperref}
}


\makeatletter
\def\blx@maxline{77}
\makeatother
\begin{document}

\subsubsection[What are the limits of crowd work]{Identifying the Limits of Crowd Work}\label{sec:complexity}
\subsubsubsection{\crowdworkpers}
\topic{Crowdsourcing research has spent the better part of a decade attempting
to prove the viability of crowdsourcing in increasingly complex work.}  % this is my topic sentence.
\citeauthor{crowdForgeKittur}
map the discussion toward this goal in their work on
crowdsourcing complex work
\cite{crowdForgeKittur}.
The broader body of work has varied significantly in type
--- providing conversational assistants,
interpreting medical data, and
telling coherent and compelling stories,
to name a few examples
\cite{Lasecki:2013:CCC:2501988.2502057,mavandadi2012distributed,KimStoria}.

\topic{This body of research has involved similar approaches to problems,
often involving insights made in Computer Science and applied to human work--flows.}
The crowd work literature typically identifies target milestones in computer science
that have presented significant challenges for researchers,
leverages some of the approaches and insights that Computer Science researchers have already made
(for example, MapReduce in the case of \citeauthor{crowdForgeKittur}'s \textit{CrowdForge}),
and arranges humans as computational black boxes within those approaches and processes
\cite[][and others]{crowdForgeKittur,foundry}.
This approach has proven a compelling one because
it leverages the in--built advantages that technology and digital media afford.
For example, \textit{Foundry}'s tools for managing and arranging expert groups into a cohort
allow researchers to convincingly argue that expert teams can be rapidly formed,
just like non--expert teams
\cite{foundry}.



\subsubsubsection{\pieceworkpers}
\topic{Piecework researchers have found themselves trying to understand what characteristics limit piecework,
or more precisely what has prevented piecework from becoming more prevalent.}
The research into piecework makes the case that
piecework has been limited principally by the challenges of human management and oversight.
\citeauthor{10.2307/23702539} describes a case study in Santa Fe Railway, which
deployed scientific management and a piecework regime in an attempt to stymie rising repair costs
\cite{10.2307/23702539}.
\citeauthor{10.2307/23702539} reports on the hiring of Harrington Emerson,
an ``efficiency expert'', who went on to develop a ``standard time'' for each task at the company
informed by ``thousands of individual operations at the Topeka shops''.
The cost of measuring workers in such excruciating detail at the turn of the \nth{20} century was undoubtedly immense,
but this ``standard time'' value, which determined the pay that workers would earn for each task they do,
was the only viable approach at the time to determine appropriate pay given the task
\cite{10.2307/23702539}.
But the repeated measurement of workers' time to complete tasks had shortcomings;
for one thing, pay rates for rarer tasks
were necessarily less certain than for more common tasks,
which had the simple benefit of a larger sample size.
\ali{Do I need ``for another thing\dots''? This paragraph is kind of huge already. Not sure how to break it up/down as it is.}
One might conclude from \citeauthor{10.2307/23702539}'s observations that
complex, creative work --- which is inherently heterogeneous and difficult to routinize ---
would be unsuitable for piecework.
% We might reflect on \citeauthor{hart2016rise}'s piecework's limitations,
% and specifically the challenge of multidimensional work
% --- tasks comprising of numerous, sometimes conflicting, goals
% \cite{hart2016rise}.

% It would be reasonable, then, to infer that work like this
% --- reasonably highly skilled work where quality is difficult to assess ---
% would be unsuitable for piecework.
% \msb{that conclusion came way too fast. give a little detail on the railway. how did they come to that conclusion?}

% \msb{this paragraph needs a topic sentence. it's a wandering paragraph currently. is the claim here that evaluation is the limiting factor?}
\topic{Determining appropriate pay rates, informed by the careful measurement of workers,
isn't the only major challenge piecework faced; evaluation proved a limiting factor as well.}
\citeauthor{10.2307/23702539} enumerates some of the roles required
to facilitate piecework in the early \nth{20} century --- among them
    ``\dots~piecework clerks, inspectors, and `experts'\dots''
\cite{10.2307/23702539}.
% \citeauthor{10.2307/23702539} and \citeauthor{hart2016rise} may seem
% to be making differing claims about the limitations of piecework, but
% we argue that \citeauthor{10.2307/23702539}
% is simply making a more concrete observation illustrating the insight that
% \citeauthor{hart2016rise} later makes.
\citeauthor{10.2307/23702539} further recognizes that
it's necessary for a successful piecework shop to employ
clerks,
inspectors, and
other experts to properly design and evaluate complex work.
\citeauthor{hart2016rise} later makes a more concrete observation of this hurdle,
as he argues an ultimate limit to how far this can go;
at some point, evaluating multidimensional work output for quality
(rather than for quantity) becomes infeasible.
In his words,
``if the quality of the output is more difficult to measure than the quantity,
perhaps because of `difficult--to--observe' production techniques,
then a piecework system is likely to encourage
an over--emphasis on quantity produced and an under-emphasis on quality''
\cite{hart2016rise}.
This, \citeauthor{hart2016rise} argues, may have fundamentally hamstrung piecework,
and ultimately precipitated its downfall,
especially with the increasing complexity of manufacturing work
over the course of the \nth{20} century.
\ali{just to be clear, this whole point is queuing up so that later I can be like
``Oh hey computers are terrific at evaluating some stuff really quickly.
And researchers in crowd work have done some work on workers evaluating other workers
(e.g. Find--Fix--Verify, arguably \textit{PeerStudio} does something along this line)
\cite{bernsteinSoylent,Kulkarni:2015:PRP:2724660.2724670}''}

\topic{The research seems to suggest that it was difficult to apply piecework to more skilled work,
particularly because maximizing on the advantages of piecework seemed to reward
smaller, more constrained, more narrowly trained tasks.}
For most of the \nth{19} century,
piecework was applied almost exclusively to farm and textile work.
% \citeauthor{grier2013computers} describes Airy's training and employment of young men to compute astronomical observations,
% mentioning that boys needed only to have ``the basic skills of mathematics''
% \cite{grier2013computers}.
% \citeauthor{hart2013rise} argue that piecework flourished in the first half of the \nth{20} century because
% tasks were decomposed --- a discussion we will return to later ---
% such that untrained workers could be trained relatively quickly on a very narrowly designed task
% \cite{hart2013rise}.
Work was simple and widely understood
--- farm workers didn't need to be trained on how to plow fields, or birth foals;
seamstresses knew how to sew together denim
\cite{10.2307/2338394,riisOtherSideLives}.
% The general argument for decomposition, we will see later,
% hinges on the promise that
% tasks can be \textit{made} less complex and time--consuming,
% if they're not already so.
% \ali{this section is here because i later talk about the complexity of work, and
% I want to keep the whole thing about how Uber, Google Maps, etc\dots make it possible to
% \begin{inlinelist}
% \item train workers more quickly (honestly just transmitting instructions over the Internet is so much quicker than what Airy had), and
% \item give workers necessary tools/information to get work done
% (e.g. ``Waze knows you should avoid the intersection 3 blocks down the road because of an accident''
% and (maybe) ``not great at math? WolframAlpha can confirm or just give you your answer'')
% \end{inlinelist}.}

\topic{This isn't to say that complex work is outside of the realm of piecework;
indeed, we've discussed complex applications of crowd work already.}
% As \citeauthor{hart2013rise} described,
While \citeauthor{hart2013rise} described a flourishing of ingenious piecework design,
much of it arose out of necessity ---
% almost strictly out of necessity ---
it was infeasible to provide new workers with
the comprehensive education that was familiar to men through apprenticeships
\cite{hart2013rise}.
While this constraint led to much more tightly scoped work
% (perhaps surprisingly at the time) more efficient allocations of workers,
who now had to specialize in extremely narrowly defined roles.
The same could be said of Airy and his \textit{computers}
--- young boys whose preparations consisted principally of
a relatively specific mathematics curriculum
\cite{grier2013computers}.
Instead, we argue that the literature suggests that
piecework is tightly limited only when the application of piecework follows
a direct, perhaps even unimaginative,
mapping from an time--based regime to an output--based one.
When the work is redesigned from the ground up 
--- as we see with mathematicians in the \nth{19} century and
with the metalworking industry during the Second World War ---
it seems that we don't yet know the limits of complexity with regard to piecework.
% \msb{so what are we to take out of this? that there is no limit? just our creativity?}
\ali{This is \textit{almost} making a prediction,
but I wanted to take something punchy away from the stuff I cite here.}




\topic{Piecework researchers also argue that,
in addition to constraints on the kind of \textit{work} that's amenable to piecework,
only certain kinds of \textit{organizations} are amenable to piecework.}
% \msb{I think a reader wouldn't understand why that claim is relevant.
% Let's invert it, something like how
% the researchers point out that
% only certain kinds of organizations can effectively make use of piecework.}
\citeauthor{Brown01041990} discusses the organizational factors necessary for piecework to thrive,
arguing that piecework
``\dots~ is less likely in jobs with
a variety of duties than in jobs with a narrow set of routinized duties''
\cite{Brown01041990}
% \msb{that's confusing, is incentive pay the same as piecework?
% can you cut the first part of the quote and
% give us the noun instead of forcing us to figure it out?}.
\citeauthor{10.2307/23702539} adds further, that
successful cases of piecework owed themselves in part to the fact that
    ``\dots~only [the largest and most wealthy railroads] had the resources to~\dots
    pay the overhead involved in installing work reorganization''
\cite{10.2307/23702539}.
Together, \citeauthor{10.2307/23702539} and \citeauthor{Brown01041990}
make a persuasive argument that piecework is limited in complexity by
the capacity to endure managerial overhead while transitioning to a new system.
\msb{wait, wasn't that in a previous paragraph?
that should be joined with the managerial overhead text above}
\ali{I'm trying to make this argument that \cite{10.2307/23702539} is making two claims;
the first is that there are limits on doing hard, ``difficult--to--observe'' work,
and the second is that the cost propagates up to management,
putting limits on who can run piecework systems.
So\dots\\
\makebox[0pt][l]{$\square$}\raisebox{.15ex}{\hspace{1em}} Rewrite\\
\makebox[0pt][l]{$\square$}\raisebox{.15ex}{\hspace{1em}} Refactor\\
\makebox[0pt][l]{$\square$}\raisebox{.15ex}{\hspace{1em}} Drop

Maybe I should just merge it with the next paragraph?}

\topic{There are other characteristics to effective complex piecework institutions, such as
appropriately designed management practices.}
\citeauthor{10.2307/2118435} describe the role of the foreman in West Virginia coal mines under the piecework model:
``The foreman had the power to hire and fire workers and allocate workplaces,
but then left the face--worker largely free to his own efforts so that
often he went all day without seeing the foreman''
\cite{10.2307/2118435}.
The general approach adopted by these West Virginia mines was,
as in other factories with active foremen,
to let the foreman be the intermediary between management and the worker.
Specifically, foremen were responsible for allocating resources and
understanding when and how to modify work as necessary
\cite{wray1949marginal}.
% \msb{Currently unclear what I'm supposed to take out of this. Summarize for me. What does this say about the complexity limits of piecework?}
The management of pieceworkers %, \citeauthor{10.2307/2118435} and later \citeauthor{wray1949marginal} argue,
demanded people in positions akin to foremen
--- intermediate managers people who were
\begin{inlinelist}
  \item familiar with and even sympathetic to the needs of workers,
  \item empowered by higher level management to make decisions, and
  \item relaxed enough in day--to--day work to allow workers to go about their work
\end{inlinelist}
\cite{wray1949marginal,10.2307/2118435}.



\subsubsubsection{\whatchanged}
\msb{Before you get into this, summarize what I am supposed to have learned from the prior section on piecework.
You're about to draw on those points to make your argument, so they need to be at the top of my mind here.}
\topic{Piecework makes a number of observations leading to the conclusion that
piecework's complexity is fundamentally bounded by several limitations,
chief among them the costs of managerial overhead and the transition thereto.}
\citeauthor{Brown01041990} and \citeauthor{10.2307/23702539}'s claims that
organizations can't adopt piecework unless
they're sufficiently large to absorb the cost of transitioning to a piecework system;
\citeauthor{10.2307/2118435} and \citeauthor{wray1949marginal}'s observations for
the importance of competent, effective managerial oversight
--- a human resource, which made the scaling cost prohibitively expensive for many
% in their times (\citeyear{10.2307/2118435} and \citeyear{wray1949marginal}, respectively)
\cite{10.2307/2118435,wray1949marginal,10.2307/23702539,Brown01041990}.
\ali{Something like this? Should I dig deeper?
I can line these points up in a way to make the next paragraph sort of obvious or inevitably,
depending on how actively engaged someone is while reading this\dots}

\topic{Digital media have expanded the scope of
viable piecework by pushing drastically on the limits
cited by piecework researchers.}
The research on piecework tells us that
we should expect piecework to thrive in industries where
the nature of the work is limited in complexity
\cite{Brown01041990}.
% \msb{I don't recall this point in the prior section:
% I remember management overhead and fixed costs of materials,
% I don't remember a paragraph about complexity.
% If we want to draw on that point, make it in the earlier section}
Given the flourishing of on--demand labor platforms such as
Uber, AMT, and others, we ask ourselves
what --- if anything --- has changed.
We argue that
the Internet has trivialized
the costs and challenges of the earlier limiting factors %  for two reasons:
% \begin{inlinelist}
  % \item 
  because technology make it easier
  \begin{inlinelist}
  \item to do complex work aided by computers and
  \item to evaluate and manage workers as they do increasingly complex work,
        even observing their work to an otherwise unprecedented granularity.
  \end{inlinelist}

%   \item The Internet allows us to leverage the benefits of
%         ``economies of scale'' at very little cost
%         to the system--designer \cite{lessig2006code,miller2011understanding}.% \msb{I don't understand that argument. What is scaling here? The people? The tools for complex work?}
% \end{inlinelist}

\topic{Technology has made it possible
for non--experts to do work that was once considered
within the domain of experts.}
\msb{I don't yet buy the following argument. If the point is that technology makes us more expert, I disagree that the CrowdCrit/Voyant systems are using technology to do this. They are building the smarts into their OWN workflows, rather than giving workers EXTERNAL tools that make them smarter. Giving workers a calculator is an external tool; the mathematical tables project already demonstrated that you can build smarts into the workflow if you don't have one.}
\citeauthor{yuanAlmost} builds on the work of others
(\textit{Voyant} and, more relevantly, \textit{CrowdCrit})
to design workflows that yield ``expert--level feedback''
\cite{yuanAlmost,Xu:2014:VGS:2531602.2531604,Luther:2014:CCA:2556420.2556788}.
This body of work identifies ways to transform a variety of duties comprising complex tasks
and distills them into ``a narrow set of routinized duties'',
informed in part by researchers --- acting as inspectors --- and experts
\cite[quotations from][]{10.2307/23702539}.
Where \citeauthor{10.2307/23702539} would call additionally for the identification of
crowdsourcing's version of ``piecework clerks'', we point out that
today algorithms manage workers as pieceworkers once did
\cite{uberAlgorithm,10.2307/23702539}.

\topic{Furthermore, technology more directly facilitates the subversion of expertise requirements
by giving non--experts access to information that would otherwise be unavailable.}
Taxi drivers in London endure rigorous training to pass a test known as ``The Knowledge''
--- a demonstration of the driver's comprehensive familiarity.
Researchers have identified significant growth of
the hippocampal regions of the brains in veteran drivers,
generally understood to be responsible for spatial functions such as navigation
\cite{Maguire11042000,Maguire2894,Skok:1999:KML:299513.299625,
      skok2000managing,Woollett1407,woollett2011acquiring}.
Services such as Google Maps \& Waze make it possible for
people entirely unfamiliar with a city
to know more about a city even than experts through
the collective data generated by other users
ranging topics such as police activity, congestion, construction, etc\dots
\cite{silva2013traffic,hind2014outsmarting}. \msb{what's the insight I should take away from this paragraph? what does this say about crowd work?}

\subsubsubsection{\implication}
\msb{This paragraph can be expanded to make a more concrete argument. What will be possible? What won't?}
\topic{The piecework literature gives us a template for pushing the boundaries of complexity in piecework, but
it also signals some of the ultimate limitations of crowd work and piecework in general.}
While the threshold preventing task requesters from utilizing piecework
has dropped thanks to affordances of the Internet, the ceiling on task complexity hasn't moved significantly. \msb{is that your prediction? I would argue against the fact that it hasn't moved significantly, crowdforge did far more complex work, as did flash teams and flash orgs}
If we're to make use of \citeauthor{Brown01041990}'s prescriptions,
we would benefit from finding ways to decompose varied tasks into homogeneous microtasks. \msb{isn't that what we've been doing all along?}

\msb{this doesn't seem like a concrete prediction. what would piecework say will happen if we didn't resolve the tension?}
We should also consider exploring the limitations that algorithmic management bring along more carefully.
While research has touched on this subject, we've yet to make out the bigger picture of this theme
\cite{uberAlgorithm}.
If we can resolve the tension between workers and perilously antagonistic managers, as \citeauthor{10.2307/2118435} suggest,
then we may be able to break a toxic cycle of mistrustful requesters \cite[for example][]{MaliciouscrowdworkersGadiraju}
and develop more considerate platforms as \citeauthor{takingAHITMcInnis} advocate
\cite{takingAHITMcInnis}.

Finally, and perhaps most importantly,
we need to replicate the success of narrowly slicing education and training for expert work
as \citeauthor{hart2013rise} and \citeauthor{grier2013computers} described in their piecework examples
\cite{hart2013rise,grier2013computers} \msb{remind us of what those were}.
That is, we need to identify new ways to train crowdworkers for uniquely narrowly defined work. \msb{I don't understand: why?}
To some extent, an argument can be made that
MOOCs and other online education resources
provide crowd workers with the resources that they need, but 
it remains to be seen whether that work will be appropriately valued, let alone
properly interpreted by task solicitors
\cite{aguaded2013mooc}.
If we can overcome this obstacle,
we might be able to empower crowd workers to do complex work such as engineering and metalworking,
rather than doom them to match girl reputations:
``brash, irregular, immoral, and uneducated''
\cite{10.2307/3827491}. \msb{how will that reduce any of the problems except the last one?}


\onlyinsubfile{
  \balance{}
  \printbibliography
}

\end{document}