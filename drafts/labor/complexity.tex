\documentclass[trackingWork]{subfiles}


\makeatletter
\def\blx@maxline{77}
\makeatother
\begin{document}

\begin{comment}
Crowdwork
  - Kittur said let's do complex stuff
  - This works by using CS techniques
  - Clear that this works in focused cases
  - More recent shift toward using experts
\end{comment}

\subsection[What are the complexity limits of crowd work]
{Identifying the Complexity Limits of Crowd Work}\label{sec:complexity}
A key question to the future of crowd work is
\textit{what} precisely will become part of this economy.
Paid crowdsourcing began with simple microtasks on platforms such as
Amazon Mechanical Turk, but
microtasks are only helpful if they build up to a larger whole.
So, our first question:
how complex can the work outcomes from crowd work be?

\subsubsection{\crowdworkpers}

\topic{Crowdsourcing research has spent the better part of a decade
proving the viability of crowdsourcing in complex work.}
Unless crowdsourcing can demonstrate viability for meaningfully complex tasks,
the argument runs,
it will be incapable of ensuring a pro--social outcome for work and workers
\cite{crowdworkFuture}.
\citeauthor{crowdForgeKittur} first opened the question of
whether crowdsourcing could be used for goals that are not simple parallel tasks
\cite{crowdForgeKittur}.
Their work demonstrated proof--of--concept crowdsourcing of
a simple encyclopedia article and news summary
--- tasks which could be verified or repeated
with reasonable expectations of similar outcomes.
Seeking to raise the complexity ceiling
\cite{myers2000past},
researchers have since created
additional proof--of--concept applications and techniques,
including conversational assistants \cite{Lasecki:2013:CCC:2501988.2502057},
medical data interpreters \cite{Lasecki:2013:CCC:2501988.2502057}, and
idea generation \cite{YuEncouragingOutside,yu2014distributed,Yu2016a},
to name a few examples.

\topic{To achieve complex work, this body of research has often applied ideas from Computer Science to design new crowdsourcing workflows.}
Beginning with a goal that has
presented significant challenges for computers,
the researcher leverages an insight from Computer Science
(for example, MapReduce
\cite{crowdForgeKittur} or
sequence alignment algorithms
\cite{lasecki2012real})
and arranges humans as computational black boxes within those approaches.
% \ali{We could quote \citeauthor{storiesIraniSilberman} and say ``human APIs'',
% but that gets a little critique--y, and we might want to save that for later.}
This approach has proven a compelling one because
it leverages the in--built advantages of
scale,
automation, and
programmability that software affords.

It is now clear that this approach works with focused complex tasks, but
the broader wicked problem that each complex goal represents remains unsolved
\cite{rittel1973dilemmas}.
As a first example,
idea generation shows promise
\cite{YuEncouragingOutside,yu2014distributed,Yu2016a},
but there is as yet no general crowdsourced solution for
the broader goal of invention and innovation
\cite{fuge2014analysis}.
Second,
focused writing tasks are now feasible
\cite{Kim2017,bernsteinSoylent,Nebeling:2016:WCW:2858036.2858169,
      writingMicroTasks,agapie2015crowdsourcing}, but
there is no general solution to create
a cross--domain, high--quality crowd--powered author. 
Third,
data analysis tasks such as
clustering \cite{chilton2013cascade},
categorization \cite{andre2014crowd}, and
outlining \cite{luther2015crowdlines}
are possible, but there is no general solution for sensemaking.
It is not yet clear what insights would be required
to enable crowdsourced solutions for these broader wicked problems.
%  and
% other crowdsourced goals such as addressing climate change
% \cite{introne2011climate}.

Restricting attention to non--expert, microtask workers proved limiting.
So, \citeauthor{foundry} introduced the idea of crowdsourcing with
online paid \textit{experts} from platforms such as Upwork.
Expert crowdsourcing enables access to a much broader set of workers,
for example designers and programmers.
The same ideas can then be applied to expert ``macro--tasks''
\cite{cheng2015break,haas2015argonaut}, enabling the crowdsourcing of goals such as user--centered design \cite{foundry},
programming \cite{latoza2014microtask,Fast2016,Chen2016}, and
mentorship \cite{suzukiAtelier}.
However, there remains the open question of
how complex the work outcomes from expert crowds can be.


\subsubsection{\pieceworkpers}

\begin{comment}
- Farm workers-->textile
- Limit: human management and oversight
- Evaluation
- Skilled work harder
- Only some organizations can use it
- Management practices
\end{comment}

\topic{\citeauthor{grier2013computers} gives early accounts of
a piecework strategy in Airy's creation of
the British Nautical Almanac
\cite{grier2013computers}.}
Airy's goal was complex
--- mathematical calculations to produce tables that would
allow sailors to locate themselves by starlight from sea.
Many of his contributors did not have high--level mathematical training,
so Airy broke down the task into simpler calculations and
distributed them by mail,
accomplishing the complex goal through piecework tasks that paid little.

However, when piecework entered the American economy,
it was not used for complex work.
One reason for low complexity was workers' skills: it was infeasible to provide new pieceworkers with the comprehensive education
that apprenticeships imparted \cite{hart2013rise}.
So, initially piecework arose for farm work, and as
\citeauthor{hughRaynbirdTaskWork} and others discuss,
the practice remained relatively obscure until
it blossomed in the textile industry
\cite{hughRaynbirdTaskWork}.
Complexity levels remained low at the turn of
the \nth{20} century as piecework saturated New York City
\cite{riisOtherSideLives}.
However, writers of the time focused their attention on
wage \cite{burton1899commercial} and
management regimes \cite{norton1900textile}
rather than training.


Measurement also limited the complexity of piecework:
only tasks that could be measured and priced could be completed via piecework. 
When \citeauthor{Brown01041990} investigated
what limited the adoption of piecework in industries that otherwise gravitated toward it
(e.g., railway engineers),
the homogeneity of tasks arose as a major contributing factor~\cite{Brown01041990}.
\citeauthor{10.2307/23702539} concurs via a case study of the Santa Fe Railway,
which used ``efficiency experts'' to develop a ``standard time''
to determine pay for each task at the company informed by
``thousands of individual operations''% removed ``at the Topeka shop'' because even I had to reach back into the back of my mind to be like ``Oh, yeah, Kansas''.
~\cite{10.2307/23702539}.
One might conclude from \citeauthor{10.2307/23702539}'s observations that
complex, creative work
--- which is inherently heterogeneous and difficult to routinize ---
would be unsuitable for piecework.

\topic{Piecework was limited to tasks that could be clearly evaluated.}
For example, the roles required to facilitate piecework in the early \nth{20} century included ``piecework clerks, inspectors, and `experts'''~\cite{10.2307/23702539}.
\citeauthor{hart2016rise} argues that evaluation causes an ultimate complexity limit:
at some point, evaluating multidimensional work for quality
(rather than for quantity) becomes infeasible.
In his words,
``if the quality of the output is more difficult to measure than the quantity [\ldots]
then a piecework system is likely to encourage
an over--emphasis on quantity produced and an under--emphasis on quality''
\cite{hart2016rise}.
Complex work, which is often subjective to evaluate, falls victim to this criteria.

\topic{This focus on measurement and tracking had consequences.}
\citeauthor{10.2307/23702539} suggests that the first sparks of scientific management
could be found in piecework:
the approach of paying workers for each piece of output necessitated
the rigorous tracking, measurement, and training of workers
for which scientific management became famous
\cite{10.2307/23702539}.
If true, 
the concurrent upswing of
scientific management and Fordism
through the first two--thirds of the \nth{20} century
alongside piecework was not only understandable, but predictable
\cite{hart2013rise}.

\topic{Piecework researchers also argue that,
in addition to constraints on the kind of \textit{work} that's amenable to piecework,
only certain kinds of \textit{organizations} were amenable to piecework.}
Researchers detail three organizational criteria.
First, \citeauthor{Brown01041990} argues that
piecework ``is less likely in jobs with a variety of duties than in
jobs with a narrow set of routinized duties''
\cite{Brown01041990}.
\citeauthor{SJOE:SJOE371} points out the phenomenon here as a market effect:
``in an environment with multi--tasking,
pay schemes based on tightly specified performance may
induce workers to neglect tasks that are less easy to measure''
\cite{SJOE:SJOE371}.
Second,
complexity was limited by access to capital to create the necessary infrastructure.
As \citeauthor{10.2307/23702539} reports,
only the largest and most wealthy railroads had the resources
necessary
% ``pay the overhead involved in installing work reorganization''
~\cite{10.2307/23702539}.
Third, organizations required capable managers in charge of the pieceworkers.
The West Virginia mines, for example, hired foremen 
to be the intermediary between upper management and the workers~\cite{10.2307/2118435}.
These foremen were responsible for allocating resources and
understanding when and how to modify work as necessary
\cite{wray1949marginal}.
So, in sum,
organizations historically could only take advantage of piecework if they had
homogeneous work to be done,
access to capital to purchase the necessary equipment, and
the ability to hire people who could serve as intermediaries between pieceworkers and management.

\topic{The research seems to suggest that it was difficult to apply piecework to more skilled work,
particularly because maximizing the advantages of piecework seemed to reward
smaller, more constrained, more narrowly--trained tasks, and only in organizations that could pay for the equipment and people to enable it.}
For most of the \nth{19} century,
piecework was applied almost exclusively to farm and textile work.
Work was simple and widely understood
--- farm workers didn't need to be trained on how to plow fields, or birth foals;
seamstresses knew how to sew together denim
\cite{10.2307/2338394,riisOtherSideLives}.



\subsubsection{\whatchanged}
\begin{comment}
  mangeerial overhead limits, so what's different
  more people can now do complex work without training (more complex)
  parts of management can be automated (more firms)
  cheaper to create the infrastructure (more complex)
\end{comment}
% \topic{Piecework makes a number of observations leading to the conclusion that
% piecework's complexity is fundamentally bounded by several limitations,
% chief among them the costs of managerial overhead and the transition thereto.}
% \citeauthor{Brown01041990} and \citeauthor{10.2307/23702539}'s claims that
% organizations can't adopt piecework unless
% they're sufficiently large to absorb the cost of transitioning to a piecework system;
% \citeauthor{10.2307/2118435} and \citeauthor{wray1949marginal}'s observations for
% the importance of competent, effective managerial oversight
% --- a human resource, which made the scaling cost prohibitively expensive for many
% \cite{10.2307/2118435,wray1949marginal,10.2307/23702539,Brown01041990}.

% \topic{Digital media have expanded the scope of
% viable piecework by pushing drastically on the limits
% cited by piecework researchers.}
The research on piecework tells us that
we should expect piecework to thrive in industries where
the nature of the work is limited in complexity
\cite{Brown01041990}.
Given the flourishing of on--demand labor platforms such as
Uber, AMT, and others, we ask ourselves
what --- if anything --- has changed.
We argue that
the Internet has trivialized
the costs and challenges of the earlier limiting factors
because technology makes it easier
\begin{inlinelist}
\item for workers to do complex work without training,
\item to manage workers in doing complex work, and 
\item to create the infrastructure necessary to manage the workers.
\end{inlinelist}

\topic{Technology increases non--experts' levels of expertise by giving access to information that would otherwise be unavailable.}
For example, taxi drivers in London endure rigorous training to pass a test known as ``The Knowledge''
--- a demonstration of the driver's comprehensive familiarity with the city's roads.
This test is so challenging that veteran drivers exhibit significantly larger
the regions of the brain associated with spatial functions such as navigation
\cite{Maguire11042000,Maguire2894,Skok:1999:KML:299513.299625,
      skok2000managing,Woollett1407,woollett2011acquiring}.
In contrast, with on--demand platforms such as Uber, services such as Google Maps \& Waze make it possible for
people entirely unfamiliar with a city
to operate profesionally~\cite{silva2013traffic,hind2014outsmarting}.
Other examples include search engines enabling information retrieval, and word processors enabling spelling and grammar checking.
By augmentating the human intellect~\cite{engelbart2001augmenting}, computing has shifted the complexity of work that is possible without training.

Algorithms have automated some tasks that previously fell to management.
Computational systems hire workers~\cite{turkitLittle,weld2010decision}, as well as direct their activities~\cite{uberAlgorithm}, and act as ``piecework clerks''~\cite{10.2307/23702539} to inspect, modify and combine work~\cite{turkopticon,takingAHITMcInnis}.
In many cases, the intermediary function has been removed as well, leading workers to need to directly email requesters for clarification and feedback~\cite{martin2014being}.
These algorithms, however, are less able than human managers to manage contingencies that were not programmed into them.

% \ali{maybe this should go away?
% \topic{Technology has made it possible
% for non--experts to do work that was once considered
% within the domain of experts.}
% \citeauthor{yuanAlmost} builds on the work of others
% (\textit{Voyant} and, more relevantly, \textit{CrowdCrit})
% to design workflows that yield ``expert--level feedback''
% \cite{yuanAlmost,Xu:2014:VGS:2531602.2531604,Luther:2014:CCA:2556420.2556788}.
% This body of work identifies ways to transform a variety of duties comprising complex tasks
% and distills them into ``a narrow set of routinized duties'',
% informed in part by researchers --- acting as inspectors --- and experts
% \cite[quotations from][]{10.2307/23702539}.
% Where \citeauthor{10.2307/23702539} would call additionally for the identification of
% crowdsourcing's version of ``piecework clerks'', we point out that
% today algorithms manage workers as pieceworkers once did
% \cite{uberAlgorithm,10.2307/23702539}.}
Finally, the organizational limit on infrastructure creation is somewhat lessened. Writing web scripts takes fewer people and fewer hours than creating physical equipment for piecework.
\citeauthor{turkitLittle}'s vision was that any user with basic programming skills could tap into on--demand human intelligence.
As better toolkits lower this threshold~\cite{myers2000past} and computational thinking diffuses, a broader population will be able to use crowd work.

\subsubsection{\implication}
% \topic{The piecework literature gives us a template for pushing the boundaries of complexity in piecework, but
% it also signals some of the ultimate limitations of crowd work and piecework in general.}
% While the threshold preventing task requesters from utilizing piecework
% has dropped thanks to affordances of the Internet, the ceiling on task complexity hasn't moved significantly.
% \msb{is that your prediction? I would argue against the fact that it hasn't moved significantly, crowdforge did far more complex work, as did flash teams and flash orgs}
% If we're to make use of \citeauthor{Brown01041990}'s prescriptions,
% we would benefit from finding ways to decompose varied tasks into homogeneous microtasks. \msb{isn't that what we've been doing all along?}
Technology's ability to support human cognition will enable stronger assumptions about workers' abilities, increasing the complexity of crowd work outcomes.
Just as the shift to expert crowdsourcing increased complexity, so too will workers with better tools increase the set of tasks possible.
Beyond this, further improvements would most likely come from replicating the success of narrowly--slicing education for expert work as \citeauthor{hart2013rise} and \citeauthor{grier2013computers} described in their piecework examples
of human computation \cite{grier2013computers} and drastically reformulating macro--tasks given the constraints of piecework \cite{hart2013rise}.
To some extent, an argument can be made that
MOOCs and other online education resources
provide crowd workers with the resources that they need, but 
it remains to be seen whether that work will be appropriately valued, let alone
properly interpreted by task solicitors
\cite{aguaded2013mooc}.
If we can overcome this obstacle,
we might be able to empower more crowd workers to do complex work such as engineering and metalworking,
rather than doom them to ``uneducated'' match girl reputations~\cite{10.2307/3827491}.
However, many such experts are already available on platforms such as Upwork, so training may not directly increase the complexity accessible to crowd work unless it makes common expertise more broadly available.

Will the shift from human managers to Turing--complete algorithms raise the complexity ceiling? 
By the Turing test, the algorithms would be at best indistinguishable from human piecework clerks and foremen.
%At worst, as now, they are far simpler~\cite{uberAlgorithm}.
So in terms of enabling coordination, algorithmic management is unlikely to directly raise the ceiling beyond what piecework could achieve.
However, as a resource constraint, algorithms are a fixed cost and not a per--person cost like human managers.
So in terms of accessibility, algorithms will allow a broader class of organizations and individuals to afford crowd work.
This shift may enable complex goals that were not cost--effective before to become feasible.
However, because algorithms remain far from replicating all of the foremen's responsibilities, most likely is a middle ground in which crowd work re--introduces the human element to management in a more targeted way~(e.g.,~\cite{haas2015argonaut,kulkarni2012mobileworks,crowdguilds}).
This move will require resolving the tension between workers and perilously antagonistic managers, as \citeauthor{10.2307/2118435} suggest, to break a toxic cycle of mistrustful requesters \cite{MaliciousCrowdworkersGadiraju}.
% and develop more considerate platforms as \citeauthor{takingAHITMcInnis} advocate
% \cite{takingAHITMcInnis}.
% \ali{What will happen if we don't do this?
% Maybe this is where I should bring up
% the ``market for lemons'' metaphor and say that
% the market will just continue to decline in quality until
% nobody is willing to rely on it either as a source of data/information (requesters), or
% a means of sustaining oneself financially (workers).}

Finally, the cost of creating piecework infrastructure has dropped. 
Expensive manufacturing equipment has been largely replaced by computer code~\cite{lessig2006code}.
As with lowered costs of management, lowered infrastructure costs will make crowd work accessible to a broader set of people and organizations.
This in and of itself does not raise the complexity ceiling, but by broadening the potential market for crowd work, it may enable a new set of goals and needs take part.

\onlyinsubfile{
  \balance{}
  \printbibliography
}

\end{document}