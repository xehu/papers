\documentclass[trackingWork]{subfiles}
\makeatletter
\def\blx@maxline{77}
\makeatother
\begin{document}
% \newcommand{\note}[1]{\onlyinsubfile{\ali{#1}}}

% lol oh my gosh i just discovered figlet and it's changing my life
% it's like i'm distributing warez


\section{Major Research Questions}
\ali{4 paragraphs on what happened in the above contexts; 2 paragraphs on best/worst outcomes \& why; ??? paragraphs on predictions for crowdsourcing}

%  _ _           _ _                __ 
% | (_)_ __ ___ (_) |_ ___    ___  / _|
% | | | '_ ` _ \| | __/ __|  / _ \| |_ 
% | | | | | | | | | |_\__ \ | (_) |  _|
% |_|_|_| |_| |_|_|\__|___/  \___/|_|  
                                     
%                            _                          _             
%   ___ _ __ _____      ____| |___  ___  _   _ _ __ ___(_)_ __   __ _ 
%  / __| '__/ _ \ \ /\ / / _` / __|/ _ \| | | | '__/ __| | '_ \ / _` |
% | (__| | | (_) \ V  V / (_| \__ \ (_) | |_| | | | (__| | | | | (_| |
%  \___|_|  \___/ \_/\_/ \__,_|___/\___/ \__,_|_|  \___|_|_| |_|\__, |
%                                                               |___/ 

\subsection{What are the limits of crowdsourcing?}
Research in crowdsourcing has spent the better part of a decade
exploring ways of growing the limits of crowdsourcing and
finding the boundaries of crowd work and microtasks;
  identifying challenges to this form of labor,
  overcoming them through novel designs of work--flows and processes,
and repeating the process
\cite{bernsteinSoylent}.
The question that has emerged
among these researchers and
through the work that they have produced then
has been driving at \textit{whether} there are limits to crowdsourcing
(and, if so, what factors determine those limits).
Through this lens, we can point to
a number of contributions to the field that have extended the boundaries of crowd work.

The exploration of crowdsourcing's potential and limits has principally looked at
manipulating and extending along three dimensions:
\begin{inlinelist}
  \item \nameref{sec:decomposition},
  \item \nameref{sec:workAbstraction}, and
  \item \nameref{sec:flexibility}
\end{inlinelist}.
% We will further discuss the ways that various contributions drive along these dimensions.
As we'll discuss these dimensions,
exploring the extents to which work can be
decomposed,
contextually abstracted, and
made more resilient to attrition of various forms,
we'll also point to corresponding piecework literature addressing these aspects.
Finally, we'll discuss how these elements will serve
to constrain the upper and lower bounds of crowdsourcing as it relates
to the question of the furthest limits of crowdsourcing.


\ali{At this point I'm trying to retain as much of the old content as possible,
but if the next few sections seem like I'm trying too hard to be lazy,
say so and I'll do a rewrite rather than refactoring.}


%      _                                          _ _   _             
%   __| | ___  ___ ___  _ __ ___  _ __   ___  ___(_) |_(_) ___  _ __  
%  / _` |/ _ \/ __/ _ \| '_ ` _ \| '_ \ / _ \/ __| | __| |/ _ \| '_ \ 
% | (_| |  __/ (_| (_) | | | | | | |_) | (_) \__ \ | |_| | (_) | | | |
%  \__,_|\___|\___\___/|_| |_| |_| .__/ \___/|___/_|\__|_|\___/|_| |_|
%                                |_|                                  
% lol i can't believe this worked i'm going to do this with every (sub)*section.

\subsubsection{Decomposition}\label{sec:decomposition}
Scholarship describing and exploring
the decomposition of tasks is perhaps the most established
of the previously mentioned three areas
to HCI researchers;
\citeauthor{crowdForgeKittur} specifically
drive at this goal by addressing the possibility of
crowdsourcing complex work
\cite{crowdForgeKittur}.
\citeauthor{cheng2015break} found that microtasks
--- though not necessarily \textit{faster} than ``macrotasks'' ---
yield higher quality work,
particularly when that work is susceptible to frequent interruptions
\cite{cheng2015break}.

Much of the research in the space of designing crowd work has
sought to illustrate the potential to take highly creative or skilled work
and generate high--quality results.
Perhaps the most notable case study here can be found in
\citeauthor{foundry}'s \textit{Foundry}, which employed
``flash teams'' to achieve expert--level outcomes via thoughtful
decomposition of work as ``modular tasks''
\cite{foundry}.

Research exploring the decomposition of work more generally
--- that is, without the constraint of an employer/requester directing workers ---
has since emerged as well.
\citeauthor{selfsourcingTeevan2014} further push the boundaries of decomposed work,
exploring ``selfsourcing'', and further this work with \citeauthor{selfsourcingTeevan2016}
\cite{selfsourcingTeevan2014,selfsourcingTeevan2016}.
While some of this work doesn't strictly fall under ``crowdsourcing'',
the topicality of this broader body of work will become more apparent
as we trace the parallels of crowdsourcing with piecework.

The work described thus far has illustrated myriad ways that
we can manage workers for the purpose of accomplishing complex tasks with
rapidly sourced workers from across the world.
Critically, the work's defining characteristic involves
the use of the Internet as a medium both 
to coordinate workers, as well as
to do the work itself.
But research into the ``decomposition'' of work more generally
illustrates the same concepts of work that ``Taylorism''
and scientific management sought to embody
---
\citeauthor{professionalCrowdworkEthics} in particular foresaw this danger and warned of it in
\citeyear{professionalCrowdworkEthics}
\cite{crowdworkFuture,professionalCrowdworkEthics,nickerson2013crowd}.
In both the historical and contemporary cases of decomposed work,
work was,
at least initially,
distributed in the form of tasks to the homes of workers;
\citeauthor{riisOtherSideLives} captured this in
in \citeyear{riisOtherSideLives}
\cite{riisOtherSideLives}.

Following this thread of work, it may be said that
much of the work in crowdsourcing has found ways of ``vertically slicing'' work
such that each person is responsible for as small a task as can be made reasonable
--- taking a minute of audio, for instance, and transcribing just that.
The aforementioned research has found many novel ways to slice work,
communicate different amounts of context of work from one worker to the next,
etc\dots
but fundamentally they follow similar patterns.
It's through this lens that we see the echoes of piecework.

%                        _ _      _ 
%  _ __   __ _ _ __ __ _| | | ___| |
% | '_ \ / _` | '__/ _` | | |/ _ \ |
% | |_) | (_| | | | (_| | | |  __/ |
% | .__/ \__,_|_|  \__,_|_|_|\___|_|
% |_|                                         

Piecework has seen work along this dimension spanning decades;
\citeauthor{thompson1913time} investigate some of the ways that
construction can benefit from the principles of scientific management.
\citeauthor{thompson1913time}'s thesis asserts that
task work is predicated on the accurate scientific management of work,
including the ``miscellaneous tasks''.
\citeauthor{thompson1913time} argues
--- as early as \citeyear{thompson1913time} ---
that ``\dots one may be challenged to find any class of work
involving labor either indoors or out--of--doors
where tasks cannot be fixed by proper time--study''
\cite{thompson1913time}.




% Piecework's growing popularity was best documented at the turn of the \nth{20}
% by \citeauthor{riisOtherSideLives} % in his work on \citetitle{riisOtherSideLives}
% \cite{riisOtherSideLives}.



Broken down in this way, work could grow to unprecedented scales,
but the quality of the work would remain relatively variable
\cite{murray1983decentralisation}.
Textile work being a salient example,
it took time for workers to acquire sufficient skill
to do every aspect of the work so that the garment would be accepted by the company soliciting that work
\cite{vezina1992light}.

A compelling solution emerged in the early \nth{20} century to break tasks down into discrete,
manageable routines that could be taught relatively easily,
and whose work output could be evaluated in abstraction from the rest of the work
\cite{restructuringPieceworkBaker}.
In Ford's assembly line, this meant that workers were not responsible for building a whole car,
but a single very narrowly defined action that needed to be done on every car
\cite{towardsGlobalFordism}.
By the mid--\nth{20} century, \citeauthor{schoenberger1988fordism} writes,
``\dots~the intensification of the labor process is argued to have hit mental, physical, and social limits.''
\cite{schoenberger1988fordism}.

This approach, ``Fordism'' (and its better--known contemporary ``Taylorism'' of similar ethos),
can be seen today in crowd work and on--demand labor through the application of micro--tasks.
\citeauthor{writingMicroTasks} highlight some of the advantages of breaking work into pieces,
facilitating evaluation and parallelization
\cite{writingMicroTasks}.
By decomposing and recomposing tasks,
and in particular by assigning similarly natured work to the same workers,
workers could become ``experts'' in a small aspect of the work that they did,
speeding their work dramatically
\cite{delayAndOrderLasecki}.
Perhaps more important, however, was that
the breaking down of work into tasks has made it more practical to evaluate work at each stage
\cite{rogstadius2011assessment}.

\subsubsection{So how does this affect crowdwork?}
The work we've seen so far 
\begin{itemize}
  \item worst case: assembling iPhones (extant)
  \item average case: railroad workers and assembly lines
  \item high (complexity) case: 
\end{itemize}

% what can be decomposed? what can be evaluated?
% technology/turuing completeness gets us a win
% predefined expertise? you can find someone almost instantly;
% ease of doing this makes it qualitatively different due to Channel Factors (diverting a channel/river)
% huddler paper cites some decomposition stuff workflows are good/bad at things
% 3-way slack w/msb and mav
% academy of management 
% van der ven paper ~1972 limits of workflows cited in daniela's flash orgs paper



%                     _            _         _                  _   _             
% __      _____  _ __| | __   __ _| |__  ___| |_ _ __ __ _  ___| |_(_) ___  _ __  
% \ \ /\ / / _ \| '__| |/ /  / _` | '_ \/ __| __| '__/ _` |/ __| __| |/ _ \| '_ \ 
%  \ V  V / (_) | |  |   <  | (_| | |_) \__ \ |_| | | (_| | (__| |_| | (_) | | | |
%   \_/\_/ \___/|_|  |_|\_\  \__,_|_.__/|___/\__|_|  \__,_|\___|\__|_|\___/|_| |_|
% lollllllllllllllll


\subsubsection{Work Abstraction}\label{sec:workAbstraction}
Decomposition allows requesters to assign tasks without concern for the broader context.
While we'll discuss this aspect of crowd work more critically later,
it's worth pointing out that discrete blocks of work containing all the relevant context for a worker
allows workers to engage with virtually any component of work without worrying that their lack of 
higher--level awareness of the goals of the requester might negatively affect their work.

\citeauthor{chilton2013cascade} perhaps best illustrated this with
\textit{Cascade} by demonstrating that it's possible to
break certain classes of tasks apart
in such a way that they yield taxonomies of various subjects,
a task previously thought to be safely within the domain of expert workers
with top--down awareness of the context of the work as a whole
\cite{chilton2013cascade}.
\citeauthor{verroios2014context} further illustrate this potential by
forming a task one might consider highly contextually dependent
--- summarizing the contents of a movie ---
in such a way that crowd workers could contribute small pieces of work without
needing to know the content of the rest of the project
\cite{verroios2014context}.

Here, \citeauthor{hu1961parallel}'s work,
saying of assembly line work that
``it is assumed that men are of equal ability and every man can do any of the $n$ jobs'',
parallels the approach that dominated early research into crowd work
--- namely, using non--expert crowds for complex work
\cite{hu1961parallel}.
This mindset in \citeauthor{hu1961parallel}'s analysis,
and indeed the study of factory and mass manufacturing labor through the \nth{20} century,
substantively owes its existence to scientific management
and the rigorous decomposition of work into tasks, discussed earlier,
and persists to this day as it colors
researchers' goals and objectives in the study and design of crowd work.

Piecework's influence on the abstraction of work into tasks,
described above, is more than just caused by the decomposition of work;
work abstraction itself makes it possible for workers to come and go flexibly,
prompting work requesters to consider ways to design these now discrete tasks in ways that
maximize flexibility, both by allowing (and even anticipating) some inconsistency in worker availability
\textit{and} allowing and anticipating some inconsistency in the quality of the work output itself.
It's to this area that we now turn our attention.










\subsubsection{Flexibility}\label{sec:flexibility}
% \msb{nothing in this subsubsection connects to the piecework literature. bring it back!}

Earlier we discussed \citeauthor{cheng2015break}'s work
measuring the impact that interruption has on worker performance.
This work both points to and embodies a broader sentiment in
both the study and practice
of crowd work that microtasks should be designed resiliently against the variability of workers,
% \msb{not sure what flexibly means here, and the rest of the sentence isn't helping me unpack it. flexibility for requester? worker? flexible algorthms?},
fully exploiting the abstracted nature of each piece of work
\cite{interruptionIqbal,delayAndOrderLasecki,vaish2014low}.
That is to say, micro--tasks should be designed such that a single worker's poor performance,
or a good worker's sudden departure,
would not significantly impact the agenda of the work as a whole.
While \citeauthor{cheng2015break} identified costs with breaking tasks into smaller components
in the form of higher cumulative time to complete
(albeit much shorter real time to complete, owing to parallelization),
\citeauthor{delayAndOrderLasecki} found that at least \textit{some} performance can be recouped by stringing 
similar tasks together.


Given the importance of consistent work results, one might intuit that
requesters would prefer high--quality workers who can be relied upon to be available
(even for contextually independent tasks),
which would appear to contradict the benefits of flexibility already discussed;
requesters have thus made significant headway toward
``embracing error'' to allow requesters to maximize the benefits of a flexible,
even transient,
workforce.

\citeauthor{embracingErrorKrishna} offer orders--of--magnitude improvements
in various binary classification tasks
on the principle that diverse workers complete these tasks
in order to accurately inform the model on the variety of delays in response times.
And rather than building tasks to \textit{tolerate} worker drop--off and attrition,
some researchers have designed work predicated on the expectation of this phenomenon:
\citeauthor{sensitiveTasks} describe ways of assigning tasks in such a way that
crowd workers would never be given enough information to piece together sensitive information about
any single topic
\cite{sensitiveTasks}.

Flexibility has been explored through the lens of Fordism, perhaps best illustrated by
\citeauthor{tolliday1986between}'s treatment describing
turnover rates rising above 300\% in the decade leading to the introduction of the assembly line in 1913.
Specifically, the utilization of ``\dots~`semi--special' machine tools which could be adapted
[and]~\dots~added flexibility through seasonal layoffs for production workers and the use of
piece rates~\dots~rather than a day wage system''
\cite{tolliday1986between}.

In the field of piecework,
the research covering this topic has both explored
a breadth of tasks that might be rendered doable by piecemeal workers
\textit{as well as} longitudinally documented the success of these approaches.
Here, we 

Points to make:
  \begin{itemize}
    \item Blossoming of piecework
    \begin{itemize}
      \item high point: consultants?
      \item most cases: auto workers/etc\dots
      \item worst case: sweatshops (especially in developing nations)
    \end{itemize}
    \item what led to these outcomes?
    \begin{itemize}
      \item ``consultant'' work came out well because the work was complex;
      this made it difficult to turn into a mass market commodity.
      We see consultants ranging skill levels like
      oDesk (implementing modules) to
      Accenture (on--demand teams of consultants).
      \item auto workers,
      working in settings where capital couldn't be moved as easily
      found themselves in the same workspace as a direct
      --- if multi--stepped ---
      result to the benefits of putting people in the same place to consolidate resources.
      Moreover,
      workers had leverage over factory owners as a result of that consolidated capital;
      operating equipment required training.
      \item sweatshop workers fared the worst,
      for reasons that may seem obvious with hindsight.
      Source materials for textiles are easier to ship than mechanical components such as engines,
      making it easier for factories to relocate to developing nations
      (where cost--of--living, and consequently wages, would be lower).
      As wages, CoLAs, and QoL rise, workers begin asking for
      (and later demanding)
      higher wages and better conditions.
      But where the auto workers have leverage,
      textile workers find only a precarious economic balance now tipped by their collective action,
      spurring manufacturers to move to a new locale
    \end{itemize}
  \end{itemize}





\subsection{What forms of work design and worker management are viable?}
\begin{itemize}
  \item researchers have looked at how to increase worker productivity
  (e.g. finding the maximal speed at which gig workers can be expected to work before making errors)
  \cite{measuringCrowdsourcingCheng}.
  \item we've also seen people ``embrace error''
  \cite{embracingErrorKrishna}.
  \item still other research has looked into ways to sandbox workers from the context of their work 
  \item but scholarship looking into the design and management of work and workers isn't new;
  lots of research into getting pieceworkers to do work more quickly
  \cite{seymour1954manual}.
  \item Researchers have even asked the age old question of \textit{what motivates} pieceworkers
  (echoing similar research on Wikipedia and Mechanical Turk)
  \cite{roy1953work,Nov:2007:MW:1297797.1297798,kaufmann2011more}
\end{itemize}


\subsection{What will work and the place of work look like for workers?}
The metaphorical mechanics of these dynamics are still at play;
workers and managers continue to interact in adversarial manners,
despite substantive work into aligning the motivations of workers and requesters



































\onlyinsubfile{
\clearpage
% \section{Case Studies}
The existing body of research has shed light on on--demand labor from various perspectives,
and revealed a number of topics that,
through our framing, are clearly situated together.
Those topics are, at a high level, as follows:
\begin{enumerate}
\item the \textbf{processes} involved in making work into tasks, or discretization;
\item the outcomes (and indeed the \textbf{fallout}) of that discretization,
both on the work itself as well as the workers;
and finally
\item the \textbf{relationships} between workers and requesters of the work
--- both \textit{cooperative} and \textit{adversarial} cases.
\end{enumerate}






% \msb{I still dont' have a clear conceptual picture of what flexibility is, and what about it sets this section apart. Ranjay's stuff is about speed...? I'm confused.}




\subsection{The Fallout of Crowd Work}\label{sec:Fallout}
% If the research exploring and documenting how we can exploit crowd \textit{work} can be described as wide--ranging,
% the scholarship discussing the ways crowd \textit{workers} have been exploited is more focused;
\citeauthor{turkopticon} point out the disillusion that companies such as Amazon foster on platforms for work like AMT
(see also \citeauthor{dynamo}'s work
continuing in the spirit of this observation to generate collective action to improve worker conditions)
\cite{turkopticon,dynamo}.
\citeauthor{uberAlgorithm}
find similarly that workers on gig work platforms are frustrated by the systems on which they work,
to say little of the policies which these systems enforce
\cite{uberAlgorithm}.

We discussed the benefits of flexibility
(both in the sense of having arbitrary workers perform tasks and
in the sense that we can design tasks to be more resilient to poor work)
in the previous section.
It's from that point in the literature that we turn our attention to
the perhaps unintended effects of crowd work
and the affordances for transience that we build into this mode of work.
We'll address two major areas of work under this subject:
\begin{inlinelist}
\item \nameref{sec:lowPay}; and
\item \nameref{sec:varQualWork}.
\end{inlinelist}

\subsubsection{Low Pay}\label{sec:lowPay}


\citeauthor{laborEconomicsOfCrowdsourcingHorton}
identified problems with crowd work wages relatively early on,
attempting to address this imbalance from a behavioral economic perspective ---
that is, identifying and presenting a model that describes a worker's
``\textit{reservation wage}''
\cite{laborEconomicsOfCrowdsourcingHorton}.
This work has largely informed much of the research into and practice of estimating crowd work compensation
\cite{incentivesShaw,paolacci2010running}.

But we turn to \citeauthor{turkopticon}'s discussion of ``\textit{Turkopticon}'',
a system they designed to interrogate worker invisibility and to promote better wages across several dimensions
\cite{turkopticon}.
Of particular relevance here,
\citeauthor{turkopticon} call to attention that ``Turkers'' are ultimately vulnerable to
wage theft and
pay rates that translate to well under minimum wage.
Returning to \citeauthor{laborEconomicsOfCrowdsourcingHorton},
we find that the median ``reservation wage'' in \citeyear{laborEconomicsOfCrowdsourcingHorton}
was \$1.38, while the mean was \$3.63
\cite{laborEconomicsOfCrowdsourcingHorton}.

Understanding workers' motivations given these conditions has thus become a goal for some researchers
\cite{whyWouldAnyoneBrewer}.
\citeauthor{Sun20111033} conclude that
``\dots~solvers participate in online tasks
not only for money
but also for enjoyment
or the sense of self--worth''
\cite{Sun20111033}.
This might have rung true in \citeyear{Sun20111033},
and certainly corroborates \citeauthor{Ross}'s findings after investigating
``who are the crowdworkers'',
but as \citeauthor{whoareNOTtheTurkers} points out
``we [have since] learned that most tasks on AMT are done by a small group of professional Turkers\dots''
\cite{Ross,whoareNOTtheTurkers}.

Now, \citeauthor{turkopticon}
and later
\citeauthor{dynamo} cite insufficient pay as a central point of frustration among workers,
via \citeauthor{irani2015cultural} and \citeauthor{dawnDigitalSweatshopCushing}'s contributions in this space
\cite{dynamo,irani2015cultural,dawnDigitalSweatshopCushing,turkopticon}.

On--demand workers were not the first to be exploited along the dimension of low pay rates.
Frustration over low (and declining) pay was one of the chief grievances among then nascent
British labor unions in the early \nth{20} century
\cite{turner1952trade}.
This, \citeauthor{ebbinghaus1999institutions} argued,
fueled the rocketing union membership rates through the mid--\nth{20} century until 1980
(to which we'll return when we discuss \citeauthor{levi2009union}'s reexamination of labor unions)
\cite{ebbinghaus1999institutions,levi2009union}.
This realization has similarly fueled a body of research into
the various incentive structures available to piecework employers
\cite{roy1953work}.


The parallels between the complaints of low pay among crowd workers and other on--demand workers
and the pieceworkers and later factory workers in the \nth{20} century
are inescapable.
We argue further that the \textit{causes} here
--- work decomposition,
work abstraction, and
flexibility ---
lead inexorably to low and declining pay for workers.
Moreover, we point out that low pay leads to other negative outcomes both
in on--demand work
as well as
in piecework and on assembly lines.

\subsubsection{Variable quality work}\label{sec:varQualWork}
Researchers have struggled with what we might generously call work of ``variable quality''
along two dimensions.
The first, to use the characterization of one of these contributions, we can call
``understanding malicious behavior''
\cite{MaliciousCrowdworkersGadiraju}.
While some work has cast workers as ``malicious'' or at least adversarial parties,
the evidence thus far suggests that
workers behave in unexpected ways as they attempt to assert some control over their interaction with the system
(a topic of discussion to which we'll return later)
\cite{uberAlgorithm}.
The second dimension of research in this space generally attempts
to eke out the highest quality work possible from workers
given the apparent difficulty in predicting work outcomes
\cite{embracingErrorKrishna}.




% Low pay yields variable quality work for a number of reasons,
% but before we discuss the causes of this effect, we should discuss the

% Variance in the quali

The effect low wages have had on piece work and factory workers is well--known;
\citeauthor{gantt1913work} discuss this exact mechanism in his book on
% \citetitle{gantt1913work}, pointing out that
``\dots~where there is no union,
the class wage is practically gauged by the wages the poor workman will accept,
and the good workman soon becomes discouraged and \textit{sets his pace by that of his less efficient neighbor},
with the result that the general tone of the shop is lowered'' (emphasis added)
\cite{gantt1913work}.

This research is similar to, but subtly different from, the notion of the ``market for `lemons'''
which \citeauthor{fort2011amazon} discuss;
specifically, \citeauthor{akerlof1970market}'s writing of a ``market for `lemons'''
describes a marketplace where the quality of the product or service is unknown to the buyer
\cite{fort2011amazon,akerlof1970market}.
The effect of this \textit{perceived} uncertainty is that
the \textit{actual} trustworthiness drops precipitously
as all of the consistent, reliable, high--quality workers capable of leaving these markets do so,
leaving only the ones who cannot or will not establish their trustworthiness.

\subsection{Relationships Between Workers and Managers}\label{sec:relationships}
Suffice it to say that poor pay and poor work are linked,
and that we should not be surprised to find this relationship play out online as strongly as it does offline.
But the poor treatment of workers by managers
--- both human and algorithmic ---
do more than affect the economic relationships between workers and employers.
Here, then, we turn to examine this facet of on--demand work
and how these dynamics strikingly replicate the relationships
researchers in labor advocacy encountered in the study of piecework and factory work.

This topic can be condensed into two major areas:
\begin{inlinelist}
\item external (scientific) management, and the evaluation of workers as functional modules; and
% \item the alienation of workers under this form of evaluation; and
\item the consequential resistance workers express due to their perceived alienation and distance from managing forces.
\end{inlinelist}

\subsubsection{External Management}
We discussed Fordism and Taylorism earlier in our discussions of
\nameref{sec:decomposition} and \nameref{sec:flexibility},
but here the core of these paradigmatic views
--- the scientific management of work ---
becomes relevant.
We use ``external'' here instead of ``scientific'', however,
to more broadly capture the disconnect between managers and workers.
By describing it as thus,
we can touch on the relationship that workers have with \textit{researchers}, as well,
even though that work is not strictly
--- or just not exclusively ---
of the same nature as the management and experience as when interacting with requesters.

First, intuitively, the variable--quality work we discussed previously has led to
a large and growing body of research attempting to evaluate workers' performance and error rates
across numerous dimensions;
for example, \citeauthor{measuringCrowdsourcingCheng} explore the error rates of workers by
operating on a sliding scale giving workers varying amounts of time to accomplish micro--tasks
\cite{measuringCrowdsourcingCheng}.
\citeauthor{storiesIraniSilberman} describe the treatment of workers
as sorts of ``human APIs'' that can, importantly, be rigorously evaluated
\cite{storiesIraniSilberman}.
\citeauthor{gevins2003neurophysiological} began to explore the neurophysiological effects of
cognitively demanding tasks on workers,
informing crowdsourcing research by suggesting the use of cognitive load assessments such as
NASA Task Load Index surveys to evaluate workers pre and post--tasks
\cite{embracingErrorKrishna,measuringCrowdsourcingCheng}.


External management comes in other forms than scientific, as previously mentioned.
Researchers in particular have noticed that their relationships with on--demand workers are,
at the least, complex.
\citeauthor{storiesIraniSilberman} point out that their relationships with Turkers are highly complex;
specifically, their interactions with field sites in which they work
as designers and mediators of change influence the relationships they have with Turkers
\cite{storiesIraniSilberman}.

The scientific management of pieceworkers has been well--studied under the umbrella of assembly line research,
and even physiological study of pieceworkers closely resembles the
research into cognitive loads and stress levels that we discussed among on--demand crowd workers
\cite{hu1961parallel,pieceworkBiologicalHarm}.
Even the complicated relationships between observers and workers themselves are not necessarily new;
\citeauthor{riisOtherSideLives}'s photodocumentary of pieceworkers has even been re--examined through an exercise asking crowd workers
to photograph themselves for similar purposes as \citeauthor{riisOtherSideLives}'s --- to document and humanize an otherwise abstracted, invisible workforce
\cite{facesOfMechanicalTurk,turkopticon,riisOtherSideLives}.

Similarly, \citeauthor{pollard1963factory}'s words on the punishment factory workers faced ---
for example, that ``unsatisfactory work was punished \dots~by fines or by dismissal'' ---
seems especially relevant given the fears we now know to be ubiquitous on platforms such as AMT, Uber, and other on--demand markets
\cite{pollard1963factory,uberAlgorithm,dynamo,turkopticon,takingAHITMcInnis}.

\subsubsection{Resistance}
It shouldn't surprise us, then, that workers have resisted the management imposed on them both by other people and their systems,
often without recourse or opportunity for feedback, let alone substantive input.
Indeed, \citeauthor{uberAlgorithm} discover of Uber drivers that 
many toggle their availability to avoid being dispatched to more distant locations,
resisting the intent of the designers of the systems and their ``algorithmic and data--driven management''
\cite{uberAlgorithm}.

Resistance has sometimes been more coordinated, as well;
we see this in \citeauthor{turkopticon}'s coverage on \textit{Turkopticon} as workers collectively accumulated information about requesters, and
in \citeauthor{dynamo}'s work on \textit{Dynamo}, which generated ``Guidelines for Academic Requesters'' written by crowd workers
\cite{turkopticon,dynamo}.

Resistance against managers in piecework and factory labor settings are deeply well--explored,
but perhaps the most relevant case study to draw on here is to be found in
\citeauthor{waldinger1996helots}'s case study of ``Justice for Janitors'',
where marginalized workers managed to raise awareness for their plight and secure support for badly needed reforms
\cite{waldinger1996helots}.
The achievements of labor advocacy groups such as labor unions as resistant,
even adversarial organizations counter--balancing the management
is somewhat well--understood
\cite{russell1982collective,craig1992behavior}.
We argue that these threads of resistance against management in various forms
are in fact one.
}


\onlyinsubfile{
  \balance{}
  \printbibliography
}

\end{document}