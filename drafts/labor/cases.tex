\documentclass[trackingWork]{subfiles}
\makeatletter
\def\blx@maxline{77}
\makeatother
\begin{document}
\section{Research Questions}

{Research in crowdsourcing has spent the better part of a decade
exploring how to grow the limits of crowdsourcing,
finding the boundaries of crowd work and microtasks.}
This has largely involved iteratively
  identifying challenges to increasing complexity and
  overcoming them through novel designs of work--flows and processes~\cite[e.g.][]{bernsteinSoylent,foundry,crowdForgeKittur}.
The question that has emerged then
has been \textit{whether} there are limits to crowdsourcing
--- and if so, what factors determine them.
Through this lens,
a number of contributions to the field have pushed the boundaries of crowd work.

The exploration of crowdsourcing's potential and limits has principally looked at
manipulating and extending along three dimensions:
\begin{Numberlist}[itemjoin*={.~And~},itemjoin={.~}]
  \item \namerefl{sec:complexity}
  \item \namerefl{sec:decomposition}
  \item \namerefl{sec:relationships}.
\end{Numberlist}
We'll explore these aspects of crowdsourcing,  discussing the extents to which work can be
decomposed,  contextually abstracted, and
made more resilient to attrition of various forms.
We'll also point to corresponding piecework literature addressing these aspects.
\msb{Our goal will be to use that literature to inform answers to the questions about crowd work (or something like that)}
Finally, we'll discuss how these elements will serve
to constrain the upper and lower bounds of crowdsourcing as it relates
to the question of the furthest limits of crowdsourcing.



\subfile{complexity.tex}
\subfile{decomposition.tex}
\subfile{relationships.tex}


\end{document}