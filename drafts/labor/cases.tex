\documentclass[trackingWork]{subfiles}
\makeatletter
\def\blx@maxline{77}
\makeatother
\begin{document}
\section{Research Questions}

\topic{Research in crowdsourcing has spent the better part of a decade
exploring how to grow the limits of crowdsourcing and
find the boundaries of crowd work and microtasks.}
This has largely involved
  identifying challenges to this form of labor,    overcoming them through novel designs of work--flows and processes, and
  repeating the process~\cite[e.g.][]{bernsteinSoylent,foundry,crowdForgeKittur}.
The question that has emerged
among these researchers and
through the work that they have produced then
has been driving at \textit{whether} there are limits to crowdsourcing
(and, if so, what factors determine those limits).
Through this lens, we can point to
a number of contributions to the field that have extended the boundaries of crowd work.

The exploration of crowdsourcing's potential and limits has principally looked at
manipulating and extending along three dimensions:
\begin{inlinelist}
  \item \nameref{sec:complexity},
  \item \nameref{sec:decomposition}, and
  \item \nameref{sec:relationships}.
\end{inlinelist}
We'll explore these aspects of crowdsourcing,  discussing the extents to which work can be
decomposed,  contextually abstracted, and
made more resilient to attrition of various forms.
We'll also point to corresponding piecework literature addressing these aspects.
\msb{Our goal will be to use that literature to inform answers to the questions about crowd work (or something like that)}
Finally, we'll discuss how these elements will serve
to constrain the upper and lower bounds of crowdsourcing as it relates
to the question of the furthest limits of crowdsourcing.



\subfile{complexity.tex}
\subfile{decomposition.tex}
\subfile{relationships.tex}


\end{document}