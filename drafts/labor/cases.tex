\documentclass[trackingWork]{subfiles}
\makeatletter
\def\blx@maxline{77}
\makeatother
\begin{document}
\section{Major Research Questions}
% \al¬i{4 paragraphs on what happened in the above contexts; 2 paragraphs on best/worst outcomes \& why; ??? paragraphs on predictions for crowdsourcing}

%  _ _           _ _                __ 
% | (_)_ __ ___ (_) |_ ___    ___  / _|
% | | | '_ ` _ \| | __/ __|  / _ \| |_ 
% | | | | | | | | | |_\__ \ | (_) |  _|
% |_|_|_| |_| |_|_|\__|___/  \___/|_|  
                                     
%                            _                          _             
%   ___ _ __ _____      ____| |___  ___  _   _ _ __ ___(_)_ __   __ _ 
%  / __| '__/ _ \ \ /\ / / _` / __|/ _ \| | | | '__/ __| | '_ \ / _` |
% | (__| | | (_) \ V  V / (_| \__ \ (_) | |_| | | | (__| | | | | (_| |
%  \___|_|  \___/ \_/\_/ \__,_|___/\___/ \__,_|_|  \___|_|_| |_|\__, |
%                                                               |___/ 


% let's do topic sentences!
\subsection{What are the limits of crowdsourcing?}
Research in crowdsourcing has spent the better part of a decade
exploring how to grow the limits of crowdsourcing and
find the boundaries of crowd work and microtasks;
this has largely involved
  identifying challenges to this form of labor,
  overcoming them through novel designs of work--flows and processes, and
  repeating the process
\cite{bernsteinSoylent}.
The question that has emerged
among these researchers and
through the work that they have produced then
has been driving at \textit{whether} there are limits to crowdsourcing
(and, if so, what factors determine those limits).
Through this lens, we can point to
a number of contributions to the field that have extended the boundaries of crowd work.

The exploration of crowdsourcing's potential and limits has principally looked at
manipulating and extending along three dimensions:
\begin{inlinelist}
  % \item \cref*{sec:Complexity},
  \item \nameref{sec:Complexity},
  \item \nameref{sec:Slicing}, and
  \item \nameref{sec:flexibility}
\end{inlinelist}.
We'll explore these aspects of crowdsourcing,
discussing the extents to which work can be
decomposed,
contextually abstracted, and
made more resilient to attrition of various forms.
We'll also point to corresponding piecework literature addressing these aspects.
Finally, we'll discuss how these elements will serve
to constrain the upper and lower bounds of crowdsourcing as it relates
to the question of the furthest limits of crowdsourcing.


%      _                                          _ _   _             
%   __| | ___  ___ ___  _ __ ___  _ __   ___  ___(_) |_(_) ___  _ __  
%  / _` |/ _ \/ __/ _ \| '_ ` _ \| '_ \ / _ \/ __| | __| |/ _ \| '_ \ 
% | (_| |  __/ (_| (_) | | | | | | |_) | (_) \__ \ | |_| | (_) | | | |
%  \__,_|\___|\___\___/|_| |_| |_| .__/ \___/|___/_|\__|_|\___/|_| |_|
%                                |_|                                  
%
% stronger topic sentences
% boosterist (???)


\subsubsection{Achieving greater complexity}\label{sec:Complexity}
\topic{Crowdsourcing research has spent the better part of a decade attempting
to prove the viability of crowdsourcing in increasingly complex work.}  % this is my topic sentence.
\citeauthor{crowdForgeKittur}
map the discussion toward this goal in their work on
crowdsourcing complex work
\cite{crowdForgeKittur}.
\citeauthor{Lasecki:2013:CCC:2501988.2502057} tackle an area of computationally assisted interaction
--- conversational artificial intelligence ---
and find ways to arrange workers and formulate a ``crowd--powered conversational assistant''
\cite{Lasecki:2013:CCC:2501988.2502057}.
Other work has followed a similar pattern,
identifying target milestones in Computer Science
that have presented significant challenges for researchers,
and demonstrating that ``crowd--powered'' systems can tackle many of these challenges
with generally greater success than purely computational attempts
\cite[][and others]{Lasecki:2013:RCL:2441776.2441912,foundry}.

% NEEDS TO GO ELSEWHERE
% \citeauthor{cheng2015break} found that microtasks
% --- though not necessarily \textit{faster} than ``macrotasks'' ---
% yield higher quality work,
% particularly when that work is susceptible to frequent interruptions
% \cite{cheng2015break}.


\topic{While much of the above work has focused on
clever arrangements of systems to manage collectives of workers,
another body of work has turned inward toward managing the self.}
\citeauthor{selfsourcingTeevan2014} push the boundaries of decomposed work,
exploring ``selfsourcing'', and further this work with \citeauthor{selfsourcingTeevan2016}
\cite{selfsourcingTeevan2014,selfsourcingTeevan2016}.
While some of this work doesn't strictly fall under ``crowdsourcing'',
the [scientific] management of the self as a worker
(of sorts)
will prove relevant as we trace the literature surrounding piecework.

\topic{A smaller but growing body of work has reflected on these bodies of work
and brought to our attention a number of ethical questions surrounding the
increasing complexity of crowdwork and the hazards that increasingly arise.}
\citeauthor{professionalCrowdworkEthics} bring some of these issues at stake
--- working for increasing amounts of time on tasks of growing complexity, only to discover that requesters are not willing to pay,
for instance ---
but these and other dangers range an enormous landscape % too informal?
\cite{crowdworkFuture,professionalCrowdworkEthics,nickerson2013crowd,dynamo}.

\topic{We summarize the work discussed so far in this way:
% crowdsourcing researchers have attempted to find the characteristics of crowdwork
researchers of crowdsourcing and on--demand labor in general
have attempted to find the characteristics which
enable and stymie successful applications of this form of work,
and how to get past or around these boundaries.}
Whether this work has focused on the self (as in the case of ``selfsourcing'')
or on others (as in \textit{CrowdForge} and \textit{Foundry}),
crowdsourcing literature has generally studied the constituent work
and attempted either to work around or indeed with the limitations of crowdwork to accomplish a difficult task
\cite{selfsourcingTeevan2014,selfsourcingTeevan2016,crowdForgeKittur,foundry}.
% Following this thread of work, it may be said that
% much of the work in crowdsourcing has found ways of ``vertically slicing'' work
% such that each person is responsible for as small a task as can be made reasonable
% --- taking a minute of audio, for instance, and transcribing that.
% The findings we take away from the crowdsourcing literature might be summed up as
% a number of ways
% to break tasks down \cite[see][]{selfsourcingTeevan2014,foundry,crowdForgeKittur} and
% measuring the costs as they relate to work completion \&
% to a lesser extent to workers
% \cite[see][]{measuringCrowdsourcingCheng,cheng2015break,crowdworkFuture,professionalCrowdworkEthics}.


\topic{Research into piecework has explored similar limits,
with particularly striking parallels in
the language used to describe factors necessary for piecework to thrive.}
\citeauthor{10.2307/23702539} describe piecework in \citeyear{10.2307/23702539} as
``\dots~based on examination of various shop jobs,
which included calculation of the standard time and compensation for each task.
Most piecework plans required additional clerical and supervisory people
(such as piecework clerks, inspectors, and ``experts'')\dots'',
allowing for what the \textit{American Machinist} would later call ``paper efficiency'',
describing savings on paper that are in fact passed on to workers
\cite{10.2307/23702539}.
In this context,
the introspection of tasks by experts and the careful work assignment \& management of workers
we see in historical piecework seem to parallel crowdsourcing.

\citeauthor{Brown01041990} draws more decisive boundaries on piece rate approaches to remuneration,
identifying in analysis of data from
the U.S. Bureau of Labor and Statistics that
``larger establishment are less likely to use piece rates and predominantly female establishments more likely to do so; unionized establishments make significantly greater use of standard--rate pay than do nonunion establishments; \textit{incentive pay is less likely in jobs with a variety of duties than in jobs with a narrow set of routinized duties}; and the ease of monitoring work quality is correlated with the use of incentive pay''
\cite{Brown01041990}.


What's different today? We can turn to \citeauthor{10.2307/23702539}'s reflection on railroad piecework
for a hint of what might have enabled the future of crowdwork in which we now live:
``\dots~only the largest and most wealthy railroads stood to gain substantially from scientific management,
because only they had the resources to eliminate union resistance and
\textit{pay the overhead involved in installing work reorganization}''
(emphasis added)
\cite{10.2307/23702539}.

We argue that crowdsourcing is more widespread today,
and has the potential to become even larger in scale,
because the underlying architecture of crowdwork is digital;
reformulating workflows to decompose tasks and institute piecework compensation programs
is easier today than it was for railroad companies which had to move physical capital to do so.




\ali{\citeauthor{10.2307/23702539} later argues that a significant obstacle to the introduction of piecework in railroad shops was the resistance of workers
(the other being the resistance of management (for different and varying reasons)).
In the cases of online platforms like AMT and digitally mediated platforms like Uber,
where in both cases workers rarely if ever interact face--to--face and opportunities for coordination and collective action are severely limited,
we should expect to see the challenges stymieing collective action that we see in \citeauthor{dynamo} and elsewhere
\cite{10.2307/23702539,dynamo}.
}


%                     _            _         _                  _   _             
% __      _____  _ __| | __   __ _| |__  ___| |_ _ __ __ _  ___| |_(_) ___  _ __  
% \ \ /\ / / _ \| '__| |/ /  / _` | '_ \/ __| __| '__/ _` |/ __| __| |/ _ \| '_ \ 
%  \ V  V / (_) | |  |   <  | (_| | |_) \__ \ |_| | | (_| | (__| |_| | (_) | | | |
%   \_/\_/ \___/|_|  |_|\_\  \__,_|_.__/|___/\__|_|  \__,_|\___|\__|_|\___/|_| |_|
%                                                                                

\subsubsection{Work Abstraction}\label{sec:Slicing}
Decomposition allows requesters to assign parallelized tasks
without undue concern for the broader context of the work.
The earlier example of transcribed audio is admittedly a carefully curated one
--- used to illustrate the ability
to parallelize work across as many workers as there are tasks ---
but even in this case problems arise;
breaking an hour--long interview into
3,600 1--second--long tasks would seem to make it impossible to do the tasks
for lack of \textit{context},
but the other extreme (one hour--long block)
offers no performance improvement over conventionally sourced work.

The question, then, becomes about how to break tasks down as much as possible
(as discussed in the previous section)
without losing the necessary context to do a task.
Stated another way, how can we design work so that as little context as possible is provided,
without leaving out context or information necessary to complete the task?
As \citeauthor{verroios2014context} frame it,
``the crowd must be able to act with
global understanding when each contributor only has access to local views''
\cite{verroios2014context}.

While some research has attempted to increase workers' awareness of the work they do in
apparent efforts to yield higher quality work,
others have turned this constraint of crowdwork into a feature ---
the abstraction from the work itself
appears to allow requesters to engage workers in work with ---
for instance ---
sensitive, confidential or otherwise personally identifying information.
We will discuss these and other avenues of research here.

\textit{Cascade} demonstrated that it's possible to
break certain classes of tasks apart
in such a way that they yield taxonomies of various subjects,
a task generally thought to be sufficiently complex that only expert workers
with top--down awareness of the task
--- in this case, with awareness of all the constituent colors ---
could complete the task
\cite{chilton2013cascade}.
\citeauthor{verroios2014context} further illustrate this potential by
forming a task one might consider highly contextually dependent
--- summarizing the contents of a movie ---
in such a way that crowd workers could contribute small pieces of work without
needing to know the content of the rest of the project
\cite{verroios2014context}.

%   __ _           _ _     _ _ _ _         
%  / _| | _____  _(_) |__ (_) (_) |_ _   _ 
% | |_| |/ _ \ \/ / | '_ \| | | | __| | | |
% |  _| |  __/>  <| | |_) | | | | |_| |_| |
% |_| |_|\___/_/\_\_|_.__/|_|_|_|\__|\__, |
%                                    |___/ 

% how much volatility and quality can we deal with?
% how to measure, get rid of people, recover from bad workers
%mason's on getting workers to be better
\subsubsection{Volatile quality and availability? How much can you deal with?}\label{sec:flexibility}
A number of researchers have identified
worker attrition,
variability of worker performance,
and uncertainty about good versus bad--faith actors as open questions of crowdwork
\cite{MaliciousCrowdworkersGadiraju,Ipeirotis:2010:QMA:1837885.1837906}.
\ali{We can and should discuss the distinction between presumably ``bad faith'' workers
\& workers who are merely responding in kind to bad requesters
--- and the broader questions surrounding
the roles that requesters as well as workers should play ---
but let it suffice to say that
requesters have been trying to understand and manage what appears to them as inconsistent work.
Their ways of responding to that variance in work quality has largely involved
making the work more flexible and resilient to work
(although some work has gone into investigating the causes, rather than treating the symptoms)}

Earlier we discussed \citeauthor{cheng2015break}'s work
measuring the impact that interruption has on worker performance
\cite{cheng2015break}.
This work illustrates a broader sentiment in
both the study and practice
of crowdwork, that microtasks should be designed resiliently against the variability of workers,
fully exploiting the abstracted nature of each piece of work
\cite{interruptionIqbal,delayAndOrderLasecki,vaish2014low}.
That is to say, micro--tasks should be designed such that a single worker's poor performance,
or a good worker's sudden departure,
does not significantly impact the agenda of the work as a whole.
While \citeauthor{cheng2015break} found costs with breaking tasks into smaller components
in the form of higher cumulative time to complete
(albeit much shorter real time to complete, owing to parallelization),
\citeauthor{delayAndOrderLasecki} found that at least \textit{some} performance can be recouped by stringing 
similar tasks together
\cite[respectively]{cheng2015break,delayAndOrderLasecki}.

\citeauthor{embracingErrorKrishna} take a different approach;
by ``embracing error'' and forming models describing the latency of workers in classifying objects at rapid speeds,
the authors offer orders--of--magnitude improvements
in various binary classification tasks
% on the principle that diverse workers complete these tasks
% in order to accurately inform the model on the variety of delays in response times.
\cite{embracingErrorKrishna}.
And rather than building tasks to \textit{tolerate} worker drop--off and attrition,
some researchers have designed work predicated on the expectation of it instead:
\citeauthor{sensitiveTasks} describe ways of assigning tasks in such a way that
crowd workers would never be given enough information to piece together sensitive information about
any single topic
\cite{sensitiveTasks}.

The work thus far seems to attempt to maximize the quality of work among workers through various means:
\begin{inlinelist}
  \item Identifying ``bad'' workers (fraught with problems as this characterization is) \cite{MaliciousCrowdworkersGadiraju},
  \item Designing tasks with break points to facilitate the on--boarding and off--boarding that happens anyway \cite{cheng2015break}, and
  \item Expecting certain levels of attrition and incorrectness and using that variability to their advantage \cite{embracingErrorKrishna}
\end{inlinelist}.


% \subsubsection{Piecework's role here}
% We address the question of the limits of crowdsourcing in three parts:
% \begin{inlinelist}
% \item decomposition
% \item work abstraction
% and \item flexibility
% \end{inlinelist}.
% The literature on piecework has offered insights into each of these topics, which we will explore now.


Flexibility has been explored through the lens of Fordism, perhaps best illustrated by
\citeauthor{tolliday1986between}'s treatment describing
turnover rates rising above 300\% in the decade leading to the introduction of the assembly line in 1913.
Specifically, the utilization of ``\dots~`semi--special' machine tools which could be adapted
[and]~\dots~added flexibility through seasonal layoffs for production workers and the use of
piece rates~\dots~rather than a day wage system''
\cite{tolliday1986between}.

In the field of piecework,
the research covering this topic has both explored
a breadth of tasks that might be rendered doable by piecemeal workers
\textit{as well as} longitudinally documented the success of these approaches.
Here, we 













Here, \citeauthor{hu1961parallel}'s work,
saying of assembly line work that
``it is assumed that men are of equal ability and every man can do any of the $n$ jobs'',
parallels the approach that dominated early research into crowd work
--- namely, using non--expert crowds for complex work
\cite{hu1961parallel}.
This mindset in \citeauthor{hu1961parallel}'s analysis,
and indeed the study of factory and mass manufacturing labor through the \nth{20} century,
substantively owes its existence to scientific management
and the rigorous decomposition of work into tasks, discussed earlier,
and persists to this day as it colors
researchers' goals and objectives in the study and design of crowd work.

Piecework's influence on the abstraction of work into tasks,
described above, is more than just caused by the decomposition of work;
work abstraction itself makes it possible for workers to come and go flexibly,
prompting work requesters to consider ways to design these now discrete tasks in ways that
maximize flexibility, both by allowing (and even anticipating) some inconsistency in worker availability
\textit{and} allowing and anticipating some inconsistency in the quality of the work output itself.
It's to this area that we now turn our attention.





















%                        _ _      _ 
%  _ __   __ _ _ __ __ _| | | ___| |
% | '_ \ / _` | '__/ _` | | |/ _ \ |
% | |_) | (_| | | | (_| | | |  __/ |
% | .__/ \__,_|_|  \__,_|_|_|\___|_|
% |_|                                         

Piecework has seen work along this dimension spanning decades;
\citeauthor{thompson1913time} investigate some of the ways that
construction can benefit from the principles of scientific management.
\citeauthor{thompson1913time}'s thesis asserts that
task work is predicated on the accurate scientific management of work,
including the ``miscellaneous tasks''.
\citeauthor{thompson1913time} argues
--- as early as \citeyear{thompson1913time} ---
that ``\dots one may be challenged to find any class of work
involving labor either indoors or out--of--doors
where tasks cannot be fixed by proper time--study''
\cite{thompson1913time}.

Broken down in this way, work could grow to unprecedented scales,
but the quality of the work would remain relatively variable
\cite{murray1983decentralisation}.
Textile work being a salient example,
it took time for workers to acquire sufficient skill
to do every aspect of the work so that the garment would be accepted by the company soliciting that work
\cite{vezina1992light}.

A compelling solution emerged in the early \nth{20} century to break tasks down into discrete,
manageable routines that could be taught relatively easily,
and whose work output could be evaluated in abstraction from the rest of the work
\cite{restructuringPieceworkBaker}.
In Ford's assembly line, this meant that workers were not responsible for building a whole car,
but a single very narrowly defined action that needed to be done on every car
\cite{towardsGlobalFordism}.
By the mid--\nth{20} century, \citeauthor{schoenberger1988fordism} writes,
``\dots~the intensification of the labor process is argued to have hit mental, physical, and social limits.''
\cite{schoenberger1988fordism}.

This approach, ``Fordism'' (and its better--known contemporary ``Taylorism'' of similar ethos),
can be seen today in crowd work and on--demand labor through the application of micro--tasks.
\citeauthor{writingMicroTasks} highlight some of the advantages of breaking work into pieces,
facilitating evaluation and parallelization
\cite{writingMicroTasks}.
By decomposing and recomposing tasks,
and in particular by assigning similarly natured work to the same workers,
workers could become ``experts'' in a small aspect of the work that they did,
speeding their work dramatically
\cite{delayAndOrderLasecki}.
Perhaps more important, however, was that
the breaking down of work into tasks has made it more practical to evaluate work at each stage
\cite{rogstadius2011assessment}.

\subsubsection{So how does this affect crowdwork?}
The work we've seen so far 
\begin{itemize}
  \item worst case: assembling iPhones (extant)
  \item average case: railroad workers and assembly lines
  \item high (complexity) case: 
\end{itemize}

% so CW?
% mechanism/limitation in pw?
% differences in cw? tech/access to intelligence?
% predicted outcome then? -> decompose effort/expertise access
% possible quirks/diffs?


% what can be decomposed? what can be evaluated?
% technology/turuing completeness gets us a win
% predefined expertise? you can find someone almost instantly;
% ease of doing this makes it qualitatively different due to Channel Factors (diverting a channel/river)
% huddler paper cites some decomposition stuff workflows are good/bad at things
% 3-way slack w/msb and mav
% academy of management 
% van der ven paper ~1972 limits of workflows cited in daniela's flash orgs paper




























\onlyinsubfile{
  \balance{}
  \printbibliography
  % \nobalance{}
  \clearpage
  \nobalance{}
  \begin{appendices}
  Points to make:
  \begin{itemize}
    \item Blossoming of piecework
    \begin{itemize}
      \item high point: consultants?
      \item most cases: auto workers/etc\dots
      \item worst case: sweatshops (especially in developing nations)
    \end{itemize}
    \item what led to these outcomes?
    \begin{itemize}
      \item ``consultant'' work came out well because the work was complex;
      this made it difficult to turn into a mass market commodity.
      We see consultants ranging skill levels like
      oDesk (implementing modules) to
      Accenture (on--demand teams of consultants).
      \item auto workers,
      working in settings where capital couldn't be moved as easily
      found themselves in the same workspace as a direct
      --- if multi--stepped ---
      result to the benefits of putting people in the same place to consolidate resources.
      Moreover,
      workers had leverage over factory owners as a result of that consolidated capital;
      operating equipment required training.
      \item sweatshop workers fared the worst,
      for reasons that may seem obvious with hindsight.
      Source materials for textiles are easier to ship than mechanical components such as engines,
      making it easier for factories to relocate to developing nations
      (where cost--of--living, and consequently wages, would be lower).
      As wages, CoLAs, and QoL rise, workers begin asking for
      (and later demanding)
      higher wages and better conditions.
      But where the auto workers have leverage,
      textile workers find only a precarious economic balance now tipped by their collective action,
      spurring manufacturers to move to a new locale
    \end{itemize}
  \end{itemize}





\subsection{What forms of work design and worker management are viable?}
\begin{itemize}
  \item researchers have looked at how to increase worker productivity
  (e.g. finding the maximal speed at which gig workers can be expected to work before making errors)
  \cite{measuringCrowdsourcingCheng}.
  \item we've also seen people ``embrace error''
  \cite{embracingErrorKrishna}.
  \item still other research has looked into ways to sandbox workers from the context of their work 
  \item but scholarship looking into the design and management of work and workers isn't new;
  lots of research into getting pieceworkers to do work more quickly
  \cite{seymour1954manual}.
  \item Researchers have even asked the age old question of \textit{what motivates} pieceworkers
  (echoing similar research on Wikipedia and Mechanical Turk)
  \cite{roy1953work,Nov:2007:MW:1297797.1297798,kaufmann2011more}
\end{itemize}


\subsection{What will work and the place of work look like for workers?}
The metaphorical mechanics of these dynamics are still at play;
workers and managers continue to interact in adversarial manners,
despite substantive work into aligning the motivations of workers and requesters






    % \section{Case Studies}
    The existing body of research has shed light on on--demand labor from various perspectives,
    and revealed a number of topics that,
    through our framing, are clearly situated together.
    Those topics are, at a high level, as follows:
    \begin{enumerate}
    \item the \textbf{processes} involved in making work into tasks, or discretization;
    \item the outcomes (and indeed the \textbf{fallout}) of that discretization,
    both on the work itself as well as the workers;
    and finally
    \item the \textbf{relationships} between workers and requesters of the work
    --- both \textit{cooperative} and \textit{adversarial} cases.
    \end{enumerate}

    \subsection{The Fallout of Crowd Work}\label{sec:Fallout}
    % If the research exploring and documenting how we can exploit crowd \textit{work} can be described as wide--ranging,
    % the scholarship discussing the ways crowd \textit{workers} have been exploited is more focused;
    \citeauthor{turkopticon} point out the disillusion that companies such as Amazon foster on platforms for work like AMT
    (see also \citeauthor{dynamo}'s work
    continuing in the spirit of this observation to generate collective action to improve worker conditions)
    \cite{turkopticon,dynamo}.
    \citeauthor{uberAlgorithm}
    find similarly that workers on gig work platforms are frustrated by the systems on which they work,
    to say little of the policies which these systems enforce
    \cite{uberAlgorithm}.

    We discussed the benefits of flexibility
    (both in the sense of having arbitrary workers perform tasks and
    in the sense that we can design tasks to be more resilient to poor work)
    in the previous section.
    It's from that point in the literature that we turn our attention to
    the perhaps unintended effects of crowd work
    and the affordances for transience that we build into this mode of work.
    We'll address two major areas of work under this subject:
    \begin{inlinelist}
    \item \nameref{sec:lowPay}; and
    \item \nameref{sec:varQualWork}.
    \end{inlinelist}

    \subsubsection{Low Pay}\label{sec:lowPay}


    \citeauthor{laborEconomicsOfCrowdsourcingHorton}
    identified problems with crowd work wages relatively early on,
    attempting to address this imbalance from a behavioral economic perspective ---
    that is, identifying and presenting a model that describes a worker's
    ``\textit{reservation wage}''
    \cite{laborEconomicsOfCrowdsourcingHorton}.
    This work has largely informed much of the research into and practice of estimating crowd work compensation
    \cite{incentivesShaw,paolacci2010running}.

    But we turn to \citeauthor{turkopticon}'s discussion of ``\textit{Turkopticon}'',
    a system they designed to interrogate worker invisibility and to promote better wages across several dimensions
    \cite{turkopticon}.
    Of particular relevance here,
    \citeauthor{turkopticon} call to attention that ``Turkers'' are ultimately vulnerable to
    wage theft and
    pay rates that translate to well under minimum wage.
    Returning to \citeauthor{laborEconomicsOfCrowdsourcingHorton},
    we find that the median ``reservation wage'' in \citeyear{laborEconomicsOfCrowdsourcingHorton}
    was \$1.38, while the mean was \$3.63
    \cite{laborEconomicsOfCrowdsourcingHorton}.

    Understanding workers' motivations given these conditions has thus become a goal for some researchers
    \cite{whyWouldAnyoneBrewer}.
    \citeauthor{Sun20111033} conclude that
    ``\dots~solvers participate in online tasks
    not only for money
    but also for enjoyment
    or the sense of self--worth''
    \cite{Sun20111033}.
    This might have rung true in \citeyear{Sun20111033},
    and certainly corroborates \citeauthor{Ross}'s findings after investigating
    ``who are the crowdworkers'',
    but as \citeauthor{whoareNOTtheTurkers} points out
    ``we [have since] learned that most tasks on AMT are done by a small group of professional Turkers\dots''
    \cite{Ross,whoareNOTtheTurkers}.

    Now, \citeauthor{turkopticon}
    and later
    \citeauthor{dynamo} cite insufficient pay as a central point of frustration among workers,
    via \citeauthor{irani2015cultural} and \citeauthor{dawnDigitalSweatshopCushing}'s contributions in this space
    \cite{dynamo,irani2015cultural,dawnDigitalSweatshopCushing,turkopticon}.

    On--demand workers were not the first to be exploited along the dimension of low pay rates.
    Frustration over low (and declining) pay was one of the chief grievances among then nascent
    British labor unions in the early \nth{20} century
    \cite{turner1952trade}.
    This, \citeauthor{ebbinghaus1999institutions} argued,
    fueled the rocketing union membership rates through the mid--\nth{20} century until 1980
    (to which we'll return when we discuss \citeauthor{levi2009union}'s reexamination of labor unions)
    \cite{ebbinghaus1999institutions,levi2009union}.
    This realization has similarly fueled a body of research into
    the various incentive structures available to piecework employers
    \cite{roy1953work}.


    The parallels between the complaints of low pay among crowd workers and other on--demand workers
    and the pieceworkers and later factory workers in the \nth{20} century
    are inescapable.
    We argue further that the \textit{causes} here
    --- work decomposition,
    work abstraction, and
    flexibility ---
    lead inexorably to low and declining pay for workers.
    Moreover, we point out that low pay leads to other negative outcomes both
    in on--demand work
    as well as
    in piecework and on assembly lines.

    \subsubsection{Variable quality work}\label{sec:varQualWork}
    Researchers have struggled with what we might generously call work of ``variable quality''
    along two dimensions.
    The first, to use the characterization of one of these contributions, we can call
    ``understanding malicious behavior''
    \cite{MaliciousCrowdworkersGadiraju}.
    While some work has cast workers as ``malicious'' or at least adversarial parties,
    the evidence thus far suggests that
    workers behave in unexpected ways as they attempt to assert some control over their interaction with the system
    (a topic of discussion to which we'll return later)
    \cite{uberAlgorithm}.
    The second dimension of research in this space generally attempts
    to eke out the highest quality work possible from workers
    given the apparent difficulty in predicting work outcomes
    \cite{embracingErrorKrishna}.




    % Low pay yields variable quality work for a number of reasons,
    % but before we discuss the causes of this effect, we should discuss the

    % Variance in the quali

    The effect low wages have had on piece work and factory workers is well--known;
    \citeauthor{gantt1913work} discuss this exact mechanism in his book on
    % \citetitle{gantt1913work}, pointing out that
    ``\dots~where there is no union,
    the class wage is practically gauged by the wages the poor workman will accept,
    and the good workman soon becomes discouraged and \textit{sets his pace by that of his less efficient neighbor},
    with the result that the general tone of the shop is lowered'' (emphasis added)
    \cite{gantt1913work}.

    This research is similar to, but subtly different from, the notion of the ``market for `lemons'''
    which \citeauthor{fort2011amazon} discuss;
    specifically, \citeauthor{akerlof1970market}'s writing of a ``market for `lemons'''
    describes a marketplace where the quality of the product or service is unknown to the buyer
    \cite{fort2011amazon,akerlof1970market}.
    The effect of this \textit{perceived} uncertainty is that
    the \textit{actual} trustworthiness drops precipitously
    as all of the consistent, reliable, high--quality workers capable of leaving these markets do so,
    leaving only the ones who cannot or will not establish their trustworthiness.

    \subsection{Relationships Between Workers and Managers}\label{sec:relationships}
    Suffice it to say that poor pay and poor work are linked,
    and that we should not be surprised to find this relationship play out online as strongly as it does offline.
    But the poor treatment of workers by managers
    --- both human and algorithmic ---
    do more than affect the economic relationships between workers and employers.
    Here, then, we turn to examine this facet of on--demand work
    and how these dynamics strikingly replicate the relationships
    researchers in labor advocacy encountered in the study of piecework and factory work.

    This topic can be condensed into two major areas:
    \begin{inlinelist}
    \item external (scientific) management, and the evaluation of workers as functional modules; and
    % \item the alienation of workers under this form of evaluation; and
    \item the consequential resistance workers express due to their perceived alienation and distance from managing forces.
    \end{inlinelist}

    \subsubsection{External Management}
    We discussed Fordism and Taylorism earlier in our discussions of
    \nameref{sec:Complexity} and \nameref{sec:flexibility},
    but here the core of these paradigmatic views
    --- the scientific management of work ---
    becomes relevant.
    We use ``external'' here instead of ``scientific'', however,
    to more broadly capture the disconnect between managers and workers.
    By describing it as thus,
    we can touch on the relationship that workers have with \textit{researchers}, as well,
    even though that work is not strictly
    --- or just not exclusively ---
    of the same nature as the management and experience as when interacting with requesters.

    First, intuitively, the variable--quality work we discussed previously has led to
    a large and growing body of research attempting to evaluate workers' performance and error rates
    across numerous dimensions;
    for example, \citeauthor{measuringCrowdsourcingCheng} explore the error rates of workers by
    operating on a sliding scale giving workers varying amounts of time to accomplish micro--tasks
    \cite{measuringCrowdsourcingCheng}.
    \citeauthor{storiesIraniSilberman} describe the treatment of workers
    as sorts of ``human APIs'' that can, importantly, be rigorously evaluated
    \cite{storiesIraniSilberman}.
    \citeauthor{gevins2003neurophysiological} began to explore the neurophysiological effects of
    cognitively demanding tasks on workers,
    informing crowdsourcing research by suggesting the use of cognitive load assessments such as
    NASA Task Load Index surveys to evaluate workers pre and post--tasks
    \cite{embracingErrorKrishna,measuringCrowdsourcingCheng}.


    External management comes in other forms than scientific, as previously mentioned.
    Researchers in particular have noticed that their relationships with on--demand workers are,
    at the least, complex.
    \citeauthor{storiesIraniSilberman} point out that their relationships with Turkers are highly complex;
    specifically, their interactions with field sites in which they work
    as designers and mediators of change influence the relationships they have with Turkers
    \cite{storiesIraniSilberman}.

    The scientific management of pieceworkers has been well--studied under the umbrella of assembly line research,
    and even physiological study of pieceworkers closely resembles the
    research into cognitive loads and stress levels that we discussed among on--demand crowd workers
    \cite{hu1961parallel,pieceworkBiologicalHarm}.
    Even the complicated relationships between observers and workers themselves are not necessarily new;
    \citeauthor{riisOtherSideLives}'s photodocumentary of pieceworkers has even been re--examined through an exercise asking crowd workers
    to photograph themselves for similar purposes as \citeauthor{riisOtherSideLives}'s --- to document and humanize an otherwise abstracted, invisible workforce
    \cite{facesOfMechanicalTurk,turkopticon,riisOtherSideLives}.

    Similarly, \citeauthor{pollard1963factory}'s words on the punishment factory workers faced ---
    for example, that ``unsatisfactory work was punished \dots~by fines or by dismissal'' ---
    seems especially relevant given the fears we now know to be ubiquitous on platforms such as AMT, Uber, and other on--demand markets
    \cite{pollard1963factory,uberAlgorithm,dynamo,turkopticon,takingAHITMcInnis}.

    \subsubsection{Resistance}
    It shouldn't surprise us, then, that workers have resisted the management imposed on them both by other people and their systems,
    often without recourse or opportunity for feedback, let alone substantive input.
    Indeed, \citeauthor{uberAlgorithm} discover of Uber drivers that 
    many toggle their availability to avoid being dispatched to more distant locations,
    resisting the intent of the designers of the systems and their ``algorithmic and data--driven management''
    \cite{uberAlgorithm}.

    Resistance has sometimes been more coordinated, as well;
    we see this in \citeauthor{turkopticon}'s coverage on \textit{Turkopticon} as workers collectively accumulated information about requesters, and
    in \citeauthor{dynamo}'s work on \textit{Dynamo}, which generated ``Guidelines for Academic Requesters'' written by crowd workers
    \cite{turkopticon,dynamo}.

    Resistance against managers in piecework and factory labor settings are deeply well--explored,
    but perhaps the most relevant case study to draw on here is to be found in
    \citeauthor{waldinger1996helots}'s case study of ``Justice for Janitors'',
    where marginalized workers managed to raise awareness for their plight and secure support for badly needed reforms
    \cite{waldinger1996helots}.
    The achievements of labor advocacy groups such as labor unions as resistant,
    even adversarial organizations counter--balancing the management
    is somewhat well--understood
    \cite{russell1982collective,craig1992behavior}.
    We argue that these threads of resistance against management in various forms
    are in fact one.


  \end{appendices}
  }


\end{document}