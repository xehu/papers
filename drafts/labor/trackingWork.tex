\documentclass[chi_draft]{sigchi}

\usepackage{xr-hyper}
% \zxrsetup{%
%   tozreflabel=false, % not needed, since `zref` is not used otherwise
%   toltxlabel=true, % classical LaTeX labels that can be referenced with \ref, \pageref
% }
% \zexternaldocument{cases.tex}% star form: only LaTeX labels are imported

\usepackage{todonotes,txfonts,balance,graphics,color,alphalph,subfiles,appendix,endnotes}
\usepackage{booktabs,textcomp,microtype,ccicons,enumitem,moreenum,nth,endnotes,cleveref,textcase}
\usepackage[normalem]{ulem}
\usepackage[citestyle=numeric,backend=bibtex]{biblatex}
\usepackage[T1]{fontenc}
\usepackage[pdflang={en-US},pdftex]{hyperref}
\usepackage{mfirstuc,lastpage}
\usepackage[en-GB]{datetime2}
\usepackage{fancyhdr}
\usepackage[all]{hypcap}  % Fixes bug in hyperref caption linking
\usepackage[utf8]{inputenc} % for a UTF8 editor only

% \bibliographystyle{SIGCHI-Reference-Format}
\renewcommand*{\notesname}{Graveyard of old paragraphs}
\makeatletter
\def\blx@maxline{77}
\makeatother


\providecommand*{\namerefl}[1]{%
\refused{#1}%
\MakeTextLowercase{%
\getrefbykeydefault{#1}{name}{??}%
}%
}


% Paper metadata (use plain text, for PDF inclusion and later
% re-using, if desired).  Use \emtpyauthor when submitting for review
% so you remain anonymous.

\newlist{inlinelist}{enumerate*}{1}
\setlist*[inlinelist,1]{%
   % label=\alph*),
 label=\arabic*),
}

\setlist{nolistsep}
\def\plaintitle{Examining Crowd Work Through The Historical Lens of Piecework}
\def\plainauthor{Ali Alkhatib, Margaret Levi, Michael Scott Bernstein}
\def\emptyauthor{}
\def\plainkeywords{Crowdsourcing; on--demand labor; gig work}
\def\plaingeneralterms{Documentation, Standardization}

% llt: Define a global style for URLs, rather that the default one
\makeatletter
\def\url@leostyle{%
  \@ifundefined{selectfont}{
    \def\UrlFont{\sf}
  }{
    \def\UrlFont{\small\bf\ttfamily}
  }}
\makeatother
\urlstyle{leo}

% To make various LaTeX processors do the right thing with page size.
\def\pprw{8.5in}
\def\pprh{11in}
\special{papersize=\pprw,\pprh}
\setlength{\paperwidth}{\pprw}
\setlength{\paperheight}{\pprh}
\setlength{\pdfpagewidth}{\pprw}
\setlength{\pdfpageheight}{\pprh}
\definecolor{PineGreen}{HTML}{008800}
\definecolor{BrickRed}{HTML}{FF0000}
\definecolor{JokeGreen}{HTML}{00C953}
\newcommand{\msb}[1]{{\color{PineGreen}[MSB: #1]}}
\newcommand{\ali}[1]{{\color{BrickRed}[al2: #1]}}
\newcommand{\TO}{\textit{Turkopticon}}
\newcommand{\DO}{\textit{Dynamo}}


% Make sure hyperref comes last of your loaded packages, to give it a
% fighting chance of not being over-written, since its job is to
% redefine many LaTeX commands.
\definecolor{linkColor}{RGB}{6,125,233}
\hypersetup{%
  pdftitle={\plaintitle},
% Use \plainauthor for final version.
%  pdfauthor={\plainauthor},
  pdfauthor={\emptyauthor},
  pdfkeywords={\plainkeywords},
  bookmarksnumbered,
  pdfstartview={FitH},
  colorlinks,
  citecolor=black,
  filecolor=black,
  linkcolor=black,
  urlcolor=linkColor,
  breaklinks=true,
  hypertexnames=false
}

% create a shortcut to typeset table headings
% \newcommand\tabhead[1]{\small\textbf{#1}}
\bibliography{references}
% \addbibresource{references}

\newcommand{\subsubsubsection}[1]{\textit{#1}}
\newcommand{\pieceworkpers}{Piecework's perspective}
\newcommand{\crowdworkpers}{Crowd work's perspective}
\newcommand{\whatchanged}{Comparing the phenomena}
\newcommand{\implication}{Implications for crowd work}
\newcommand{\pieceworkdefinition}{piecework is the paradigm of renumeration that is largely made compelling for its use of
``payment for results'',
leading to and leveraging narrower skill sets required for narrowly defined tasks,
and affording workers some amount of freedom to complete tasks
(and consequently earn money)
at at whatever rate and in whichever manner they wish} % do i use this more than once? even that much?

\newcommand{\onlyinsubfile}[1]{#1}
\newcommand{\notinsubfile}[1]{}
\newcommand{\note}[1]{\footnote{#1}}
\newcommand{\topic}[1]{\textbf{#1}} % if you're in a subfile, bold the \topic sentences to make it easier to skim/diagnose logical issues
\newcommand{\joke}[1]{{\color{JokeGreen}#1}} % if you're in a subfile, make the joke JokeGreen colored so people know I'm being funny!

\let\internalNoteali\ali
\let\internalNotemsb\msb
\onlyinsubfile{\renewcommand{\ali}[1]{\onlyinsubfile{\internalNoteali{#1}}}}
\onlyinsubfile{\renewcommand{\msb}[1]{\onlyinsubfile{\internalNotemsb{#1}}}}

% \DTMsetdatestyle{mmddyyyy}
% \DTMsetup{datesep=/}
% \pagestyle{fancy}
% \lfoot{Last compiled \today~at \DTMcurrenttime}
% \cfoot{}
% \rfoot{page \thepage/\pageref*{LastPage}}
% \fancyhead{}
% \pagenumbering{arabic}
\renewcommand{\headrulewidth}{0pt}

% \onlyinsubfile{
%   \usepackage{xr-hyper}
%   \usepackage{hyperref}
% }


\renewcommand{\topic}[1]{#1} % don't do anything with \topic sentences; this is just for drafting 
\renewcommand{\footnotesize}{\normalsize} % i don't need small text! (no but really remove this before submitting)
\begin{document}
\renewcommand{\topic}[1]{#1} % don't do anything with \topic sentences; this is just for drafting 
\renewcommand{\joke}[1]{} % don't do anything with \jokes! Just throw them away! (Better to drop it than accidentally include it)
\renewcommand{\onlyinsubfile}[1]{}
\renewcommand{\notinsubfile}[1]{#1}
\renewcommand{\note}[1]{\onlyinsubfile{\note{#1}}}



\title{\plaintitle}

\numberofauthors{3}
\author{%
  \alignauthor{Ali Alkhatib\\
    \affaddr{Stanford University}\\
    \affaddr{Stanford, USA}\\
    \email{ali.alkhatib@cs.stanford.edu}}\\
  \alignauthor{Michael Scott Bernstein\\
    \affaddr{Stanford University}\\
    \affaddr{Stanford, USA}\\
    \email{msb@cs.stanford.edu}}\\
  \alignauthor{Margaret Levi\\
    \affaddr{CASBS}\\
    \affaddr{Stanford, USA}\\
    \email{mlevi@stanford.edu}}\\
}
\author{
\alignauthor{Anonymous for submission\\
\affaddr{Affiliation fields filled}\\
    \affaddr{to take representative space}\\
    \email{of final document}}\\
}
\maketitle
\subfile{abstract.tex}

\category{H.5.3.}{Information Interfaces and Presentation
  (e.g. HCI)}{Group and Organization Interfaces}

\keywords{\plainkeywords}
\subfile{introduction.tex}
\subfile{piecework_lit.tex}
\subfile{cases.tex}
\subfile{discussion.tex}
\subfile{conclusion.tex}

\balance{}
\printbibliography

\clearpage
\nobalance{}
\endnotetext{
  % \section{The Bleak Future of Crowd Work}
  % We've traced a path from piecework itself through
  % the processes that describe the design and implementation of piece work and crowd work as part of the same thread;
  % in tracing this process, we touched on the relationships between decomposition,
  % work \& worker abstraction,
  % flexibility, and followed through both the general fallout of crowd work in the research community
  % as well as the fallout between workers and the managers and other external parties
  % --- including researchers.

  % Throughout these case studies, we have pointed out the parallels between
  % the contemporary research in on--demand labor and
  % the much larger body of research constituting our understanding of topics such as
  % piecework, factory work, and laborer relations.
  % If we agree that this framing is useful and informative, then
  % several topics emerge as relatively open questions in the study of crowd work and on--demand labor.
  % Two of the most pressing questions are
  % \begin{inlinelist}
  % \item the beginnings of factories, and 
  % \item the decline of relevance of worker advocacy organizations.
  % \end{inlinelist}
  % We will discuss those questions here.



  % \subsection{The beginnings of factories}\label{sec:Factorization}
  % We established earlier that abstracted work and low wages tend to result in variable outcomes,
  % which presents problems for employers.
  % % So work output is variable, and that's bad. What are we doing about it?
  % Historically, this is what led to factories; by employing a cohort of known workers,
  % we can be reasonably assured that the quality of the work will be better than random.
  % Furthermore, we can invest more resources in training workers and
  % get workers to do more complex work with more context.
  % % you would pay some experts who understood what they needed to do
  % % and they would form a cohort into which you invest more.

  

  % This, then, suggests that the beginning of the regularization of workforces
  % --- a sort of coalescence of factories ---
  % is already happening.
  % If our framing of on--demand labor is accurately describing an underlying relationship with piecework,
  % then we should watch for the emergence and popularization of persistent teams of workers.
  % While this field site will likely prove exceedingly difficult to 
  % Social computing researchers should seek to understand this process more
  % fully, as we a

  % \subsection{The decline of advocacy organizations}
  

  % \section{Implications for Design}
  % If it's agreed that
  % the major topics we've discussed thus far are related and
  % --- at least to \textit{some} extent ---
  % precipitated in the fashion we argue,
  % then we have a rare opportunity as researchers,
  % and as agents of change in the communities we study,
  % to affect change on the dynamics of crowd and on--demand work
  % as they continue to develop.

  % Without claiming to have easy, cut--and--dry solutions to these problems,
  % we can nevertheless bring to attention a number of critical opportunities to
  % learn from historical parallels in piecework and factory labor,
  % and make informed decisions regarding whether
  % (or indeed how)
  % we may want to influence outcomes.
  % The challenges we bring to attention here are as follows:
  % \begin{inlinelist}
  %   \item codifying investment toward collective goods into the designs of systems;
  %   \item (re--)decentralizing the internet; and
  %   \item enabling reputation transferral.
  % \end{inlinelist}


  % \subsection{Codify the common good}
  % As \citeauthor{lessig2006code} points out in his book,
  % digital media give designers the opportunity to design and build into the systems
  % policies and practices to contribute to the collective benefit of the people therein
  % \cite{lessig2006code}.
  % Historically, the confluence of forces \citeauthor{lessig2006code} describes
  % would ultimately result in outcomes such as benefits for workers,
  % funds for sick leave and vacation, and other conveniences.
  % The transient nature of on--demand work would seem to problematize this arrangement,
  % but we can discuss and explore the viability of building into systems the mechanisms necessary
  % to save a portion of payment from every gig,
  % record taxable income, or
  % myriad other generally administrative tasks automatically.


  % \subsection{Decentralize the internet --- again}
  % Digitally mediated on--demand labor markets have historically been insular and incompatible with one another,
  % forcing workers either to choose one or juggle participation in these markets with great difficulty.
  % An ``API'' for on--demand labor markets could make it possible for any person
  % or organization to instantiate their own marketplace and inter--operate with.
  % This can be changed, and indeed must, if we are to realize the hopes of early researchers who advocated the
  % democratizing nature and power of the internet
  % \cite{barlow2009declaration,lanier2014owns}.

  % \subsection{Deal with reputation}
  % Reputation systems in on--demand labor markets are fundamentally broken.
  % To say nothing of the fact that information workers (such as those on AMT) can't transfer their reputations to qualitatively different forms of labor
  % like driving--for--hire (e.g. Uber), even within the same industry
  % it's currently not feasible for workers to transfer their reputations or other information from one place to another.
  % This affects more than the reputation and trustworthiness of workers;
  % accounting for things such as taxes, benefits, etc\dots is all but left to the individual workers, who struggle with myriad bureaucratic obstacles.
  % We can design systems that facilitate the aggregation and, more importantly, the transferral of reputation, income, and other features of work.



  % \section{Discussion}
  % We've discussed a number of aspects of on--demand work that
  % offer parallels with historical piecework.
  % Perhaps more importantly, we've hopefully demonstrated that the dynamics we observe in on--demand work
  % are interrelated and follow from one another just as necessarily as they did in
  % the development and maturation of piecework and factory work through the \nth{20} century.
  % This framing on on--demand work should, we hope, provide us with the necessary historical context
  % to make better--informed design decisions about how we want ``the future of crowd work'' to look.

  % \section{Conclusion}
  % \citeauthor{crowdworkFuture} discussed many of the challenges and problems in crowd work in \citeyear{crowdworkFuture},
  % but didn't necessarily situate the notion of crowd work in a broader context.
  % This paper attempts to fill that gap, and in doing so hopes
  % to give the research community theoretical grounding to work with and within on--demand labor more successfully.
  % But more than that, we hope to have addressed important questions to inform how we actually might make crowd work
  % a career in which we want our children to work.
}
% \theendnotes

\end{document}

%%% Local Variables:
%%% mode: latex
%%% TeX-master: t
%%% End:
