\documentclass[trackingWork]{subfiles}
\makeatletter
\def\blx@maxline{77}
\makeatother
\begin{document}

\subsubsection[slicing work into smaller parts]{Slicing work smaller
\joke{ --- or ``putting the {\tiny `micro'} in microwork''}
}\label{sec:Slicing}

Concurrent with the body of work on the complexity of crowdsourcing
has been a thread of work exploring the decomposition of work.
In this section we'll discuss some of the approaches and the results 
the crowdsourcing community have taken from their work.
Then, as previously,
we'll discuss some of the findings that piecework research
has yielded.
Finally, we'll discuss similarities and differences between crowdwork and piecework,
and how crowdwork's limitations and potential differ from piecework's.

\subsubsubsection{\crowdworkpers}
\topic{The research on crowdwork and decomposition has generally driven toward finding new ways
to strip unnecessary context from work,
allowing tasks to be broken down into smaller parts.}
As \citeauthor{verroios2014context} frame it,
``the crowd must be able to act with
global understanding when each contributor only has access to local views''
\cite{verroios2014context}.
Researchers have thus investigated the role of
distributed, contextually independent work
in the contexts of collaborative writing of various forms,
software development,
classification, and myriad other purposes
\cite{Kittur:2009:CCI:1518701.1518928,Baecker:1993:UID:169059.169312,
      10.2307/1562247,karnin2010crowdsourcing,bragg2013crowdsourcing,
      chilton2013cascade}.


\topic{One of the emergent properties of micro--tasks has been the relative cost of
\textit{finding} worthwhile tasks.}
The research community has documented and to some extent attempted to intervene in
the discovery of worthwhile tasks
\cite{taskSearch}.
\citeauthor{Cosley:2007:SUI:1216295.1216309}
attempts to address this by
directing workers to tasks through
``intelligent task routing''
\cite{Cosley:2007:SUI:1216295.1216309}.
Much of this work and the work at the periphery of this space, then,
has focused on
minimizing the amount of time that people need to spend doing
anything other than the work for which they are paid.


\subsubsubsection{\pieceworkpers}
The beginnings of systematized task decomposition stretch back as far as the \nth{17} century,
when Airy employed young boys at the Greenwich Observatory who
``possessed the basic skills of mathematics, including
`Arithmetic, the use of Logarithms, and Elementary Algebra'~''
to \textit{compute} astronomical phenomena
\cite{grier2013computers}.
Airy's tasks were unique at the time for several reasons:
\begin{enumerate}
  \item each task was quickly verifiable by a qualified [human] computer,
  \item tasks were discrete --- independent from one another, and
  \item knowledge of the full scope of the project --- indeed, knowledge of anything more than the problem set at hand ---
  was wholly unnecessary.
\end{enumerate}


\subsubsubsection{\whatchanged}
\ali{todo}



\onlyinsubfile{
  \balance{}
  \printbibliography
  
  \clearpage
  \nobalance{}
  \begin{appendices}
    \section{Paragraph graveyard}
    The finding here is that
    crowdwork can be more carefully micro--managed than piecework could be,
    and that this is a double--edged sword:
    we can effectively give feedback to workers on everything they do,
    but this is emboldening us to try to
    over--manage workers just as piecework tried to do.

    My goal for this section is to make two points:
    \begin{enumerate}
      \item show how this is related to the assembly line and scientific management,
            and how piecework literature tried to measure everything,
            but found it untenable given the extra equipment that was necessary
            (but generally which didn't exist)
            to track every movement and action that workers took.
      \item show how this work was enabled by the ``verifiability'' of work output(?)
    \end{enumerate}
  \end{appendices}
}

\end{document}