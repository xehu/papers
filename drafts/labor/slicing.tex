\documentclass[trackingWork]{subfiles}
\makeatletter
\def\blx@maxline{77}
\makeatother
\begin{document}

\subsubsection{Slicing work smaller
\joke{ --- or ``putting the {\tiny `micro'} in microwork''}
}\label{sec:Slicing}

Concurrent with the body of work on the complexity of crowdsourcing
has been a thread of work exploring the decomposition of work.
In this section we'll discuss some of the approaches and the results 
the crowdsourcing community have taken from their work.
Then, as previously,
we'll discuss some of the findings that piecework research
has yielded.
Finally, we'll discuss similarities and differences between crowdwork and piecework,
and how crowdwork's limitations and potential differ from piecework's.

\subsubsubsection{Crowdwork's perspective.}
\topic{The research on crowdwork and decomposition has generally driven toward finding new ways
to strip unnecessary context from work,
allowing tasks to be broken down into smaller parts.}
As \citeauthor{verroios2014context} frame it,
``the crowd must be able to act with
global understanding when each contributor only has access to local views''
\cite{verroios2014context}.
% the decomposition of work allows requesters to assign tasks to workers in parallel,
% theoretically scaling to as many workers as there are tasks.
% As the nature of the work has grown in complexity,
% so too has the challenge of parallelizing and decontextualizing work;
Researchers have thus investigated the role of
distributed, contextually independent work
in the contexts of collaborative writing of various forms,
software development,
classification, and myriad other purposes
\cite{Kittur:2009:CCI:1518701.1518928,Baecker:1993:UID:169059.169312,
      10.2307/1562247,karnin2010crowdsourcing,bragg2013crowdsourcing,
      chilton2013cascade}.

% \topic{i was going to say something here about breaking work apart to the point that nobody can piece things together,
% but i think this would be more interesting as a commentary on trusting workers or something.}
% Given that much of the focus in crowdsourcing has focused on information work,
% \citeauthor{sensitiveTasks}, for instance,
% illustrate one approach to ensuring that workers never become aware of
% enough information to piece together sensitive data
% \cite{sensitiveTasks}.


\topic{One of the emergent properties of micro--tasks has been the relative cost of
\textit{finding} worthwhile tasks.}
The research community has documented and to some extent attempted to intervene in
the discovery of worthwhile tasks
\cite{taskSearch}.
\citeauthor{Cosley:2007:SUI:1216295.1216309}
attempts to address this by
directing workers to tasks through
``intelligent task routing''
\cite{Cosley:2007:SUI:1216295.1216309}.
Much of this work and the work at the periphery of this space, then,
has focused on
minimizing the amount of time that people need to spend doing
anything other than the work for which they are paid.

{
\subsubsubsection{Piecework's perspective.}
My goal for this section is to make this all about the assembly line and
scientific management, and how piecework literature tried to measure everything,
but found it untenable given the extra equipment that was necessary
(but generally which didn't exist) to track every movement and action that workers took.

\subsubsubsection{What's changed.}
The finding here is that
crowdwork can be more carefully micro--managed than piecework could be,
and that this is a double--edged sword:
we can effectively give feedback to workers on everything they do,
but this is emboldening us to try to
over--manage workers just as piecework tried to do.}

% The question, then, becomes about how to break tasks down as much as possible
% (as discussed in the previous section)
% without losing the necessary context to do a task.
% ow can we design work so that as little context as possible is provided,
% without leaving out context or information necessary to complete the task?

% While some research has attempted to increase workers' awareness of the work they do in
% apparent efforts to yield higher quality work,
% others have turned this constraint of crowdwork into a feature ---
% the abstraction from the work itself
% appears to allow requesters to engage workers in work with ---
% for instance ---
% sensitive, confidential or otherwise personally identifying information.
% We will discuss these and other avenues of research here.

% \textit{Cascade} demonstrated that it's possible to
% break certain classes of tasks apart
% in such a way that they yield taxonomies of various subjects,
% a task generally thought to be sufficiently complex that only expert workers
% with top--down awareness of the task
% --- in this case, with awareness of all the constituent colors ---
% could complete the task
% \cite{chilton2013cascade}.
% \citeauthor{verroios2014context} further illustrate this potential by
% forming a task one might consider highly contextually dependent
% --- summarizing the contents of a movie ---
% in such a way that crowd workers could contribute small pieces of work without
% needing to know the content of the rest of the project
% \cite{verroios2014context}.


\onlyinsubfile{
  % \balance{}
  \printbibliography
}

\end{document}