\documentclass[trackingWork]{subfiles}
\makeatletter
\def\blx@maxline{77}
\makeatother
\begin{document}


\subsubsection[What will work and the place of work look like for workers]{The Relationships of Workers to Work, Peers, and Others}
\label{sec:relationships}

\subsubsubsection{\crowdworkpers}
\topic{The relationships of workers
% with their work,
with their peers and
with requesters
are nuanced and not especially well--understood.}
Researchers have begun to appreciate
the sociality of crowd workers in labor markets;
still, the study of these communities is made more challenging by
the limited access to workers on these sites of work
inherent to digital spaces made without social affordances
\cite{crowdcollab,miller2011understanding}.
We can break this general body of work into two subgroups:
workers' relationships
\begin{inlinelist}
% \item with \textit{work}, % \nameref{ch:workRelationship},
\item with \textit{requesters}, and
\item with \textit{other workers}.
\end{inlinelist}
We'll look at workers' relationships with work itself, which we'll discover
gives us insight into why people engage in crowd work in the first place.

\topic{\ali{some topic sentence that brings together the debate where
one side blames Turkers for being bad at Turking and
another side blames requesters for not understanding Turkers as a culture of people.}}
Some research frames this tension as the Turker's problem
(see, for example, \citeauthor{MaliciousCrowdworkersGadiraju}'s work,
which frames the problem of unpredictable work as the result of ``malicious'' crowd workers),
\cite{MaliciousCrowdworkersGadiraju,Sheng:2008:GLI:1401890.1401965,Ipeirotis:2010:QMA:1837885.1837906}.

Early on, \citeauthor{turkopticon} highlighted
the information asymmetry between workers and requesters on AMT,
leading to the creation of \TO, a site which allows
Turkers to rate and review requesters
\cite{turkopticon}.
\citeauthor{dynamo} took this critique on information asymmetry and power imbalances a step further,
designing \DO~to facilitate
Turkers acting collectively to bring about changes to their circumstances
--- this led to the Academic Requester Guidelines
\cite{dynamo}.
This unbridled power that requesters have over workers and
the resultant stress and frustration that this generates
has been part of the undercurrent of research into
the tense relationships between workers and requesters
\cite{fixingChaos,dynamo}.

\topic{The frustration that workers experience dealing with requesters
seems to precipitate frustration and mistrust between crowd workers, as well.}
\citeauthor{dynamo} describes ``mega--drama'' among workers on forums for Turkers;
\citeauthor{irani2015cultural} and \citeauthor{storiesIraniSilberman} discuss
the culture of crowd work and the study thereof.
\citeauthor{crowdcollab} quantifies and maps this social network of Turkers.
\citeauthor{takingAHITMcInnis} takes these observations and considers
what a crowd work platform might look like if it were to be designed more inclusively
\cite{dynamo,irani2015cultural,storiesIraniSilberman,crowdcollab,takingAHITMcInnis}.
The overarching theme of the research in this space has been
documenting the struggle of crowd workers
and attempting to intervene in constructive ways, while walking the balancing act
(especially in the cases of \citeauthor{irani2015cultural} and later \citeauthor{storiesIraniSilberman})
as we think about the culture of crowd workers.



\subsubsubsection{\pieceworkpers}
\topic{The questions surrounding
the ways pieceworkers related to managers might be best answered by
the work that has been done in the emergence and proliferation of labor unions.}
The primary avenue for workers to interact with managers has been through
laborer advocacy groups such as the American Federation of Labor,
(one of the forerunners of the largest and
most politically influential labor union in the United States).
Looking through that lens, we find copious research on
the relationships between workers and requesters
\cite{levi2009union,ahlquist2013interest,
      mccallum2013global,jacoby1983union}.
One component of collectively negotiating with managers has been the process
of collective action, a topic which has been substantively explored
but is not quite yet answered
\cite{russell1982collective,olsonlogic}.


\topic{Answering how workers related to one another is arguably more challenging
for a number of reasons.}
For one thing, the research methods we typically associate with the exploratory study of cultures
--- Anthropology, and namely participant--observation, ethnography, etc\dots ---
didn't exist quite as we know them at the turn of the \nth{20} century, and
wouldn't for several more decades.
Still, we can look at primary sources, like \citetitle{american1921problem}
to give us some hint of how they related to each other
\cite{american1921problem}.


The driving force of American labor advocacy organizations was to get piecework railroad workers
to identify
``not only as railroad employees but also as members of the larger life of the community''
\cite{american1921problem}.
Doing this, 
\citeauthor{ostrom1990governing} and others argued,
would facilitate collective action and perhaps collective governance
\cite{ostrom1990governing,russell1982collective,olsonlogic}.
\citeauthor{riisOtherSideLives} had contributed to this sense of shared struggle and endurance
by the time \citetitle{american1921problem} was published
by documenting pieceworkers in their home--workplaces,
literally bringing to light the grim circumstances in which pieceworkers lived and worked
\cite{riisOtherSideLives}.

% This can also foreshadow crowd sourcing efforts like
% \TO~and \DO.



\subsubsubsection{\whatchanged}
\topic{The differences between crowd workers and pieceworkers seem defined largely by
the differences in the places of work.}
Whereas it arguably became inevitable that workers would have a place to
meet, discuss, and collaborate
when they began sharing places of work,
online spaces make it much harder to do so.
Crowd workers can ``lurk'' and do tasks, or just do the occasional one--off task,
without any affiliation with
--- or even knowledge of ---
communities of peers
% \ali{multiple citations that labor unions came out of factories here}
\cite{miller2011understanding,mcinnis2016one,earl2011digitally}.

\topic{We further find the sources of differences between crowd work and piecework
in the nature of the relationship between workers and requesters
--- or rather, the lack thereof.}
While historically the management of workers had to be done through a foreman
(who necessarily had an intuitive
--- perhaps sympathetic ---
relationship with workers),
the foreman of the \nth{20} century has largely been replaced
by algorithms of the \nth{21} century
\cite{uberAlgorithm}.
The result of this change is that the agents managing work are now
cold and logical, if unforgiving.
Where a person might recognize that the ``attention check'' questions 
proposed by \citeauthor{le2010ensuring} ensure that malicious and inattentive are stopped,
some implementations of these approaches
only seem to antagonize workers \ali{shots fired}
\cite{le2010ensuring,MaliciousCrowdworkersGadiraju}.
% We've known for decades that workers prefer human management 
\citeauthor{10.2307/2555446} told us more than 30 years ago
--- in \citeyear{10.2307/2555446} --- that
``\dots~when performance is difficult to evaluate,
imperfect input measures and
a manager's subjective judgment are preferable to
defective (simple, observable) output measures''
\cite{10.2307/2555446}.
\ali{here's an idea. feel free to push back or revise or something:\\
``Should we be surprised that
management approaches we've known to
frustrate pieceworkers since \citeyear{10.2307/2555446}
would only aggravate crowd workers in many of the same ways?''}


\subsubsubsection{\implication}
\topic{What we've done in the field of crowd work might be able to tell us something about piecework
just as piecework has told us so much about crowd work.}
Crowd work research doesn't just benefit from
digital media allowing us to make
relationship networks like \citeauthor{crowdcollab} do;
we benefit from the firmer theoretical basis of Anthropology that
existed in a radically different form at the turn of the \nth{20} century,
when piecework began to emerge.
\citeauthor{malinowski2002argonauts,boas1940race,mead1973coming} and
other luminaries throughout the first half of the \nth{20} century
effectively defined Cultural Anthropology as we know it today;
\textit{participant--observation},
the \textit{etic} and the \textit{emic} understanding of culture, and
\textit{reflexivity}
didn't take even a resemblance of their contemporary forms until these works
\cite{malinowski2002argonauts,boas1940race,mead1973coming}.

\topic{The research on piecework still offers to guide us on perhaps the most rudimentary aspects of worker management}
\citeauthor{10.2307/2555446} drew a dichotomous line between
``defective (simple, observable) output measures'' and
``a manager's subjective judgment'',
but such a dichotomy need not necessarily represent our work management styles
\cite{10.2307/2555446}.
We can develop tools that better inform humans, rather than
(perhaps futilely) attempt to delegate all worker management to machines.
This is an area we should pursue, but haven't yet.
If the literature on piecework is to be believed,
more considerate \textit{human} management may resolve
many of the tensions we've discovered among among crowd workers.







% \paragraph[relationships to work]{Relationships to work}\label{ch:workRelationship}
% \topic{A number of ethical questions surrounding the
% increasing complexity of crowd work have arisen in recent years.}
% \citeauthor{professionalcrowdworkEthics} bring some of these issues to light
% --- working for increasing amounts of time on tasks of growing complexity%
% , only to discover that requesters are not willing to pay%
% , for instance ---
% but these and other dangers range an enormous landscape
% \cite{professionalcrowdworkEthics}.
% \citeauthor{crowdworkFuture}
% list a few of the problems they identified in \citeyear{crowdworkFuture}
% --- motivation, feedback, reputation, quality control, to name a few ---
% while others discuss challenges such as fostering collective action
% and the opportunity for learning and career advancement
% \cite{crowdworkFuture,nickerson2013crowd,dynamo}.



% Some research already looks at research such as investing in workers, and
% informally, we know that this happens among industry requesters
% \cite{jonBrelig,shepherdingDow}.
% AMT, meanwhile, offers requesters the ability to create tasks which are
% not just hidden from unqualified workers by default, but completely.
% Requesters have taken to using lists of worker IDs which reference
% workers who have proven their reliability,
% representing a sort of proto--organization of loosely connected workers.


% \subsubsubsection{\pieceworkpers}
% The rise of labor unions in the \nth{20} century seems to have been precipitated by
% egregiously unjust conditions imposed on workers in factories and elsewhere
% \cite{ebbinghaus1999institutions}.
% Incidents broadly describing this dynamic can be found in research on AMT
% \cite{turkopticon,dynamo}.
% If these are prototypical labor advocacy organizations of contemporary on--demand work,
% the next question we should look to is if
% --- and indeed \textit{how} ---
% these institutions might face challenges in the future.

% For insight on this, we return to \citeyear{levi2009union}'s study of labor unions,
% and identify that
% ``Scholars who evaluate union governance by procedural criteria generally find that oligarchy tends to arise and persist even when democratic procedures are in place''
% \cite{levi2009union}.
% Indeed, \citeauthor{levi2009union} writes about the general perception that labor unions were either
% This perception already appears to be emerging in digitally mediated peer--governed organizations,
% as \citeauthor{keegan2010egalitarians} and others have illustratively documented
% \cite{beschastnikh2008wikipedian,keegan2010egalitarians}.
% If these organizations and others are to avoid the same fate that labor unions faced,
% they should take care to study this phenomenon and attempt to avoid it.


% \subsubsubsection{\whatchanged}
% \topic{While online and distributed workers can be harder to find,
% other features make the study thereof rewarding for the substantively different ways
% that we can approach their study.}
% The longitudinal analysis of everyday discourse, for instance,
% is trivial in the study of online communities


% \subsubsubsection{\implication}



\onlyinsubfile{
\printbibliography
}
\end{document}