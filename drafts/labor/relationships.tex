\documentclass[trackingWork]{subfiles}
\makeatletter
\def\blx@maxline{77}
\makeatother
\begin{document}


\subsection[What will work and the place of work look like for workers]{Workers' Relationships to their Work}\label{sec:relationships}

HCI and CSCW have historically framed themselves around supporting work.
While all artifacts have politics,
the recent shift into computational labor systems has directly impacted
the lives and livelihoods of workers in new ways.
So, it's imperative to ask:
What will the future look like for the workers who use these systems?

\subsubsection{\crowdworkpers}
\begin{comment}
	- Workers do it for money
	- Workers coordinate and communicate (gray, being a turker, team stuff). there is mega--drama
	- Workers don't like requesters. have tried to organize
\end{comment}

Who are the crowd workers and what draws them to crowd work?
Early literature emphasized motivations like fun and spare change, but this narrative soon shifted to emphasize that many workers use platforms such as Amazon Mechanical Turk as a primary source of income~\cite{kaufmann2011more,ipeirotis2010demographics,Antin2012a}.
Despite this, Mechanical Turk is a disappointingly low--wage worksite for most people in the United States~\cite{ipeirotis2010demographics,martin2014being,gupta2014turk}.
Thus, those who choose to opt out of the traditional labor force and spend significant time on Mechanical Turk are especially motivated by the opportunity for autonomy and transience between tasks~\cite{kaufmann2011more}.
While some describe Turkers as powerless victims or even unaware of what's going on,
this framing is increasingly being rejected by workers and designers as
``cast[ing] Turkers as dopes in the system.''~\cite{storiesIraniSilberman}.



Workers' relationships with requesters are fraught.
The unbridled power that requesters have over workers, and
the resultant frustration that this generates,
has motivated research into
the tense relationships between workers and requesters~\cite{fixingChaos,dynamo}.
Workers are often blamed for any low--quality work, regardless of whether they are responsible~\cite{martin2014being,takingAHITMcInnis}.
Some research is extremely open about this position, blaming unpredictable work on ``malicious'' workers~\cite{MaliciousCrowdworkersGadiraju} or those with ``a lack of expertise, dedication [or] interest''~\cite{Sheng:2008:GLI:1401890.1401965}.
Workers resent this position, and for good reason.
\citeauthor{turkopticon} highlighted the information asymmetry between workers and requesters on AMT,
which led to the creation of Turkopticon, a site which allows Turkers to rate and review requesters~\cite{turkopticon}.
Dynamo then took this critique on information asymmetry and power imbalances further, designing a platform to facilitate
collective action among Turkers to changes to their circumstances~\cite{dynamo}.


Researchers have also begun to appreciate the sociality of crowd workers.
Because the platforms do not typically include social spaces, workers instead congregate off--platform in forums and mailing lists.
There, Turkers exchange advice on high--paying work, talk about their earnings, build social connections, and discuss requesters~\cite{martin2014being}.
Many crowd workers know each other through offline and online connections, coordinating behind--the--scenes despite the platforms encouraging independent work~\cite{crowdcollab,yin2016communication}.
However, the frustration and mistrust that workers experience with requesters does occasionally boil over on the forums. 
%This behavior has come to be known as ``mega--drama'' amongst such workers~\cite{dynamo}.
%Still, the study of these communities is made challenging because most of these platforms do not themselves include social affordances for workers~\cite{miller2011understanding}.


\subsubsection{\pieceworkpers}

\begin{comment}
notes: what info do i assume the reader has seen already?
- Clark: pieceworkers work harder, more diligently, etc...
- Riis saw terrible conditions, documented and communicated it to the world
- Worker advocacy groups arose to speak out against piecework
\end{comment}

\topic{Early observers believed that workers were strongly motivated by the autonomy of working in the piecework model}.
\citeauthor{clark1908cotton} observed textile mill pieceworkers and reported,
``When he works by the day the Italian operative wishes to leave before the whistle blows,
but if he works by the piece he will work as many hours as it is possible for him to stand''~\cite{clark1908cotton}.
However, the emergent trend contrasted with this early rhetoric, as
when workers began instituting ``The Fix'', deliberately slow work to game efficiency experts~\cite{roy1954efficiency}.
Piece workers, \citeauthor{roy1954efficiency} found, would form acrimonious relationships with their managers.

\topic{Soon, workers began resisting piecework regimes.}
The match--girls engaged in their famous strike of 1888, particularly pushing to abolish the fines that were taken out of their wages.
Soon others followed suit, including women in the garment industry in Philadelphia who established collective bargaining rights~\cite{10.2307/41829256} and national coal miners who effected an individual minimum wage in 1912~\cite{10.2307/2221944}.


Many worker organizations began weighing in against
% (or, more precisely, against)
piecework and the myriad oversights it made in valuing workers' time~\cite{american1921problem,richards1904anything}.
As mounting attention increasingly revealed problems in piecework's treatment of workers,
workers themselves began to speak out about their frustration with this new regime.
Organizations representing
railway workers,
mechanical engineers, and
others began to mount advocacy in defense of workers~\cite{american1921problem,richards1904anything}.
Pieceworkers' relationships with their employers eventually developed a pattern of using 
laborer advocacy groups~\cite{levi2009union,ahlquist2013interest,mccallum2013global,jacoby1983union}.
Following the template of the match--girls, collective action grew to become a central component of negotiating with managers~\cite{russell1982collective,olsonlogic}.

\topic{Relative to the modern on--demand workers, there is a noticeable dearth of information on the interpersonal relationships among pieceworkers
beyond the match--girls at the end of the \nth{19} century.}
Nevertheless, we can offer some observations:
primary sources indicate that labor organizations wished for workers to identify as a collective group, 
``not only as railroad employees but also as members of the larger life of the community''~\cite{american1921problem}.
Doing this, 
\citeauthor{ostrom1990governing} and others later argued,
would facilitate collective action and perhaps collective governance~\cite{ostrom1990governing,russell1982collective,olsonlogic}.
\citeauthor{riisOtherSideLives} also contributed to this sense of shared struggle and endurance
by documenting pieceworkers in their home--workplaces,
literally bringing to light the grim circumstances in which pieceworkers lived and worked~\cite{riisOtherSideLives}.


\subsubsection{\whatchanged}

\begin{comment}
	*workers make little money but love autonomy --- workers make little money
	workers blamed for quality --- ???
	both cases, sociality is hard
	*collective action hard --- collective action succeeded
	- algorithms, not managers
\end{comment}

There was generally less written about work quality concerns for historical pieceworkers than there is in modern on--demand work. 
Why the difference? 
One possibility is that, by writing web scripts and applying them to many tasks,
a small number of spammers have an outsized influence on the perception of bad actors.
Another possibility is that historical pieceworkers faced much more risk in shirking:
it was much harder for pieceworkers to move to a new location and find a new job.
Today, Mechanical Turk workers can work for a dozen or more different groups in the span of a day.
A third possibility: online anonymity breeds distrust~\cite{friedman2000trust}, and
where pieceworkers could be directly observed by foremen and known to them,
online workers are known by little more than an inscrutible alphanumeric string, like A2XJMS2J2FMVXK.



\topic{The relationship between workers and employers has also shifted: while historically the management of workers had to be done through a foreman,
% (who necessarily had an intuitive
% --- perhaps sympathetic ---
% relationship with workers),
foremen of the \nth{20} century have largely been replaced
by algorithms of the \nth{21} century~\cite{uberAlgorithm}.}
% The result of this change is that
Consequently,
the agents managing work are now
cold, logical, and unforgiving.
While a person might recognize that the ``attention check'' questions
proposed by \citeauthor{le2010ensuring} and others ensure that
malicious and inattentive workers are stopped~\cite{le2010ensuring,AAAIW113995},
some implementations of these approaches
only seem to antagonize workers~\cite{takingAHITMcInnis}.
As \citeauthor{10.2307/2555446} wrote in \citeyear{10.2307/2555446},
``when performance is difficult to evaluate,
imperfect input measures and
a manager's subjective judgment are preferable to
defective (simple, observable) output measures''~\cite{10.2307/2555446}.
This frustration has only grown as requesters have had to rely on automatic management mechanisms.
Only a few use the equivalent of human foremen~\cite{haas2015argonaut,kulkarni2012mobileworks}.

Relative to the history of collective action for pieceworkers,
on--demand workers have struggled to make their voices heard~\cite{dynamo,storiesIraniSilberman,turkopticon}.
With workers constantly drifting through these platforms, and with many part--time members,
it's extremely difficult to corral the group to make a collective decision~\cite{dynamo}.
Even when they can, enforcement remains a challenge:
while pieceworkers could physically block access to a site of production and convince other workers to join them,
online labor markets provide no facilities for workers to change the experience of other workers.
This is a key limitation --- without it, workers cannot enforce a strike.


\subsubsection{\implication}
The decentralization and anonymization of on--demand work, especially online crowd work, will continue to make many of its social relationships a struggle.
While some workers get to know each other well on forums~\cite{martin2014being,crowdcollab}, many never engage in these social spaces.
Without intervention, worker relationships and collectivism are likely to be inhibited by this decentralized design.
One option is to build worker centralizing points into the platform, for example asking workers to vote on each others' reputation or allowing groups of workers to collectively reject a task from the platform~\cite{crowdguilds}.

\topic{The history of piecework further suggests that
relationships between workers and employers might be improved
if employers engaged in more human management styles.}
Instead of delegating as many management tasks as possible to an algorithm,
it might be possible to build dashboards and
other information tools that empower modern crowd work foremen~\cite{kulkarni2012mobileworks}.
If the literature on piecework is to be believed,
more considerate \textit{human} management may resolve
many of the tensions.

\topic{Reciprocally, crowd work may be able to inform piecework research in this domain.}
There exists far less literature about piece workers' relationships than there does today about on--demand workers' relationships.
Two reasons stand out: first, modern platforms are visible to researchers in ways that the sites of piece work labor were not.
Second, Anthropology stands on a firmer theoretical and methodological basis than it did at the turn of the \nth{20} century.
Malinowski, Boas, Mead, and
other luminaries throughout the first half of the \nth{20} century
effectively defined Cultural Anthropology as we know it today;
\textit{participant--observation},
the \textit{etic} and the \textit{emic} understanding of culture, and
\textit{reflexivity}
didn't take even a resemblance of their contemporary forms until these works~\cite{malinowski2002argonauts,boas1940race,mead1973coming}.
On--demand labor today may give us an opportunity to revisit open questions in piecework with a more refined lens.





% \onlyinsubfile{
% \balance{}
% \printbibliography
% }
\end{document}
