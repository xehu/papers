\documentclass[trackingWork]{subfiles}
\makeatletter
\def\blx@maxline{77}
\makeatother
\begin{document}


\subsubsection[the relationships of workers]{The Relationships of Workers to Work, Peers, and Others}
\label{sec:relationships}

\subsubsubsection{\crowdworkpers}
\topic{The relationships of workers with their work, with their peers, and with others
are complex and not especially well--understood.}
While researchers have begun to appreciate
the sociality of crowd workers in labor markets,
the study of these communities is made enormously more challenging by
the limited access of online spaces 
\cite{crowdcollab}.
Nevertheless, a number of interesting findings have
already emerged in the crowd work literature%
, which we can report here.




\topic{A number of ethical questions surrounding the
increasing complexity of crowd work have arisen in recent years.}
\citeauthor{professionalcrowdworkEthics} bring some of these issues at stake
--- working for increasing amounts of time on tasks of growing complexity%
, only to discover that requesters are not willing to pay%
, for instance ---
but these and other dangers range an enormous landscape
\cite{crowdworkFuture,professionalcrowdworkEthics,nickerson2013crowd,dynamo}.


Some research already looks at research such as investing in workers, and
informally, we know that this happens among industry requesters
\cite{jonBrelig,shepherdingDow}.
AMT, meanwhile, offers requesters the ability to create tasks which are
not just hidden from unqualified workers by default, but completely.
Requesters have taken to using lists of worker IDs which reference
workers who have proven their reliability,
representing a sort of proto--organization of loosely connected workers.


\subsubsubsection{\pieceworkpers}
The rise of labor unions in the \nth{20} century seems to have been precipitated by
egregiously unjust conditions imposed on workers in factories and elsewhere
\cite{ebbinghaus1999institutions}.
Incidents broadly describing this dynamic can be found in research on AMT
\cite{turkopticon,dynamo}.
If these are prototypical labor advocacy organizations of contemporary on--demand work,
the next question we should look to is if
--- and indeed \textit{how} ---
these institutions might face challenges in the future.

For insight on this, we return to \citeyear{levi2009union}'s study of labor unions,
and identify that
``Scholars who evaluate union governance by procedural criteria generally find that oligarchy tends to arise and persist even when democratic procedures are in place''
\cite{levi2009union}.
Indeed, \citeauthor{levi2009union} writes about the general perception that labor unions were either
% unable to satisfy their needs or otherwise more invested in the 
This perception already appears to be emerging in digitally mediated peer--governed organizations,
as \citeauthor{keegan2010egalitarians} and others have illustratively documented
\cite{beschastnikh2008wikipedian,keegan2010egalitarians}.
If these organizations and others are to avoid the same fate that labor unions faced,
they should take care to study this phenomenon and attempt to avoid it.


\subsubsubsection{\whatchanged}
\topic{While online and distributed workers can be harder to find,
other features make the study thereof rewarding for the substantively different ways
that we can approach their study.}
The longitudinal analysis of everyday discourse, for instance,
is trivial in the study of online communities


\subsubsubsection{\implication}



\onlyinsubfile{
\printbibliography
}
\end{document}