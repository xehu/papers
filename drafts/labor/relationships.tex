\documentclass[trackingWork]{subfiles}
\makeatletter
\def\blx@maxline{77}
\makeatother
\begin{document}


\subsubsection[the relationships of workers]{???}
\label{sec:relationships}


\ali{TODO:
\begin{enumerate}
  \item \crowdworkpers
  \item \pieceworkpers
  \item \whatchanged
\end{enumerate}
}



\onlyinsubfile{
  % \balance{}
  \printbibliography
  \clearpage
  % \nobalance{}
  }
  \endnotetext{
    \section{Flexibility notes}
    A number of researchers have identified
    worker attrition,
    variability of worker performance,
    and uncertainty about good versus bad--faith actors as open questions of crowdwork
    \cite{MaliciousCrowdworkersGadiraju,Ipeirotis:2010:QMA:1837885.1837906}.
    \ali{We can and should discuss the distinction between presumably ``bad faith'' workers
    \& workers who are merely responding in kind to bad requesters
    --- and the broader questions surrounding
    the roles that requesters as well as workers should play ---
    but let it suffice to say that
    requesters have been trying to understand and manage what appears to them as inconsistent work.
    Their ways of responding to that variance in work quality has largely involved
    making the work more flexible and resilient to work
    (although some work has gone into investigating the causes, rather than treating the symptoms)}

    Earlier we discussed \citeauthor{cheng2015break}'s work
    measuring the impact that interruption has on worker performance
    \cite{cheng2015break}.
    This work illustrates a broader sentiment in
    both the study and practice
    of crowdwork, that microtasks should be designed resiliently against the variability of workers,
    fully exploiting the abstracted nature of each piece of work
    \cite{interruptionIqbal,delayAndOrderLasecki,vaish2014low}.
    That is to say, micro--tasks should be designed such that a single worker's poor performance,
    or a good worker's sudden departure,
    does not significantly impact the agenda of the work as a whole.
    While \citeauthor{cheng2015break} found costs with breaking tasks into smaller components
    in the form of higher cumulative time to complete
    (albeit much shorter real time to complete, owing to parallelization),
    \citeauthor{delayAndOrderLasecki} found that at least \textit{some} performance can be recouped by stringing 
    similar tasks together
    \cite[respectively]{cheng2015break,delayAndOrderLasecki}.

    \citeauthor{embracingErrorKrishna} take a different approach;
    by ``embracing error'' and forming models describing the latency of workers in classifying objects at rapid speeds,
    the authors offer orders--of--magnitude improvements
    in various binary classification tasks
    \cite{embracingErrorKrishna}.
    And rather than building tasks to \textit{tolerate} worker drop--off and attrition,
    some researchers have designed work predicated on the expectation of it instead:
    \citeauthor{sensitiveTasks} describe ways of assigning tasks in such a way that
    crowd workers would never be given enough information to piece together sensitive information about
    any single topic
    \cite{sensitiveTasks}.

    The work thus far seems to attempt to maximize the quality of work among workers through various means:
    \begin{inlinelist}
      \item Identifying ``bad'' workers (fraught with problems as this characterization is) \cite{MaliciousCrowdworkersGadiraju},
      \item Designing tasks with break points to facilitate the on--boarding and off--boarding that happens anyway \cite{cheng2015break}, and
      \item Expecting certain levels of attrition and incorrectness and using that variability to their advantage \cite{embracingErrorKrishna}
    \end{inlinelist}.


    Flexibility has been explored through the lens of Fordism, perhaps best illustrated by
    \citeauthor{tolliday1986between}'s treatment describing
    turnover rates rising above 300\% in the decade leading to the introduction of the assembly line in 1913.
    Specifically, the utilization of ``\dots~`semi--special' machine tools which could be adapted
    [and]~\dots~added flexibility through seasonal layoffs for production workers and the use of
    piece rates~\dots~rather than a day wage system''
    \cite{tolliday1986between}.

    In the field of piecework,
    the research covering this topic has both explored
    a breadth of tasks that might be rendered doable by piecemeal workers
    \textit{as well as} longitudinally documented the success of these approaches.
    Here, we \ali{\dots?}
}
\theendnotes
\end{document}