\documentclass[trackingWork]{subfiles}
\makeatletter
\def\blx@maxline{77}
\makeatother
\begin{document}


\subsection[What will work and the place of work look like for workers]{The Relationships of Workers to Work, Peers, and Others}
\label{sec:relationships}

HCI and CSCW have largely framed themselves around supporting work rather than becoming an infrastructural layer enabling it. While all artifacts have politics, this shift into computational labor systems has directly impacted the lives and livelihood of workers. So, it is important to understand: what will the future look like for the workers who use these systems?

\subsubsection{\crowdworkpers}
\begin{comment}
	- Workers do it for money
	- Workers coordinate and communicate (gray, being a turker, team stuff). there is mega-drama
	- Workers don't like requesters. have tried to organize
\end{comment}

One of the initial questions that researchers asked was, who are the crowd workers and what draws them to crowd work?
Early literature emphasized motivations like fun and spare change, but this narrative soon shifted to emphasize that many workers use platforms such as Amazon Mechanical Turk as a primary source of income~\cite{kaufmann2011more,ipeirotis2010demographics,Antin2012a}.
Despite this, Mechanical Turk is a low-wage affair for most workers in the United States~\cite{ipeirotis2010demographics,martin2014being,gupta2014turk}.
Thus, those who choose to opt out of the traditional labor force and spend significant time on Mechanical Turk are especially motivated by the opportunity for autonomy and skill variety~\cite{kaufmann2011more}.
Due to valuing autonomy, it is tempting to ascribe attitudes of ``pity the workers'' to Turkers, but this frame is increasingly rejected by workers and designers as patronizing~\cite{storiesIraniSilberman}.


% \topic{The relationships of workers
% % with their work,
% with their peers and
% with requesters
% are nuanced and not especially well--understood.}
% We can break this general body of work into two subgroups:
% workers' relationships
% \begin{inlinelist}
% % \item with \textit{work}, % \nameref{ch:workRelationship},
% \item with \textit{requesters}, and
% \item with \textit{other workers}.
% \end{inlinelist}
% We'll look at workers' relationships with work itself, which we'll discover
% gives us insight into why people engage in crowd work in the first place.

Workers' relationships with requesters are fraught.
% \topic{\ali{some topic sentence that brings together the debate where
% one side blames Turkers for being bad at Turking and
% another side blames requesters for not understanding Turkers as a culture of people.}}
Workers are often blamed for any low-quality work, regardless of whether they are responsible~\cite{martin2014being,takingAHITMcInnis}.
Some research is extremely open about this position, blaming unpredictable work on ``malicious'' workers~\cite{MaliciousCrowdworkersGadiraju} or those with ``a lack of expertise, dedication [or] interest''~\cite{Sheng:2008:GLI:1401890.1401965}.
% (see, for example, \citeauthor{MaliciousCrowdworkersGadiraju}'s work,
% which frames the problem of unpredictable work as the result of ``malicious'' crowd workers),
% \cite{MaliciousCrowdworkersGadiraju,Sheng:2008:GLI:1401890.1401965,Ipeirotis:2010:QMA:1837885.1837906}.
Workers resent this position --- for good reason.
\citeauthor{turkopticon} highlighted the information asymmetry between workers and requesters on AMT, leading to the creation of \TO, a site which allows Turkers to rate and review requesters~\cite{turkopticon}.
Dynamo then took this critique on information asymmetry and power imbalances a step further, designing a platform to facilitate
Turkers acting collectively to bring about changes to their circumstances~\cite{dynamo}.
This unbridled power that requesters have over workers and
the resultant stress and frustration that this generates
has been part of the undercurrent of research into
the tense relationships between workers and requesters
\cite{fixingChaos,dynamo}.

Researchers have also begun to appreciate the sociality of crowd workers.
Because the platforms do not typically include social spaces, workers instead congregate off-platform in forums and mailing lists.
There, Turkers exchange advice on high-paying work, talk about their earnings, build social connections, and discuss requesters~\cite{martin2014being}.
Many crowd workers know each other through offline and online connections, coordinating behind-the-scenes despite the platforms encouraging independent work~\cite{crowdcollab,yin2016communication}.
However, the frustration and mistrust that workers experience with requesters does occasionally boil over on the forums. 
This behavior has come to be known as ``mega--drama'' amongst such workers~\cite{dynamo}.
Still, the study of these communities is made challenging because most of these platforms do not themselves include social affordances for workers~\cite{miller2011understanding}.

% \topic{The frustration that workers experience dealing with requesters
% seems to precipitate frustration and mistrust between crowd workers, as well.}
% \citeauthor{dynamo} describes ``mega--drama'' among workers on forums for Turkers;
% \citeauthor{irani2015cultural} and \citeauthor{storiesIraniSilberman} discuss
% the culture of crowd work and the study thereof.
% \citeauthor{crowdcollab} quantifies and maps this social network of Turkers.
% \citeauthor{takingAHITMcInnis} takes these observations and considers
% what a crowd work platform might look like if it were to be designed more inclusively
% \cite{dynamo,irani2015cultural,storiesIraniSilberman,crowdcollab,takingAHITMcInnis}.
% The overarching theme of the research in this space has been
% documenting the struggle of crowd workers
% and attempting to intervene in constructive ways, while walking the balancing act
% (especially in the cases of \citeauthor{irani2015cultural} and later \citeauthor{storiesIraniSilberman})
% as we think about the culture of crowd workers.



\subsubsection{\pieceworkpers}
Early observers believed that workers were strongly motivated by the piecework model.
\citeauthor{clark1908cotton} observed textile mill pieceworkers and reported,
``When he works by the day the Italian operative wishes to leave before the whistle blows,
but if he works by the piece he will work as many hours as it is possible for him to stand.''
Workers' situations were quite dire: \citeauthor{riisOtherSideLives}  documented abhorrent working and living conditions of pieceworkers in New York City~\cite{riisOtherSideLives}.

Workers' relationships with employers quickly soured.
The match--girls strike of 1888 was one of the earliest and most famous successful worker strikes and perhaps the beginning of ``militant trade unionism''~\cite{10.2307/3827491}.
As \citeauthor{weyer1894history} described, ``the match--girls' victory turned a new leaf in Trade Union annals''~\cite{weyer1894history}: in the 30 years after the match--girls strike, the Trade Union Movement enrollment grew from 20\% of eligible workers to over 60\%.
% in the 30 years since the match--girls strike,
% the Trade Union Movement grew from
% ``20 per cent of adult male manual--working wage--earners [to] over 60 per cent''
% \cite{webb1920history}.
This strike was followed by others, including women in the garment industry in Philadelphia who established collective bargaining rights~\cite{10.2307/41829256} and national coal miners who effected an individual minimum wage in 1912~\cite{10.2307/2221944}.
The adoption of piecework's time--studies and other Taylorist and scientific management approaches reliably precipitated strikes and more generally gave workers a clear enemy against which to rally
\cite{jacoby1983union}.


Soon, many worker organizations were weighing in on (or, more precisely, against) piecework and the myriad oversights it made in valuing workers' time~\cite{american1921problem,richards1904anything}.
As mounting attention increasingly revealed problems in piecework's treatment of workers, theworkers themselves began to speak out about their frustration with this new regime.
Organizations representing railway workers, mechanical engineers, and others began to mount advocacy in defense of workers
\cite{american1921problem,richards1904anything}.


% \topic{While many workers participated in piecework, worker sentiment toward the practice was --- by all accounts --- mostly negative.}
% The match girls strikes which \citeauthor{10.2307/3827491} describes were just one early
% --- albeit critical ---
% case study in this space;
% the national coal strike of 1912 led to an overwhelming vote among federated coal miner pieceworkers
% to strike for
% an individual minimum wage, among other demands
% \cite{10.2307/2221944}.
% \citeauthor{10.2307/41829256} documents a series of efforts among women in the garment industries in Philadelphia to negotiate collective bargaining rights and recognition of their own labor union
% \cite{10.2307/41829256}.
% The adoption of piecework's time--studies and other Taylorist and scientific management approaches reliably precipitated strikes and more generally gave workers a clear enemy against which to rally
% \cite{jacoby1983union}.

Pieceworkers' relationships with their employers eventually developed a pattern of using 
% \topic{The questions surrounding
% the ways pieceworkers related to managers might be best answered by
% the work that has been done in the emergence and proliferation of labor unions.}
% The primary avenue for workers to interact with managers has been through
laborer advocacy groups~\cite{levi2009union,ahlquist2013interest,
      mccallum2013global,jacoby1983union}.
% (one of the forerunners of the largest and
% most politically influential labor union in the United States).
% Looking through that lens, we find copious research on
% the relationships between workers and requesters
% \cite{levi2009union,ahlquist2013interest,
%       mccallum2013global,jacoby1983union}.
Collective action grew to become a central component of negotiating with managers~\cite{russell1982collective,olsonlogic}.

\topic{Less is known about how pieceworkers related to each other.}
For one thing, the research methods we typically associate with the exploratory study of cultures
--- Anthropology, and namely participant--observation, ethnography, etc\dots ---
didn't exist quite as we know them at the turn of the \nth{20} century, and wouldn't for several more decades.
% Still, we can look at primary sources, like \citetitle{american1921problem}
% to give us some hint of how they related to each other
% \cite{american1921problem}.
Primary sources indicate that labor organizations wished for workers to identify as a collective group, 
% The driving force of American labor advocacy organizations was to get piecework railroad workers
% to identify
``not only as railroad employees but also as members of the larger life of the community''
\cite{american1921problem}.
Doing this, 
\citeauthor{ostrom1990governing} and others argued,
would facilitate collective action and perhaps collective governance
\cite{ostrom1990governing,russell1982collective,olsonlogic}.
\citeauthor{riisOtherSideLives} also contributed to this sense of shared struggle and endurance
% by the time \citetitle{american1921problem} was published
by documenting pieceworkers in their home--workplaces,
literally bringing to light the grim circumstances in which pieceworkers lived and worked
\cite{riisOtherSideLives}.

% This can also foreshadow crowd sourcing efforts like
% \TO~and \DO.



\subsubsection{\whatchanged}

\begin{comment}
	*workers make little money but love autonomy --- workers make little money
	workers blamed for quality --- ???
	both cases, sociality is hard
	*collective action hard --- collective action succeeded

	- algorithms, not managers
\end{comment}

While historical pieceworkers could be looked down on, as the match-stick girls were characterized by ``brashness, irregularity, low morality, and little education'', there was generally less written about quality concerns for historical pieceworkers than there is in modern crowd work. 
Why the difference? 
One possibility is that, through writing web scripts and applying them to many tasks, it is possible for a small number of spammers have an outsized influence.
Historically, it was much harder for such workers to move and get new jobs --- today, they can simply accept a different task on Mechanical Turk.
Another possibility: online anonymity breeds distrust~\cite{friedman2000trust}, and where pieceworkers could be directly observed by foremen, online workers are known by little more than an account ID.



% \topic{The differences between crowd workers and pieceworkers seem defined largely by
% the differences in the places of work.}
% Whereas it arguably became inevitable that workers would have a place to
% meet, discuss, and collaborate
% when they began sharing places of work,
% online spaces make it much harder to do so.
% Crowd workers can ``lurk'' and do tasks, or just do the occasional one--off task,
% without any affiliation with
% --- or even knowledge of ---
% communities of peers
% % \ali{multiple citations that labor unions came out of factories here}
% \cite{miller2011understanding,mcinnis2016one,earl2011digitally}.

\topic{The relationship between workers and employers has also shifted: while historically the management of workers had to be done through a foreman
(who necessarily had an intuitive
--- perhaps sympathetic ---
relationship with workers),
the foreman of the \nth{20} century has largely been replaced
by algorithms of the \nth{21} century
\cite{uberAlgorithm}.}
The result of this change is that the agents managing work are now
cold, logical, and unforgiving.
Where a person might recognize that the ``attention check'' questions 
proposed by \citeauthor{le2010ensuring} ensure that malicious and inattentive workers are stopped,
some implementations of these approaches
only seem to antagonize workers.
More than 30 years ago, \citeauthor{10.2307/2555446} wrote:
``When performance is difficult to evaluate,
imperfect input measures and
a manager's subjective judgment are preferable to
defective (simple, observable) output measures''
\cite{10.2307/2555446}.
This frustration has only grown as requesters have had to rely on automatic management mechanisms. Only a few use the equivalent of human foremen~\cite{haas2015argonaut,kulkarni2012mobileworks}.

% \ali{here's an idea. feel free to push back or revise or something:\\
% ``Should we be surprised that
% management approaches we've known to
% frustrate pieceworkers since \citeyear{10.2307/2555446}
% would only aggravate crowd workers in many of the same ways?''}

Relative to the mature state of collective action for pieceworkers offline, crowd workers have struggled to make their voices heard~\cite{dynamo,storiesIraniSilberman,turkopticon}.
Both pieceworkers and crowd workers have struggled at times to form a collective identity necessary to organize.
With workers joining and leaving the crowd labor force continuously, and with many part-time members, it is extremely difficult to corral the group to make a collective decision~\cite{dynamo}.
However, even when they can: whereas pieceworkers could physically block access to a site of production, online labor markets provide no facilities for workers to change the experience of other workers.
This is a key limitation --- without it, workers cannot enforce a strike.


\subsubsection{\implication}
\topic{What we've done in the field of crowd work might be able to tell us something about piecework
just as piecework has told us so much about crowd work.}
Crowd work research doesn't just benefit from
digital media allowing us to make
relationship networks like \citeauthor{crowdcollab} do;
we benefit from the firmer theoretical basis of Anthropology that
existed in a radically different form at the turn of the \nth{20} century,
when piecework began to emerge.
\citeauthor{malinowski2002argonauts,boas1940race,mead1973coming} and
other luminaries throughout the first half of the \nth{20} century
effectively defined Cultural Anthropology as we know it today;
\textit{participant--observation},
the \textit{etic} and the \textit{emic} understanding of culture, and
\textit{reflexivity}
didn't take even a resemblance of their contemporary forms until these works
\cite{malinowski2002argonauts,boas1940race,mead1973coming}.

\topic{The research on piecework still offers to guide us on perhaps the most rudimentary aspects of worker management}
\citeauthor{10.2307/2555446} drew a dichotomous line between
``defective (simple, observable) output measures'' and
``a manager's subjective judgment'',
but such a dichotomy need not necessarily represent our work management styles
\cite{10.2307/2555446}.
We can develop tools that better inform humans, rather than
(perhaps futilely) attempt to delegate all worker management to machines.
This is an area we should pursue, but haven't yet.
If the literature on piecework is to be believed,
more considerate \textit{human} management may resolve
many of the tensions we've discovered among among crowd workers.







% \paragraph[relationships to work]{Relationships to work}\label{ch:workRelationship}
% \topic{A number of ethical questions surrounding the
% increasing complexity of crowd work have arisen in recent years.}
% \citeauthor{professionalcrowdworkEthics} bring some of these issues to light
% --- working for increasing amounts of time on tasks of growing complexity%
% , only to discover that requesters are not willing to pay%
% , for instance ---
% but these and other dangers range an enormous landscape
% \cite{professionalcrowdworkEthics}.
% \citeauthor{crowdworkFuture}
% list a few of the problems they identified in \citeyear{crowdworkFuture}
% --- motivation, feedback, reputation, quality control, to name a few ---
% while others discuss challenges such as fostering collective action
% and the opportunity for learning and career advancement
% \cite{crowdworkFuture,nickerson2013crowd,dynamo}.



% Some research already looks at research such as investing in workers, and
% informally, we know that this happens among industry requesters
% \cite{jonBrelig,shepherdingDow}.
% AMT, meanwhile, offers requesters the ability to create tasks which are
% not just hidden from unqualified workers by default, but completely.
% Requesters have taken to using lists of worker IDs which reference
% workers who have proven their reliability,
% representing a sort of proto--organization of loosely connected workers.


% \subsubsubsection{\pieceworkpers}
% The rise of labor unions in the \nth{20} century seems to have been precipitated by
% egregiously unjust conditions imposed on workers in factories and elsewhere
% \cite{ebbinghaus1999institutions}.
% Incidents broadly describing this dynamic can be found in research on AMT
% \cite{turkopticon,dynamo}.
% If these are prototypical labor advocacy organizations of contemporary on--demand work,
% the next question we should look to is if
% --- and indeed \textit{how} ---
% these institutions might face challenges in the future.

% For insight on this, we return to \citeyear{levi2009union}'s study of labor unions,
% and identify that
% ``Scholars who evaluate union governance by procedural criteria generally find that oligarchy tends to arise and persist even when democratic procedures are in place''
% \cite{levi2009union}.
% Indeed, \citeauthor{levi2009union} writes about the general perception that labor unions were either
% This perception already appears to be emerging in digitally mediated peer--governed organizations,
% as \citeauthor{keegan2010egalitarians} and others have illustratively documented
% \cite{beschastnikh2008wikipedian,keegan2010egalitarians}.
% If these organizations and others are to avoid the same fate that labor unions faced,
% they should take care to study this phenomenon and attempt to avoid it.


% \subsubsubsection{\whatchanged}
% \topic{While online and distributed workers can be harder to find,
% other features make the study thereof rewarding for the substantively different ways
% that we can approach their study.}
% The longitudinal analysis of everyday discourse, for instance,
% is trivial in the study of online communities


% \subsubsubsection{\implication}



% \onlyinsubfile{
% \balance{}
% \printbibliography
% }
\end{document}