\documentclass[trackingWork]{subfiles}

\makeatletter
\def\blx@maxline{77}
\makeatother

\begin{document}

\begin{abstract}
The Internet is enabling the rise of
crowd work,
gig work, and
other forms of on--demand labor.
A large and growing body of scholarship has sought to predict
the socio--technical outcomes of this shift, especially along three threads: %. This question has been dominated by three threads:
\begin{inlinelist}
\item \nameref{sec:complexity};
\item \nameref{sec:decomposition}; and
\item \nameref{sec:relationships}.
\end{inlinelist}
In this paper, we look to the historical scholarship on piecework
--- a similar trend of
work decomposition,
distribution, and
payment that was popular at the turn of the \nth{20} century ---
to understand how these questions might play out with modern crowd work.
To do so, we identify the mechanisms that limited piecework historically, and
identify whether crowd work faces the same mechanism limits or might differentiate itself.
This approach to understanding crowd work provides additional theoretical framing
to make sense of a topic with which researchers continue to struggle to understand.
\end{abstract}
\end{document}