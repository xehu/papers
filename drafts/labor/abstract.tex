\documentclass[trackingWork]{subfiles}

\makeatletter
\def\blx@maxline{77}
\makeatother

\begin{document}

\begin{abstract}
Networked computation is enabling the rise of
crowdwork,
gig work, and
other forms of on--demand labor.
A large and growing body of scholarship has sought to predict
the socio--technical outcomes of this shift, especially
\nameref{sec:complexity} how complex and interdependent crowdwork's tasks can be,
\nameref{sec:decomposition} how thinly crowdwork can be sliced and modularized, and
\nameref{sec:relationships} what the collective outcomes will be for crowdworkers.
In this paper, we look to the historical scholarship on piecework
--- a strikingly similar trend of
work decomposition,
distribution, and
payment that was popular at the turn of the \nth{20} century ---
to understand how these questions might play out with modern crowdwork.
To do so, we identify the mechanisms that limited piecework historically, and
identify whether crowdwork faces the same mechanism limits or might differentiate itself.

% We consider the aforementioned open questions in the body of research on crowdwork
% under the following framework:
% \begin{inlinelist}
% \item we look to the familiar literature in human computation;
%       then
% \item we investigate the scholarship on piecework;
% \item we identify what (if anything) has changed in crowdwork to differentiate it from piecework;
%       and finally
% \item we attempt to provide a concise prediction on these open questions,
%       informed by piecework's insights.
% \end{inlinelist}

% With growing attention toward on--demand labor
% --- ranging from the ``sharing economy'' to information work ---
% scholars have made connections to various frameworks 
% and mechanisms such as worker advocacy, empowerment, and Taylorism,
% to make sense of our observations of
% on--demand work and the workers that power this movement.
% We argue that the literature surrounding ``piecework''
% informs and even predicts both the contributions
% that have been made toward the development of on--demand labor and crowd work
% as well as the fallout among workers and researchers
% with regard to the disillusionment and alienation of work.

% After evaluating this framing
% through a series of case studies,
% we look to the future
% to identify worthwhile questions and
% points of inquiry, such as the movement toward factories,
% that researchers in social computing should consider
% as we attempt to anticipate and perhaps shape the future of work.
\end{abstract}


% \onlyinsubfile{
% \clearpage
% \subfile{cases}
% }

\end{document}