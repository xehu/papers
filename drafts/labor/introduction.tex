\documentclass[trackingWork]{subfiles}
\onlyinsubfile{
  \usepackage{xr-hyper}
  \usepackage{hyperref}
  \externaldocument{complexity}
  \externaldocument{relationships}
  \externaldocument{decomposition}
  \externaldocument{piecework_lit}
}


% \onlyinsubfile{
%   \hypersetup{
%     citecolor  = blue,
%     linkcolor  = JokeGreen,
%   }
% }
% \renewcommand{\topic}[1]{#1}


\makeatletter
\def\blx@maxline{77}
\makeatother

\begin{document}

\section{Introduction}\label{sec:introduction}
\msb{Please take a full pass through for consistent terminology. In the intro, we align on on--demand work and call it the union of crowd work and gig work. But then, in the title and the research questions, everything is crowd work, crowd work, crowd work. I suggest either using crowd work in the introduction and explaining that you're using that to mean both traditional crowd work as well as modern gig work, or using on--demand work in the title and in the research questions. I see the point ``(which we will refer to as \textit{crowd work} subsequently, for consistency with prior literature)'', but it feels like a cop out. Which one do we want to use, really? Let's pick one and be consistent throughout.}

\ali{\textit{On--demand work} is more ``correct'', but I'm worried that crowdsourcing researchers will skim it and feel there's nothing relevant to them. Is that giving them too little credit?}

\msb{I think this is a question of which community you want to be influencing. Do you want people who study crowdsourcing to be the primary readers+citers of this paper? Or people who study Uber and algorithms at work? One way to decide this is to name five people who you really hope will read and like this paper, and note which community they are a part of.}

\topic{The past decade has seen a flourishing of computationally-mediated labor.}
The framing of work into modular components
has enabled computational hiring and management of workers at scale~\cite{howe2008crowdsourcing,Bigham2014,crowdworkFuture}.
Distributed workers engage in work whenever their schedules allow,
often with little to no awareness of the broader context of the work, and
often with fleeting identities and associations~\cite{martin2014being,uberAlgorithm}.

For years, such labor was limited to information work such as data annotation and surveys~\cite{CrowdsourcingUserStudies,movieSummarizationWu,yuenSurvey,geiger2011managing,quinnbedersonTaxonomy}.
Howver, physically embodied work such as driving and cleaning have now spawned multiple online labor markets as well~\cite{uberAlgorithm,uberOfficial,zaarlyOfficial,taskrabbitOfficial}.
In this paper we will use the term \textit{on--demand work} to capture this pair of related phenomena:
first, \textit{crowd work}~\cite{crowdworkFuture}, on platforms such as Amazon Mechanical Turk (AMT) and other sites of (predominantly) information work;
and second, \textit{gig work}, often as platforms for one--off jobs, like driving, courier services, and administrative support.
%\topic{As on--demand work has grown, it has given rise to scholarship exploring the domains it occurs in and the people who engage in it.}
%When \citeauthor{howe2008crowdsourcing} coined crowdsourcing, he defined it in relationship to paid labor:
%``taking a function once performed by employees and outsourcing to an undefined, generally large group of people in the form of an open call''~\cite{howe2008crowdsourcing}.

The realization that complex goals can be accomplished by directing  crowds of workers has spurred the technology industry to flock to sites of labor
such as AMT to explore the limits of this distributed, on--demand workforce.
Researchers have also taken to the space in earnest,
developing systems that enable new forms of production
(e.g. \cite{bernsteinSoylent,vizwiz,paolacci2010running}) and pursuing social scientific inquiry into the workers on these platforms~\cite{Ross,whoareNOTtheTurkers}.
This research has called out identified the sociality of gig work~\cite{crowdcollab},
as well as the frustration and disenfranchisement that these systems embody~\cite{turkopticon,martin2014being,takingAHITMcInnis}.
Others have focused on the outcomes of this frustration,
reflecting on the resistance that workers express against digitally-mediated labor markets~\cite{uberAlgorithm,dynamo}.


\topic{This body of research has broadly sought to answer one central question:
What does the future hold for on--demand work and those who do it?}
Researchers have offered insights on this question along three major threads:
First, \namerefl{sec:complexity}~\cite{crowdForgeKittur}?
      Specifically, how complex are the goals that crowd work can accomplish, and what kinds of goals and industries may eventually utilize it?
      \cite{foundry,suzukiAtelier,KimStoria,yuanAlmost,YuEncouragingOutside,
            Nebeling:2016:WCW:2858036.2858169,
            Hahn:2016:KAB:2858036.2858364}.
Second, \namerefl{sec:decomposition}
      \cite{embracingErrorKrishna,bernsteinSoylent,sensitiveTasks,
            LykourentzouPersonalityMatters,KucherbaevReLauncher,
            Law:2016:CKC:2858036.2858144,Cai:2016:CRI:2858036.2858237,
            Chang:2016:ACC:2858036.2858411,Newell:2016:OMA:2858036.2858490}?
And third, \namerefl{sec:relationships}
      \cite{turkopticon,storiesIraniSilberman,dynamo,crowdcollab,
            whyWouldAnyoneBrewer,takingAHITMcInnis}?
% \end{numberlist}

This research has largely sought to answer these questions by examining extant on--demand work phenomena.
So far, it has not offered a framing for holistically explaining
the developments in worker processes that researchers have developed, or
the emergent phenomena in social environments;
nor has any research
gone so far as to directly predict future developments.

\begin{table*}[t]
  \centering
  \begin{tabularx}{\textwidth}{l X X X}
    \toprule
    & \textit{Observations in piecework} & \textit{Mechanism} & \textit{Implications for On--demand Work} \\
    \midrule
    Complexity &
    \small{Growth from simple tasks such as sewing to more complex composite outcomes on the assembly line floor.} &
    \small{Complexity was limited by what could be easily measured and evaluated for payment by the piece.} &
    \small{Measurement and verification remain challenges and will limit complexity unless solved.} \\ \hline
    
    Decomposition &
    \small{Work began sliced such that non-experts could perfom each piece, but over time was sliced such that non-overlapping expertise was required for each step.} &
    \small{Scientific Management and Taylorism informed and drove decomposition by measuring and facilitating the optimization of smaller tasks.} &
    \small{After scientific management matured, piecework began  specialized training to create experts in narrow tasks. A similar shift seems feasible with on--demand work.} \\ \hline
    %The research has reached an ``atomic'' level of decomposition, exhausting the cognitive limits of the human brain, past which point task--switching becomes a substantial cost.} \\ \hline
    
    Workers &
    \small{Firms antagonized and exploited workers, leading workers to support one another independently, ultimately resulting in strong advocacy groups counterbalancing firms.} &
    \small{The features of piecework (independence and transience) were both the fulcrum managers used to exploit workers as well as the focal point around which workers bonded.} &
    \small{While worker frustrations are similar, the decentralized nature of on--demand work will limit collective action until there exist platforms to coordinate and exert pressure.} \\ 
    \bottomrule
  \end{tabularx}
  \caption{Piecework and on--demand work have both wrestled with questions of how complex work can get, how finely-sliced tasks can become, and what the workplace looks like for workers. We connect piecework's history (left) to the mechansims that impacted it (center) to derive predictions for modern on--demand work (right). }
\end{table*}

\subsection{Piecework as a lens to understand crowdsourcing}
In this paper, we offer a framing for on--demand work as a contemporary instantiation of \textit{piecework},
a work and payment structure which breaks tasks down into discrete jobs,
wherein payment is made for work \textit{output}, rather than for \textit{time}.
Piecework use as a lens on on--demand work is not new.
\citeauthor{crowdworkFuture} in \citeyear{crowdworkFuture}
referenced crowd work as piecework briefly
as a loose analogy~\cite{crowdworkFuture}.
But more than this,
the framing of on--demand labor as a re--instantiation of piecework
gives us years of historical material to make sense of this new form of work, and allows us to reflect on--demand work through a mature theoretical lens, informed by decades of rigorous, empirical research.

More concretely, by looking at on--demand work as
an instantiation (or even a continuation) of piecework,
and by interrogating patterns that the historical literature identifies, we can achieve three goals.
First, we can make sense of past events as part of a much larger series of an interrelated phenomenon.
Second, we can reflect on differences in the features that impacted piecework historically and on--demand work today.
And third, to some extent, we can use these differences to offer some predictions of what on--demand work researchers,
and workers themselves,
might expect to see on the horizon.
For example, we will draw on the piecework literature regarding task decomposition,
which was historically limited by shortcomings in measurement and instrumentation, and
leverage that understanding to suggest how modern technology affects this mechanism in on--demand work
--- namely, by enabling precise tracking and measurement of workers via algorithms and software.

We organize this paper as follows:
first, we review the definition and history of piecework
to make clear the analogy to on--demand work;
second, we interrogate the three major research questions above using the lens of piecework. 
For each question, we will survey piecework's historical perspective as well as on--demand work's current perspective, call out similarities and differences, and then hazard predictions on what piecework's history might suggest about the answers to these persistent questions for on--demand work.

\onlyinsubfile{
  \clearpage
  \balance{}
  \printbibliography
}


\end{document}
