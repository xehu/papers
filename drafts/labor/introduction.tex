\documentclass[trackingWork]{subfiles}
\onlyinsubfile{
  \usepackage{xr-hyper}
  \usepackage{hyperref}
  \externaldocument{complexity}
  \externaldocument{relationships}
  \externaldocument{decomposition}
  \externaldocument{piecework_lit}
}


% \onlyinsubfile{
%   \hypersetup{
%     citecolor  = blue,
%     linkcolor  = JokeGreen,
%   }
% }
% \renewcommand{\topic}[1]{#1}


\makeatletter
\def\blx@maxline{77}
\makeatother
\begin{document}

\section{Introduction}\label{sec:introduction}

\topic{The past decade has seen a flourishing of \textit{on--demand work},
largely driven by the reformulation of work as
the constituent parts of larger tasks.}
This framing of work into de--contextualized, modular blocks
enables computation to hire workers at scale through open calls on the Internet
\cite{howe2008crowdsourcing,Bigham2014,crowdworkFuture}.
Distributed paid participants then engage in the work whenever their schedules allow,
with little to no awareness of the broader context of the work, and 
with (often) fleeting identities and associations
\cite{martin2014being,uberAlgorithm}.
In this paper, we use the term on--demand work to join a pair of related phenomena:
\begin{inlinelist}
\item \textit{crowd work}, on platforms such as Amazon Mechanical Turk (AMT) and other sites of (predominantly) information work; and
\item \textit{gig work}, typically involving platforms for one--off jobs, like driving, courier services, or administrative support.
\end{inlinelist}
The realization that complex tasks can be accomplished by directing and managing these crowds of workers spurred industry to flock to sites of labor
like AMT and Uber to explore the limits of this distributed, on--demand workforce.
Researchers have also taken to the space in earnest,
developing systems and designs that enable new forms of production
(e.g., \cite{bernsteinSoylent,vizwiz,paolacci2010running}).

\topic{As on--demand work has grown far beyond the domain of information work from which it first sprang, it has given rise to an increasingly complicated and conflicted culture amongst both the workers who enable it and the researchers who empower it.}
Originally, \citeauthor{howe2008crowdsourcing} described crowdsourcing in general terms as
``outsourcing [work] to an undefined, generally large group of people in the form of an open call''.
However, for years its instantiation was limited to the utilization of
human intelligence to process data, participate in scientific studies, and perform information work
\cite{CrowdsourcingUserStudies,movieSummarizationWu,
      yuenSurvey,geiger2011managing,quinnbedersonTaxonomy}.
More recently, crowdsourcing of physically embodied work
--- driving and cleaning, for instance ---
has become a focus for on--demand labor markets
\cite{uberAlgorithm,uberOfficial,zaarlyOfficial,taskrabbitOfficial}.
This growth prompted increasing efforts to understand the workers who gravitate toward these platforms~\cite{Ross,whoareNOTtheTurkers}.
Some of this research has been motivated by the identification of the sociality of gig work,
and the frustration and disenfranchisement that these systems embody
\cite{turkopticon,martin2014being,takingAHITMcInnis}.
Other work has focused on the \textit{outcomes} of this frustration,
reflecting on the resistance workers express against digitally mediated labor markets
\cite{uberAlgorithm,dynamo}.

\topic{This body of research has sought to answer one central question:
What does the future hold for on--demand work and those that do it?}
Researchers have offered their input on this open question along three major threads:
\begin{enumerate}
  \item \nameref{sec:complexity}?
        Specifically,
        \begin{inlinelist}[label=(\alph*)]
          \item How complex are the goals that crowd work can accomplish?, and
          \item What kinds of goals and industries may eventually utilize it?
        \end{inlinelist}
        \cite{foundry,suzukiAtelier,KimStoria,yuanAlmost,YuEncouragingOutside,
              Nebeling:2016:WCW:2858036.2858169,
              Hahn:2016:KAB:2858036.2858364};
  \item \nameref{sec:decomposition}?
        \cite{embracingErrorKrishna,bernsteinSoylent,sensitiveTasks,
              LykourentzouPersonalityMatters,KucherbaevReLauncher,
              Law:2016:CKC:2858036.2858144,Cai:2016:CRI:2858036.2858237,
              Chang:2016:ACC:2858036.2858411,Newell:2016:OMA:2858036.2858490}; and%
  \item \nameref{sec:relationships}?
        \cite{turkopticon,storiesIraniSilberman,dynamo,crowdcollab,
              whyWouldAnyoneBrewer,takingAHITMcInnis}
\end{enumerate}
This research literature has largely sought to answer these questions by examining the present phenomenon.
So far, it has not offered a framing for holistically explaining
the developments in worker processes that researchers have developed, or
the emergent phenomena in social environments;
nor has any research, to our knowledge,
gone as far as predict future developments.

\subsection{Piecework as a lens to understand crowdsourcing}
% MSB Cutting because it seems like it's just attacking the literature unnecessarily...
% ----
%This large and growing body of research has conversed
%to varying degrees with labor scholarship,
% ...
%To the extent that the research community has explored the boundaries of crowd work,
%it nevertheless has not reflected on the underlying mechanisms determining those boundaries.
%Similarly, work directed toward the decomposition of tasks has begun to appreciate the limits of decomposition,
%but on the whole it hasn't considered the relationship between the decomposition of work and the complexity of work.
% \ali{Mixed feelings about this because I think I'll need to go back into decomposition and
% give a whole thing on \citeauthor{marx2012economic}
% if I bring this up as a real failing of the crowd work research to date.
% Thoughts?}
%While researchers are quickly picking up on the importance of
%the relationships between workers in these platforms,
%this research seems to be unfolding without the benefit of any of
%the inter--personal labor research that has been informing modern work for the better part of a century.
% \ali{Overall: too long? too vague? overstating the shortcomings?}
In this paper, we offer a framing for on--demand work as a contemporary instantiation of \textit{piecework}: %\namerefl{sec:pieceworkArc} ---
a work and payment structure which breaks tasks down into standalone contracts,
wherein payment is made for \textit{work output}, rather than for \textit{time}.
Piecework as a metaphor for crowd work is not new.
Indeed,
\citeauthor{crowdworkFuture} in \citeyear{crowdworkFuture}
referenced crowd work as ``piecework'' briefly
as a loose analogy to the form of work emerging at the time
\cite{crowdworkFuture}.
But more than this,
the framing of on--demand labor as a re--instantiation of piecework
gives us years of historical material to make sense of the broader research on this new form of work, and allows us to reflect on--demand work through a mature theoretical lens, informed by decades of rigorous, empirically based research.

More concretely, by looking at on--demand work as
an instantiation (or even a continuation) of piecework,
and by looking for patterns of behavior that the corresponding literature predicts
on this basis, we can
\begin{inlinelist}
  \item make sense of the phenomena so far as part of a much larger series of interrelated events;
  \item reflect on similarities in the ongoing work among
        workers,
        system--designers, and
        researchers in this space;
        and finally,
  \item to the extent that history repeats itself, offer predictions of what on--demand work researchers,
        and workers themselves,
        should expect to see on the horizon.
\end{inlinelist}
For example, we will draw on the piecework literature such as case studies of the Santa Fe Railway to understand the historical complexity limits in piecework, and leverage that understanding to suggest which modern complexity limits in crowd work~\cite{crowdworkFuture} may be fundamental and which may be overcome.
%we draw upon the piecework literature to focus the research crowd work researchers have done on the complexity of work,
%and come up with theoretically grounded mechanisms explaining the limitations researchers have struggled to overcome thus far.
% \msb{what I think would really help this paragraph is a single great example, in one or two sentences, of how this kind of framing would help. ``For example, we draw on ... to explain ...''. Reader goes AHA I see why you're doing this now!}


We organize this paper as follows: we first review the literature on piecework to lay  groundwork and make clear the analogy to on--demand work. Then, we interrogate the three major research questions above from a piecework frame. We will 
% We will look at a broad range of cases under a number of major themes \msb{a broad range under a number of themes is an empty calorie sentence. get specific.}
% we propose as broadly describing the types of research being done in crowd work
% and more generally in what we argue is contemporary piecework.
%We will look at the above questions in crowd work through the lens
%first of the body of research made by crowd work researchers,
%and then of the body of work piecework researchers have offered.
%After validating this lens as a way of reasoning about crowd work and on--demand labor as a whole,
we will identify similarities and differences between piecework as historically understood and
on--demand work as we experience it today.
Finally, we will make predictions of future developments based on how those similarities and differences influenced piecework.
% we will use this perspective to suggest areas of research worth anticipating,
% and developments we should expect to see in the maturation of digitally mediated work.
Finally, we will offer design implications for researchers and practitioners based on our results.


\onlyinsubfile{
  \balance{}
  \printbibliography
}

\end{document}
