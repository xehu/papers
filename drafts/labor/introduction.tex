\documentclass[trackingWork]{subfiles}
\makeatletter
\def\blx@maxline{77}
\makeatother
\begin{document}

\newcommand{\note}[1]{\onlyinsubfile{\ali{#1}}}

\section{Introduction}
The past decade has seen a flourishing of on--demand work
where the statuses of workers have become so fleeting that 
these workers
(known colloquially as ``Turkers'')
have been described as ``transient''
\cite{kargerIterativeLearning,mitraComparingStrategies,latozaCrowdDev}.
The realization
that tasks can be accomplished by directing and managing this crowd of workers
has spurred the research and and industry communities to flock to sites of labor
like Amazon's Mechanical Turk (AMT)
to explore the limits of this distributed, seemingly ephemeral labor force.
Researchers in particular have taken to the space in earnest,
finding opportunities to enable new forms of work
as well as using Turkers
as representative populations of the public
\cite{bernsteinSoylent,redballoon,paolacci2010running}.

The many sites of work replicating and extending on
the general style of labor popularized by AMT
have predominantly involved work done on a computer
or involving the human processing of data,
leading many to call this ``information work"
\cite{turkopticon,professionalCrowdworkEthics,IraniFromCriticalDesign,OlsonMakeDistanceWorkWork}.
\citeauthor{howe2008crowdsourcing} defined ``crowdsourcing'' in \citeyear{howe2008crowdsourcing}
as
``taking a job traditionally performed by a designated agent
(usually an employee)
and outsourcing it to an undefined,
generally large group of people in the form of an open call''
\cite{howe2008crowdsourcing}.

In the years since,
scholars have generated taxonomies for the work done by many distributed workers
in an attempt to better categorize and reason about the many forms of work done 
on information work platforms such as AMT, oDesk, etc\dots
\cite{yuenSurvey,geiger2011managing,quinnbedersonTaxonomy}.
We add that,
under \citeauthor{howe2008crowdsourcing}'s constraints,
even \textit{more} new forms of work fall squarely under
the metaphorical umbrella we collectively call ``crowdsourcing''.

Indeed, this on--demand workforce has sparked interest
across industries ranging
livery (driving for hire, for example Uber),
house--cleaning (Handy),
and generalized services (TaskRabbit)
\cite{uberOfficial,zaarlyOfficial,taskrabbitOfficial}.
Today, a rapidly growing transient workforce is forming,
itself assembling piece--by--piece as industries and researchers find yet more unexpected ways
to benefit from a latent pool of previously vetted workers
\cite{pewSharing}.

% Broadly, this work has imagined the various applications of ``gig''--based work,
% and making it work for ``requesters''
% --- academics, industries, and to some extent end--users
% who might benefit from the product of crowd--powered work
% \cite{hong2015group,jonBrelig,paolacci2010running}.
% % ??? -->
% Since then,
% the application of digitally mediated environments acting as rapid ``hiring halls''
% for on--demand tasks
% has found purchase far beyond the context of ``information work'' like Amazon Mechanical Turk (AMT);
% livery services (Uber), cleaning (Handy), and various other services (TaskRabbit)
% % parentheticals giving examples?
% have garnered the interest of both industry and research.

Researchers have made efforts to understand the people
that have gravitated toward crowdsourcing platforms
since its emergence and popularization,
but as the form of work has grown and changed, so too have the demographics of workers
\cite{Ross,whoareNOTtheTurkers}.
Some of this research has been motivated by the realization of the sociality of gig work,
and the frustration and disenfranchisement that these systems embody
\cite{turkopticon,dynamo}.
Other work has focused on the outcomes of work,
reflecting on the resistance workers express against digitally mediated labor markets
\cite{uberAlgorithm}.

% Bringing this broader body of work together, perhaps we can think of this as documenting the ``fallout'' of on--demand labor;
% workers increasingly expressing frustration with the disenfranchisement of the work--flows pioneered by researchers in the decade leading up till now,
% and researchers beginning to appreciate the unforeseen costs of transient labor and the impacts both on the work and the workers.

% \note{this probably doesn't belong here but i need to put it somewhere!}

% From a wide range of academic and industrial perspectives
% and increasingly identifying the many stakeholders in digitally mediated work,
The extant body of work has ostensibly sought to answer one underlying question:
What does the future hold for work and those that do it? % expand on ...?
Researchers have offered their input on this open question along three major threads of scholarship:
% the driving motivation behind much of the research in this space has led us to ask,
% at a high level, three important questions:
\begin{enumerate}
  \item What are the limits of crowdsourcing?
  Perhaps more tightly constrained,
  what can and cannot be done by crowd workers?
  % (what are the \textbf{complexity limits}) \note{(hospitals at high level?) -- team scaffolds} \note{why is there a high limit? is that different with crowds?}
    \cite[for example, see][]{foundry,suzukiAtelier,KimStoria,yuanAlmost,YuEncouragingOutside,embracingErrorKrishna,Nebeling:2016:WCW:2858036.2858169,Hahn:2016:KAB:2858036.2858364};
  \item What forms of work design, and worker management and arrangement, are viable?
  % \note{time in motion studies; job design; } or is this more about managers, coops, etc...
    \cite[for example, see][]{bernsteinSoylent,sensitiveTasks,LykourentzouPersonalityMatters,KucherbaevReLauncher,Law:2016:CKC:2858036.2858144,Cai:2016:CRI:2858036.2858237,Chang:2016:ACC:2858036.2858411,Newell:2016:OMA:2858036.2858490}; and%
  \item What will work and the place of work look like for the workers?
    \cite[for example, see][]{turkopticon,storiesIraniSilberman,dynamo,crowdcollab,whyWouldAnyoneBrewer,takingAHITMcInnis}
    % \note{these questions are all a little guilty of vagueness but this feels especially ``meh''}
\end{enumerate}








\subsection{Piecework as a lens to understand crowdsourcing}
This large and growing body of research has conversed
to varying degrees with labor scholarship,
but has not offered a persuasive framing for holistically explaining
the developments in worker processes that researchers have developer, or
the phenomena in social environments we have observed;
nor has any research, to our knowledge,
gone as far as predict future developments.

% These bodies of literature have conversed,
% and indeed begun to identify that their theories are interrelated in some way,
% but thus far a convincing model for making sense of
% --- let alone predicting ---
% the phenomena in on--demand labor has proven elusive.
% For instance, 
% making sense of why workers on AMT (or ``Turkers'') are frustrated with their workplace
% is understandable in its own context, but
% the relationship between the work itself (``turking'')
% and the myriad social outcomes has only briefly been explored
% \cite{crowdcollab}.

We offer a framing for crowd work spanning the aforementioned industries
collectively as a contemporary instantiation of ``piecework''.
Piecework as a metaphor for the type of work at hand is not new.
Indeed,
\citeauthor{crowdworkFuture}
in
\citeyear{crowdworkFuture}
referenced crowd work as ``piecework'' briefly
as a loose analogy to the form of work emerging at the time
% but not given as much weight as it perhaps could have been
\cite{crowdworkFuture}.
% ``Piecework'' here makes use of
% a term historically used to describe work done in the home,
% in manageable tasks,
% often involving clear instructions
% and payment only for work completed, not work done
% (the differentiation then being that
% one would be paid for the \textit{output} of the work,
% not the \textit{duration}).

% This framing prompts us to note several immediately apparent similarities:
% \begin{inlinelist}
%   \item this form of work began in the home
%   \item the worker is paid for each discrete piece of work done, regardless of time or effort; and
%   \item the worker's status
%   (not only socially, but also economically)
%   is ambiguous, or at least the subject of some controversy.
% \end{inlinelist}

But more than this,
the framing of on--demand labor as piecework (re--instantiated)
allows us to attempt to make sense of the broader research on this new form of work
by evaluating this work through a much more refined lens.
More concretely, by looking at task--based or ``gig'' work as
an instantiation (or even a continuation) of piecework,
and by looking for patterns of behavior that the corresponding literature predicts
on this basis, we can
\begin{inlinelist}
  \item make sense of the phenomena so far as part of a much larger series of interrelated events;
  \item bring into focus the ongoing work among workers, system--designers, and researchers in this space; and finally,
  \item offer predictions of what social computing researchers,
        and workers themselves,
        should expect to see on the horizon of on--demand work.
\end{inlinelist}

We'll look at a broad range of cases under a number of major themes
we propose as broadly describing the types of research being done in crowd work
and more generally in what we argue is contemporary piecework.
After validating this lens as a way of reasoning about on--demand labor,
we'll attempt to use this perspective to suggest areas of research worth anticipating,
and developments we should expect to see in the maturation of digitally mediated work.
Finally, we will offer design implications based on this research\dots
% \note{???}

\note{
\textbf{notes/what i want to get across\dots}

\begin{itemize}
  \item on--demand labor has consumed an increasing proportion of the labor market in the past ten years [\textit{Pew research on on--demand labor}];
  the work  has ranged from Turk work processes
  \item some of it has been about resisting Uber \& other gig labor platforms \cite{bernsteinSoylent,uberAlgorithm};
  some researchers have tackled understanding this new instantiation of work, but come up short (\textit{how}).

  \item (what binds this together)
  \item we offer a framing on this topic that situates the research so far on a timeline of the maturation of another form of work that emerged approximately 150 years ago
  --- that is, piecework --- which convincingly suggests that on--demand labor in the form of Turking, driving for Uber, etc\dots are in fact little more than piecework re--surging today.

  \item this paper will trace the body of research describing microwork --- and later gig work --- and place it in the context of historical piecework and the industrial revolution as a whole.
  \item informed by this new lens, we'll then turn our attention to the future of this re--emergent form of work, and suggest ways that researchers and users of on--demand labor might influence the outcomes we predict in this paper.
\end{itemize}
\upshape}


\onlyinsubfile{
  \balance{}
  \printbibliography
}

\end{document}
