\documentclass[trackingWork]{subfiles}
\onlyinsubfile{
  \usepackage{xr-hyper}
  \usepackage{hyperref}
  \externaldocument{complexity}
  \externaldocument{relationships}
  \externaldocument{decomposition}
  \externaldocument{piecework_lit}
}


% \onlyinsubfile{
%   \hypersetup{
%     citecolor  = blue,
%     linkcolor  = JokeGreen,
%   }
% }
% \renewcommand{\topic}[1]{#1}


\makeatletter
\def\blx@maxline{77}
\makeatother
\begin{document}

\section{Introduction}\label{sec:introduction}

\topic{The past decade has seen a flourishing of on--demand work,
largely driven by the reformulation of work as
the constituent parts of larger tasks.}
This framing of work into modular blocks
has allowed people to engage in work despite
limited time,
little to no awareness of the broader context of the work, and 
(often) fleeting identities and associations
\cite{kargerIterativeLearning,mitraComparingStrategies,latozaCrowdDev}.
We've seen on--demand work passing by a number of aliases:
as \textit{crowd work} --- on platforms such as Amazon Mechanical Turk (AMT) and other sites of (predominantly) information work;
as \textit{gig work} --- typically involving platforms for one--off jobs, like driving, courier services, etc\dots;
as the \textit{sharing economy} --- where participants can rent out their valuables (such as a car, a home, etc\dots)
on an ad hoc basis when they don't need it.
The realization
that complex tasks can be accomplished by directing and managing crowds of workers
has spurred the research and industry communities to flock to sites of labor
like AMT
to explore the limits of this distributed, on--demand workforce.
Researchers have taken to the space in earnest,
finding opportunities to enable new forms of work
using this population of ``Turkers''
\cite{bernsteinSoylent,redballoon,paolacci2010running}.

\ali{this mega--paragraph may need to be reduced since it started as two, did very little/nothing to reduce it.}
\topic{This form of work has grown considerably in size,
far beyond the domain of ``information work'' from which it first sprang.}
While \citeauthor{howe2008crowdsourcing} described crowdsourcing as
``outsourcing [work] to an undefined, generally large group of people in the form of an open call'',
for years the instantiation of this work was limited to the utilization of
human intelligence to process data and act on information
\cite{CrowdsourcingUserStudies,movieSummarizationWu,
      yuenSurvey,geiger2011managing,quinnbedersonTaxonomy}.
More recently, crowdsourcing of embodied work
--- driving, cleaning, for instance ---
has become a focus of on--demand labor markets
\cite{uberAlgorithm,uberOfficial,zaarlyOfficial,taskrabbitOfficial}.
For all the growth we've observed in this labor market,
we have also seen a complicated and conflicted culture emerge among its constituent workers.
Researchers have made efforts to understand the people
that have gravitated toward crowdsourcing platforms
since its emergence and popularization,
but as the form of work has grown and changed, so too have the demographics of workers
\cite{Ross,whoareNOTtheTurkers}.
Some of this research has been motivated by the identification of the sociality of gig work,
and the frustration and disenfranchisement that these systems embody
\cite{turkopticon,dynamo}.
Other work has focused on the \textit{outcomes} of this frustration,
reflecting on the resistance workers express against digitally mediated labor markets
\cite{uberAlgorithm}.

\topic{The extant body of work has ostensibly sought to answer one underlying question:
What does the future hold for crowd work and those that do it?}
Researchers have offered their input on this open question along three major threads:
\begin{enumerate}
  \item \nameref{sec:complexity}?
        Specifically,
        \begin{inlinelist}
          \item how complex are the goals that crowd work can accomplish? and
          \item how far will crowd work reach into the everyday lives of people?
          \msb{which people? workers? requesters? do you mean, will recruiting crowd work be the domain of experts or everyday end users?}
          \ali{I think I'm trying to ask how much of the world will do crowd work --- the whole world, or just representative slices of lots of different types of people (experts, non--experts, etc\dots)}
        \end{inlinelist}
        \cite{foundry,suzukiAtelier,KimStoria,yuanAlmost,YuEncouragingOutside,
              Nebeling:2016:WCW:2858036.2858169,
              Hahn:2016:KAB:2858036.2858364};
  \item \nameref{sec:decomposition}?
        \cite{embracingErrorKrishna,bernsteinSoylent,sensitiveTasks,
              LykourentzouPersonalityMatters,KucherbaevReLauncher,
              Law:2016:CKC:2858036.2858144,Cai:2016:CRI:2858036.2858237,
              Chang:2016:ACC:2858036.2858411,Newell:2016:OMA:2858036.2858490}; and%
  \item \nameref{sec:relationships}?
        \cite{turkopticon,storiesIraniSilberman,dynamo,crowdcollab,
              whyWouldAnyoneBrewer,takingAHITMcInnis}
        % \ali{is this about collective action? yes; also, career growth? education? governance? sociality? wages?}
\end{enumerate}


\subsection{Piecework as a lens to understand crowdsourcing}
This large and growing body of research has conversed
to varying degrees with labor scholarship,
but has not offered a persuasive framing for holistically explaining
the developments in worker processes that researchers have developed, or
the phenomena in social environments we have observed;
nor has any research, to our knowledge,
gone as far as predict future developments.
To the extent that the research community has explored the boundaries of crowd work,
it nevertheless has not reflected on the underlying mechanisms determining those boundaries.
Similarly, work directed toward the decomposition of tasks has begun to appreciate the limits of decomposition,
but on the whole it hasn't considered the relationship between the decomposition of work and the complexity of work.
% \ali{Mixed feelings about this because I think I'll need to go back into decomposition and
% give a whole thing on \citeauthor{marx2012economic}
% if I bring this up as a real failing of the crowd work research to date.
% Thoughts?}
While researchers are quickly picking up on the importance of
the relationships between workers in these platforms,
this research seems to be unfolding without the benefit of any of
the inter--personal labor research that has been informing modern work for the better part of a century.
% \ali{Overall: too long? too vague? overstating the shortcomings?}

We offer a framing for crowd work spanning the aforementioned industries
collectively as a contemporary instantiation of \namerefl{sec:pieceworkArc} ---
a work and payment structure which breaks tasks down into standalone contracts,
wherein payment is made for \textit{work output}, rather than for \textit{time}.
Piecework as a metaphor for crowd work is not new.
Indeed,
\citeauthor{crowdworkFuture} in \citeyear{crowdworkFuture}
referenced crowd work as ``piecework'' briefly
as a loose analogy to the form of work emerging at the time
\cite{crowdworkFuture}.
But more than this,
the framing of on--demand labor as a re--instantiation of piecework
gives us more material to make sense of the broader research on this new form of work
by evaluating this work through a much more refined theoretical lens,
informed by decades of rigorous, empirically based research.

More concretely, by looking at crowd work as
an instantiation (or even a continuation) of piecework,
and by looking for patterns of behavior that the corresponding literature predicts
on this basis, we can
\begin{inlinelist}
  \item make sense of the phenomena so far as part of a much larger series of interrelated events;
  \item bring into focus the ongoing work among
        workers,
        system--designers, and
        researchers in this space;
        and finally,
  \item offer predictions of what crowd work researchers,
        and workers themselves,
        should expect to see on the horizon of on--demand work.
\end{inlinelist}
For example,
we draw upon the piecework literature to focus the research crowd work researchers have done on the complexity of work,
and come up with theoretically grounded mechanisms explaining the limitations researchers have struggled to overcome thus far.
% \msb{what I think would really help this paragraph is a single great example, in one or two sentences, of how this kind of framing would help. ``For example, we draw on ... to explain ...''. Reader goes AHA I see why you're doing this now!}


% We will look at a broad range of cases under a number of major themes \msb{a broad range under a number of themes is an empty calorie sentence. get specific.}
% we propose as broadly describing the types of research being done in crowd work
% and more generally in what we argue is contemporary piecework.
We will look at the above questions in crowd work through the lens
first of the body of research made by crowd work researchers,
and then of the body of work piecework researchers have offered.
After validating this lens as a way of reasoning about crowd work and on--demand labor as a whole,
we will identify differences between piecework as historically understood and
digitally mediated crowd work as we experience it today,
and how those differences influence the predictions that researchers in piecework have already made.
% we will use this perspective to suggest areas of research worth anticipating,
% and developments we should expect to see in the maturation of digitally mediated work.
Finally, we will offer design implications based on this research.


\onlyinsubfile{
  \balance{}
  \printbibliography
}

\end{document}
