\documentclass[trackingWork]{subfiles}
\onlyinsubfile{
  \hypersetup{
    citecolor  = blue,
    linkcolor  = JokeGreen,
  }
}

\makeatletter
\def\blx@maxline{77}
\makeatother
\begin{document}
\section{Introduction}
\topic{The past decade has seen a flourishing of on--demand work,
largely driven by the reformulation of work as
the constituent parts of larger tasks.}
This framing of work into abstract blocks
has allowed people to engage in work despite
limited time,
little to no awareness of the broader context of the work, and 
(often) fleeting identities and associations
\cite{kargerIterativeLearning,mitraComparingStrategies,latozaCrowdDev}.
The realization
that complex tasks can be accomplished by directing and managing crowds of workers
has spurred the research and and industry communities to flock to sites of labor
like Amazon's Mechanical Turk (AMT)
to explore the limits of this distributed, fleeting workforce.
Researchers in particular have taken to the space in earnest,
finding opportunities to enable new forms of work
using this population of ``Turkers''
\note{as test subjects in a sort of at--scale laboratory for economic and other behavioral studies}
\cite{bernsteinSoylent,redballoon,paolacci2010running}.

\topic{This form of work has grown considerably in size,
far beyond the domain of ``information work'' from which it first sprang.}
While \citeauthor{howe2008crowdsourcing} described crowdsourcing as
``outsourcing [work] to an undefined, generally large group of people in the form of an open call'',
for years the instantiation of this work was limited to the utilization of
human intelligence to process data and act on information
\cite{CrowdsourcingUserStudies,movieSummarizationWu,
      yuenSurvey,geiger2011managing,quinnbedersonTaxonomy}.
More recently, crowdsourcing of embodied work
--- driving, cleaning, for instance ---
has become a focus of on--demand labor markets
\cite{uberAlgorithm,uberOfficial,zaarlyOfficial,taskrabbitOfficial}.
Today, on--demand work promises to become a \joke{yuge industry}
\joke{[citation needed]}.

% \topic{This matters because\dots WHYYYYYYYYYYYYYYY\\}
% Indeed, this on--demand workforce has sparked interest
% across industries ranging
% driving for hire (for example Uber),
% house--cleaning (Handy),
% and generalized services (TaskRabbit)
% \cite{uberOfficial,zaarlyOfficial,taskrabbitOfficial}.
% Today, a rapidly growing transient workforce is forming,
% itself assembling piece--by--piece as industries and researchers find yet more unexpected ways
% to benefit from a latent pool of previously vetted workers
% \cite{pewSharing}.

\topic{For all the growth we've observed in this labor market,
we have also seen a complicated and conflicted culture emerge among its constituent workers.}
Researchers have made efforts to understand the people
that have gravitated toward crowdsourcing platforms
since its emergence and popularization,
but as the form of work has grown and changed, so too have the demographics of workers
\cite{Ross,whoareNOTtheTurkers}.
Some of this research has been motivated by the identification of the sociality of gig work,
and the frustration and disenfranchisement that these systems embody
\cite{turkopticon,dynamo}.
Other work has focused on the \textit{outcomes} of this frustration,
reflecting on the resistance workers express against digitally mediated labor markets
\cite{uberAlgorithm}.

\topic{The extant body of work has ostensibly sought to answer one underlying question:
What does the future hold for work and those that do it?}
Researchers have offered their input on this open question along three major threads:
\begin{enumerate}
  \item What are the limits of crowdsourcing
        \cite{foundry,suzukiAtelier,KimStoria,yuanAlmost,YuEncouragingOutside,embracingErrorKrishna,
              Nebeling:2016:WCW:2858036.2858169,Hahn:2016:KAB:2858036.2858364}?
        \note{Perhaps to be more precise, we ask two questions here:
                \begin{enumerate}
                  \item how far will crowdsourcing reach into the everyday lives of people, and
                  \item how complex can crowdwork become?
                \end{enumerate}}
  \item What forms of work design, and worker management and arrangement, are viable?
        \cite{bernsteinSoylent,sensitiveTasks,LykourentzouPersonalityMatters,KucherbaevReLauncher,
              Law:2016:CKC:2858036.2858144,Cai:2016:CRI:2858036.2858237,
              Chang:2016:ACC:2858036.2858411,Newell:2016:OMA:2858036.2858490}; and%
  \item What will work and the place of work look like for the workers?
        \cite{turkopticon,storiesIraniSilberman,dynamo,crowdcollab,whyWouldAnyoneBrewer,
              takingAHITMcInnis}
\end{enumerate}


\subsection{Piecework as a lens to understand crowdsourcing}
This large and growing body of research has conversed
to varying degrees with labor scholarship,
but has not offered a persuasive framing for holistically explaining
the developments in worker processes that researchers have developer, or
the phenomena in social environments we have observed;
nor has any research, to our knowledge,
gone as far as predict future developments.

We offer a framing for crowd work spanning the aforementioned industries
collectively as a contemporary instantiation of ``piecework''.
Piecework as a metaphor for the type of work at hand is not new.
Indeed,
\citeauthor{crowdworkFuture} in \citeyear{crowdworkFuture}
referenced crowd work as ``piecework'' briefly
as a loose analogy to the form of work emerging at the time
\cite{crowdworkFuture}.
But more than this,
the framing of on--demand labor as a re--instantiation of piecework
gives us more material to make sense of the broader research on this new form of work
by evaluating this work through a much more refined theoretical lens,
informed by decades of rigorous, empirically based research.

More concretely, by looking at task--based or ``gig'' work as
an instantiation (or even a continuation) of piecework,
and by looking for patterns of behavior that the corresponding literature predicts
on this basis, we can
\begin{inlinelist}
  \item make sense of the phenomena so far as part of a much larger series of interrelated events;
  \item bring into focus the ongoing work among
        workers,
        system--designers, and
        researchers in this space;
        and finally,
  \item offer predictions of what social computing researchers,
        and workers themselves,
        should expect to see on the horizon of on--demand work.
\end{inlinelist}

We'll look at a broad range of cases under a number of major themes
we propose as broadly describing the types of research being done in crowd work
and more generally in what we argue is contemporary piecework.
After validating this lens as a way of reasoning about on--demand labor,
we'll attempt to use this perspective to suggest areas of research worth anticipating,
and developments we should expect to see in the maturation of digitally mediated work.
Finally, we will offer design implications based on this research.


\onlyinsubfile{
  \balance{}
  \printbibliography
  \nobalance{}
  \clearpage
  \begin{appendices}
  \section{Sleeping kittens}
  \theendnotes
  \end{appendices}

}

\end{document}
