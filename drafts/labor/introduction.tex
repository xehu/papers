\documentclass[trackingWork]{subfiles}
\onlyinsubfile{
  \usepackage{xr-hyper}
  \usepackage{hyperref}
  \externaldocument{complexity}
  \externaldocument{relationships}
  \externaldocument{decomposition}
  \externaldocument{piecework_lit}
}


% \onlyinsubfile{
%   \hypersetup{
%     citecolor  = blue,
%     linkcolor  = JokeGreen,
%   }
% }
% \renewcommand{\topic}[1]{#1}


\makeatletter
\def\blx@maxline{77}
\makeatother
\begin{document}

\section{Introduction}\label{sec:introduction}
\msb{Please take a full pass through for consistent terminology. In the intro, we align on on--demand work and call it the union of crowd work and gig work. But then, in the title and the research questions, everything is crowd work, crowd work, crowd work. I suggest either using crowd work in the introduction and explaining that you're using that to mean both traditional crowd work as well as modern gig work, or using on--demand work in the title and in the research questions. I see the point ``(which we will refer to as \textit{crowd work} subsequently, for consistency with prior literature)'', but it feels like a cop out. Which one do we want to use, really? Let's pick one and be consistent throughout.}

\ali{\textit{On--demand work} is more ``correct'', but I'm worried that crowdsourcing researchers will skim it and feel there's nothing relevant to them. Is that giving them too little credit?}

\msb{I think this is a question of which community you want to be influencing. Do you want people who study crowdsourcing to be the primary readers+citers of this paper? Or people who study Uber and algorithms at work? One way to decide this is to name five people who you really hope will read and like this paper, and note which community they are a part of.}

\topic{The past decade has seen a flourishing of \textit{on--demand work},
largely driven by the reformulation of work as
constituent parts of larger tasks.}
This framing of work into modular \msb{missing word?}
has enabled the computational hiring \& management of workers at scale~\cite{howe2008crowdsourcing,Bigham2014,crowdworkFuture}.
Distributed paid participants then engage in work whenever their schedules allow,
often with little to no awareness of the broader context of the work, and
often with fleeting identities and associations~\cite{martin2014being,uberAlgorithm}.
In this paper, we use ``on--demand work'' to capture a pair of related phenomena:
first, \textit{crowd work}, on platforms such as Amazon Mechanical Turk (AMT) and other sites of (predominantly) information work;
and second, \textit{gig work}, often as platforms for one--off jobs, like driving, courier services, or administrative support.
The realization that complex goals can be accomplished by directing and managing crowds of workers spurred industry to flock to sites of labor
such as AMT to explore the limits of this distributed, on--demand workforce.
Researchers have also taken to the space in earnest,
developing systems and designs that enable new forms of production
(e.g. \cite{bernsteinSoylent,vizwiz,paolacci2010running}).

\topic{As on--demand work has grown far beyond information work, it has given rise to an increasingly complex, conflicted culture amongst both the workers who enable it and the researchers who study it.}
\msb{This paragraph's topic sentence is about the conflicted culture, but the content seems to be about the evolution of the phenomenon. What's the goal here?}
\citeauthor{howe2008crowdsourcing} first described crowdsourcing broadly as
``outsourcing to an undefined, generally large group of people in the form of an open call''~\cite{howe2008crowdsourcing}.
However, for years its instantiation was limited to the utilization of
human intelligence to process data, participate in scientific studies, and perform information work~\cite{CrowdsourcingUserStudies,movieSummarizationWu,yuenSurvey,geiger2011managing,quinnbedersonTaxonomy}.
More recently, crowdsourcing of physically embodied work
--- driving and cleaning, for instance ---
has become a focus for on--demand labor markets~\cite{uberAlgorithm,uberOfficial,zaarlyOfficial,taskrabbitOfficial}.
This growth prompted efforts to understand not just the work, but also the workers on these platforms~\cite{Ross,whoareNOTtheTurkers}.
Some of this research has been motivated by the identification of the sociality of gig work~\cite{crowdcollab},
and the frustration and disenfranchisement that these systems embody~\cite{turkopticon,martin2014being,takingAHITMcInnis}.
Other work has focused on the outcomes of this frustration,
reflecting on the resistance workers express against digitally mediated labor markets~\cite{uberAlgorithm,dynamo}.

\topic{This body of research has broadly sought to answer one central question:
What does the future hold for on--demand work and those who do it?}
Researchers have offered insights on this question along three major threads:
First, \nameref{sec:complexity}?
      Specifically,
      \begin{inlinelist}[label=(\alph*)]
        \item How complex are the goals that crowd work can accomplish?, and
        \item What kinds of goals and industries may eventually utilize it?
      \end{inlinelist}
      \cite{foundry,suzukiAtelier,KimStoria,yuanAlmost,YuEncouragingOutside,
            Nebeling:2016:WCW:2858036.2858169,
            Hahn:2016:KAB:2858036.2858364}.
Second, \nameref{sec:decomposition}
      \cite{embracingErrorKrishna,bernsteinSoylent,sensitiveTasks,
            LykourentzouPersonalityMatters,KucherbaevReLauncher,
            Law:2016:CKC:2858036.2858144,Cai:2016:CRI:2858036.2858237,
            Chang:2016:ACC:2858036.2858411,Newell:2016:OMA:2858036.2858490}.
And third, \nameref{sec:relationships}?
      \cite{turkopticon,storiesIraniSilberman,dynamo,crowdcollab,
            whyWouldAnyoneBrewer,takingAHITMcInnis}.
% \end{numberlist}

This research has largely sought to answer these questions by examining extant on--demand work phenomena.
So far, it has not offered a framing for holistically explaining
the developments in worker processes that researchers have developed, or
the emergent phenomena in social environments;
nor has any research
gone so far as to directly predict future developments.

\subsection{Piecework as a lens to understand crowdsourcing}
In this paper, we offer a framing for on--demand work as a contemporary instantiation of \textit{piecework},
a work and payment structure which breaks tasks down into standalone contracts,
wherein payment is made for work output, rather than for time.
Piecework as a metaphor for crowd work is not new.
Indeed,
\citeauthor{crowdworkFuture} in \citeyear{crowdworkFuture}
referenced crowd work as piecework briefly
as a loose analogy~\cite{crowdworkFuture}.
But more than this,
the framing of on--demand labor as a re--instantiation of piecework
gives us years of historical material to make sense of this new form of work, and allows us to reflect on--demand work through a mature theoretical lens, informed by decades of rigorous, empirical research.

More concretely, by looking at on--demand work as
an instantiation (or even a continuation) of piecework,
and by interrogating patterns that the corresponding literature predicts
on this basis, we can
first, make sense of past events as part of a much larger series of an interrelated phenomenon;
second, reflect on differences in the factors that impacted piecework historically and their impact on--demand work today;
and third, to some extent, offer predictions of what on--demand work researchers,
and workers themselves,
might expect to see on the horizon.
% \end{numberlist}
For example, we will draw on the piecework literature regarding task decomposition,
which was historically limited by shortcomings in measurement and instrumentation, and
leverage that understanding to suggest how modern technology affects this mechanism in on--demand work
--- namely, by enabling precise tracking and measurement of workers via algorithms and software. \msb{leading to... (let's close the loop)}


We organize this paper as follows:
first, we review the definition and historical arc of piecework
to lay groundwork and make clear the analogy to on--demand work
(which we'll call \textit{crowd work} hereon
for consistency with the existing literature);
second, we interrogate the three major research questions above using the lens of piecework. 
We will identify similarities and differences between piecework as historically understood and
on--demand work as we know it today;
third, we will make predictions of future developments based on how those similarities and differences influenced piecework;
finally, we will offer implications for researchers and practitioners based on our results.
% \end{numberlist}


% \onlyinsubfile{
%   \balance{}
%   \printbibliography
% }

\end{document}
