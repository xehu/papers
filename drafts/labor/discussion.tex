\documentclass[trackingWork]{subfiles}
\makeatletter
\def\blx@maxline{77}
\makeatother
\onlyinsubfile{
\usepackage{xr-hyper}
\usepackage{hyperref}
\externaldocument{complexity}
\externaldocument{relationships}
\externaldocument{decomposition}
}

\begin{document}
\section{Discussion}
\topic{Having taken a comprehensive look toward crowd work through the piecework lens,
we can't help but take a step back to consider a number of meta--issues that arose in our analysis.}
Stated briefly, these issues are 
\begin{inlinelist}
  \item \namerefl{sec:perilousProblemsPredicting},
  \item \namerefl{sec:polarizationOfCrowdWork}, and
  \item \namerefl{sec:whatShouldBeTheFuture}.
\end{inlinelist}
We will attempt to grapple with these questions here based on
what we brought up in the earlier case studies.


\subsection{The Hazards of Predicting the Future}\label{sec:perilousProblemsPredicting}
The past can't be a perfect predictor for the future;
as \cite{scholz2012digital} points out,
``it would be wrong to conclude that
in the realm of digital labor
there is nothing new under the sun''
\cite{scholz2012digital}.
Many of the challenges that \DO~overcame
--- determining trustworthiness, ensuring anonymity, etc\dots
\cite{dynamo} ---
were relatively unique challenges precipitated by the affordances of the Internet, specifically
that people could (and often do) contribute to communities in a one--off manner
\cite{mcinnis2016one}.
The Internet seemed to make this kind of loose affiliation feasible where before it wasn't
\cite{catalyst,dynamo}.
% articulated in the discussion about
% organizing people and the challenges therein
% \cite{dynamo}, as well as for
% many of the rhizomatic features \citeauthor{miller2011understanding}
% discusses
% \cite{miller2011understanding}).


But this does not mean that
attempting to draw meaningfully from historical scholarship would be folly;
as we have shown in the preceding sections,
enough of piecework can and does inform crowdsourcing that
we should take this as a cue to look first for
historical framings on phenomena that we encounter,
as this work can outline both the mechanisms that we should expect to play out
as well as one of the (perhaps many) possible futures.

In particular, the predictions that have emerged surrounding crowd work have run the spectrum
from deep pessimism to exuberant optimism. \ali{should I cite this?}
In the next section, we will use
the piecework foundation which informed our case studies
to trace out possible Dystopian and Utopian futures for crowd work.

\subsection{Polarizing Tendencies}\label{sec:polarizationOfCrowdWork}
\topic{It would be easy
to think about the future of crowd work and
end up at one of two extremes.}
On one hand, crowd work researchers
imagine the application of crowdsourcing as
a potentially bright future that enables the achievement of near--impossible goals
\cite{redballoon,crowdworkFuture,vizwiz}; % \ali{and who else should I cite here?}
on the other hand, researchers
warn against crowd work as potentially exploitative sites of dispossession \cite{scholz2012digital},
a site of racial discrimination \cite{edelman2015racial},
and invisible, deeply frustrated workers \cite{turkopticon,bighamHalfWorkday}.

\topic{A uniquely challenging facet of this topic of inquiry is the public attention that this domain has attracted.}
Activists have described speculative work as having
``essentially been turned into modern--day slaves''
\cite{activistsHuffPoLawsuit}.
Meanwhile, advocates describe it as
``a project of sharing
aimed at providing ordinary people
with more economic opportunities and
improving their lives\dots''
\cite{uberPropaganda}.
\ali{these are salient, but with the last paragraph it feels really flimsy.
I'd really like to get this Uber quote in, but
not if it means this paragraph gets that much weaker.}

% \topic{There is, of course, a kernel of truth in everything here.}
\topic{There is evidence to support the claim that
a Dystopian world awaits us at the end of the tunnel.}
The arbitrary nature of on--demand work may permeate our lives, inducing us
``to neglect tasks that are less easy to measure'' \cite{SJOE:SJOE371}
rewarding us not for creativity but predictability;
payment for this work may ultimately be determined by
an algorithm that fundamentally doesn't understand us;
the layers between us and our managers might increasingly become
``defective (simple, observable)'' algorithms \cite{10.2307/2555446},
just like those which already frustrate
on--demand workers
\cite{uberAlgorithm,dynamo,turkopticon}.

\topic{On the other hand, there is some evidence that offers hope.}
Yes, piecework's nascent years were arguably grim, but
they precipitated a century of the strongest labor advocacy the world has ever seen
\cite{hart2013rise,mccallum2013global}.
Even today, the geist that came out of the labor union revolution
inspires collective action and worker empowerment around the world:
in India, workers across the nation engaged in
the largest labor strike in human history
--- perhaps as many as 150 million
\cite{indiaStrikeRealNews}.
If labor advocacy groups can find ways to permeate on--demand labor markets,
as some have called for
\cite{futureUnions},
then the future of crowd work may follow
the same trajectory of worker empowerment that piecework \textit{later} found.
\ali{The Internet may make this possible!}
% Arguably the Internet has made collective action on an unprecedented scale much easier.

\topic{Perhaps the most pressing question is this:
is the future of crowd work going to look more \textit{Dystopian} or \textit{Utopian}?}
If we have learn nothing, crowd work will probably end up somewhere in the middle.
It's possible that the difficulties of enforcing laws on multinational corporations is
tipping the balance of power in the direction of corporations, in which case
things will get worse;
but arguably we can affect change on the trajectory of crowd work,
benefiting from everything piecework scholars have learned about
collective action and governance
among workers, while also avoiding some of the perils they faced.
\ali{I really don't like this paragraph, but I felt like we needed to plant a flag on the question in the topic sentence}



\subsection{Determining our Research Agenda}\label{sec:whatShouldBeTheFuture}
Taking a step back from the work that's been done so far and
thinking about where the body of crowd work research is attempting to go may help us identify
areas of research on which we should be pushing forward on.
We return to the three questions that motivated this paper:
\begin{inlinelist}
  \item ``\namerefl{sec:complexity}?''
  \item ``\namerefl{sec:decomposition}?''
        and
  \item ``\namerefl{sec:relationships}?''
\end{inlinelist}.

While we have arguably outpaced piecework with regard to the limits on the complexity of work,
we have yet to find insurmountable limits to crowdsourcing.
Still, we can learn from the piecework literature as it relates to
the stymieing effect that mismanagement has on workers;
research into the complexity limits should emphasize on finding new ways to manage workers,
in particular using humans --- perhaps other crowd workers ---
to act as modern foremen.

Piecework researchers looking into decomposition pointed out long ago that
piecework is saddled by a lower limit on decomposition:
``piecework does not compensate workers for time spent switching tasks''
\cite{bewley1999wages}.
We've since studied this phenomenon in crowd work to great length both
observationally \cite{taskSearch} and
experimentally \cite{delayAndOrderLasecki}.
We should consider whether this remains a worthwhile area to explore; %  unless
% we're actually finding ways to minimize the many forms of costs of switching tasks.
unless the work we put forth directly affects the costs of task--switching
--- for instance, the cost of suboptimal task search, or the cognitive burden of changing tasks ---
we may only make incremental advances in micro--task decomposition.

Finally, we turn to the relationships of crowd workers.
The crowd work literature here can convincingly speak back to
the piecework scholarship perhaps more than in the other sections.
The tools that are available to us today
--- not just technical, but \textit{methodological} ---
make it possible to
discover,
study, and
empower \ali{I feel like Lily would take some umbrage at this because she seemed to not like the narrative of us researchers just unilaterally bequeathing power upon Turkers (well, Turkers certainly disliked it). Thoughts on refactoring?}
crowd workers in ways that were unimaginable to piecework researchers.
A professor engages in crowd work \cite{bighamHalfWorkday}
not just because it's possible, but because our community
(of mostly computer scientists)
appreciates the importance of approaches such as participant--observation and ethnography as a whole
\cite{olson2014ways}.
% Whereas piecework didn't have the benefit of Anthropology as we know it or the myriad ways of discovering, learning about, and empowering workers
% What questions are we trying to answer?
% If, as we assumed in
% the previous section on \namerefl{sec:polarizationOfCrowdWork},
% our driving motivation is to empower workers, we 

% \end{enumerate}

\endnotetext{
\begin{enumerate}
  \item the past is no guarantee of the future; what could be unforeseen?
  \item it's easy to fall into the (dys|u)--topian camp of what piecework/crowd work will look like
  \item what should we be focusing on? the future of crowd work
\end{enumerate}
}


\onlyinsubfile{
  \balance{}
  \printbibliography
}

\end{document}