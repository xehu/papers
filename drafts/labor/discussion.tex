\documentclass[trackingWork]{subfiles}
\makeatletter
\def\blx@maxline{77}
\makeatother
\begin{document}
\section{Discussion}
\topic{Having taken a comprehensive look toward crowd work through the piecework lens,
we can't help but take a step back to consider a number of meta--issues that arose in our analysis.}
Stated briefly, these issues are 
\begin{inlinelist}
  \item \namerefl{sec:perilousProblemsPredicting},
  \item \namerefl{sec:polarizationOfCrowdWork}, and
  \item \namerefl{sec:whatShouldBeTheFuture}.
\end{inlinelist}
We will attempt to grapple with these questions here based on
what we brought up in the earlier case studies.


\subsection{The Hazards of Predicting the Future}\label{sec:perilousProblemsPredicting}
\begin{enumerate}
  \item The past isn't a perfect indicator of the future
  \item ``it would be wrong to conclude that
        in the realm of digital labor
        there is nothing new under the sun''
        \cite{scholz2012digital}.
  \item Digital sites of labor are so radically different
        (for many of the reasons that \DO
        articulated in the discussion about
        organizing people and the challenges therein
        \cite{dynamo}, as well as for
        many of the rhizomatic features \citeauthor{miller2011understanding}
        discusses
        \cite{miller2011understanding}).
\end{enumerate}


\subsection{Polarizing Tendencies}\label{sec:polarizationOfCrowdWork}
\begin{enumerate}
  \item It's easy to think crowd work as heading toward some extreme
  \item Activists describe speculative workers as having
        ``essentially been turned into modern--day slaves''
        \cite{activistsHuffPoLawsuit}.
  \item Meanwhile, advocates describe it as
        ``a project of sharing
        aimed at providing ordinary people
        with more economic opportunities and
        improving their lives\dots''
        \cite{uberPropaganda}.
  \item There may be truth buried in both claims, but the important takeaway is that
        the closest we can get to a single truth is somewhere in the middled,
        buried in nuance.
  \item There may be a Utopian world at the end of the tunnel that is our world now:
        all work that we engage in could become speculative and risky;
        the layers between us and our managers might increasingly become
        ``defective (simple, observable)'' algorithms \cite{10.2307/2555446}
        --- the same ones which already seem to frustrate
        on--demand workers on Uber and other markets for labor
        \cite{uberAlgorithm,dynamo,turkopticon}.
  \item On the other hand, the future of crowd work could be eminently promising:
        piecework's nascent years were nothing short of grim, but
        they precipitated a century of the strongest labor advocacy the world has ever seen.
        In India, workers across the nation engaged in
        the largest labor strike in human history
        % \cite{indiaStrikeIndustriall,indiaStrikeAlJazeera}
        --- perhaps as many as 150 million
        \cite{indiaStrikeRealNews}.
        Arguably the Internet has made collective action on an unprecedented scale much easier.
  \item So are we heading toward \textit{Mad Max} or\dots
        \ali{is there a Utopian future movie?}
        If we do nothing, probably somewhere in the middle.
        It's possible that the difficulties of enforcing laws on multinational corporations is
        tipping the balance of power in the direction of corporations, in which case
        things will get worse;
        but arguably we can affect change on the trajectory of crowd work,
        benefiting from everything piecework scholars have learned about
        collective action and governance
        among workers, while also avoiding some of the perils they faced.
\end{enumerate}

\subsection{Deciding our Research Agenda}\label{sec:whatShouldBeTheFuture}
\begin{enumerate}
  \item We need to take a step back from the work that's been done so far and
        think about where we are going with the crowd work research agenda.
  \item Piecework researchers pointed out long ago that piecework is problematized by the fact that
        ``piecework does not compensate workers for time spent switching tasks''
        \cite{bewley1999wages} --- we've studied this phenomenon in crowd work
        \cite{delayAndOrderLasecki,taskSearch}
        and we should consider whether this remains a worthwhile area to explore unless
        we're actually finding ways to minimize the many forms of costs of switching tasks.
  \item What questions are we trying to answer?
        If, as we assumed in
        the previous section on \namerefl{sec:polarizationOfCrowdWork},
        our driving motivation is to empower workers, we 

\end{enumerate}

\endnotetext{
\begin{enumerate}
  \item the past is no guarantee of the future; what could be unforeseen?
  \item it's easy to fall into the (dys|u)--topian camp of what piecework/crowd work will look like
  \item what should we be focusing on? the future of crowd work
\end{enumerate}
}


\onlyinsubfile{
  \balance{}
  \printbibliography
}

\end{document}