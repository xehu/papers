\documentclass[trackingWork]{subfiles}
\makeatletter
\def\blx@maxline{77}
\makeatother
\onlyinsubfile{
\usepackage{xr-hyper}
\usepackage{hyperref}
\externaldocument{complexity}
\externaldocument{relationships}
\externaldocument{decomposition}
}

\begin{document}
\section{Discussion}
\topic{In our analysis of crowd work via the piecework lens, three issues arose:}
\begin{inlinelist}
  \item \namerefl{sec:perilousProblemsPredicting},
  \item \namerefl{sec:polarizationOfCrowdWork}, and
  \item \namerefl{sec:whatShouldBeTheFuture}.
\end{inlinelist}
We will attempt to grapple with these questions here explicitly.


\subsection{The Hazards of Predicting the Future}\label{sec:perilousProblemsPredicting}
The past can't be a perfect predictor for the future;
as \cite{scholz2012digital} points out,
``it would be wrong to conclude that
in the realm of digital labor
there is nothing new under the sun''
\cite{scholz2012digital}.
Our analysis is limited by the differences, foreseen and unforeseen, between historical piecework and modern crowd work.
For example, many of the challenges that \DO~overcame in crowd collective action, such as designing for trustworthiness and ensuring anonymity, were relatively unique challenges precipitated by the affordances of the internet.
For example, unlike physical work environments, people can (and often do) contribute to online communities in a one--off manner
\cite{mcinnis2016one}.
The Internet makes this kind of loose affiliation feasible.
While we have attempted to understand the likely overlaps and differences between history and modern day, no analysis is perfect.
% articulated in the discussion about
% organizing people and the challenges therein
% \cite{dynamo}, as well as for
% many of the rhizomatic features \citeauthor{miller2011understanding}
% discusses
% \cite{miller2011understanding}).


But this does not mean that
attempting to draw meaningfully from historical scholarship would be folly;
enough of piecework can and does inform crowdsourcing that
% we should take this as a cue to look first for
HCI and CSCW might seek out historical framings for other phenomena of study as well.
% as this work can outline both the mechanisms that we should expect to play out
While we can only speculate one of (perhaps many) possible futures, history does allow us to articulate and bound which futures appear more likely.

In particular, the predictions that have emerged surrounding crowd work have run the spectrum
from deep pessimism~\cite{fort2011amazon} to exuberant optimism~\cite{crowdworkFuture}.
In the next section, we will use
the piecework foundation which informed our case studies
to trace out possible dystopian and utopian futures for crowd work.


\subsection{Utopian and Dystopian Visions}\label{sec:polarizationOfCrowdWork}
\topic{An easy narrative is to characterize the future of crowd work  at one of two extremes.}
On one hand, crowd work researchers
imagine the application of crowdsourcing as
a potentially bright future that enables the achievement of near--impossible goals and career opportunities~\cite{redballoon,crowdworkFuture,vizwiz,suzukiAtelier}. % \ali{and who else should I cite here?}
On the other hand, researchers
warn that crowd work will create exploitative sites of dispossession~\cite{scholz2012digital},
racial discrimination~\cite{edelman2015racial},
and invisible, deeply frustrated workers~\cite{turkopticon,bighamHalfWorkday}.

\topic{A uniquely challenging facet of this topic of inquiry is the public attention that this domain has attracted.}
Activists have described speculative work as having
``essentially been turned into modern--day slaves''~\cite{activistsHuffPoLawsuit}.
Meanwhile, advocates describe it as
``a project of sharing
aimed at providing ordinary people
with more economic opportunities and
improving their lives''~\cite{uberPropaganda}.

% \topic{There is, of course, a kernel of truth in everything here.}
\topic{Piecework teaches us that, without appropriate norms and policy, the dystopian outcome has happened and will happen again.}
The piecework nature of on--demand work induces us
``to neglect tasks that are less easy to measure'' \cite{SJOE:SJOE371},
rewarding us not for creativity but predictability;
payment for this work may ultimately be determined by
an algorithm that fundamentally doesn't understand people;
the layers between us and our managers might increasingly become
``defective (simple, observable)'' algorithms \cite{10.2307/2555446},
just like those which already frustrate
on--demand workers
\cite{uberAlgorithm,dynamo,turkopticon}.
However, social policy has advanced since the early 1900s, so as crowd work gains popularity a repeat of \textit{How the Other Half Lives}~\cite{riisOtherSideLives} seems less likely.

\topic{On the other hand, while piecework's nascent years were grim, 
they precipitated a century of extremely strong labor advocacy~\cite{hart2013rise,mccallum2013global}.}
Even today, the geist that came out of the labor union revolution
inspires collective action and worker empowerment around the world:
in India, workers across the nation recently engaged in
the largest labor strike in human history
--- perhaps as many as 150 million
\cite{indiaStrikeRealNews}.
If labor advocacy groups can find ways to permeate on--demand labor markets as some have called for
\cite{futureUnions},
then the future of crowd work may follow
the same trajectory of worker empowerment that piecework \textit{later} found.

% Arguably the Internet has made collective action on an unprecedented scale much easier.

\topic{The history of piecework suggests that the utopian and dystopian outcomes will \textit{both} occur, in different parts of the world and to different people.}
When piecework plummeted in the United States, outsourcing rose --- creating major labor issues around the world.
It is entirely possible that we will create a new brand of flexible online career in developed countries, while simultaneously fueling an unskilled decentralized labor force in developing nations.
As designers and researchers, this prompts the question: which outcome are we attempting to promote or avoid for who?

% \topic{Perhaps the most pressing question is this:
% is the future of crowd work going to look more \textit{Dystopian} or \textit{Utopian}?}
% If we have learn nothing, crowd work will probably end up somewhere in the middle.
% It's possible that the difficulties of enforcing laws on multinational corporations is
% tipping the balance of power in the direction of corporations, in which case
% things will get worse;
% but arguably we can affect change on the trajectory of crowd work,
% benefiting from everything piecework scholars have learned about
% collective action and governance
% among workers, while also avoiding some of the perils they faced.
% \ali{I really don't like this paragraph, but I felt like we needed to plant a flag on the question in the topic sentence}



\subsection{A Research Agenda}\label{sec:whatShouldBeTheFuture}
Piecework also helps bring into focus the
% Taking a step back from the work that's been done so far and
% thinking about where the body of crowd work research is attempting to go may help us identify
areas of research that might bear the most fruit.
We return to the three questions that motivated this paper:
\begin{inlinelist}
  \item ``\namerefl{sec:complexity}?''
  \item ``\namerefl{sec:decomposition}?''
        and
  \item ``\namerefl{sec:relationships}?''
\end{inlinelist}.

While we have arguably outpaced piecework with regard to the limits on the complexity of work,
the most complex and open-ended wicked problems~\cite{rittel1973dilemmas} remain the domain of older human collectives such as governments and organizations. 
% we have yet to find insurmountable limits to crowdsourcing.
In addition, we can learn from the piecework literature as it relates to
the stymieing effect that mismanagement has on workers;
research into the complexity limits should emphasize on finding new ways to manage workers,
in particular using humans --- perhaps other crowd workers ---
to act as modern foremen.

Piecework researchers looking into decomposition pointed out long ago that
piecework is saddled by a lower limit on decomposition:
``piecework does not compensate workers for time spent switching tasks''~\cite{bewley1999wages}.
We've since studied this phenomenon in crowd work to great length both
observationally \cite{taskSearch} and
experimentally \cite{delayAndOrderLasecki}.
We should consider whether this remains a worthwhile area to explore; %  unless
% we're actually finding ways to minimize the many forms of costs of switching tasks.
unless the work we put forth directly affects the costs of task--switching
--- for instance, the cost of suboptimal task search, or the cognitive burden of changing tasks ---
we may only make incremental advances in micro--task decomposition.
When the cognitive cost of understanding a task and its inputs outstrips the effort required to complete the task, decomposition seems a poor choice.

Finally, we turn to the relationships of crowd workers.
The crowd work literature here can convincingly speak back to
the piecework scholarship perhaps more than in the other sections.
The tools that are available to us today
--- not just technical, but \textit{methodological} ---
make it possible to
discover,
study, and
partner with 
% \ali{I feel like Lily would take some umbrage at this because she seemed to not like the narrative of us researchers just unilaterally bequeathing power upon Turkers (well, Turkers certainly disliked it). Thoughts on refactoring?}
crowd workers in ways that were unimaginable to piecework researchers.
A professor engages in crowd work~\cite{bighamHalfWorkday}
not just because it's possible, but because our community
% (of mostly computer scientists)
appreciates the importance of approaches such as participant--observation and ethnography as a whole~\cite{olson2014ways}.
% Whereas piecework didn't have the benefit of Anthropology as we know it or the myriad ways of discovering, learning about, and empowering workers
% What questions are we trying to answer?
% If, as we assumed in
% the previous section on \namerefl{sec:polarizationOfCrowdWork},
% our driving motivation is to empower workers, we 

% \end{enumerate}

\endnotetext{
\begin{enumerate}
  \item the past is no guarantee of the future; what could be unforeseen?
  \item it's easy to fall into the (dys|u)--topian camp of what piecework/crowd work will look like
  \item what should we be focusing on? the future of crowd work
\end{enumerate}
}


\onlyinsubfile{
  \balance{}
  \printbibliography
}

\end{document}