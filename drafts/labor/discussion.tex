\documentclass[trackingWork]{subfiles}
\makeatletter
\def\blx@maxline{77}
\makeatother
\onlyinsubfile{
\usepackage{xr-hyper}
\usepackage{hyperref}
\externaldocument{complexity}
\externaldocument{relationships}
\externaldocument{decomposition}
}

\begin{document}

\section{Discussion}
\topic{In our analysis of on--demand work via the piecework lens, three issues arise:}
\begin{inlinelist}
  \item \namerefl{sec:perilousProblemsPredicting},
  \item \namerefl{sec:polarizationOfCrowdWork}, and
  \item \namerefl{sec:whatShouldBeTheFuture}.
\end{inlinelist}
We will attempt to grapple with these questions here explicitly.


\subsection{The Hazards of Predicting the Future}\label{sec:perilousProblemsPredicting}
The past isn't a perfect predictor for the future;
as \citeauthor{scholz2012digital} cautions,
``it would be wrong to conclude that in the realm of digital labor there is nothing new under the sun''~\cite{scholz2012digital}.
Our analysis is limited by the differences, foreseen and unforeseen, between historical piecework and modern on--demand work.
For example, unlike physical work environments, people can (and often do) make one--off contributions to online communities~\cite{mcinnis2016one}.
While we have attempted to identify some likely parallels and divergences between piecework and on--demand work,
we can't claim to have accounted for everything.

But this does not mean that attempting to draw meaningfully from historical scholarship would be folly;
enough of piecework can and does inform on--demand work that
HCI and CSCW might seek out historical framings for other phenomena of study as well.
While we can only speculate one of (perhaps many) possible futures, history does allow us to articulate and bound which futures appear more likely.

\citeauthor{rosenberg1994exploring} and others have contributed substantially to the practice in part by clearly limiting the extent of their claims
--- only offering, for instance,
``to narrow our estimates and thus to concentrate resources in directions that are more likely to have useful payoffs''~\cite{rosenberg1994exploring}.
Using this approach,
our method of relating history to modern socio--technical systems may be
a useful tool for researchers attempting
to make sense of ostensibly new phenomena.
In other words, offering ``that past history is an indispensable source of information
to anyone interested in characterizing technologies''~\cite{rosenberg1982inside}.


\subsection{Utopian and Dystopian Visions}\label{sec:polarizationOfCrowdWork}
\topic{An easy narrative is to characterize the future of on--demand labor at one of two extremes.}
On one hand, crowd work researchers
imagine the application of crowdsourcing as
a potentially bright future that enables the achievement of near--impossible goals and career opportunities~\cite{redballoon,crowdworkFuture,vizwiz,suzukiAtelier}.
On the other hand, researchers
warn that on--demand labor will create exploitative sites of dispossession~\cite{scholz2012digital},
racial discrimination~\cite{edelman2015racial},
and invisible, deeply frustrated workers~\cite{turkopticon,bighamHalfWorkday}.

\topic{A uniquely challenging facet of this domain is the public attention that it has garnered.}
Activists have described speculative work as having
``essentially been turned into modern--day slaves''~\cite{activistsHuffPoLawsuit}.
Meanwhile, advocates have described it as
``a project of sharing aimed at providing ordinary people with more economic opportunities and improving their lives''~\cite{uberPropaganda}.

\topic{Piecework teaches us that, without appropriate norms and policies, the dystopian outcome has happened and may happen again.}
The piecework nature of on--demand work induces us
``to neglect tasks that are less easy to measure''~\cite{SJOE:SJOE371},
rewarding us not for creativity but predictability;
payment for this work may ultimately be determined by
algorithms that fundamentally don't understand people;
the layers between us and our managers might eventually become
``defective (simple, observable)'' algorithms~\cite{10.2307/2555446},
just like those which already frustrate
on--demand workers~\cite{uberAlgorithm,dynamo,turkopticon}.
However, social policy has advanced since the early 1900s, so as on--demand work grows, a repeat of \textit{How the Other Half Lives}~\cite{riisOtherSideLives} seems less likely.

\topic{On the other hand, while piecework's nascent years were grim, 
they precipitated a century of some of the most potent labor advocacy organizations in modern history~\cite{hart2013rise,mccallum2013global}.}
Even today, the \textit{geist} of the labor union revolution
inspires collective action and worker empowerment around the world.
Recently, in India, workers across the nation engaged in
the largest labor strike in human history
\cite{indiaStrikeRealNews}.
If labor advocacy groups can find ways to effect change in on--demand work as some have called for
\cite{futureUnions},
then the future of on--demand labor may follow
the same trajectory of worker empowerment that piecework saw.


\topic{The history of piecework suggests that the utopian and dystopian outcomes will \textit{both} occur, in different parts of the world and to different people.}
When piecework plummeted in the United States, outsourcing rose --- creating major labor issues around the world.
It is entirely possible that we will create a new brand of flexible online career in developed countries, while simultaneously fueling an unskilled decentralized labor force in developing nations.
As designers and researchers, this prompts the question: which outcome are we attempting to promote or avoid for who?


\subsection{A Research Agenda}\label{sec:whatShouldBeTheFuture}
Piecework also helps bring into focus the
areas of research that might bear the most fruit.
We return to the three questions that motivated this paper:
\begin{inlinelist}
  \item \nameref{sec:complexity}?
  \item \nameref{sec:decomposition}?
  \item \nameref{sec:relationships}?
\end{inlinelist}

While we have arguably outpaced piecework with regard to the limits on the complexity of work,
the most complex and open-ended wicked problems~\cite{rittel1973dilemmas} remain the domain of older human collectives such as governments and organizations. 
In addition, we can learn from the piecework literature as it relates to
the stymieing effect that mismanagement has on workers;
research into the complexity limits should emphasize on finding new ways to manage workers,
in particular using humans --- perhaps other crowd workers ---
to act as modern foremen.

Piecework researchers looking into decomposition pointed out long ago that
piecework is saddled by a lower limit on decomposition:
``piecework does not compensate workers for time spent switching tasks''~\cite{bewley1999wages}. % Chinmay wanted me to specifically call out that this is from chapter 6. Is that necessary...?
\ali{Chinmay: <suggestion to look at \cite{stigler1962information} and friction> --- not sure if worth getting into.}
We've since studied this phenomenon in crowd work to great length both
observationally \cite{taskSearch} and
experimentally \cite{delayAndOrderLasecki}.
We should consider whether this remains a worthwhile area to explore;
unless the work we put forth directly affects the costs of task--switching
--- for instance, the cost of suboptimal task search, or the cognitive burden of changing tasks ---
we may only make incremental advances in micro--task decomposition.
When the cognitive cost of understanding a task and its inputs outstrips the effort required to complete the task, decomposition seems a poor choice.

Finally, we turn to the relationships of crowd workers.
The crowd work literature here can convincingly speak back to
the piecework scholarship perhaps more than in the other sections.
The tools that are available to us today
--- not just technical, but \textit{methodological} ---
make it possible to
discover,
study, and
partner with 
crowd workers in ways that were unimaginable to piecework researchers.
A professor engages in crowd work~\cite{bighamHalfWorkday}
not just because it's possible, but because our community
appreciates the importance of approaches such as participant--observation and ethnography as a whole~\cite{olson2014ways}.

\ali{
Discussions from Slack:

Chinmay: I think the most provocative claim is ``The history of piecework suggests that the utopian and dystopian outcomes will both occur, in different parts of the world and to different people.''\\
seems contrary to the idea that marketplaces are global\\
I would also like some more thought into the Research Agenda section\\

Ali: so the dystopian and utopian claim is partly influenced by the fact that piecework seemed to affect some industries differently than others\\

Chinmay: ah ok\\
i think you could claim the same about online piecework\\
differnt industries\\

Ali: garment-makers were pretty thoroughly marginalized, but industrial metalwork led to this whole WWII phenomenon with expert training\\
mmmm\\
but now it's not just about the industry, but about the design of the labor market\\
we can influence the ``architecture'' in a way that we couldn't before\\
this is all lessig and stuff\\

Chinmay: oh\\
i understand\\
hmm something there\\
might poke more :)

\dotfill

Chinmay> i wish you had a little more criticality in taylorism but I'm a downer :)

Ali> actually a reviewer wanted more on taylorism but i was afraid that we just didn't have the space to really give anyone into taylorism what they want

Chinmay> essentially that would be a great place to talk about how those three questions you mentioned interact

Ali> oh
that might be worth fitting in
i mean we inexplicably got it down to 9.5 pages or so
or 9.75
we were over for like 99% of the drafting and then inexplicably something just imploded and we got like half a page
so yeah i'll try and write something giving taylorism more space and using that to talk about the confluence of these three questions
}

\end{document}