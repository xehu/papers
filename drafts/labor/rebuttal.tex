\documentclass[11pt]{article}
\usepackage{balance,graphics,setspace,parskip,times,hyperref,nth}
\usepackage[margin=1in]{geometry}
\usepackage[inline]{enumitem}
\usepackage[tiny,compact]{titlesec}
\usepackage[citestyle=numeric,backend=bibtex]{biblatex}
\bibliography{references}
\newlist{inlinelist}{enumerate*}{1}
\setlist*[inlinelist,1]{%
   % label=\alph*),
 label=\arabic*),
}
\pagenumbering{gobble}
\begin{document}
We'd sincerely like to thank the reviewers for the feedback they generously offered
--- your comments have pointed out some opportunities for clarification
and potential changes in our framing that we think will help strengthen the paper.

1AC and R5 bring up that one of the potential hazards with
historical analysis is that
it tends to strip away context in pursuit of ahistoric conclusions.
This is an excellent point, and
the reference to~\citeauthor{rosenberg1994exploring}'s
work in particular is especially beneficial to us, as his work
(in~\citeyear{rosenberg1982inside} and later~\citeyear{rosenberg1994exploring})
offers a concise description of the spirit of the method we adopted here,
and articulates a limited use case for this approach.

In our revision, we'll attempt to clarify two things:
\begin{enumerate}
  \item that this particular contribution offers
        to fill some of the gaps in crowd work and suggests potential futures
        (perhaps specifically offering
            ``to narrow our estimates
              and thus to concentrate resources
              in directions that are more likely to have useful payoffs''~\cite{rosenberg1994exploring});
  and
  \item that the overarching method here may be
        a powerful tool for researchers attempting
        to make sense of (seemingly) new phenomena
        (for example,
        arguing ``that past history is
                  an indispensable source of information
                  to anyone interested in characterizing technologies''~\cite{rosenberg1982inside}).
\end{enumerate}


% To quote~\citeauthor{rosenberg1994exploring},
% ``Science will often provide the capability to acquire information
% \dots that we do not presently possess, but
% \textit{science does not make the acquisition of this information costless}''
%~\cite{rosenberg1994exploring}.
% In that spirit, 
% This is absolutely an excellent point, and
% a shortcoming which we struggled to articulate in our discussion section.
% We'll add more to the discussion section to highlight the potential hazards
% of applying historical analyses due to the different
% politics, economies, cultures, and other factors which
% may influence any phenomena.

% \textit{There's almost certainly something by Rosenberg about how
%         historical analysis is flawed but not totally unhelpful.
%         I'll quote it.}

R5 also points out that we give attention to~\citeauthor{grier2013computers}'s work and
the case study of human computers,
perhaps at the expense of the other case studies
(that is, the cases of the matchstick girls and railroad workers).
While we occasionally bring to light other cases
(such as the industrial workers during the Second World War),
we'll attempt to bring the two major case studies to
a similar level of attention as we afford Airy's human computers,
% which~\citeauthor{grier2013computers} offers~\cite{grier2013computers},
making these three case studies into
more equal threads which recur throughout the paper.

R3 noted our decision to cluster crowd work research in a particular way,
which consolidated some research
--- for instance, the work into ``professional development'' ---
into the broader topic of ``complexity'';
the concern R3 raises is well taken, and
we'll dedicate some space to reflect on
the decisions we made with regard to clustering research topics.

R4 and R5 offer a number of works
(e.g. by~\textbf{Hounsell}, \textbf{Roe Smith}, and~\citeauthor{williamson2016},
separately)
for a more comprehensive discussion of scientific management.
We agree that these works will
substantively add to a reader's understanding of scientific management.
% We'll attempt to bring into discussion these works, but
% our original concern was that a satisfactory review of scientific management
% would be difficult to articulate concisely.
We'll attempt to crystallize the body of work concisely, and
point out that there's much more to be said about these topics.


% We found ourselves struggling to reconcile
% how adequately to bring up some of the leading thinkers in
% scientific management
% --- and their contributions ---
% given the tight constraints.
% We can only gesture to~\textbf{Hounsell}, \textbf{Roe Smith}, and~\citeauthor{williamson2016}
% and point out that there's substantially more to scientific management than we
% can ``afford'' to communicate.
% Given this feedback, we'll attempt to carefully balance the desire
% to satisfactorily discuss leading minds in these fields with
% the limited space afforded to provide readers with context
% We struggled with this topic in the process of writing,
% as we wanted to communicate a high--level overview of some topics only as they
% were relevant to the core argument of piecework.
% We worried that bringing \textbf{Hounsell}, \textbf{Roe Smith}, and others into
% the section mentioning scientific management would have necessitated
% a much more involved discussion than
% this venue's format would afford.
% Given this feedback, we'll
% acknowledge this influential work and point out that there's
% significantly more detail and nuance to scientific management than
% we could afford to explore, given our prerogative.

% {\LARGE
% Dear reviewers,

% You da real MVPs

% Your feedback was {\Huge😎} I'm all like {\Huge🙇}

% But R5 I mean {\Huge😒}

% Nah I'm just kidding {\Huge😂} keep it {\Huge💯}!
% }

\printbibliography{}
\end{document}