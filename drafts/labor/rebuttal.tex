\documentclass[11pt]{article}
\usepackage{balance,graphics,setspace,parskip,times,hyperref,nth}
\usepackage[margin=1in]{geometry}
\usepackage[inline]{enumitem}
\usepackage[tiny,compact]{titlesec}
\usepackage[citestyle=numeric,backend=bibtex]{biblatex}
\bibliography{references}
% \setlength{\parskip}{.4em}
\pagenumbering{gobble}
\begin{document}
We'd like to thank the reviewers for the feedback they offered
--- their input will absolutely strengthen this paper.

1AC and R5 bring up that one of the issues with historical analysis such as ours
is that the method strips away context in pursuit of ahistoric conclusions.
This is absolutely an excellent point, and
a shortcoming which we struggled to articulate in our discussion section.
We'll add more to the discussion section to highlight the potential hazards
of applying historical analyses due to the different
politics, economies, cultures, and other factors which
may influence any phenomena.

\textit{There's almost certainly something by Rosenberg about how
        historical analysis is flawed but not totally unhelpful.
        I'll quote it.}

R5 also points out that we give attention to \citeauthor{grier2013computers}'s work and
the case study of human computers,
perhaps at the expense of the other case studies
(that is, the cases of the matchstick girls and railroad workers).
While we'll occasionally reference other cases
(such as the industrial workers outside of
the railroad industry in the early and mid--\nth{20} century),
we'll bring the other two case studies to
a similar level of scrutiny that the case study of Airy's human computers
(as described by \citeauthor{grier2013computers}~\cite{grier2013computers}) is afforded.

R3 noted that we decided to cluster crowd work research in a particular way
which consolidated some research
--- for instance, into ``professional development'' --- into
the broader topic of ``complexity'';
the observation is well taken, and we'll dedicate some space to reflect on
the decisions we made with regard to clustering.

R4 and R5 suggest a number of works in scientific management
(\textbf{Hounsell}, \textbf{Roe Smith}, and \citeauthor{williamson2016}) for a more complete view on scientific management.
We struggled with this topic in the process of writing,
as we wanted to communicate a high--level overview of some topics only as they
were relevant to the core argument of piecework.
We worried that bringing \textbf{Hounsell}, \textbf{Roe Smith}, and others into
the section mentioning scientific management would have necessitated
a much more involved discussion than
this venue's format would afford.
Given this feedback, we'll
acknowledge this influential work and point out that there's
significantly more detail and nuance to scientific management than
we could afford to explore, given our prerogative.

% {\LARGE
% Dear reviewers,

% You da real MVPs

% Your feedback was {\Huge😎} I'm all like {\Huge🙇}

% But R5 I mean {\Huge😒}

% Nah I'm just kidding {\Huge😂} keep it {\Huge💯}!
% }

\printbibliography{}
\end{document}