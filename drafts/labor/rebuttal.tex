\documentclass[11pt]{article}
\usepackage{balance,graphics,setspace,parskip,times,hyperref,nth}
\usepackage[margin=1in]{geometry}
\usepackage[inline]{enumitem}
\usepackage[tiny,compact]{titlesec}
\usepackage[citestyle=authoryear,backend=bibtex]{biblatex}
\bibliography{references}
\newlist{inlinelist}{enumerate*}{1}
\setlist*[inlinelist,1]{%
   % label=\alph*),
 label=\arabic*),
}
\pagenumbering{gobble}
\begin{document}
We sincerely thank the reviewers for the feedback they offered. Your comments have identified several opportunities for clarification and potential changes in our framing that we think will help strengthen the paper.

\section*{RISKS OF HISTORICAL ANALYSIS}
1AC \& R5 bring up that one of the potential hazards with
historical analysis is that
it tends to strip away context in pursuit of ahistoric conclusions.
This is an excellent point, and
the pointers to~\citeauthor{rosenberg1994exploring}'s works
in particular are especially beneficial;
his~\citeyear{rosenberg1982inside} and~\citeyear{rosenberg1994exploring} works
offer concise descriptions of the spirit of the method we adopted here,
and articulates a use case for this approach.
In our revision, we will clarify that:
\begin{enumerate}
  \item Our contribution %offers to fill some of the gaps in crowdwork and 
  suggests potential futures, offering
            ``to narrow our estimates
              and thus to concentrate resources
              in directions that are more likely to have useful payoffs''~(\cite{rosenberg1994exploring});
  and
  \item Our method of relating history to modern socio--technical systems may be
        a useful tool for researchers attempting
        to make sense of ostensibly new phenomena.
        In other words, offering ``that past history is an indispensable source of information
                          to anyone interested in characterizing technologies''~(\cite{rosenberg1982inside}).
\end{enumerate}


% To quote~\citeauthor{rosenberg1994exploring},
% ``Science will often provide the capability to acquire information
% \dots that we do not presently possess, but
% \textit{science does not make the acquisition of this information costless}''
%~(\cite{rosenberg1994exploring}).
% In that spirit, 
% This is absolutely an excellent point, and
% a shortcoming which we struggled to articulate in our discussion section.
% We'll add more to the discussion section to highlight the potential hazards
% of applying historical analyses due to the different
% politics, economies, cultures, and other factors which
% may influence any phenomena.

% \textit{There's almost certainly something by Rosenberg about how
%         historical analysis is flawed but not totally unhelpful.
%         I'll quote it.}

\section*{CASE STUDY FOCUS}
R5 points out that we give attention to~\citeauthor{grier2013computers}'s~(\citeyear{grier2013computers}) work and
the case of human computers,
perhaps at the expense of the other case studies.
Our goal had been to equally highlight the case studies of
the match--girls and railroad workers.
While we occasionally bring to light other cases
(such as the industrial workers during the Second World War),
% we'll attempt to bring the major case studies to
% a similar level of attention as we afforded the human computers,
% % which~\citeauthor{grier2013computers} offers~(\cite{grier2013computers}),
% making these three case studies into
% threads of peer importance, recurring through the paper.
% We can elaborate on a number of aspects of match--girls' work and
% how it relates to crowdwork under the case studies of
% decomposition and worker relations in the following ways:
% \subsection*{Complexity}
% \subsection*{Decomposition}
we'll bring more focus on these ``major'' case studies
(the match--girls and the railroad workers)
to afford them
the scrutiny human computers earned.

To use the match--girls as an example, we'll address
the reorientation of their payment for work from
time--centric to
output--centric.
This reorientation led to further discretization and decomposition,
allowing managers
to track the match--girls more minutely,
evaluate their working styles \& resultant performance,
and encourage or discourage certain behaviors much more granularly via
precise disciplinary measures.
This analysis will also
allow us to make deeper conclusions about
the implications of piecework decomposition on crowdwork.

% We can delve into the parallels between
% Airy's computers, match--girls, farm work, and railroad work in the context of
% the decomposition of work by looking at
% the compensation models which directed attention toward
% discrete, measurable output (namely, rather than remuneration by time).
% We can further flesh this topic out by highlighting
% the overarching driving force to discretize work output and
% render payment for each individual unit of output,
% in some cases (e.g. match--girls' work manufacturing individual matches)
% more effectively than others (e.g. railroad workers renewing a coupler, or
%                                    farm workers birthing a foal).
% Hypothesizing
% (and in the case of railroad work, deriving some insight from
% the critiques outlined by~\citeauthor{american1921problem}'s~(\citeyear{american1921problem}) grievances)
% from these cases, we might infer that inadequate measurement approaches
% --- particularly those which leave open the opportunity for systematized gaming ---
% stymied piecework's popularity in some cases, while in others it flourished.

% \subsection*{Relationships}

% With regard to the relationships among workers and between workers and their managers,
We'll also return to literature on the relationships
among match--girls in their nascent labor movement and
the notably adversarial relationships they had with their managers,
whose disciplinary methods later took on
punitive, arbitrary qualities.
This discussion will expose further parallels between
the internal and external relationships
of crowdworkers and pieceworkers.

% we see ample opportunity to further develop the case studies of match--girls in both
% the context of their cooperation
% (that is, how they collectively acted to initiate a workers' strike), and
% in the context of their relationships with managers
% (that is, a decidedly adversarial one, fraught with
% arbitrary and sometimes punitive payment deductions).
% Regarding railroad workers (and to a lesser extent farm workers),
% our sources gave some insight to the relationships between workers and their managers,
% but ``anthropological'' studies of the sociality of workers and
% their relationships with managers and each other was, as we conclude later,
% notably sparse relative to what we now know about crowdwork and its culture.




\section*{TOPIC SELECTION}
R3 \& R4 note that our decision to cluster crowdwork research around three questions
consolidated some research
--- for instance,
the ``quantity-quality dilemma'' (R4),
``professional development'' (R3),
and ``incentive structures'' (R3) --- into other broader topics.
This critique is well taken.
We will dedicate some space to reflect on
the decisions we made with regard to clustering research topics.

\section*{RELATED WORK}
R4 \& R5 offer a number of works
(\cite[e.g.][]{williamson2016})
for a more comprehensive discussion of scientific management.
We agree that these works will
substantively add to a reader's understanding of scientific management.
% We'll attempt to bring into discussion these works, but
% our original concern was that a satisfactory review of scientific management
% would be difficult to articulate concisely.
We'll attempt to crystallize the body of work concisely, and
point out that there's much more to be said about these topics.

\section*{ETHICS}
R4 asks whether our analysis can shed any light onto the question of
``whether it is ethical, to make use of crowdwork in HCI research''.
We had two interpretations of this question:
\begin{inlinelist}
  \item whether piecework itself is ethical, or
  \item whether research on crowdwork is ethical. 
\end{inlinelist}
We'll engage here with the former.

We will add this topic to the Discussion of our paper,
integrating the resource R4 offered~(\cite{williamson2016}).
Briefly, the literature on the history of labor does not
frame the question as ``whether piecework is inherently ethical or unethical'',
instead asking what conditions render it exploitative.

This literature we brought to bear suggests that exploitation occurs when
conditions harm workers directly or indirectly, such as
in sweatshops and agricultural work with pesticides, or where
employers systematically underpay or overwork laborers by contemporary standards.

The question then is
whether socio--technical infrastructure like Mechanical Turk (AMT)
and other marketplaces
similarly harm, underpay, or overwork workers.
AMT itself does not directly require any amount of payment or work,
but its design encourages employers to engage in such behaviors:
piecework rates, for example, undervalue workers' task search time, and
task design interfaces undeniably frame
workers as unreliable by recommending
replication with multiple workers rather than
trusting and paying individual workers more.
% let alone individuating them as people.


% We found ourselves struggling to reconcile
% how adequately to bring up some of the leading thinkers in
% scientific management
% --- and their contributions ---
% given the tight constraints.
% We can only gesture to~\textbf{Hounsell}, \textbf{Roe Smith}, and~\citeauthor{williamson2016}
% and point out that there's substantially more to scientific management than we
% can ``afford'' to communicate.
% Given this feedback, we'll attempt to carefully balance the desire
% to satisfactorily discuss leading minds in these fields with
% the limited space afforded to provide readers with context
% We struggled with this topic in the process of writing,
% as we wanted to communicate a high--level overview of some topics only as they
% were relevant to the core argument of piecework.
% We worried that bringing \textbf{Hounsell}, \textbf{Roe Smith}, and others into
% the section mentioning scientific management would have necessitated
% a much more involved discussion than
% this venue's format would afford.
% Given this feedback, we'll
% acknowledge this influential work and point out that there's
% significantly more detail and nuance to scientific management than
% we could afford to explore, given our prerogative.

% {\LARGE
% Dear reviewers,

% You da real MVPs

% Your feedback was {\Huge😎} I'm all like {\Huge🙇}

% But R5 I mean {\Huge😒}

% Nah I'm just kidding {\Huge😂} keep it {\Huge💯}!
% }

% \printbibliography{}
\end{document}