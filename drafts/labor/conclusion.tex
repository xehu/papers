\documentclass[trackingWork]{subfiles}
\makeatletter
\def\blx@maxline{77}
\makeatother

\onlyinsubfile{
\usepackage{xr-hyper}
\usepackage{hyperref}
\externaldocument{complexity}
\externaldocument{decomposition}
\externaldocument{relationships}
}

\begin{document}
\section{Conclusion}
Crowd work is not new: it is a modern instantiation of piecework.
In this paper, we reconsidered three major research questions in crowd work using the lens of piecework: \begin{inlinelist}
  \item ``\namerefl{sec:complexity}?'';
  \item ``\namerefl{sec:decomposition}?'';
        and
  \item ``\namerefl{sec:relationships}?''
\end{inlinelist}
To do so, we drew on piecework scholarship to inform analyses of what has changed, what hasn't, and may change change soon. 
% predictions and to contextualize the answers that crowd work researchers have already uncovered.
% We subsequently take this historical framing to ask broader questions of crowd work, including among others what the future of crowd work holds for workers.

\end{document}