\documentclass[trackingWork]{subfiles}
\makeatletter
\def\blx@maxline{77}
\makeatother

\onlyinsubfile{
\usepackage{xr-hyper}
\usepackage{hyperref}
\externaldocument{complexity}
\externaldocument{decomposition}
\externaldocument{relationships}
}

\begin{document}
\section{Conclusion}
Crowd work and on--demand work are not new: they are contemporary instantiations of piecework.
In this paper, we reconsider three major research questions in crowd work using the lens of piecework: \begin{inlinelist}
  \item ``\namerefl{sec:complexity}?'';
  \item ``\namerefl{sec:decomposition}?'';
        and
  \item ``\namerefl{sec:relationships}?''
\end{inlinelist}
To do so, we draw on piecework scholarship to inform analyses of what has changed, what hasn't, and may change change soon. 
% predictions and to contextualize the answers that crowd work researchers have already uncovered.
% We subsequently take this historical framing to ask broader questions of crowd work, including among others what the future of crowd work holds for workers.
Reciprocally, we believe that modern crowd work will teach us about the broader phenomenon of piecework as well.
If history really does repeat itself, the best we can do is be prepared.

\end{document}