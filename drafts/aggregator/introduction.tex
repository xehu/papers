\documentclass[main]{subfiles}

\makeatletter
\def\blx@maxline{77}
\makeatother
\begin{document}

\section{Introduction}\label{sec:introduction}

\ali{
  Components of a good intro (anything else to add?):
  \begin{itemize}
    \item state of the world
    \item what's the problem?
    \item why this problem is such a big deal(?)
    \item what's making this problem so persistent?
    \item what's the insight we bring?
    \item what we built/did
    \item what did we find?
  \end{itemize}
}

\begin{enumerate}
  % state of the world
  \item on--demand work has grown significantly
  \item task diversity is growing
  \item new platforms continue to emerge
  
  % what's wrong?
  \item workers are struggling to signal quality:
  \begin{enumerate}
     \item new workers --- cold start problem
     \item old workers --- too much noise
  \end{enumerate}
  \item requesters have problems as well:
  \begin{enumerate}
    \onlyinsubfile{
      \item did trump really quote Bane in his inaugural?
      \item omg like really
    }
    
    % why is this problem a big deal?
    \item requesters don't know whom to trust and have to bootstrap their own trustworthiness evaluations
          (e.g. qualifications exams, tentatively and very cautiously trusting coarse heuristic data)
    \item requesters tend to undervalue workers --- at least at first --- leading to a ``market for lemons''
  \end{enumerate}
  
  % Why is this so hard?
  \item a few problems frustrate the design of interventions
  \begin{enumerate}
    \item platforms have little incentive to facilitate external agents of intervention
    \item the stakeholders that platforms are more interested in seem to be those with the money (i.e. requesters)
          \ali{maybe too argumentative?}
    \item on--demand workers gravitate toward these platforms in large part because of the transience and ephemeral nature of the work
          (therefore, working against that will meet with resistance (and possibly attrition) from the very workers who constitute the platform today)
  \end{enumerate}

  % What's our insight?
  \item in ``conventional'' work, the problem of signaling a worker's quality was solved by using past worker performance as a proxy for future performance
  \item on--demand work platforms have tried to do this with approval and rating systems, but they're generally too vague to be useful
        % (on AMT, requesters limit work to those with approval ratings higher than 95\% or even 99\%, or workers with thousands or tens of thousands of tasks completed, or both)
  \item conventional work r\'{e}sum\'{e}s usually allow the worker to emphasize and omit information that's not relevant --- \textbf{what if we facilitated that?}
  
  % So what did we do?
  \item we built a system that collected data from workers' profiles on online crowdsourcing platforms
  (first AMT, then others), and parsed that data to present more meaningful analytic data on things like
  \begin{enumerate}
    \item the nature of the task; objective (i.e. factually based and evaluable) versus subjective (e.g. survey \ali{maybe also object labeling?})
    \item the type of requester; academic versus industry
    \item \textit{others}
  \end{enumerate}
  \item we also built a proof--of--concept qualifications management system abstracted from both our worker profile system and the work platform itself
  (to explore the potential to abstract worker qualifications and credentials from the system itself, allowing other parties to specialize in this task)
  
  % What did we find/how did it go?
  \item we evaluated this system from two perspectives:
  \begin{enumerate}
    \item did requesters who used this system find that the worker data better--informed their qualifications management and worker solicitation?
          in other words, was the output of the work more reliably accurate when they relied on the analytic data that we provided versus the coarse approval/rejection rating?
    \item did \textit{workers} benefit from having high level data providing some insight into their work trends? did this lead them to better work more quickly? did they take a more active role in the management of their reputation / ``worker profile'' / r\'{e}sum\'{e}?
  \end{enumerate}
  \item we also explored whether the externalized worker qualification system worked effectively, and considered the potential advantages and disadvantages of a set of decoupled systems rather than an integrated one (market, qualifications assessment system, payment platform, etc\dots).
\end{enumerate}





\ali{
On--demand work has so far been a compelling way to arrange work and imagine workers;
the fleeting nature of work seems to appeal greatly to workers, and
the looser expectations of responsibility toward workers appeals to employers and clients.
But in the exchange we've made for more fleeting relationships,
we've instantiated --- or at least added fuel to ---
another class of challenges and problems.
Among them remains the challenge of ensuring that workers
--- first crowd workers, and later gig workers --- both
\begin{inlinelist}
  \item know the skills necessary to do the immediate task and
  \item will continue to produce work at the desired level of quality.
\end{inlinelist}
By far, two approaches have taken hold:
First, qualification exams (e.g. Amazon Mechanical Turk and Upwork),
       which ask workers to take a test of some sort to demonstrate that they both
       understand the task in question \textit{have the ability} to do the task.
Second, relying on outside certification bodies (e.g. Uber, \ali{do we know of others that match this profile?}),
        which assumes that outside bodies
        --- in Uber's case, local DMVs and similar bodies ---
        have sufficiently evaluated a person's ability to do the tasks necessary.

Other approaches exist, but have not (yet) garnered the interest of on--demand firms in general.}


% Research has looked into this question from a handful of perspectives;
% some have tackled the qualifications problem as a pure assessment problem.
% others have looked to
% [\textit{Krishna et al. on long--term workers}]

\ali{I want citations on \textbf{worker qualifications} and on \textbf{worker effort and management}.
I think Ranjay's paper on looking forward etc\dots~at CSCW 2017 would be right here.
When do bibtex things come out?}



% Online labor markets ranging a variety of industries of (largely) service work
% --- e.g. on--demand drivers (Uber, Lyft, Sidecar, etc\dots) and
%     information work (Amazon Mechanical Turk, Upwork, CrowdFlower, etc\dots) ---
% have struggled for some time with pernicious, in some cases \textit{growing},
% difficulty assessing the quality of its workforce,
% especially as it applies to a given task.
% In other words, whether a worker on Amazon Mechanical Turk (AMT) is likely
% to complete a task ``accurately'' remains an open question,
% leading researchers to address this pervasive problem in a number of ways


% Labor market platforms ranging industries and styles of work
% (e.g. Uber, TaskRabbit, UpWork, and Amazon Mechanical Turk)
% have struggled for years with persistent
% (generally growing)
% challenges relating to worker qualifications.
% These issues range a wide spectrum:
% in some cases, worker qualifications are non--transferable,
% leading requesters to ``re--invent the wheel'' as they attempt to determine
% in their own way whether a potential worker is qualified and reliable.

% Problems determining a worker's qualifications start on day 1;
% labor markets begin their relationships with new workers
% almost entirely uncertain about the worker's competence in any type of task.
% Gathering this information through qualification exams is generally
% time--consuming and costly.

% Challenges mount as workers' skill sets develop;
% work requiring more training and skill (for example, translating or programming)
% are either verified by individual \textit{requesters} (e.g. Amazon Mechanical Turk)
% or are verified by the \textit{platform} itself (e.g. UpWork).
% While the \textit{UpWork} model avoids needless repetitive work by
% generally consolidating qualifying exams at the platform level,
% these labor platforms nevertheless find themselves in the unenviable
% (and often unexpected) position of
% having to develop new qualifications exams to outpace would--be cheaters.


\end{document}
