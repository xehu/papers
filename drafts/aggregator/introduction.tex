\documentclass[main]{subfiles}

\makeatletter
\def\blx@maxline{77}
\makeatother
\begin{document}

\section{Introduction}\label{sec:introduction}

On--demand work has so far been a compelling way to arrange work and imagine workers;
the fleeting nature of work seems to appeal greatly to workers, and
the looser expectations of responsibility toward workers appeals to employers and clients.
But in the exchange we've made for more fleeting relationships,
we've instantiated --- or at least added fuel to ---
another class of challenges and problems.
Among them remains the challenge of ensuring that workers
--- first crowd workers, and later gig workers --- both
\begin{inlinelist}
\item know the skills necessary to do the immediate task and
\item will continue to produce work at the desired level of quality.
\end{inlinelist}
By far, two approaches have taken hold:
First, qualification exams (e.g. Amazon Mechanical Turk and Upwork),
       which ask workers to take a test of some sort to demonstrate that they both
       understand the task in question \textit{have the ability} to do the task.
Second, relying on outside certification bodies (e.g. Uber, \ali{do we know of others that match this profile?}),
        which assumes that outside bodies
        --- in Uber's case, local DMVs and similar bodies ---
        have sufficiently evaluated a person's ability to do the tasks necessary.

Other approaches exist, but have not (yet) garnered the interest of on--demand firms in general.


% Research has looked into this question from a handful of perspectives;
% some have tackled the qualifications problem as a pure assessment problem.
% others have looked to
% [\textit{Krishna et al. on long--term workers}]

\ali{I want citations on \textbf{worker qualifications} and on \textbf{worker effort and management}.
I think Ranjay's paper on looking forward etc\dots~at CSCW 2017 would be right here.
When do bibtex things come out?}



% Online labor markets ranging a variety of industries of (largely) service work
% --- e.g. on--demand drivers (Uber, Lyft, Sidecar, etc\dots) and
%     information work (Amazon Mechanical Turk, Upwork, CrowdFlower, etc\dots) ---
% have struggled for some time with pernicious, in some cases \textit{growing},
% difficulty assessing the quality of its workforce,
% especially as it applies to a given task.
% In other words, whether a worker on Amazon Mechanical Turk (AMT) is likely
% to complete a task ``accurately'' remains an open question,
% leading researchers to address this pervasive problem in a number of ways


% Labor market platforms ranging industries and styles of work
% (e.g. Uber, TaskRabbit, UpWork, and Amazon Mechanical Turk)
% have struggled for years with persistent
% (generally growing)
% challenges relating to worker qualifications.
% These issues range a wide spectrum:
% in some cases, worker qualifications are non--transferable,
% leading requesters to ``re--invent the wheel'' as they attempt to determine
% in their own way whether a potential worker is qualified and reliable.

% Problems determining a worker's qualifications start on day 1;
% labor markets begin their relationships with new workers
% almost entirely uncertain about the worker's competence in any type of task.
% Gathering this information through qualification exams is generally
% time--consuming and costly.

% Challenges mount as workers' skill sets develop;
% work requiring more training and skill (for example, translating or programming)
% are either verified by individual \textit{requesters} (e.g. Amazon Mechanical Turk)
% or are verified by the \textit{platform} itself (e.g. UpWork).
% While the \textit{UpWork} model avoids needless repetitive work by
% generally consolidating qualifying exams at the platform level,
% these labor platforms nevertheless find themselves in the unenviable
% (and often unexpected) position of
% having to develop new qualifications exams to outpace would--be cheaters.


\end{document}
