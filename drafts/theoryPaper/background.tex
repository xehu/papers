%!TEX root = proceedings.tex
\section{Background \& Theory}

The existing knowledge from which we draw in this research can primarily be traced to two pools of research;
the first comes from the extensive body of research surrounding collective action,
and the work that has gone into describing, categorizing, predicting, and even designing to foster campaigns of collective action.
The second major source of our knowledge comes from the study of on-demand or gig markets.
By synthesizing the discoveries from these sources of research, we identify \textit{a priori} guidance at this intersection of two subjects.

\subsection{The Long History of Collective Action}
Grassroots, community-led organizations are not new;
a substantial body of literature illustrates myriad approaches to guiding communities and assisting in collective action.
Specifically, when we consider the role of insight into collective action, we refer to what Hardin describes as ``directed at an ongoing problem"
\cite{russell1982collective}.
The implication here, he argues, is that the guidance on this form of collective action is dramatically more nuanced than ``one-shot" collective action, demanding an ``anthropological investigation of minute interrelationships".
We might call this ``ongoing" collective action.
Economist Mancur Olson proposes, in part, that collective action depends on some large, generally inactive group in order to succeed
\cite{olsonlogic}.

After decades of observations, Hardin posits that collective action is too commonplace for Olson's thesis to hold;
He suggests that the requirements for collective action which Olson theorizes may have changed
--- specifically, lowering the threshold ---
as a result of myriad factors outside of the scope of this research, except to point out that recent work in online collective action prompts further scrutiny of Olson's thesis and the critiques later researchers have levied.
We suggest an alternative consideration: that much of the research in collective action in the space of HCI in fact corroborates the latent community requirement Olson recommends.
Myriad collective action endeavors seem to succeed in part because they precipitate a collective of latent, willing participants in some form of community action
\cite{catalyst,dynamo,foundry}.

% Given this interpretation of successful cases of collective action, we infer that designing a system specifically with the precipitation of a crowd as future participants may be a requisite factor in successful collective action.

\subsection{On-Demand Markets and Their Workers}

A robust and growing body of research exploring collective action and movements enabled by the Internet adds to a body of knowledge previously uninformed by the tools the Internet affords.
Where collective action research coordinated and executed offline describes challenges symptomatic of social structures,
researchers of online communities can and do offer design guidance for the structure of online communities
\cite{Hirsch:2009:FLA:1516016.1516024}.
Substantial contributions deeply investigating online communities, such as Wikipedia, have lent system designers guidance in designing communities geared toward some ongoing collective action online
\cite{Nov:2007:MW:1297797.1297798,wikipediansBornNotMade,whyWikipedians}.
In this last case, studies of Wikipedia and its users
--- known colloquially as ``Wikipedians" ---
is especially instructive, as it addresses the distinction Olson makes regarding ``one-shot" collective action and what we will call ``ongoing" collective action.

Researchers have described the potential of crowd work, and perhaps more importantly outlined various problems with said markets
\cite{crowdworkFuture}.
More recently, researchers explored the impact of the sharing economy among socioeconomically disadvantaged communities,
and studied norms and behaviors among the workers of individual markets like Uber
\cite{DillahuntPromise,uberAlgorithm}.
The directives offered by Kittur et al. provide substantial guidance in the conceptual design of a ``crowd market",
but the direct application of this research to a communally led organization
--- a worker-run labor market --- merits further consideration.
% but the broader transformative effects of the sharing economy passed scrutiny relatively quietly.


The focus of technologically enabled peer economies is well-explored;
researchers have pointedly identified the efforts of both for and non-profit online peer markets and the challenges workers and market operators face
\cite{uberAlgorithm,Bellotti:2014:TCS:2611247.2557061,Thebault-Spieker:2015:ASS:2675133.2675278}.
Some research has implemented independent micro-work markets for their own purposes
\cite{Alt:2010:LCE:1868914.1868921},
though the purpose of this research was not to explore the idea of a worker-centric or worker-led market.

Computational social scientists have documented workers' efforts to circumvent the systems imposed on them by market operators
\cite{uberAlgorithm},
and more directly researchers have observed the continuing effort to resist and critique markets for ``gig work", in these cases in the context of online labor, where micro-work on Amazon Mechanical Turk (AMT) predates offline gig work companies such as Uber
\cite{turkopticon,dynamo}.

Nevertheless, frustrations with these marketplaces persist,
and the trends among emerging marketplaces seem to commoditize workers more and more aggressively.
These ``patches" of existing markets appear to have only marginal effects on the qualities of these markets;
Uber drivers continue to resist the algorithmic matching while walking a fine line to avoid retribution from management, and some of the most frustrating requesters on AMT continue to antagonize Turkers.

\subsection{Juxtaposed, but not Synthesized}
Broadly, these bodies of research have overlapped in limited cases
\cite{dynamo,turkopticon}.
The intersection of ``community-driven action as a mode of designing" \& ``design of technologically-enabled peer markets" represents a field site ready for the application of existing knowledge, and perhaps the production of new.
We begin to bridge this research by applying our learnings regarding collective action (both offline and online) to what we have learned from research surrounding the ``gig economy" in its various names.
Drawing on the spirit of critical theory research and the grievances found among gig workers \cite{turkopticon,uberAlgorithm}, we began our fieldwork thinking about the viability of a self-directed crowd, informed perhaps by the 