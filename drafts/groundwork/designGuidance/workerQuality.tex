%!TEX root = ../proceedings.tex
% \subsection{Worker Quality \& Ratings}
\textit{Groups use social pressure to encourage good actors or discourage bad actors \& ensure high quality, but rarely both.}

% \textit{there should be a whole subsection where I talk about how stressful quantitative ratings are, and I can even reference some research on eBay seller profiles and how useless they are}

Another issue that arose during fieldwork surrounded the verification that a worker was of high quality.
To reference existing platforms again and using Uber and Lyft in this case, the quality of a worker can be illustrated simplistically by the quantitative ratings that drivers often have.
The distinction between qualification and quality is important;
to use driving as an example once again, a driver may be qualified (typically, licensed to drive \& insured) but not high quality
--- for example, struggling to navigate in tricky roads, unfamiliar with the area, etc\dots

We found two primary ways of communicating worker quality:
\begin{itemize} \itemsep0pt \parskip0pt
  \item guaranteeing outcomes
  \item vouching for good work
\end{itemize}

At organizations with high barriers to membership, work can be guaranteed by the organization of workers.
This guarantee gives customers relative confidence that any issues will be rectified at a cost fully absorbed by the organization, rather than the customer.
To cover these costs, work groups generally tax workers for each job to form a ``rainy day fund" in anticipation of such an expense.

In the case of organization A, if work is not done satisfactorily well, organization A's guarantee involves paying for a new worker to do the job correctly at the organization's expense.
Workers whose work requires fixing are punished in various ways depending on the nature of the error and whether the worker has done inadequate work in the recent past;
workers essentially face increasingly severe punishments as they repeatedly make mistakes (or make mistakes of greater cost).

It's important to note that these organizations can make this guarantee for two reasons.
The first reason is that membership exposes workers to work opportunities that make attempting to gain membership worthwhile.
The second reason is that membership eligibility is non-trivial to achieve, and transgressing community norms and risking expulsion carries significant consequences, especially given the work opportunities that they would be jeopardizing.

Organization E addressed this challenge using community pressure fostered by shared emotional investment in the organization.
Through democratic administrative systems and frequent meetings mandatory for all active workers, a sense of communal buy-in seemed to take form among workers who felt that their failure to do good work would let down the rest of their peers.
Organization E further stoked this sense of community investment by reading feedback from customers, made anonymous by organization E, to the group during all-hands meetings.
Workers reportedly felt a sense of shared success when good feedback arrived, and similarly felt a shared failure when feedback was critical.

We discovered that quantitative ratings weigh heavily on workers,
causing anxiety over arbitrary and opaque feedback negatively affecting their eligibility as workers.
Platforms like Uber and Lyft warn their workers that a user rating below a certain threshold
(for example, rolling average under 4.8 on a 1-5 scale) may result in disciplinary action.
Drivers, for instance, told us of courses they had to take if the average of their ratings fell below 4.6/5.

We can --- and will --- discuss ways to make quantitative rating systems less damaging in practice,
but we find that numeric ratings overwhelmingly tend to harm workers;
feedback --- rather than ratings --- proves a more effective way to improve workers
\cite{numericCritique},
and customers' responses to negative ratings seem to be exaggerated
\cite{ebayRatings}.
Given these apparent weaknesses in numeric ratings, we propose that efforts to communicate reputation should attempt to maximize qualitative data, rather than minimize it as many systems (e.g. Uber, Lyft, etc\dots) do.

This sense of shared reputation and the desire among individuals to uphold it was reportedly very strong.
We discovered that workers, fearing their refusal to do certain kinds of work would reflect poorly on organization E, would agree to do work even when it was inappropriate to do so.
Examples of impropriety include using unfamiliar mechanical equipment and working in hazardous settings with inadequate protection.
In light of this, the organization has had to remind workers to take a firm stance when customers make unreasonable requests of them.

Whether worker motivation stemmed genuinely from investment in their community
or from a more practical desire not to lose the job they've gotten that day is unclear.
However, considering the high demand we found during our fieldwork
we suspect that this practical desire was not significantly influential.
In other words,
workers could turn down work and confidently expect to be offered another worthwhile job,
if they were only concerned with earning money.