%!TEX root = ../proceedings.tex
% \subsection{Worker identity versus ``gig workers"}
\textit{The very features that make gig work appealing also stymie attempts to coalesce stable worker advocacy groups.}

At organization E, organization A, organization C, and organization B we observed a subtle but important characteristic;
everyone at these organizations identified primarily by the work that they did.
Members of organization E self-identified as house cleaners and day laborers;
People at organization A were workers, first and foremost.
The same fundamental sentiment was shared by nurses affiliated with organization B.

We found that this carried an important effect;
at organization E the effect was most salient that workers identified together and recognized the importance of maintaining organization E's positive reputation among customers.
Every week, organization E administrative staff would read reviews of workers,
made anonymous beforehand,
allowing the workers to reflect on these frequent successes and occasional failures.

Perhaps most surprising was a shared sense of success and failure when workers listened to this feedback;
people we spoke to felt that they had upheld or let down their peers and the organization of which they were part,
even if the review was not of their own work.
This strong sense of group identity seemed to drive workers to work hard to maintain the already good reputation that organization E enjoyed.

When we spoke to drivers in the gig economy, some told us that they had previously worked for competing yellow cab companies.
They explained that they gave up on conventional cab companies was because requirements to drive for yellow cabs felt onerous.
Drivers worked for Uber and Lyft because those markets allowed them to drive as much or, importantly, as little as they wanted.
At some yellow cab companies, a monthly fee was required for access to the cabs themselves.

This presents a potential challenge for those hoping to build communities fostering collective decision-making and action.
As Kraut et al. point out, a crucial component in designing successful online communities is identification with the group
\cite{successfulOnlineCommunities}.
This is already a difficult goal to achieve, now exacerbated by the nature of the community involved.
Features that makes gig work appealing
--- the freedom to come and go as one pleases, for instance ---
make it that much more difficult to form a shared sense of community based on shared identities.

Indeed, workers in the gig economy tended not to think of themselves as identified by the work that they do ---
many drivers, for instance, volunteered that they were musicians, security guards, and even elementary school teachers.
Their work as drivers, then, was not substantively important to who they are.
Workers at organization E, meanwhile, identified as cleaners, day laborers, etc\dots without qualification.

Among many of the ``gig workers" that we spoke to, the sense of freedom and independence turned out to be an important feature of their identity.
The relative freedom over when and indeed whether to work was, it seemed, powerfully appealing.

This represents a difficult dilemma for labor unions and other conventional worker advocacy groups,
which rely on the shared sense of identity of workers,
who are now increasingly detaching their identities from the work that they do.

We assume that a worker-centric market needs emotional investment from its participants in order to succeed,
and the evidence seems to corroborate this sense \cite{dynamo,olsonlogic,russell1982collective}.
Given this premise, it seems that workers in a system must appreciate their reliance on their group's collective success.
Balancing this need with the desire that workers have expressed to remain unburdened by various requirements might prove to be a difficult act, but a necessary one.

We have little concrete guidance for this challenge
except to to say that system designers might strive to find ways to make it palatable for gig workers to identify collectively
and to recognize that their individual success is collectively determined by the quality of all of their work.
By forming a sense of shared identity and investment in that community,
a constructive sort of self-policing might emerge, as briefly discussed by Lave \& Wenger \cite{lave1998communities}.