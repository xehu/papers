%!TEX root = proceedings.tex
\section{Discussion}
We spent several months working alongside workers in various industries,
iterating on mockups and prototypes,
and getting feedback from members and administrators of worker-advocacy groups like organization C, organization B, and organization D.
With their insights, we identified a number of aspects of gig work that potentially marginalize workers,
and in doing so we begin to articulate ways to avoid such an outcome.
These potentially marginalizing aspects in the peer economy consist of the following:

\begin{enumerate}\itemsep0pt \parskip0pt
  \item Constructive feedback
  \item Assigning work fairly
  \item Managing customer expectations
  \item Protecting vulnerable workers
  \item Reconciling worker identities
  \item Assessing worker qualifications
  \item Communicating worker quality
  % \item providing workers with substantive constructive feedback,
  % \item fairly assigning work in groups
  % \item managing customer expectations
  % \item protecting vulnerable workers
  % \item respecting worker identity
  % \item assessing worker qualifications
  % \item communicating worker quality
\end{enumerate}

We discussed the importance of such a system given the existing trends in technologically enabled labor markets,
and suggest an alternative market design wherein workers might operate their marketplace collectively.
Finally, informed by ethnographic fieldwork and interviews, we distill what we believe conscientious system-builders would need to create a worker-centric peer economy.

It's worth noting that while we worked with workers from various industries,
seeking to find common threads between each community,
we can't speak to the applicability of these findings across all conceivable industries and modes of work.
Information workers such as those on Amazon Mechanical Turk may differ in significant,
fundamental ways compared to drivers-for-hire such as Uber drivers.

Above all, we hope to convey the potential insight one can gain by working with stakeholders,
and the importance of collaboratively designing labor markets with the workers themselves.

As Kittur et al. point out, as researchers, computer scientists, and participants in the technological community we ask whether
\textit{``\dots we foresee a future crowd workplace in which we would want our children to participate''} \cite{crowdworkFuture}.
Indeed, we must answer this question, by articulating an economy that empowers and respects workers, rather than marginalizes and exploits them.