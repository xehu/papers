%!TEX root = proceedings.tex
\section{Method \& Position}
We approach this research with some amount of reflexivity, as described by Geertz and elaborated by others
\cite{geertz1988works,marcus1999anthropology},
to acknowledge the role we play as participants in the culture we study.
Reflexivity affords other benefits, however;
it allows us to avoid an undue and perhaps hopeless struggle to separate ourselves and objectify the communities we studied, and engage with participants more transparently, rather than attempting to abandon our preconceptions and existing mental models.
% I want to make a brief reference to reflexivity as an asset for social scientists, but if the argument is too opaque here then the answer is to make that point more fully so it's clearer, but that's not really a good option, so...

% We learned several constructive lessons in our fieldwork, especially relevant to a researcher whose intent is to design and implement a system for the members of a given community.
% do I list the subsections here? I feel like that would be a lot clearer, but I'm thinking maybe I should just dive into it

\subsection{Finding Participants, Making Partners}
Our process began by finding organizations with demonstrated success in areas such as worker advocacy and collective action, hoping that their experience in offline work would inform online working groups.
We found a variety of organizations ranging small worker cooperatives to national-scale labor unions, and began to make contact with several.
After several critical introductions by mutual contacts, we had a diverse range of partners ranging in size and spanning several industries.

We worked with 5 organizations: Organization A, organization B, organization C, organization D, and organization E.
Organization A is a labor union that spans the nation, although we specifically worked with a union chapter local to our area;
workers in this group are regulated by state laws describing qualifications and eligibility to work.
This represented one of the more formalized, organized groups that we worked with.
Organization B represents more than one million workers across various industries.
The specific vertical of work we studied involved health care, and the regulations involved with this group was even greater than with organization A.
Organization C is a national labor union and worker advocacy group which supports tens of thousands of workers, many of whom women and of minority groups.
We worked with a local group that organization C supports named organization E, which helps domestic workers and house cleaners find work our approximate geographic area.
Finally, we worked briefly with organization D,
a labor union that broadly advocates on behalf of many organizations of workers and expressed interest in representing existing workers in the ``gig economy'',
such as Uber and Lyft drivers.

% We spent more than three months making initial contact, introducing our backgrounds and motivations, and working with these groups.
With each group, we explained our interest and motivation as researchers exploring the idea of a worker-centric labor market;
we also attempted to describe what we hoped to contribute to this space.
At the time, we imagined a ``worker cooperative"
--- a worker-owned organization that often makes decisions democratically ---
and we found surprising (at the time) resistance from some worker-led organizations with which we spoke.
But this presented us with opportunities to learn about the perceived shortcomings of various organization structures.

As our interviews and discussions progressed, we mocked up designs based on what we learned.
By bringing these mock-ups to subsequent meetings, and iterating quickly on the feedback workers provided us,
we were able to demonstrate in our own ways as computer scientists and designers that we were both listening to and learning from their input.
This process, time-consuming and labor-intensive though it was, made important strides to prove the overarching claim that we were invested in these groups and respected the guidance they provided as equal partners.

Mock-ups offered another benefit that may be uniquely accessible to designers.
As we found ourselves struggling to explain concepts such as privacy and trustworthiness in tangible ways, and thus failing to get substantive answers to questions around these issues, we realized that we could illustrate various paradigms, extrapolate on them, and learn how workers felt about design choices through their preference and especially from their suggestions for revision.

We ultimately spent more than three months making contact with, and learning from, these groups.
For several weeks (admittedly fleeting, in the context of ethnographic fieldwork),
we learned through participant-observation:
answering phone dispatch requests and triaging the worker assignment system when interactions mediated by existing technologies broke down.
We gained significantly deeper insights about workers and customers as a result of this time,
but engaging in the uncomfortable, time-intensive, and tiring work of liaising with customers and triaging worker dispatch proved our level of commitment as greater in scope than a short-lived research project.