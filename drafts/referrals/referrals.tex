\documentclass{sigchi}

\usepackage{todonotes,txfonts,balance,graphics,color}
\usepackage{booktabs,textcomp,microtype,ccicons}
% \usepackage{babel}
% \usepackage{csquotes}
\usepackage[citestyle=numeric,backend=bibtex,bibencoding=ascii]{biblatex}
\usepackage{enumitem}
\usepackage[T1]{fontenc}
\usepackage[super]{nth}
\usepackage[pdftex,pdfpagelabels=false]{hyperref}
\usepackage[all]{hypcap}  % Fixes bug in hyperref caption linking
\usepackage[utf8]{inputenc} % for a UTF8 editor only

% Paper metadata (use plain text, for PDF inclusion and later
% re-using, if desired).  Use \emtpyauthor when submitting for review
% so you remain anonymous.

\newlist{inlinelist}{enumerate*}{1}
\setlist*[inlinelist,1]{%
  label=\arabic*),
}


\def\plaintitle{Naming Things is Hard: Real Title Following Colon}
\def\plainauthor{All the people (Ali, Margaret, MSB, who else?)}
\def\emptyauthor{}
\def\plainkeywords{Please don't make me pick keywords.
This is like asking a teacher to give the bullet points of
what a student missed in lecture.}
\def\plaingeneralterms{Documentation, Standardization}

% llt: Define a global style for URLs, rather that the default one
\makeatletter
\def\url@leostyle{%
  \@ifundefined{selectfont}{
    \def\UrlFont{\sf}
  }{
    \def\UrlFont{\small\bf\ttfamily}
  }}
\makeatother
\urlstyle{leo}

% To make various LaTeX processors do the right thing with page size.
\def\pprw{8.5in}
\def\pprh{11in}
\special{papersize=\pprw,\pprh}
\setlength{\paperwidth}{\pprw}
\setlength{\paperheight}{\pprh}
\setlength{\pdfpagewidth}{\pprw}
\setlength{\pdfpageheight}{\pprh}

% Make sure hyperref comes last of your loaded packages, to give it a
% fighting chance of not being over-written, since its job is to
% redefine many LaTeX commands.
\definecolor{linkColor}{RGB}{6,125,233}
\hypersetup{%
  pdftitle={\plaintitle},
% Use \plainauthor for final version.
%  pdfauthor={\plainauthor},
  pdfauthor={\emptyauthor},
  pdfkeywords={\plainkeywords},
  bookmarksnumbered,
  pdfstartview={FitH},
  colorlinks,
  citecolor=black,
  filecolor=black,
  linkcolor=black,
  urlcolor=linkColor,
  breaklinks=true,
  hypertexnames=false
}

% create a shortcut to typeset table headings
% \newcommand\tabhead[1]{\small\textbf{#1}}
\bibliography{references}

% End of preamble. Here it comes the document.
\begin{document}
\balance{}
\title{\plaintitle}

\numberofauthors{3}
\author{%
  \alignauthor{Leave Authors Anonymous\\
    \affaddr{for Submission}\\
    \affaddr{City, Country}\\
    \email{e-mail address}}\\
  \alignauthor{Leave Authors Anonymous\\
    \affaddr{for Submission}\\
    \affaddr{City, Country}\\
    \email{e-mail address}}\\
  \alignauthor{Leave Authors Anonymous\\
    \affaddr{for Submission}\\
    \affaddr{City, Country}\\
    \email{e-mail address}}\\
}

\maketitle

\begin{abstract}
  This paper is going to be about giving workers an opportunity to refer other workers.
  Some of the old literature on job referring suggests that
  it's a good way for networked workers to do stuff.
  \citeauthor{crowdcollab} found that \citetitle{crowdcollab},
  so why not take advantage of that?

  We'll try various incentive structures
  (no penalty/no rejection, task approval, referral commission, and referral bonus)
\end{abstract}

\category{H.5.m.}{Information Interfaces and Presentation
  (e.g. HCI)}{Miscellaneous} \category{See
  \url{http://acm.org/about/class/1998/} for the full list of ACM
  classifiers. This section is required.}{}{}

\keywords{\plainkeywords}
\section{Introduction}






\printbibliography


\end{document}

%%% Local Variables:
%%% mode: latex
%%% TeX-master: t
%%% End:
