\documentclass[11pt]{article}
\usepackage{balance,graphics,setspace,parskip,sourceserifpro,hyperref,titling,microtype}
\usepackage[margin=1in]{geometry}
\usepackage[inline]{enumitem}
\usepackage[compact]{titlesec}
\usepackage[citestyle=numeric,backend=biber]{biblatex}
\bibliography{references}
\makeatletter
\renewcommand{\maketitle}{\bgroup\setlength{\parindent}{0pt}
\begin{flushleft}
  \textbf{\LARGE{\@title}}

  \@author
\end{flushleft}\egroup
}
\makeatother
\title{Externalizing Worker Qualifications}
\author{Ali Alkhatib}

\begin{document}
\maketitle{}

\section*{Background}
Labor market platforms ranging industries and styles of work
(e.g. Uber, TaskRabbit, UpWork, and Amazon Mechanical Turk)
have struggled for years with persistent and
in some cases growing
challenges relating to worker qualifications.
These issues range a wide spectrum:
in some cases, worker qualifications are non--transferable,
leading requesters to ``re--invent the wheel'' as they attempt to determine
in their own way whether a potential worker is qualified and reliable.

Problems determining a worker's qualifications start on day 1;
labor markets begin their relationships with new workers
almost entirely uncertain about the worker's competence in any type of task.
Gathering this information through qualification exams is generally
time--consuming and costly.

Challenges mount as workers' skill sets develop;
work requiring more training and skill (for example, translating or programming)
are either verified by individual \textbf{requesters} (e.g. Amazon Mechanical Turk)
or are verified by the \textbf{platform} itself (e.g. UpWork).
While the \textbf{UpWork} model avoids needless repetitive work by
generally consolidating qualifying exams at the platform level,
these labor platforms nevertheless find themselves in the unenviable
(and often unexpected) position of
having to develop new qualifications exams to outpace would--be cheaters.



% \textbf{I want to communicate something along the lines of how
% these labor markets shouldn't be expected
% to handle these problems very well, because
% they've been focusing on sparking a form of commerce that hasn't existed before,
% etc\dots
% but I feel like this sounds too much like
% ``It's no wonder you all have failed!'' and
% I feel like there's no way to dig myself out of that hole.
% Thoughts?}


\section*{A (Potential) Solution}
In offline labor markets involving skilled workers,
credentials are sometimes managed by
external, trusted organizations:
the state requires electricians
to serve in apprenticeships and
pass licensing exams;
lawyers take exams administered by the American Bar Association (ABA);
doctors take similar exams given by their own oversight organization (the NBME).

This approach may prove useful in
alleviating the burden online labor markets are increasingly taking on.
By externalizing worker qualifications, a number of benefits can emerge:
\begin{enumerate}
  \item New workers can provide some evidence of their work history to labor platforms,
        mitigating or even resolving the ``cold start'' problem of
        not knowing the trustworthiness or competence of a new user.
  \item Aggregated (even heterogeneous)
        ratings sourced from differing labor markets can provide
        a more holistic picture of a worker's areas of competence.
        Workers can use this information to identify specialization more easily,
        reducing wasted time \textbf{searching for} or \textbf{working on} suboptimal tasks.
        Requesters can benefit from this information by
        making more informed decisions about which worker to contract.
  \item By turning the reputation management of workers into an external entity
        (agnostic to labor platform),
        designers and developers can focus more fully on the marketplace itself.
\end{enumerate}

% Rather than invest time and effort into worker reputation systems,
% qualifications exams, and all of the overhead associated with these challenges,
% labor platforms could
% externalize worker reputations and qualifications to other entities.
% Most labor markets 

\section*{Measurable Variables}


\section*{First Steps}

% \printbibliography{}
\end{document}