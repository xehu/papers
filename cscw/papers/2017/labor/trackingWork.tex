\documentclass{sigchi}

\usepackage{todonotes,txfonts,balance,graphics,color}
\usepackage{booktabs,textcomp,microtype,ccicons}
% \usepackage{babel}
% \usepackage{csquotes}
\usepackage[citestyle=numeric,backend=bibtex,bibencoding=ascii]{biblatex}
\usepackage{enumitem}
\usepackage[T1]{fontenc}
\usepackage[super]{nth}
\usepackage[pdftex,pdfpagelabels=false]{hyperref}
\usepackage[all]{hypcap}  % Fixes bug in hyperref caption linking
\usepackage[utf8]{inputenc} % for a UTF8 editor only

% Paper metadata (use plain text, for PDF inclusion and later
% re-using, if desired).  Use \emtpyauthor when submitting for review
% so you remain anonymous.

\newlist{inlinelist}{enumerate*}{1}
\setlist*[inlinelist,1]{%
  label=\arabic*),
}


\def\plaintitle{Naming Things is Hard: Real Title Following Colon}
\def\plainauthor{All the people (Ali, Margaret, MSB, who else?)}
\def\emptyauthor{}
\def\plainkeywords{Please don't make me pick keywords.
This is like asking a teacher to give the bullet points of
what a student missed in lecture.}
\def\plaingeneralterms{Documentation, Standardization}

% llt: Define a global style for URLs, rather that the default one
\makeatletter
\def\url@leostyle{%
  \@ifundefined{selectfont}{
    \def\UrlFont{\sf}
  }{
    \def\UrlFont{\small\bf\ttfamily}
  }}
\makeatother
\urlstyle{leo}

% To make various LaTeX processors do the right thing with page size.
\def\pprw{8.5in}
\def\pprh{11in}
\special{papersize=\pprw,\pprh}
\setlength{\paperwidth}{\pprw}
\setlength{\paperheight}{\pprh}
\setlength{\pdfpagewidth}{\pprw}
\setlength{\pdfpageheight}{\pprh}

% Make sure hyperref comes last of your loaded packages, to give it a
% fighting chance of not being over-written, since its job is to
% redefine many LaTeX commands.
\definecolor{linkColor}{RGB}{6,125,233}
\hypersetup{%
  pdftitle={\plaintitle},
% Use \plainauthor for final version.
%  pdfauthor={\plainauthor},
  pdfauthor={\emptyauthor},
  pdfkeywords={\plainkeywords},
  bookmarksnumbered,
  pdfstartview={FitH},
  colorlinks,
  citecolor=black,
  filecolor=black,
  linkcolor=black,
  urlcolor=linkColor,
  breaklinks=true,
  hypertexnames=false
}

% create a shortcut to typeset table headings
% \newcommand\tabhead[1]{\small\textbf{#1}}
\bibliography{../../../../references}

% End of preamble. Here it comes the document.
\begin{document}
\balance{}
\title{\plaintitle}

\numberofauthors{3}
\author{%
  \alignauthor{Leave Authors Anonymous\\
    \affaddr{for Submission}\\
    \affaddr{City, Country}\\
    \email{e-mail address}}\\
  \alignauthor{Leave Authors Anonymous\\
    \affaddr{for Submission}\\
    \affaddr{City, Country}\\
    \email{e-mail address}}\\
  \alignauthor{Leave Authors Anonymous\\
    \affaddr{for Submission}\\
    \affaddr{City, Country}\\
    \email{e-mail address}}\\
}

\maketitle

\begin{abstract}
  With growing attention on gig work
  --- ranging from the ``sharing economy'' to microtasks ---
  scholars have made connections to frameworks like Taylorism,
  and mechanisms such as worker advocacy and empowerment,
  to make sense of our observations of
  on--demand work and the workers that power this movement.
  We argue that our the underlying trend towards ``piecework''
  --- driven in part by the discretization,
  routinization,
  and external management of said work ---
  not only suggests, but in fact generates what we have observed:
  members of this transient workforce increasingly feeling
  disempowered,
  marginalized,
  and frustrated by the systems and platforms on which they work.
  
  After evaluating this framing
  through a series of case studies
  in various industries falling broadly under the ``gig work'' category,
  we turn our theoretical lens to look to the future,
  to identify worthwhile questions and
  points of inquiry that researchers in social computing should consider
  as we attempt to anticipate and perhaps shape the future of work.
\end{abstract}

\category{H.5.m.}{Information Interfaces and Presentation
  (e.g. HCI)}{Miscellaneous} \category{See
  \url{http://acm.org/about/class/1998/} for the full list of ACM
  classifiers. This section is required.}{}{}

\keywords{\plainkeywords}

\section{Introduction}

% \itshape
% What am I trying to say here?
% I want to get across that researchers have looked at gig work in various ways,
% but not come up with very persuasive overarching explanations for this whole process.
% That is, there's no broader theory making sense of why all of this is happening.

% Rather than make the argument that
% these things are all independently similar to the piecework
% (something to which researchers have already made the connection),
% I want to frame this as a huge process, involving:
% \begin{enumerate}
%   \item distribution of work,
%   \textbf{which leads to\dots}
%   \item discretization and routinization of work,
%   \textbf{which allows\dots}
%   \item management (scientific \& external),
%   \textbf{which causes workers to feel\dots}
%   \item alienation from work (Marx's idea) and resistance to management (also Marx probably but a more general labor advocacy theory too)
% \end{enumerate}
% okay, let's try this:
% \upshape

The past decade has seen
microtasks, the ``sharing economy'', and other instantiations of on--demand contract work
grow to occupy the fascination of both academic circles and our culture as a whole
\cite{Ross,CrowdsourcingUserStudies,scholz2012digital}.
The research community has identified myriad connections between this emergent form of work
and the historically situated ``piecework'',
as well as a number of observations relating to
the frustration workers feel stemming from the management of this work
\cite{turkopticon,dynamo,uberAlgorithm}.
This body of work represents the cornerstones of how we understand crowd workers and on--demand labor more generally.

While much of this work appears to gesture toward
the parallels between contemporary on--demand work and piecework,
it's proven difficult to bring the totality of these observations into focus using one theoretical lens.
This paper will attempt to do so by arguing that
the topics social computing researchers have investigated are
not just parallel to historical piecework and
the process of factorization that took place in the early \nth{20} century;
indeed, these phenomena are directly related, and that they precipitate
the social outcomes we have observed,
especially with regard to worker frustration, coordination, and resistance.

% This section should outline the trend of contemporary gig work in the last 10 years,
% in particular tracking the changes since
% \citeauthor{crowdworkFuture} discussed
% \citetitle{crowdworkFuture}
% in \citeyear{crowdworkFuture}
% \cite{crowdworkFuture},.
% In particular, we point to \citeauthor{crowdworkFuture}'s discussion of worthwhile
% questions to ask regarding \citetitle{crowdworkFuture}
% \cite{crowdworkFuture}.
% This would review the literature since then,
% then point to the movements toward worker unionization, advocacy, etc\dots
% that have happened in the last year or two.

In the reflection on the literature published in the last 5 years since
\citeauthor{crowdworkFuture}'s 
\citetitle{crowdworkFuture},
we notice a broader trend describing the change in work that's being done
% motivated by this realization,
% we look at the emergent phenomena in piecework both as interconnected
\cite{crowdworkFuture}.
Crowd--powered work is not just distributed across homes,
but dispersed among drivers, cleaners and other on--demand workers.
By looking at the movements toward
distributing work,
routinizing and breaking down tasks,
and externally managing workers
% discretization,
% routinization,
% and management
as linked to one another, 
we give ourselves a framing of contemporary piecework that explains,
and arguably predicts,
what we have seen thus far
--- and perhaps what we should expect to see going forward.


% The work involved often meant sewing together pieces of denim,
% was largely done by women, and turned 

Crowd work,
they argued in \citeyear{crowdworkFuture},
represents a threat to the future of work inasmuch as it marginalized
and perhaps even harmed
workers;
their driving question, then, was
whether the future of
(crowd, but arguably all discretized)
work might be one in which they wish their children to participate someday;
it seems strikingly like one that might have been asked
as patches of denim were first being delivered to the homes of early pieceworkers.

This paper will argue that the familiarity is more than passing;
it was an inevitable milestone in the maturation of the type of work that is reemerging today.
``But'',
as \citeauthor{scholz2012digital} points out,
``it would be wrong to conclude that
in the realm of digital labor there is nothing new under the sun''
\cite{scholz2012digital}.
Indeed, we will attempt to reason about more than the larger--scale progression from
the discretization of work toward scientific management and future steps;
we will attempt to make sense of how the modern processes in digital labor,
in particular the use of the Internet as a medium to manage as well as \textit{do} work,
affect these unfolding events.


% we engaged in ethnographic fieldwork
% to attempt to understand why gig work appeals to those that engage in it,
% and to gain a better understanding of how they relate to the marketplaces in which they work.


% \textit{I'd also like to bring up piecework here,
% and briefly make the case that
% the kind of work we're seeing should be thought of as a continuation
% or resurgence
% of piecework, with some important distinctions that differentiate it.}

\subsection{Piecework as a lens to understand gig work}
% It's not new to discuss gig work as an extension or even a resurgence of piecework
% \cite{crowdworkFuture,fixingChaos}.
% Nevertheless,
% it's sufficiently core to the argument here that a review is worthwhile.
\citeauthor{crowdworkFuture}
investigate the future of crowd work by
situating and interrogating it through ``piecework'',
a term almost lost to history, but which for a time described
work done in the home,
in manageable tasks,
often involving clear instructions
and payment only for work completed, not work done
(the differentiation, here, being that
one would be paid for the \textit{output} of the work,
not the \textit{duration})
\cite{crowdworkFuture}.
This work was largely in textiles and was predominantly done by women,
which by itself is a topic worthy of unpacking
\cite{scott1975women}.
Given the scope, 
we'll leave this topic with the superficial takeaway that
piecework and microtasks share some structural similarities:
\begin{inlinelist}
\item the work is done in the home;
\item the worker is paid for each discrete piece of work done, regardless of time or effort; and
\item the worker's status
(not only socially, but also economically)
is ambiguous, or at least the subject of some controversy.
\end{inlinelist}
It may be said that
on--demand work shares, if not inspiration, then many of the same outcomes as piecework.


\subsection{Context}
Importantly, since the reemergence of piecework we've seen substantial frustration
and resistance among the workers in this area
\cite{uberAlgorithm,turkopticon,dynamo}.
This paper attempts to make sense of the broader research on this piecework,
or ``gig work'', by framing this as one of several steps in the marginalization of workers,
starting with the discretization of tasks,
followed by routinization and the rise of workflows,
and finally the external management of workers.
All of this is to say that these milestones follow sequentially,
not coincidentally but necessarily,
and that by tracing this path using the corpus of scholarship on labor and workers
we can both make sense of past events and perhaps reasonably predict next steps.

We explore each topic
--- discretization, routinization, \& management ---
by looking at case studies in social computing.
Having validated this lens as a way of reasoning about contemporary piecework, or ``gig work'',
we turn to look ahead, envisioning future areas that researchers in social computing
--- and particularly digitally mediated work ---
should explore.

\section{Case Studies}
The existing body of research has shed light on on--demand from various perspectives,
and revealed a number of topics that,
through our framing, are clearly situated together.
Those topics, at a very high level, are
\begin{enumerate}
\item the \textbf{distribution} of work,
\item the \textbf{routinization} and \textbf{discretization} of work, and finally
\item the \textbf{relationships} between the workers and the product of their work,
      as well as workers and managers.
\end{enumerate}

We will follow these topics, using them as prompts for case studies in
the emergence and
development of contemporary piecework.

\subsection{Distributing Work}
In both the historical and contemporary cases,
distributed, speculative work
has promised
to diffuse the costs of capital--intensive resources
(whether they were sewing machines or laboratory equipment).

\textit{There's a reference to something that's \textbf{not} 99designs,
that describes a call for scientific solutions/designs to solve some problem
that the requesters have.
Honestly it's killing me that I can't remember the name,
and I need to find this before I do anything productive ever again.}

The HCI community is perhaps most familiar with examples such as
99designs,
Uber, Lyft,
and Amazon Mechanical Turk (AMT),
which all allow requesters in various forms to tap into
resources such as
cars,
computers, and above all
``cognitive surplus''
with relative ease
\cite{DillahuntPromise,storiesIraniSilberman,shirky2010cognitive}.
This insight, that
workers can be geographically distributed,
% (especially given careful central management of those workers, discussed later),
has proven remarkably compelling
\cite{sensitiveTasks}.

The insight here seems to have been that
a substantial amount of work does not necessarily need to be done in collocated space,
but can be done from one's own home or other space arranged by the worker.
Moreover, by shifting the workplace into the worker's domain,
the cost of capital involved in that work can more easily be placed on workers' backs.
While yellow--cab organizations often manage drivers in similar ways to platforms such as Uber and Lyft
(inasmuch as drivers are often considered contractors, or otherwise exempt from the formal status of ``employee''),
these new systems go further,
by framing ownership and responsibility for costs
(e.g. automotive maintenance)
on drivers.
In our fieldwork, we discovered many longtime ``on--demand'' drivers
(affiliated with platforms such as Uber and Lyft)
who readily voiced frustration with the costs of
frequent oil changes and the unexpected replacement of expensive parts.
Distributing labor had similar effects on pieceworkers at the turn of the \nth{20} century;
textile workers were given the source material with which to work,
but were generally expected to use their own sewing materials and work in their own homes
\cite{hapke2004sweatshop}.

% the change seemed to begin with
% the redistribution of work into the homes and onto the diffuse capital of workers.
% 99designs may represent one of the first illustrative cases of
% speculative, distributed labor in the contemporary form,
% but speculative labor and even on--demand design work is not necessarily new;
% \cite{99designs}.


% distributed; uber, etc... ->
% routinization; crowdwork - > optimization
% - > accommodation and resistance, alienation?
% tom malone


% This framing of labor worked for some time in the late \nth{19} and early \nth{20} centuries,


% This work is only made to seem new by framing the units of work as the output of ``task--completion APIs'',
% making it possible
% (and indeed advantageous)
% to allocate work on the fly to people in different places
% \cite{storiesIraniSilberman}.

\subsection{Discretization and Routinization}

Distributing work in method described above may be thought of as vertically slicing work such that 
each person is responsible for the whole task
--- making a whole garment or being responsible for driving in a single neighborhood.
Broken down in this way, work could grow to unprecedented scales,
but the quality of the work would remain relatively variable.
Textile work being a salient example,
some seamstresses might be better than others at the same task,
making this framing of the work initially problematic.

A compelling solution emerged in the early \nth{20} century to break tasks down into discrete,
manageable routines that could be taught relatively easily,
and whose work output could be evaluated in abstraction from the rest of the work.
In Ford's assembly line, this meant that workers were not responsible for building a whole car,
but a single very narrowly defined action that needed to be done on every car.

This approach paralleled what would be known as ``Taylorism'',
and its influences can be seen today in crowd work and micro--tasks.
\textit{talk about breaking tasks down both in general (Cheng) and even for the self (Teevan);
horizontal cuts instead of vertical}
\cite{cheng2015break,writingMicroTasks}.
By decomposing and recomposing tasks,
and in particular by assigning similarly natured work to the same workers,
% and especially as the nature of the work could be aligned with other, similar tasks,
workers could become ``experts'' in a small aspect of the work that they did,
speeding their work dramatically
\cite{delayAndOrderLasecki}.
Perhaps more important, however, was that
the breaking down of work into tasks has made it more practical to evaluate work at each stage
[\textit{citation needed}]







Taylorism enabled
--- and would later become synonymous with --- 
scientific management, which we will discuss later
% and might generally be regarded collectively as ``scientific management''
% The compelling solution that emerged in piecework became what we now know as Taylorism;
% rather than

% I never talk about factorization it's too soon ahhhh
% With piecework and factorization making mass manufacturing more viable,
% businesses could produce at scales that were previously unimaginable.
% Nevertheless, this approach to labor and the production capacity it offered was not without its hurdles.
% One of the issues that challenged mass production at the turn of the \nth{20} century,
% as well as today in crowd work and elsewhere,
% has been that of ensuring that the quality of the work is sufficiently good.


% hey so i need to bring in citations on piecework for this stuff because I didn't exactly live through this time

% is ANY of this good?

% The approach to piecework described above hinged in some sense on
% the shared understanding of the work to be done;
% workers generally understood what was required of sewing together pieces of fabric.
% Still, the work was speculative in that inadequately sewn clothing would not earn payment.
% This represented a risk for pieceworkers,
% who might spend time working on 10 garments only to be paid for 2 or 3.
% It also represented a potential cost for management,
% who would need to salvage the material
% (if it was even possible),
% and give that material to another pieceworker to do it correctly.

% i need to find the source for this, because it's honestly only a vague quote in my head,
% so... maybe i imagined it?

% One of the solutions that 


% Broadly speaking, the solution to this problem is to make instructions clearer and easier to follow.
% Further, by establishing a shared understanding
% --- a ``collective conscious'' or even a corporate culture regarding work ---


\textit{I want to argue that modern gig work is discretized,
and that this parallels some of the piecework that we're historically familiar with}

Work has become discretized in the last 10 years, particularly in the information work sector.
Advances such as
``find--fix--verify'' % (described in \citetitle{bernstein2015soylent})
and
``Cascade'' % (described in \citetitle{chilton2013cascade})
make it easier to assign smaller components of work to myriad workers and
recompose the constituent parts into something more complex
\cite{bernstein2015soylent,chilton2013cascade}.
Some work has taken special interest in the process of breaking tasks down into more manageable parts
% Some work has even turned to break it down'',
inasmuch as decomposing what we might call ``macro'' tasks into microtasks,
both for the purposes of crowd--sourcing in the sense of out--sourcing work
as well as ``selfsourcing''
\cite{cheng2015break,teevan2014selfsourcing}.

The parallels between this kind of work and
historical piecework of the turn of the \nth{20} century
is by no means new
[\textit{citation needed --- who said this first? most notably? Turner?}].
\textit{Elaborate on gig work, piecework, etc\dots}

% \subsection{routinization}
The \textbf{discretization} and decomposition of work certainly facilitates
breaking tasks down,
parallelizing work,
and getting more done across a broader array of people;
but the key advantage in turning macro tasks 
into many micro tasks is that we can make that work \textit{routine}.

By making larger bodies of work less contextually situated,
we can define processes that make that work
and the instructions thereof
useful for virtually anyone with shared cultural or intellectual background.

routinization can be seen in the contemporary in workflows
\cite{foundry,bernstein2015soylent}
[\textit{There are various papers that talk about creating a workflow;
maybe this is where FFV belongs? Foundry?}].

% \textit{Harken back to piecework
% and the instructions that denim workers had to make clothing;
% the instructions were clear and scoped enough that you could repeat the task ad infinitum.}


\subsection{External Management}
\textbf{Discretization}, enabling \textbf{routinization}, thus allows us to
manage work more meaningfully.
Routinization in particular made it possible
(and indeed preferable) to assume that people
% as \citeauthor{hu1961parallel} describes the premise of this sort of work,
``\dots are of equal ability and \dots can do any of the $n$ jobs''
\cite{hu1961parallel}.
\citeauthor{storiesIraniSilberman}
find similarly that
this work is only made to seem new by framing the units of work as the output of ``task--completion APIs''
% making it possible
% (and indeed advantageous)
% to allocate work on the fly to people in different places
\cite{storiesIraniSilberman}.

Crucially,
\citeauthor{hu1961parallel}'s case study assumed that workers were similarly trained and capable,
which in \citeyear{hu1961parallel}
meant that workers would at least have to be trained in person, if not instructed as well.
Throughout the industrial revolution, workers flocked to centralized locales of production,
and physically situated factories made it both
difficult to ignore the general pattern of mistreatment and
relatively easy to organize.

% construction and manufacturing in various forms
% (from textiles to automobiles)
% encourage the consolidation of capital--intensive resources like factories.
% The prohibitively expensive nature of factory equipment may have been a key factor in 
\citeauthor{storiesIraniSilberman} describe is that the ``APIs'' in the latter case


% evaluate work in abstraction.
% This means that we can 

\textit{I want to cite Justin's work on crowdsourcing effort to highlight that
we can evaluate workers for the work they do with increasingly finite amounts of time;
also, Ranjay's work that embraces failure because
we can essentially algorithmically make sense of failure (let's call it noise)
and re--capture signal.}


\subsection{Resistance}
So it's not surprising, then, that workers are frustrated:
Turkopticon, Dynamo, and many other pieces of work point this out
\cite{turkopticon,dynamo,uberAlgorithm}
[\textit{also want to cite Brian's work from CHI this year}].
In popular culture, reporters have inquired about
what ``the future of labor unions'' will look like
[\textit{Imagine I had cited an online article and
it wasn't totally inappropriate for a conference}].

\textit{The parallels here would be difficult to ignore;
Lily and others have pointed this out,
but it's worth thinking about the broader trend of
resisting management,
driven by routinization,
enabled by discretization.
Do that here.}



\subsection{Looking forward}
If we agree that
discretization,
routinization,
management,
and even resistance
necessarily follow one another according to this theoretical lens,
then we have to use it to attempt to envision
what comes next.



\textit{I've been going back and forth regarding how to frame this paper;
the approach that tries to look at
things that are similar vs things that are different
doesn't seem to work, but
I've left the thoughts here because it's not all completely bad.}

\section{Things stay the same}
How is gig work the same as it's been historically?

I think this section would be compelling to draw parallels between
the narratives drivers gave about the flexibility, autonomy, etc\dots
and that which we might have seen among pieceworkers
(predominantly women, who benefited from being able to work from home).

\subsection{Flexibility}
Are there cultural differences between the people that did piecework and
the people that do gig work now?
I'm not sure there are significant differences that have affected the outcomes so far.

Many of the workers to whom piecework appealed were mothers, wives, etc\dots
who mostly stayed at home for various reasons
(certainly largely it was cultural
--- women weren't afforded equal access to labor opportunities,
making in--home job opportunities not only compelling, but also one of few available options).

Gig workers are in some senses similarly constrained:
workers on Amazon Mechanical Turk
--- those that use it as a primary source of income, at least ---
report being homebound for various reasons (e.g. medical, parenting, etc\dots).
Society and circumstance have made it difficult
or impossible
to join the contemporary, conventional workforce;
gig work re--opens that door.

But there are differences; during our research over the summer,
we spoke to drivers on Uber and Lyft, cleaners, and other gig workers.
Many of them told us about their home lives
--- about children, spouses, and other commitments ---
to which they wanted to dedicate more time.
One driver
(let's call him Ra\'{u}l)
told me about how he drove for Lyft
after working as an inventory manager at a hospital for more than nine years.

I asked him why he quit that job and forewent
the benefits,
predictability,
and career growth opportunity
that his old job offered.
He told me that when his daughter was born,
he was overcome with a desire to spend more time with her.
No longer satisfied with work
where he often left before his daughter would wake up and return after she fell asleep,
Ra\'{u}l decided to start driving for Lyft,
because he could drive in the evenings when his daughter was asleep.

Other drivers reported similar benefits;
gig work affords its workers flexibility that conventional careers don't allow.

\subsection{routinization of work}
This leads into the next section, but I want to bring up the process of
making work about mass--manufacturing, at least inasmuch as
the instructions are the same for everyone.

The Internet arguably has made it much easier to broadcast those instructions,
but it's had this deeper effect of enabling some amount of back--and--forth
between the worker and the (algorithmic) manager.

\subsection{Taylorism}
The routinization of work makes it possible to measure that process,
optimize it for certain characteristics, and ultimately
lead to Taylorism and scientific management.
This is not new;
researchers have studied and written about the slow creep of
algorithmic management and 
discretization \& routinization of work tasks.

\textit{We hope to take a step back from the context in which this work is often applied, and
look for its place in the larger trends and theories
to make sense of the trends of gig work at large.}

Industrialization and the automobile assembly line makes this famous,
but piecework functioned on the principle that everyone was making similar
or identical
garments and other products.

Now, we see Turkers being evaluated on the outcome of their work conforming to norms,
sometimes bootstrapped,
as in Ranjay's talk on 
``Embracing Error to Enable Rapid Crowdsourcing'',
but more conventionally in work flows like ``Find--Fix--Verify''.

\textit{This might be an opportunity to reflect on
how pieceworkers internalized the work they were doing,
responded to the stress of the uncertainty of potentially rejected work,
etc\dots but I'm not familiar with research in that space.}

\subsection{The emergence of decentralized workplaces}
The practice of in--home piecework was consumed by
the centralization of factories
(the effect of which we'll talk about later, since
we can talk about how this made unions more practical),
but for a time many pieceworkers
at the turn of the \nth{20} century
worked out of their homes.
Strikingly, many of the gig workers
\citeauthor{crowdworkFuture}
discussed in
\citeyear{crowdworkFuture},
and indeed many more continue to do what we might call ``information work''
--- that is, work that predominantly demands human computation ---
but increasingly we're seeing the movement toward transient work that largely requires embodied presence
\cite{uberAlgorithm} (and others).

\subsection{What was this about?}
I was going to check in quickly and take a photo of your whiteboard
(hoping that you'd kept our conversation notes around)
but your room was \textit{literally} full of people lol.


\section{Things are changing}
The medium on which this work is being done
--- and to an extent the medium used to manage workers ---
has dramatically changed things as well, however;
workers are distributed around the world,
working out of their cars in the cases of livery services
(notably, never returning to a base of operations)
across and between cities as well as nations,
or in their homes
(paralleling the trend of piecework even more closely).

Trying to understand how gig work has differed from piecework should at least
start with looking at the different characteristics of the work involved.
After that, we should think about how
the demography and culture of the people engaging in this kind of work
have changed versus that of the pieceworkers.

\subsection{Differences in the work itself}
Gig work has all of the above similarities with piecework,
but there are key differences.

Piecework emerged at the turn of the \nth{20} century,
right at the time that telecommunications began to boom in the United States. % too far a reach?
\textit{Did telecom enable remote management?}

Gig work in its contemporary formation is largely mediated by
ubiquitously accessible digital media
(the Internet, telephony, etc\dots)
and importantly has relied on this technology to facilitate the remote management of workers
\cite{uberAlgorithm}.

\textit{Has the work fundamentally changed,
or are we just being managed remotely in different ways?
I'm not entirely sure.}

How does this work differ from the experience of being ``on--call'' that is so familiar to retail employees?
How does this work differ from the work in which truckers, taxi drivers, and other
independent contractors have been participating for decades?
We argue that the substantive difference in these markets is the speed of the market itself,
motivated by the technology which mediates it.
Because workers can be sourced and dispatched virtually instantly,
businesses that engage in this kind of work
(e.g. Uber, Amazon Mechanical Turk, etc\dots)
have taken to removing other bottlenecks,
like vetting workers upfront


\section{More deliberate work}
% \subsection{Differences in the people that do the work}
% ???


\printbibliography


\end{document}

%%% Local Variables:
%%% mode: latex
%%% TeX-master: t
%%% End:
