\documentclass{sigchi}

\usepackage{todonotes,txfonts,balance,graphics,color,alphalph}
\usepackage{booktabs,textcomp,microtype,ccicons}
\usepackage[normalem]{ulem}
% \usepackage{babel}
% \usepackage{csquotes}
\usepackage[citestyle=numeric,backend=bibtex,bibencoding=ascii]{biblatex}
\usepackage{enumitem}
\usepackage{moreenum}
\usepackage[T1]{fontenc}
\usepackage{nth}
\usepackage[pdftex,pdfpagelabels=false]{hyperref}
\usepackage[all]{hypcap}  % Fixes bug in hyperref caption linking
\usepackage[utf8]{inputenc} % for a UTF8 editor only

% Paper metadata (use plain text, for PDF inclusion and later
% re-using, if desired).  Use \emtpyauthor when submitting for review
% so you remain anonymous.

\newlist{inlinelist}{enumerate*}{1}
\setlist*[inlinelist,1]{%
  label=\arabic*),
}


\def\plaintitle{Naming Things is Hard: Real Title Following Colon}
\def\plainauthor{All the people (Ali, Margaret, MSB, who else?)}
\def\emptyauthor{}
\def\plainkeywords{Please don't make me pick keywords.
This is like asking a teacher to give the bullet points of
what a student missed in lecture.}
\def\plaingeneralterms{Documentation, Standardization}

% llt: Define a global style for URLs, rather that the default one
\makeatletter
\def\url@leostyle{%
  \@ifundefined{selectfont}{
    \def\UrlFont{\sf}
  }{
    \def\UrlFont{\small\bf\ttfamily}
  }}
\makeatother
\urlstyle{leo}

% To make various LaTeX processors do the right thing with page size.
\def\pprw{8.5in}
\def\pprh{11in}
\special{papersize=\pprw,\pprh}
\setlength{\paperwidth}{\pprw}
\setlength{\paperheight}{\pprh}
\setlength{\pdfpagewidth}{\pprw}
\setlength{\pdfpageheight}{\pprh}
\definecolor{PineGreen}{HTML}{008800}
\definecolor{BrickRed}{HTML}{FF0000}
\newcommand{\msb}[1]{{\color{PineGreen}[MSB: #1]}}
\newcommand{\ali}[1]{{\color{BrickRed}[\itshape al2: #1\upshape]}}

% Make sure hyperref comes last of your loaded packages, to give it a
% fighting chance of not being over-written, since its job is to
% redefine many LaTeX commands.
\definecolor{linkColor}{RGB}{6,125,233}
\hypersetup{%
  pdftitle={\plaintitle},
% Use \plainauthor for final version.
%  pdfauthor={\plainauthor},
  pdfauthor={\emptyauthor},
  pdfkeywords={\plainkeywords},
  bookmarksnumbered,
  pdfstartview={FitH},
  colorlinks,
  citecolor=black,
  filecolor=black,
  linkcolor=black,
  urlcolor=linkColor,
  breaklinks=true,
  hypertexnames=false
}

% create a shortcut to typeset table headings
% \newcommand\tabhead[1]{\small\textbf{#1}}
\bibliography{../../../../references}

% End of preamble. Here it comes the document.
\begin{document}
\title{\plaintitle}

\numberofauthors{3}
\author{%
  \alignauthor{Leave Authors Anonymous\\
    \affaddr{for Submission}\\
    \affaddr{City, Country}\\
    \email{e-mail address}}\\
  \alignauthor{Leave Authors Anonymous\\
    \affaddr{for Submission}\\
    \affaddr{City, Country}\\
    \email{e-mail address}}\\
  \alignauthor{Leave Authors Anonymous\\
    \affaddr{for Submission}\\
    \affaddr{City, Country}\\
    \email{e-mail address}}\\
}

\maketitle

\begin{abstract}
  With growing attention on gig work
  --- ranging from the ``sharing economy'' to microtasks ---
  scholars have made connections to frameworks like Taylorism,
  and mechanisms such as worker advocacy and empowerment,
  to make sense of our observations of
  on--demand work and the workers that power this movement.
  We argue that our the underlying trend towards ``piecework''
  --- driven in part by the discretization,
  routinization,
  and external management of said work ---
  not only suggests, but in fact generates what we have observed:
  members of this transient workforce increasingly feeling
  disempowered,
  marginalized,
  and frustrated by the systems and platforms on which they work.
  
  After evaluating this framing
  through a series of case studies
  in various industries falling broadly under the ``gig work'' category,
  we turn our theoretical lens to look to the future,
  to identify worthwhile questions and
  points of inquiry that researchers in social computing should consider
  as we attempt to anticipate and perhaps shape the future of work.
\end{abstract}

\category{H.5.m.}{Information Interfaces and Presentation
  (e.g. HCI)}{Miscellaneous} \category{See
  \url{http://acm.org/about/class/1998/} for the full list of ACM
  classifiers. This section is required.}{}{}

\keywords{\plainkeywords}

\section{Introduction}
% \msb{I think a contribution of this paper can be, and should be, unifiying crowd work, sharing economy, and gig economy all under one theoretical banner: piecework. Come out and say that super visibly somewhere in this intro as a major contribution.}

The past decade has seen
paid microtasking, the ``sharing economy'', and other instantiations of on--demand contract work
grow to occupy the fascination of both academic circles and our culture as a whole
\cite{Ross,CrowdsourcingUserStudies,scholz2012digital}.
The research community has made connections between this emergent form of work
and the historically situated ``piecework'',
as well as a number of observations relating to
the frustration workers feel stemming from the management of this work,
but situating the causes of these frustrations either in the larger body of scholarship on crowd work,
or crowd work itself,
has proven difficult;
% as a result,
the connection between crowd work and a historical parallel contextualizing its ongoing developments 
hasn't been deeply made yet
\cite{turkopticon,dynamo,uberAlgorithm}.
% \msb{A weak differentiating sentence: }
% But the connections between
% historical piecework and
% contemporary crowd work
% haven't been deeply made as yet.
% um, note to self: say more.
% \msb{This paragraph doesn't set up the tension and release effectively. I suggest focusing this paragraph on a brief summary of the things that people have pointed out (low pay, Taylorism, difficult collective action). Basically paint them as a bunch of disconnected observations --- which is what they have been. Then at the start of the next paragraph, come out with a clear and visible thesis: say that all of these things fall under the banner of piecework. (and now you can finally mention that while piecework has been brought up, nobody has made the historical connections yet) Then you can move on to the next paragraph below, which is about how it's basically a necessary outcome.}

While much of this work appears to gesture toward
the parallels between contemporary on--demand work and piecework,
it's proven difficult to bring the totality of these observations into focus using one theoretical lens.
This paper will attempt to do so by arguing that
the topics social computing researchers have investigated are
not just parallel to historical piecework and
the process of factorization that took place in the early \nth{20} century;
indeed, these phenomena \textit{precipitate and reinforce each other}.

In the reflection on the literature published in the last 5 years since
\citeauthor{crowdworkFuture}'s 
\citetitle{crowdworkFuture},
we notice a broader trend describing the change in work that's being done
\cite{crowdworkFuture}.
Following the improvements in
breaking down tasks,
delegating work, and
managing workers more broadly,
crowd--powered work has continued to parallel historical piecework's trajectory
by outgrowing ad--hoc worker groups and coalescing into more formal working groups,
for example through the use of a ``white list'' of familiar workers.
We might call this the regularization of factory work.

Making sense of the broader field of crowd work
by looking at the movements toward
distributing work,
routinizing and breaking down tasks,
and externally managing workers
as linked to one another, 
we give ourselves a framing of contemporary piecework that explains,
and arguably predicts,
what we have seen thus far
--- and perhaps what we should expect to see going forward.

This paper will attempt to demonstrate that the familiarity
that we incidentally observe between what we generally call crowd work \& gig work
and the historical practice of piecework and subsequent factory work
is more than passing;
that the phenomena researchers have observed in on--demand digitally mediated labor markets
were inevitable milestones in the birth and life of piecework.
We will set out to show that this perspective of crowd work as an embodiment of piecework
predicts the myriad outcomes of contemporary crowd work, including
our developments of work--flows further abstracting work,
the troubling effects of those developments on factors such as pay and work quality, and
many if not all of the frustrations researchers have discovered among workers
in the study of crowd and gig work.

``But'',
as \citeauthor{scholz2012digital} points out,
``it would be wrong to conclude that
in the realm of digital labor there is nothing new under the sun''
\cite{scholz2012digital}.
For one, information work
(e.g. Amazon Mechanical Turk (AMT) \& Upwork)
and digitally managed work in general make it substantially easier
to keep workers geographically dispersed
compared to the factory workers that followed home--based piecework.
This aspect of work renders workers virtually invisible to the task solicitors
--- or ``requesters'' ---
despite being significantly more connected than historical pieceworkers were % this is a weak point...
\cite{turkopticon}.
Secondly, while
factory foremen and other middle--men
traditionally mediated the relationships between workers and managers,
today the visible agents are \textit{systems};
rather than employing individuals who ``personified the functions of management''
and can thus negotiate workers' needs,
socio--technical systems mediate these interactions % often with unrewarding outcomes
\cite{wray1949marginal}.
% \cite{storiesIraniSilberman}.
Finally, this characteristic of modern piecework makes
worker coordination for collective advocacy and action significantly more difficult,
necessitating special consideration to maintain the inertia of collective will
while focusing that energy productively
\cite{dynamo}.
% \ali{I really don't like this paragraph\dots}
% here are undoubtedly more examples of differences.
% How do I not stake a claim that this is comprehensive?
% Or should I really flesh this out and make it (comprehensive)?


\subsection{Piecework as a lens to understand gig work}

\citeauthor{crowdworkFuture}
write of the future of crowd work in an effort to investigate
whether crowd work will be a worthwhile form of work or another instantiation of ``piecework''
\cite{crowdworkFuture}.
``Piecework'' here makes use of
a term historically used to describe work done in the home,
in manageable tasks,
often involving clear instructions
and payment only for work completed, not work done
(the differentiation, here, being that
one would be paid for the \textit{output} of the work,
not the \textit{duration}).
% \msb{split up that sentence. One is the def'n of piecework, the other is the comparison you're drawing}
Given the scope,
we can frame piecework and on--demand labor
% \msb{
% \sout{wait, are you using gig work as the umbrella term? I think you need a category term that covers all of them}
% \ali{let's use ``on--demand labor'', since
% that seems general enough without accidentally including historical piecework}}
as sharing these important similarities:
\begin{inlinelist}
\item this form of work began in the home
\item the worker is paid for each discrete piece of work done, regardless of time or effort; and
\item the worker's status
(not only socially, but also economically)
is ambiguous, or at least the subject of some controversy.
\end{inlinelist}. 

% \subsection{Context}
In the past decade, researchers have observed frustration
grow among on--demand workers,
with expression of this frustration spanning a wide range of tactics
\cite{uberAlgorithm,turkopticon,dynamo}.
Attempting to make sense of these case studies has been challenging
in part because
a wholly encompassing framework for understanding this form of work
has thus proven difficult to capture.


This paper attempts to make sense of the broader research on this new form of work,
\textit{on--demand labor}, by evaluating this work through a more familiar lens: piecework.
More concretely, by looking at task--based or ``gig'' work as
an instantiation and continuation of piecework,
and by looking for patterns of behavior that the corresponding literature predicts
on this basis, we can do the following:
\begin{inlinelist}
  \item we can make sense of the phenomena so far as part of a much larger series of interrelated events;
  \item we can bring into focus the ongoing work among workers, system--designers, and researchers in this space; and finally,
  \item we can offer predictions of what social computing researchers, and workers themselves, should expect to see on the horizon of on--demand work.
\end{inlinelist}

We'll look at a broad range of cases under a number of major themes
we propose as broadly describing the types of research being done in crowd work
and more generally in what we argue is contemporary piecework.
After validating this lens as a way of reasoning about on--demand labor,
we'll attempt to use this perspective to suggest areas of research worth anticipating,
and developments we should expect to see in the maturation of digitally mediated work.

\section{Case Studies}
The existing body of research has shed light on on--demand labor from various perspectives,
and revealed a number of topics that,
through our framing, are clearly situated together.
Those topics are, at a high level, as follows:
\begin{enumerate}
\item the \textbf{processes} involved in making work into tasks, or discretization;
\item the outcomes (and indeed the \textbf{fallout}) of that discretization,
both on the work itself as well as the workers;
and finally
\item the \textbf{relationships} between workers and requesters of the work
--- both \textit{cooperative} and \textit{adversarial} cases.
\end{enumerate}

% We will follow these topics, using them as prompts for case studies in
% the emergence and
% development of contemporary piecework.

\subsection{The Processes of Making Gig(s) Work}\label{sec:MakingGigsWork}
% this is phase 1: the decomposition & abstraction of work;
% all of the research that has gone into work--flows and
% all the things we can do with crowd work.

The HCI community is perhaps most familiar with examples of task--based work such as
99designs,
Amazon Mechanical Turk (AMT),
and increasingly
Uber \& Lyft,
which all allow requesters in various forms to tap into
resources such as
cars,
computers, and above all
``cognitive surplus''
with relative ease
\cite{howe2006rise,DillahuntPromise,storiesIraniSilberman,shirky2010cognitive}.
This insight, that
workers can be geographically distributed and tasks decomposed,
% (especially given careful central management of those workers, discussed later),
has proven remarkably compelling
and an effective fulcrum for leveraging the Internet for highly scalable work
\cite{sensitiveTasks,embracingErrorKrishna}.


This section will largely discuss the processes at work that make distributed,
digitally managed work both possible and indeed preferable for ``requesters''
(in other words, the employers who solicit workers).
This body of research spans a broad field
within the CSCW and broader HCI community.
In this context,
we'll look at the body of research through the lens of
highlighting contributions which expand what we
(as ``requesters'' of work) can do by managing workers in novel ways.
This work broadly consists of three areas:
\begin{inlinelist}
\item \nameref{sec:decomposition},
\item \nameref{sec:workAbstraction}, and
\item \nameref{sec:flexibility}
\end{inlinelist}.
As we explore this work, we'll
attempt to relate the advances in the design of crowd work to
the research contributions made in the research of assembly line manufacturing,
Taylorism, and
scientific management 
during the \nth{20} century
\cite{hu1961parallel}.


\subsubsection{Decomposition}\label{sec:decomposition}
Piecework may be thought of as vertically slicing work such that 
each person is responsible for the whole task
--- making a whole garment, in this case.
Broken down in this way, work could grow to unprecedented scales,
but the quality of the work would remain relatively variable
\cite{murray1983decentralisation}.
Textile work being a salient example,
it took time for workers to acquire sufficient skill
to do every aspect of the work so that the garment would be accepted by the company soliciting that work
\cite{vezina1992light}.

A compelling solution emerged in the early \nth{20} century to break tasks down into discrete,
manageable routines that could be taught relatively easily,
and whose work output could be evaluated in abstraction from the rest of the work
\cite{restructuringPieceworkBaker}.
In Ford's assembly line, this meant that workers were not responsible for building a whole car,
but a single very narrowly defined action that needed to be done on every car
\cite{towardsGlobalFordism}.
By the mid--\nth{20} century, \citeauthor{schoenberger1988fordism} writes,
``\dots~the intensification of the labor process is argued to have hit mental, physical, and social limits.''
\cite{schoenberger1988fordism}.


This approach, ``Fordism'' (and its better--known contemporary ``Taylorism'' of similar ethos),
can be seen today in crowd work and on--demand labor through the application of micro--tasks.
\citeauthor{writingMicroTasks} highlight some of the advantages of breaking work into pieces,
facilitating evaluation and parallelization
\cite{writingMicroTasks}.
By decomposing and recomposing tasks,
and in particular by assigning similarly natured work to the same workers,
workers could become ``experts'' in a small aspect of the work that they did,
speeding their work dramatically
\cite{delayAndOrderLasecki}.
Perhaps more important, however, was that
the breaking down of work into tasks has made it more practical to evaluate work at each stage
\cite{rogstadius2011assessment}.

Scholarship describing and exploring
the decomposition of tasks is perhaps the most established of the above areas among HCI researchers;
\citeauthor{crowdForgeKittur} specifically drive at this goal by addressing the possibility of
``crowdsourcing complex work''
\cite{crowdForgeKittur}.
\citeauthor{cheng2015break} found that microtasks
--- though not necessarily \textit{faster} than ``macrotasks'' ---
yield higher quality work,
particularly when that work might be readily interrupted
\cite{cheng2015break}.
\citeauthor{selfsourcingTeevan2014} further push the boundaries of decomposed work
by exploring ``selfsourcing'', and further this work with \citeauthor{selfsourcingTeevan2016}
\cite{selfsourcingTeevan2014,selfsourcingTeevan2016}.
While this work doesn't strictly fall under ``crowdsourcing'',
the major contributions here
seem uncontroversially to be inspired by the design of crowd work.

Much of the research in the space of designing crowd work has
sought to illustrate the potential to take highly creative or skilled work
and generate high--quality results.
Perhaps the most notable case study here can be found in
\citeauthor{foundry}'s \textit{Foundry}, which employed
``flash teams'' to achieve expert--level outcomes via thoughtful
decomposition of work as ``modular tasks''
\cite{foundry}.

Work decomposition, then, is far from new;
``decomposition'' generally illustrates the same concepts of work that ``Taylorism''
and scientific management sought to embody
---
\citeauthor{silberman2010ethics} in particular foresaw this danger and warned of it in
\citeyear{silberman2010ethics}
\cite{crowdworkFuture,silberman2010ethics,nickerson2013crowd}.
In both the historical and contemporary cases of decomposed work,
work was,
at least initially,
distributed in the form of tasks to the homes of workers;
\citeauthor{riisOtherSideLives} captured this in his documentary work
\citetitle{riisOtherSideLives}
in \citeyear{riisOtherSideLives}
\cite{riisOtherSideLives}.

\subsubsection{Work Abstraction}\label{sec:workAbstraction}
Decomposition allows requesters to assign tasks without concern for the broader context.
While we'll discuss this aspect of crowd work more critically later,
it's worth pointing out that discrete blocks of work containing all the relevant context for a worker
allows workers to engage with virtually any component of work without worrying that their lack of 
higher--level awareness of the goals of the requester might negatively affect their work.

\citeauthor{chilton2013cascade} perhaps best illustrated this with
\textit{Cascade} by demonstrating that it's possible to
break certain classes of tasks apart
in such a way that they yield taxonomies of various subjects,
a task previously thought to be safely within the domain of expert workers
with top--down awareness of the context of the work as a whole
\cite{chilton2013cascade}.
\citeauthor{verroios2014context} further illustrate this potential by
forming a task one might consider highly contextually dependent
--- summarizing the contents of a movie ---
in such a way that crowd workers could contribute small pieces of work without
needing to know the content of the rest of the project
\cite{verroios2014context}.

Here, \citeauthor{hu1961parallel}'s work,
saying of assembly line work that
``it is assumed that men are of equal ability and every man can do any of the $n$ jobs'',
parallels the approach that dominated early research into crowd work
--- namely, using non--expert crowds for complex work
\cite{hu1961parallel}.
This mindset in \citeauthor{hu1961parallel}'s analysis,
and indeed the study of factory and mass manufacturing labor through the \nth{20} century,
substantively owes its existence to scientific management
and the rigorous decomposition of work into tasks, discussed earlier,
and persists to this day as it colors
researchers' goals and objectives in the study and design of crowd work.

Piecework's influence on the abstraction of work into tasks,
described above, is more than just caused by the decomposition of work;
work abstraction itself makes it possible for workers to come and go flexibly,
prompting work requesters to consider ways to design these now discrete tasks in ways that
maximize flexibility, both by allowing (and even anticipating) some inconsistency in worker availability
\textit{and} allowing and anticipating some inconsistency in the quality of the work output itself.
It's to this area that we now turn our attention.

\subsubsection{Flexibility}\label{sec:flexibility}
% \msb{nothing in this subsubsection connects to the piecework literature. bring it back!}

Earlier we discussed \citeauthor{cheng2015break}'s work
measuring the impact that interruption has on worker performance.
This work both points to and embodies a broader sentiment in
both the study and practice
of crowd work that microtasks should be designed resiliently against the variability of workers,
% \msb{not sure what flexibly means here, and the rest of the sentence isn't helping me unpack it. flexibility for requester? worker? flexible algorthms?},
fully exploiting the abstracted nature of each piece of work
\cite{interruptionIqbal,delayAndOrderLasecki,vaish2014low}.
That is to say, micro--tasks should be designed such that a single worker's poor performance,
or a good worker's sudden departure,
would not significantly impact the agenda of the work as a whole.
While \citeauthor{cheng2015break} identified costs with breaking tasks into smaller components
in the form of higher cumulative time to complete
(albeit much shorter real time to complete, owing to parallelization),
\citeauthor{delayAndOrderLasecki} found that at least \textit{some} performance can be recouped by stringing 
similar tasks together.


Given the importance of consistent work results, one might intuit that
requesters would prefer high--quality workers who can be relied upon to be available
(even for contextually independent tasks),
which would appear to contradict the benefits of flexibility already discussed;
requesters have thus made significant headway toward
``embracing error'' to allow requesters to maximize the benefits of a flexible,
even transient,
workforce.

\citeauthor{embracingErrorKrishna} offer orders--of--magnitude improvements
in various binary classification tasks
on the principle that diverse workers complete these tasks
in order to accurately inform the model on the variety of delays in response times.
And rather than building tasks to \textit{tolerate} worker drop--off and attrition,
some researchers have designed work predicated on the expectation of this phenomenon:
\citeauthor{sensitiveTasks} describe ways of assigning tasks in such a way that
crowd workers would never be given enough information to piece together sensitive information about
any single topic
\cite{sensitiveTasks}.

Flexibility has been explored through the lens of Fordism, perhaps best illustrated by
\citeauthor{tolliday1986between}'s treatment describing
turnover rates rising above 300\% in the decade leading to the introduction of the assembly line in 1913.
Specifically, the utilization of ``\dots~`semi--special' machine tools which could be adapted
[and]~\dots~added flexibility through seasonal layoffs for production workers and the use of
piece rates~\dots~rather than a day wage system''
\cite{tolliday1986between}.



% \msb{I still dont' have a clear conceptual picture of what flexibility is, and what about it sets this section apart. Ranjay's stuff is about speed...? I'm confused.}




\subsection{The Fallout of Crowd Work}\label{sec:Fallout}
% If the research exploring and documenting how we can exploit crowd \textit{work} can be described as wide--ranging,
% the scholarship discussing the ways crowd \textit{workers} have been exploited is more focused;
\citeauthor{turkopticon} point out the disillusion that companies such as Amazon foster on platforms for work like AMT
(see also \citeauthor{dynamo}'s work
continuing in the spirit of this observation to generate collective action to improve worker conditions)
\cite{turkopticon,dynamo}.
\citeauthor{uberAlgorithm}
find similarly that workers on gig work platforms are frustrated by the systems on which they work,
to say little of the policies which these systems enforce
\cite{uberAlgorithm}.

We discussed the benefits of flexibility
(both in the sense of having arbitrary workers perform tasks and
in the sense that we can design tasks to be more resilient to poor work)
in the previous section.
It's from that point in the literature that we turn our attention to
the perhaps unintended effects of crowd work
and the affordances for transience that we build into this mode of work.
We'll address two major areas of work under this subject:
\begin{inlinelist}
\item \nameref{sec:lowPay}; and
\item \nameref{sec:varQualWork}.
\end{inlinelist}

\subsubsection{Low Pay}\label{sec:lowPay}
% \msb{nothing in this subsubsection connects to the piecework literature. bring it back!}

\citeauthor{laborEconomicsOfCrowdsourcingHorton}
identified problems with crowd work wages relatively early on,
attempting to address this imbalance from a behavioral economic perspective ---
that is, identifying and presenting a model that describes a worker's
``\textit{reservation wage}''
\cite{laborEconomicsOfCrowdsourcingHorton}.
This work has largely informed much of the research into and practice of estimating crowd work compensation
\cite{incentivesShaw,paolacci2010running}.
More cynically,
we would describe this work as identifying optimal levels to activate race conditions \msb{???} among piece workers.

But we turn to \citeauthor{turkopticon}'s discussion of ``\textit{Turkopticon}'',
a system they designed to interrogate worker invisibility and to promote better wages across several dimensions
\cite{turkopticon}.
Of particular relevance here,
\citeauthor{turkopticon} call to attention that ``Turkers'' are ultimately vulnerable to
wage theft and
pay rates that translate to well under minimum wage.
Returning to \citeauthor{laborEconomicsOfCrowdsourcingHorton},
we find that the median ``reservation wage'' in \citeyear{laborEconomicsOfCrowdsourcingHorton}
was \$1.38, while the mean was \$3.63
\cite{laborEconomicsOfCrowdsourcingHorton}.

Understanding workers' motivations given these conditions has thus become a goal for some researchers
\cite{whyWouldAnyoneBrewer}.
\citeauthor{Sun20111033} conclude that
``\dots~solvers participate in online tasks
not only for money
but also for enjoyment
or the sense of self--worth''
\cite{Sun20111033}.
This might have rung true in \citeyear{Sun20111033},
and certainly corroborates \citeauthor{Ross}'s findings after investigating
``who are the crowdworkers'',
but as \citeauthor{whoareNOTtheTurkers} points out
``we [have since] learned that most tasks on AMT are done by a small group of professional Turkers\dots''
\cite{Ross,whoareNOTtheTurkers}.

Now, \citeauthor{turkopticon}
and later
\citeauthor{dynamo} cite insufficient pay as a central point of frustration among workers,
via \citeauthor{irani2015cultural} and \citeauthor{dawnDigitalSweatshopCushing}'s work
\cite{dynamo,irani2015cultural,dawnDigitalSweatshopCushing,turkopticon}.


Turmoil over low --- and declining --- pay was among the chief grievances among then nascent
British labor unions in the early \nth{20} century
\cite{turner1952trade}.
This, \citeauthor{ebbinghaus1999institutions} argued,
fueled the rocketing union membership rates through the mid--\nth{20} century until 1980
(to which we'll return later when we discuss \citeauthor{levi2009union}'s reexamination of labor advocacy organizations)
\cite{ebbinghaus1999institutions}.
This realization has similarly fueled a body of research into
the various incentive structures available to piecework employers
\cite{roy1953work}.


The parallels between the complaints of low pay among crowd workers and other on--demand workers
and the pieceworkers and later factory workers in the \nth{20} century
are inescapable.
We argue further that the \textit{causes} here
--- work decomposition,
work abstraction, and
flexibility ---
lead inexorably to low and declining pay for workers.

\subsubsection{Variable quality work}\label{sec:varQualWork}
Researchers have struggled with what we might generously call work of ``variable quality''
along two dimensions:
First, to use the characterization of one of these contributions,
``understanding malicious behavior''
\cite{MaliciousCrowdworkersGadiraju}.
While some work has cast workers as ``malicious'' agents,
the evidence thus far suggests that
workers behave in unexpected ways as they attempt to assert some control over their interaction with the system
(a topic of discussion to which we'll return later)
\cite{uberAlgorithm}.
The second dimension of research in this space generally attempts
to eke out the highest quality work possible from workers
given the apparent difficulty in predicting work outcomes.



% Low pay yields variable quality work for a number of reasons,
% but before we discuss the causes of this effect, we should discuss the

% Variance in the quali

The effect that poor wages have had on piece work and factory workers is hardly unknown or novel;
\citeauthor{gantt1913work} discuss this exact mechanism in his book on
\citetitle{gantt1913work}, pointing out that
``\dots~where there is no union,
the class wage is practically gauged by the wages the poor workman will accept,
and the good workman soon becomes discouraged and \textit{sets his pace by that of his less efficient neighbor},
with the result that the general tone of the shop is lowered'' (emphasis added)
\cite{gantt1913work}.
\ali{Do I need to follow up or does this basically speak for itself?} \msb{looks good}

This research is similar to, but subtly different from, the notion of the ``market for `lemons'''
which \citeauthor{fort2011amazon} discuss;
specifically, \citeauthor{akerlof1970market}'s writing of a ``market for `lemons'''
describes a marketplace where the quality of the product or service is unknown to the buyer
\cite{fort2011amazon,akerlof1970market}.
The effect of this \textit{perceived} uncertainty is that
the \textit{actual} trustworthiness drops precipitously
as all of the consistent, reliable, high--quality workers capable of leaving these markets do so,
leaving only the ones who cannot or will not establish their trustworthiness.

Suffice it to say, then, that poor pay and poor work are linked,
and that we should not be surprised to find this relationship play out online as strongly as it does offline.
Indeed, \msb{finished sentence? \ali{Indeed.}}





\subsection{Relationships Between Workers and Managers}\label{sec:relationships}


\subsubsection{External Management}
\itshape

Scientific management in other words; should I just call it that and get right into it?
I kind of want the focus to be on the adversarial nature of the management,
and ``external'' seems to connote that more strongly than ``scientific'',
which has a tone of being more\dots~Measured? Objective? Better?

There's a fair amount to say here, and lots to cite, so I'm not super concerned.
I do want to bring up \citeauthor{storiesIraniSilberman}'s
\citetitle{storiesIraniSilberman} \cite{storiesIraniSilberman}
as a way of pointing out that even in our effort to study piecework ethnographically,
and despite our attempts to be reflexive and conscientious of
the power dynamics that we bring into any discussion
(as engineers, as people with megaphones, etc\dots)
we still manage to influence and perhaps steer.

\upshape

% Taylorism enabled
% --- and would later become synonymous with --- 
% scientific management, which we will discuss later
% and might generally be regarded collectively as ``scientific management''
% The compelling solution that emerged in piecework became what we now know as Taylorism;
% rather than


\subsubsection{Alienated Workers}
\itshape

Marx's big entrance. Lily has lots of work pointing to this, and
lots of that work points to yet more work gesturing in this direction.
I want to bring a lot of that work
that hasn't made it into the discussion yet
into the conversation here.

Note to self: examples of citations would help make it clearer for MSB which ones I'm overlooking,
if nothing else

\upshape


\subsubsection{Resistance}

\citeauthor{uberAlgorithm}'s paper on Uber drivers
\cite{uberAlgorithm}.

\citeauthor{dynamo} on \textit{Dynamo};
\citeauthor{turkopticon} on \textit{Turkopticon}
\cite{turkopticon,dynamo}.


% \subsection{Resistance}
% So it's not surprising, then, that workers are frustrated:
% Turkopticon, Dynamo, and many other pieces of work point this out
% \cite{turkopticon,dynamo,uberAlgorithm}
% [\textit{also want to cite Brian's work from CHI this year}].
% In popular culture, reporters have inquired about
% what ``the future of labor unions'' will look like
% [\textit{Imagine I had cited an online article and
% it wasn't totally inappropriate for a conference}].

% \textit{The parallels here would be difficult to ignore;
% Lily and others have pointed this out,
% but it's worth thinking about the broader trend of
% resisting management,
% driven by routinization,
% enabled by discretization.
% Do that here.}




\section{Bleak Futures}
We've traced a path from piecework itself through
the processes that describe the design and implementation of piece work and crowd work as part of the same thread;
from there, we followed the trail from decomposition to work abstraction,
and the potential to break work away from workers themselves,
which in turn facilitates flexibility for requesters as well as workers



\subsection{Factories}\label{sec:Factorization}

\itshape
So work output is variable, and that's bad. What are we doing about it?
Historically, this is what led to factories; you would pay some experts who understood what they needed to do
and they would form a cohort into which you invest more.

Some research already looks at stuff like investing in workers
\cite{shepherdingDow}.
What else is there that kind of goes in this direction?
Informally, we know that this happens a lot
\cite{jonBrelig}.
AMT, meanwhile, offers requesters the ability to create tasks which are
not just hidden from unqualified workers by default, but completely.
Requesters have taken to using lists of worker IDs which reference
workers who have proven their reliability,
representing a sort of proto--organization of loosely connected workers.
\upshape



\section{Implications for Design}
If it's agreed that
the major topics we've discussed thus far are related and
--- at least to \textit{some} extent ---
precipitated in the fashion we argue,
then we have a rare opportunity as researchers,
and as agents of change in the communities we study,
to affect change on the dynamics of crowd and on--demand work
as they continue to develop.

Without claiming to have easy, cut--and--dry solutions to these problems,
we can nevertheless bring to attention a number of critical opportunities to
learn from historical parallels in piecework and factory labor,
and make informed decisions regarding whether
(or indeed how)
we may want to influence outcomes.
The challenges we bring to attention here are as follows:
\begin{inlinelist}
  \item codifying investment toward collective goods into the designs of systems;
  \item (re--)decentralizing the internet (hey, it's lowercase now!); and
  \item enabling reputation transferral.
\end{inlinelist}


\subsection{Codify the common good}
As \citeauthor{lessig2006code} points out in his book,
digital media give designers the opportunity to design and build into the systems
policies and practices to contribute to the collective benefit of the people therein
\cite{lessig2006code}.
Historically, the confluence of forces \citeauthor{lessig2006code} describes
would ultimately result in outcomes such as benefits for workers,
funds for sick leave and vacation, and other conveniences.
The transient nature of on--demand work would seem to problematize this arrangement,
but we can discuss and explore the viability of building into systems the mechanisms necessary
to save a portion of payment from every gig,
record taxable income, or
myriad other generally administrative tasks automatically.


\subsection{Decentralize the internet (again)}
Digitally mediated on--demand labor markets have historically been insular and incompatible with one another,
forcing workers either to choose one or juggle participation in these markets with great difficulty.
Drivers 

\subsection{Deal with reputation}







% \rule{\linewidth}{1pt}


% \citeauthor{storiesIraniSilberman} describe is that the ``APIs'' in the latter case




% \textit{I've been going back and forth regarding how to frame this paper;
% the approach that tries to look at
% things that are similar vs things that are different
% doesn't seem to work, but
% I've left the thoughts here because it's not all completely bad.}

% \section{Things stay the same}
% How is gig work the same as it's been historically?

% I think this section would be compelling to draw parallels between
% the narratives drivers gave about the flexibility, autonomy, etc\dots
% and that which we might have seen among pieceworkers
% (predominantly women, who benefited from being able to work from home).

% \subsection{Flexibility}
% Are there cultural differences between the people that did piecework and
% the people that do gig work now?
% I'm not sure there are significant differences that have affected the outcomes so far.

% Many of the workers to whom piecework appealed were mothers, wives, etc\dots
% who mostly stayed at home for various reasons
% (certainly largely it was cultural
% --- women weren't afforded equal access to labor opportunities,
% making in--home job opportunities not only compelling, but also one of few available options).

% Gig workers are in some senses similarly constrained:
% workers on Amazon Mechanical Turk
% --- those that use it as a primary source of income, at least ---
% report being homebound for various reasons (e.g. medical, parenting, etc\dots).
% Society and circumstance have made it difficult
% or impossible
% to join the contemporary, conventional workforce;
% gig work re--opens that door.

% But there are differences; during our research over the summer,
% we spoke to drivers on Uber and Lyft, cleaners, and other gig workers.
% Many of them told us about their home lives
% --- about children, spouses, and other commitments ---
% to which they wanted to dedicate more time.
% One driver
% (let's call him Ra\'{u}l)
% told me about how he drove for Lyft
% after working as an inventory manager at a hospital for more than nine years.

% I asked him why he quit that job and forewent
% the benefits,
% predictability,
% and career growth opportunity
% that his old job offered.
% He told me that when his daughter was born,
% he was overcome with a desire to spend more time with her.
% No longer satisfied with work
% where he often left before his daughter would wake up and return after she fell asleep,
% Ra\'{u}l decided to start driving for Lyft,
% because he could drive in the evenings when his daughter was asleep.

% Other drivers reported similar benefits;
% gig work affords its workers flexibility that conventional careers don't allow.

% \subsection{routinization of work}
% This leads into the next section, but I want to bring up the process of
% making work about mass--manufacturing, at least inasmuch as
% the instructions are the same for everyone.

% The Internet arguably has made it much easier to broadcast those instructions,
% but it's had this deeper effect of enabling some amount of back--and--forth
% between the worker and the (algorithmic) manager.

% \subsection{Taylorism}
% The routinization of work makes it possible to measure that process,
% optimize it for certain characteristics, and ultimately
% lead to Taylorism and scientific management.
% This is not new;
% researchers have studied and written about the slow creep of
% algorithmic management and 
% discretization \& routinization of work tasks.

% \textit{We hope to take a step back from the context in which this work is often applied, and
% look for its place in the larger trends and theories
% to make sense of the trends of gig work at large.}

% Industrialization and the automobile assembly line makes this famous,
% but piecework functioned on the principle that everyone was making similar
% or identical
% garments and other products.

% Now, we see Turkers being evaluated on the outcome of their work conforming to norms,
% sometimes bootstrapped,
% as in Ranjay's talk on 
% ``Embracing Error to Enable Rapid Crowdsourcing'',
% but more conventionally in work flows like ``Find--Fix--Verify''.

% \textit{This might be an opportunity to reflect on
% how pieceworkers internalized the work they were doing,
% responded to the stress of the uncertainty of potentially rejected work,
% etc\dots but I'm not familiar with research in that space.}

% \subsection{The emergence of decentralized workplaces}
% The practice of in--home piecework was consumed by
% the centralization of factories
% (the effect of which we'll talk about later, since
% we can talk about how this made unions more practical),
% but for a time many pieceworkers
% at the turn of the \nth{20} century
% worked out of their homes.
% Strikingly, many of the gig workers
% \citeauthor{crowdworkFuture}
% discussed in
% \citeyear{crowdworkFuture},
% and indeed many more continue to do what we might call ``information work''
% --- that is, work that predominantly demands human computation ---
% but increasingly we're seeing the movement toward transient work that largely requires embodied presence
% \cite{uberAlgorithm} (and others).

% \subsection{What was this about?}
% I was going to check in quickly and take a photo of your whiteboard
% (hoping that you'd kept our conversation notes around)
% but your room was \textit{literally} full of people lol.


% \section{Things are changing}
% The medium on which this work is being done
% --- and to an extent the medium used to manage workers ---
% has dramatically changed things as well, however;
% workers are distributed around the world,
% working out of their cars in the cases of livery services
% (notably, never returning to a base of operations)
% across and between cities as well as nations,
% or in their homes
% (paralleling the trend of piecework even more closely).

% Trying to understand how gig work has differed from piecework should at least
% start with looking at the different characteristics of the work involved.
% After that, we should think about how
% the demography and culture of the people engaging in this kind of work
% have changed versus that of the pieceworkers.

% \subsection{Differences in the work itself}
% Gig work has all of the above similarities with piecework,
% but there are key differences.

% % Piecework emerged at the turn of the \nth{20} century,
% % right at the time that telecommunications began to boom in the United States. % too far a reach?
% % \textit{Did telecom enable remote management?}

% Gig work in its contemporary formation is largely mediated by
% ubiquitously accessible digital media
% (the Internet, telephony, etc\dots)
% and importantly has relied on this technology to facilitate the remote management of workers
% \cite{uberAlgorithm}.

% \textit{Has the work fundamentally changed,
% or are we just being managed remotely in different ways?
% I'm not entirely sure.}

% How does this work differ from the experience of being ``on--call'' that is so familiar to retail employees?
% How does this work differ from the work in which truckers, taxi drivers, and other
% independent contractors have been participating for decades?
% We argue that the substantive difference in these markets is the speed of the market itself,
% motivated by the technology which mediates it.
% Because workers can be sourced and dispatched virtually instantly,
% businesses that engage in this kind of work
% (e.g. Uber, Amazon Mechanical Turk, etc\dots)
% have taken to removing other bottlenecks,
% like vetting workers upfront


% \section{More deliberate work}
% \subsection{Differences in the people that do the work}
% ???

\balance{}
\printbibliography


\end{document}

%%% Local Variables:
%%% mode: latex
%%% TeX-master: t
%%% End:
