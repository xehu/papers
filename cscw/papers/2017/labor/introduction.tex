% %!TEX root = trackingWork.tex
\documentclass[trackingWork]{subfiles}
\makeatletter
\def\blx@maxline{77}
\makeatother
\begin{document}
\section{Introduction}
% each sentence feels like a thought that's unrelated to the next one
In the past 10 years, TEST TEST TEST
Amazon Mechanical Turk (AMT) has been the site of a flourishing economy for ``information work''
\cite{paolacci2010running}.

The past decade has seen a proliferation of a new type of labor measured in discrete, atomic tasks and on--off gigs;
the workers that do these tasks float into and out of the labor market so freely that it has been called ``transient''
by some
\onlyinsubfile{\ali{who talks about this? definitely \cite{solinger1995chinese} but not the right context! (although, super interesting)}}.
Stemming from the recognition that
complex tasks might be achievable
--- and indeed trivial ---
through the effective management of the ``crowd'',
researchers have rapidly begun exploring
the space for opportunities to facilitate new (and revisit familiar) forms of work
\cite{bernsteinSoylent,redballoon}.
%  ??? -->
Broadly, this work has imagined the various applications of ``gig''--based work,
and making it work for ``requesters''
--- academics, industries, and to some extent end--users
who might benefit from the product of crowd--powered work
\cite{hong2015group,jonBrelig,paolacci2010running}.
% ??? -->
Since then,
the application of digitally mediated environments acting as rapid ``hiring halls''
for on--demand tasks
has found purchase far beyond the context of ``information work'' like Amazon Mechanical Turk (AMT);
livery services (Uber), cleaning (Handy), and various other services (TaskRabbit)
% parentheticals giving examples?
have garnered the interest of both industry and research.

Researchers have made efforts to understand the \textit{people} that have gravitated toward on--demand work
since its emergence and popularization,
but as the form of work has grown and changed, so too have the demographics of workers
\cite{Ross,whoareNOTtheTurkers}.
Some of this research has been motivated by the realization of the sociality of gig work,
and the frustration and disenfranchisement that these systems embody
\cite{turkopticon,dynamo}.
Other work has focused on the outcomes of work,
reflecting on the resistance workers express against digitally mediated labor markets
\cite{uberAlgorithm}.
Bringing this broader body of work together, perhaps we can think of this as documenting the ``fallout'' of on--demand labor;
workers increasingly expressing frustration with the disenfranchisement of the work--flows pioneered by researchers in the decade leading up till now,
and researchers beginning to appreciate the unforeseen costs of transient labor and the impacts both on the work and the workers.

\subsection{Piecework as a lens to understand gig work}
These bodies of literature have conversed,
and indeed begun to identify that their theories are interrelated in some way,
but thus far a convincing model for making sense of
--- let alone predicting ---
the phenomena in on--demand labor has proven elusive.
For instance, 
making sense of why workers on AMT (or ``Turkers'') are frustrated with their workplace
is understandable in its own context, but
the relationship between the work itself (``turking'')
and the myriad social outcomes has only briefly been explored
\cite{crowdcollab}.

We offer a framing for \textit{all} on--demand work as a modern instantiation of ``piecework'',
referenced briefly by
\citeauthor{crowdworkFuture}
in
\citeyear{crowdworkFuture}
as a loose analogy to the form of work emerging at the time,
but not given as much weight as it perhaps could have been
\cite{crowdworkFuture}.
``Piecework'' here makes use of
a term historically used to describe work done in the home,
in manageable tasks,
often involving clear instructions
and payment only for work completed, not work done
(the differentiation then being that
one would be paid for the \textit{output} of the work,
not the \textit{duration}).

This framing prompts us to note several immediately apparent similarities:
% Given the scope,
% we can frame piecework and on--demand labor
% as sharing these important similarities:
\begin{inlinelist}
\item this form of work began in the home
\item the worker is paid for each discrete piece of work done, regardless of time or effort; and
\item the worker's status
(not only socially, but also economically)
is ambiguous, or at least the subject of some controversy.
\end{inlinelist}

% In the past decade, researchers have observed frustration
% grow among on--demand workers,
% with expression of this frustration spanning a wide range of tactics
% \cite{uberAlgorithm,turkopticon,dynamo}.
% Attempting to make sense of these case studies has been challenging
% in part because
% a wholly encompassing framework for understanding this form of work
% has thus proven difficult to capture.


But more than this,
the framing of on--demand labor as piecework (re--instantiated)
allows us to attempt to make sense of the broader research on this new form of work
by evaluating this work through a much more refined lens.
More concretely, by looking at task--based or ``gig'' work as
an instantiation (or even a continuation) of piecework,
and by looking for patterns of behavior that the corresponding literature predicts
on this basis, we can
\begin{inlinelist}
  \item make sense of the phenomena so far as part of a much larger series of interrelated events;
  \item bring into focus the ongoing work among workers, system--designers, and researchers in this space; and finally,
  \item offer predictions of what social computing researchers,
        and workers themselves,
        should expect to see on the horizon of on--demand work.
\end{inlinelist}

We'll look at a broad range of cases under a number of major themes
we propose as broadly describing the types of research being done in crowd work
and more generally in what we argue is contemporary piecework.
After validating this lens as a way of reasoning about on--demand labor,
we'll attempt to use this perspective to suggest areas of research worth anticipating,
and developments we should expect to see in the maturation of digitally mediated work.
Finally, we will offer design implications based on this research\dots \ali{???}

\onlyinsubfile{
\itshape
\subsection{notes/what i want to get across\dots}
\begin{itemize}
  \item on--demand labor has consumed an increasing proportion of the labor market in the past ten years [\textit{Pew research on on--demand labor}];
  the work  has ranged from Turk work processes
  \item some of it has been about resisting Uber \& other gig labor platforms \cite{bernsteinSoylent,uberAlgorithm};
  some researchers have tackled understanding this new instantiation of work, but come up short (\textit{how}).

  \item (what binds this together)
  \item we offer a framing on this topic that situates the research so far on a timeline of the maturation of another form of work that emerged approximately 150 years ago
  --- that is, piecework --- which convincingly suggests that on--demand labor in the form of Turking, driving for Uber, etc\dots are in fact little more than piecework re--surging today.

  \item this paper will trace the body of research describing microwork --- and later gig work --- and place it in the context of historical piecework and the industrial revolution as a whole.
  \item informed by this new lens, we'll then turn our attention to the future of this re--emergent form of work, and suggest ways that researchers and users of on--demand labor might influence the outcomes we predict in this paper.
\end{itemize}
\upshape
}

\end{document}