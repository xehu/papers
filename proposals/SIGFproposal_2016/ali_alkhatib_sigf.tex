% !TEX program = xelatex
\documentclass{article}
\usepackage{balance,graphics,setspace,hyperref,etoolbox,refcount}
\usepackage{graphicx,sidecap,fancyhdr,titling,blindtext,fontspec}
\usepackage{xltxtra,indentfirst,xunicode,xesearch}
\usepackage[margin=1in]{geometry}
\usepackage[inline]{enumitem}
\usepackage[tiny,compact]{titlesec}
\usepackage[square,numbers]{natbib}
\bibliographystyle{newapa}

\titlespacing*{\section}
{0pt}{4pt}{0pt}

\titleformat*{\subsection}{\it}

\setmainfont[
Ligatures=TeX,
Extension=.otf,
UprightFont= *-regular,
BoldFont=*-bold,
ItalicFont=*-italic,
BoldItalicFont=*-bolditalic]{texgyretermes}

\setmonofont{Source Code Pro}

\newcounter{countitems}
\newcounter{nextitemizecount}
\newcommand{\setupcountitems}{%
  \stepcounter{nextitemizecount}%
  \setcounter{countitems}{0}%
  \preto\item{\stepcounter{countitems}}%
}
\makeatletter
\newcommand{\computecountitems}{%
  \edef\@currentlabel{\number\c@countitems}%
  \label{countitems@\number\numexpr\value{nextitemizecount}-1\relax}%
}
\newcommand{\nextitemizecount}{%
  \getrefnumber{countitems@\number\c@nextitemizecount}%
}
\makeatother
\AtBeginEnvironment{inlinelist}{\setupcountitems}
\AtEndEnvironment{inlinelist}{\computecountitems}
\AtBeginEnvironment{enumerate}{\setupcountitems}
\AtEndEnvironment{enumerate}{\computecountitems}
\AtBeginEnvironment{itemize}{\setupcountitems}
\AtEndEnvironment{itemize}{\computecountitems}

\pagestyle{fancy}

\newcounter{words}
\newenvironment{counted}{%
  \setcounter{words}{0}
  \SearchList!{wordcount}{\stepcounter{words}}
    {a?,b?,c?,d?,e?,f?,g?,h?,i?,j?,k?,l?,m?,
    n?,o?,p?,q?,r?,s?,t?,u?,v?,w?,x?,y?,z?}
  \UndoBoundary{'}
  \SearchOrder{p;}}{%
  \StopSearching}


\lhead{} % controls the left corner of the header
% \chead{} % controls the center of the header
\rhead{Ali Alkhatib} % controls the right corner of the header
\lfoot{} % controls the left corner of the footer
\cfoot{} % controls the center of the footer
\rfoot{Page~\thepage} % controls the right corner of the footer
\renewcommand{\headrulewidth}{0.0pt}
\renewcommand{\footrulewidth}{0.0pt}

% \thispagestyle{empty}% Removes the header from the first page. Change plain to empty to remove the numbering entirely.

\setlength{\parskip}{.2em}

\newlist{inlinelist}{enumerate*}{1}
\setlist*[inlinelist,1]{
  label=\arabic*),
}

\setlist[itemize]{itemsep=0pt,topsep=0pt,parsep=0pt}
\setenumerate{itemsep=0pt,topsep=0pt,parsep=0pt}
\newcommand{\sectitle}[1]{\textit{#1}}
\title{Designing for Collective Governance in the Gig Economy}
\author{Ali Alkhatib}
\date{Submitted \today}


\begin{document}
\begin{titlingpage}
    \maketitle
    % \begin{counted}
    \begin{abstract}
Digitally mediated marketplaces have led to a resurgence of ``gig'' work,
wherein workers are connected by the Internet rather than geography.
Without shared physical spaces,
disenfranchised workers struggle to coordinate and take collective action. I will engage in fieldwork with digitally managed workers in information work,
domestic services,
and on--demand driving to understand how workers resist algorithmic systems and act collectively. I will then design and launch a new cooperatively--governed platform for work,
in collaboration with the National Domestic Workers Alliance (NDWA),
to develop frameworks and tools to facilitate collective action and governance.
    \end{abstract}
    % \end{counted}
% \footnotesize{word count: \thewords~(100 max)}
\end{titlingpage}


\section*{Background}

Today, with the growth of digitally mediated systems to manage workers and decontextualized pieces of work,
\textit{gig} work
--- characterized by the discrete jobs (or \textit{gigs}) which workers agree to do  ---
has increasingly become the focus of public and academic interest
\citep{martin2014being,uberAlgorithm,hall2015analysis}.
But gig work platforms have achieved some level of notoriety as settings of disempowerment
fueled by information asymmetry and opaque algorithmic management;
while researchers have made efforts to empower workers,
perhaps most notably on Amazon Mechanical Turk (AMT),
these efforts have thus far only ameliorated,
rather than resolved,
tension between
the various stakeholders of these systems
\citep{turkopticon,dynamo}.
This persisting frustration among \textit{Turkers} (the colloquial name for workers on AMT) parallels the frustration
workers in the \textit{gig} economy
--- namely, on--demand services such as Uber and Handy
\citep{dillahunt2016does,uberRiots,fedsUber,hall2015analysis}.

With gig work increasingly representing the labor paradigm under which people work,
and with this mode of labor being exported to other parts of the world,
it is not only important that we attempt to understand the culture of those engaged in the discretization of work,
but that we attempt to participate in this process with design interventions
to improve the circumstances of workers by empowering them.

Working with the National Domestic Workers Alliance (NDWA),
I began to explore methods of empowering workers and ensuring their benefits, safety, and well--being.
Historically, labor unions have advocated on behalf of workers,
typically representing an industry or company's workforce.
In the case of gig work,
the geographic diffusion associated with digitally mediated systems
makes collective organization difficult.
More importantly, however,
the findings we uncovered among Turkers during our work on Dynamo
suggests that gig workers are reluctant regarding what they perceive labor unions to represent.
As one Turker described the sentiment I found more generally,
``\dots~I would not feel comfortable taking part [in a labor union].
It runs against my grain because I am an individualist.
[\dots] I consider myself self--employed\dots~not working for anyone in particular''
\citep{dynamo}.

This cultural disconnect between workers and the perceived qualities of labor unions presented us with an opportunity
to explore more creative ways of dealing with the adversarial relationships between workers and managers.
In this search, we found that \textit{worker cooperatives}
--- which have struggled to reach large scales in the past in part due to the mounting struggle associated with growing collectively governed groups ---
may in fact be feasible given the unique features of the Internet, 
described by Miller and others,
which make it possible for people to collaborate rapidly and cheaply,
obviating many of the concerns that typically stymie efforts to organize,
such as geography
\citep{miller2011understanding}.

I propose to conduct ethnographic research on gig work in digitally mediated platforms,
through the historical lens of \textit{piece work},
to better understand the characteristics that influence and affect the culture of gig workers today.
The ethnographic research we conduct will yield an \textit{emic} understanding of
the values gig workers hold and their sense of self as they relate to the work they do.
With this ethnographic data,
I will then design and build a worker cooperative in collaboration with gig workers and the NDWA.
This system will reflect on the ethnographic findings we will have made from the first study,
as well as test guidelines toward enabling and fostering collective governance strategies.


While researchers have contributed to our understanding of worker cooperatives and participatory democracy,
we nevertheless find that worker cooperatives struggle to grow to the scales that firm--owned companies do.
Hardin, Olson, and others offer general framing of collective governance,
especially as it relates to the instances of collective action I explored in our work on Dynamo
\citep{russell1982collective,olsonlogic}.
Ostrom in particular provides substantive guidance on the mechanisms necessary to facilitate collective governance
\citep{ostrom1990governing}.

Ultimately, while we know how to design and engineer systems for collaboration,
and in particular we can generate instances of collective action where the target action is finite and focused,
we do not yet know how to design online systems for the sustained collective governance that we see only occasionally
(e.g. Wikipedia)
\citep{catalyst,dynamo,foundry}.
The core questions in this research, exploring the mechanisms of collective governance on the Internet, are social ones.

\section*{Research plan}
As stated previously,
a critical component of the research proposal consists of
understanding the culture of gig workers.
Indeed, this proposal consists of two distinct research endeavors:
\begin{inlinelist}
\item Ethnographic study of gig workers, to understand the ways that workers do and indeed may support one another, and
\item design interventions consisting of the creation of a cooperative labor market designed collaboratively with workers.
\end{inlinelist}

\subsection*{Study 1: Ethnographic study of gig workers}
The first study will consist of ethnographic fieldwork studying workers in modern piece work labor markets 
--- e.g., Mechanical Turk, Uber, and Handy ---
to understand how social coping strategies emerge among workers.
How do workers air their grievances?
How do they come together to push for change?
How have they adapted to the working conditions of these digital hiring halls?

I will perform one year of fieldwork with digital piece workers.
To understand their breadth of experience, I will sample roughly fifty workers from information work
(for example,
Mechanical Turk and CrowdFlower),
and another fifty workers in the ``sharing economy''
(including Uber,
Lyft,
and AirBnB),
and fifty in gig work
(namely Handy and TaskRabbit).
Given existing relationships and reputations as credible researchers in communities of Turkers, I can recruit participants through known existing channels.
In the cases of platforms such as Uber,
AirBnB,
and Handy, I can recruit workers for nominally legitimate work
(for example,
requesting that a driver drives for 15 minutes to a previously selected destination)
and invite them to participate in our interview.

\subsection*{Study 2: Design and engineering of a cooperative labor market}
The second component of this research will involve
the engineering and launch a cooperatively owned ``gig" marketplace for domestic workers
in collaboration with the National Domestic Workers Alliance
(NDWA).
We will contribute the first digitally mediated gig labor market collaboratively designed with
workers and labor advocacy groups.
% My prior work established a collective action platform for crowd workers
% \citep{dynamo}.
Informed by my work on Dynamo and the ethnographic fieldwork in the first study,
I will then seek to augment this exploration of union--style collective by creating a cooperative that can scale up and set its own standards in the marketplace.
The NDWA's Fair Care Labs are providing access to interested domestic workers
(house cleaners), and we are designing and implementing the platform with a pilot group.

The goal of this research is
to understand how to design for a set of distributed workers to achieve consensus on issues such as
management, service fees, and arbitration.
Through the creation of this labor market and platform, we will explore and test theories behind
enabling collective governance, consensus--building, and participatory democracy.
This study will yield guidelines for system--designers to spark and foster these values.

Together with the Fair Care Labs, we are designing and launching a cooperative gig labor market for domestic workers (house cleaners).
This cooperative labor market, Alia, is designed for domestic workers in the California Bay Area.
However, our interest (and Fair Care Labs') is to generalize its insights to other workers and industries.
While the driving domain is already crowded with services such as Lyft and Uber, domestic work is a more ideal area to launch a research prototype.
In addition, the NDWA Fair Care Labs' support and network will help us gain rapid traction.


\section*{Interdisciplinarity}
For this research to succeed, the methodological approaches of Political Science
and the engineering skills of Computer Science must not only be in conversation, but in concert.
The challenge in designing for collective governance,
as Hardin (1982) differentiates it from collective action,
is such that collective governance
``takes on the character of a nearly anthropological investigation of minute interrelationships''
\citep{russell1982collective}.
In order to design a system that empowers workers,
a researchers must first attempt to understand a wide range of workers' circumstances:
The needs of workers;
the contexts in which they work;
their relationships with one another, with other groups, \& with institutions such as governments;
and more generally the paradigmatic views of gig workers.
Only then can one reasonably hope to design a system consistent with the views and broader culture of gig work.

Lee et al. and Gray et al. have identified a number of ways that workers subvert and circumvent the intents of system--designers,
both in digital workplaces and where work is simply mediated digitally
\citep{uberAlgorithm,crowdcollab}.
These patterns of behavior elude algorithmic tracking and measurement because they deliberately avoid the \textit{a priori} assumptions made by the designers of systems,
who attempt to structure these sites of work in ways to create incentives for preferred behavior.
Identifying and understanding the details of this behavior thus begins with a qualitative,
ethnographic endeavor.

\section*{Anticipated Contributions \& Implications}
This research is important given the recent trends toward the discretization of work.
With increasing numbers of people joining the gig economy,
and lacking sufficient work toward digitally mediated forms of collective action and governance,
disenfranchised and marginalized laborers in particular stand to lose significantly.
Indeed, advances made by labor advocacy groups in the past century have already eroded
due to the classification of many of these workers as ``contractors'' rather than ``employees'',
signaling the larger trend of the reversal of benefits and protections for which workers have fought.

More broadly, this project's will speak to the critique that the Internet,
far from the democratizing force that Barlow predicted it would be,
has become an infrastructure facilitating the corporate centralization of power
\citep{barlow2009declaration,jones2011does,EboCybertopia}.
The success of this project will therefore not only
bring a more nuanced understanding of the shifting climate and culture of labor,
but may
--- in a small way ---
deliver on the promise Barlow made that
cyberspace and the Internet would be a tool for democratizing access to information, and with that,
empowering more than the tech elite who engineered it.
% but will add to the increasingly minority claim that the Internet should be a democratizing force,
% rather than a centralizing one;
% \citep{barlow2009declaration}.

% To ensure that system--designers have reasonable access to our findings,
% our research targets will be top--tier venues in Human--Computer Interaction
% which support open access publication paradigms
% (e.g., ACM CHI, ACM CSCW).
% A second set of outlets will be through the NDWA's relationships with labor advocates and the press.

\pagebreak

% \bibliographystyle{acm}
\bibliography{../../references.bib}
\end{document}