\documentclass{article}
\usepackage{xr-hyper}
\usepackage{todonotes,txfonts,balance,graphics,color,alphalph,subfiles,appendix,endnotes}
\usepackage{booktabs,textcomp,microtype,ccicons,enumitem,nth,cleveref,textcase,comment}
\usepackage[normalem]{ulem}
\usepackage{url}
% \usepackage{natbib}
\usepackage[citestyle=numeric,backend=bibtex]{biblatex}
\usepackage[T1]{fontenc}
\usepackage[pdflang={en-US},pdftex]{hyperref}
\usepackage{mfirstuc,lastpage}
\usepackage[en-GB]{datetime2}
\usepackage{fancyhdr,tabularx}
\usepackage[all]{hypcap}  % Fixes bug in hyperref caption linking
\usepackage[utf8]{inputenc} % for a UTF8 editor only


\usepackage{mfirstuc,lastpage}
\bibliography{references}
\usepackage[margin=1in]{geometry}
\pagestyle{fancy}


\lhead{} % controls the left corner of the header
% \chead{} % controls the center of the header
\rhead{Ali Alkhatib} % controls the right corner of the header
\lfoot{} % controls the left corner of the footer
\cfoot{} % controls the center of the footer
\rfoot{Page~\thepage} % controls the right corner of the footer
\renewcommand{\headrulewidth}{0.0pt}
\renewcommand{\footrulewidth}{0.0pt}

\thispagestyle{empty}% Removes the header from the first page. Change plain to empty to remove the numbering entirely.

% \setlength{\parskip}{.2em}

% \newlist{inlinelist}{enumerate*}{1}
% \setlist*[inlinelist,1]{
%   label=\arabic*),
% }

% \setlist[itemize]{itemsep=0pt,topsep=0pt,parsep=0pt}
% \setenumerate{itemsep=0pt,topsep=0pt,parsep=0pt}
% \newcommand{\sectitle}[1]{\textit{#1}}
\title{Designing for Complex Gig Work and Careers}
\author{Ali Alkhatib}
\date{Submitted \today}


% take half 
% motivating work: historical analysis of crowd 
% gig work instead of on--demand work
%
%
%
%
\definecolor{Blue}{HTML}{0000FF}
\newcommand{\topic}[1]{{\color{Blue}#1}}
\definecolor{Gray}{HTML}{BABABA}
\newcommand{\ugh}[1]{{\color{Gray}\textit{#1}}}
\newenvironment{blah}{\par\color{Gray}\itshape}{\par}


\begin{document}
  % \maketitle
  \begin{abstract}
Gig work has become a new instantiation of piecework.
Further, gig work platforms allow employers to geographically disperse workers and silo workers' information,
making organization and coordination all but impossible,
and stymieing work specialization and expertise.
Our research will draw on our past work, using historical analysis of piecework to inform this contemporary phenomenon,
to design and develop a ``digital hiring hall''.
Our research will
shed light on the potential complexity limits of gig work,
point to possible futures of labor advocacy for gig workers,
and offer some insights into the best practices of principally online collective governance.

% The results of this research will make it easier for gig workers to identify and focus on specializations,
% fostering rewarding careers in gig work and giving workers spaces to organize and coordinate.

% Our goal is to both understand the factors that drive collective behavior in supposedly distributed labor platforms,
% and to build infrastructures that enable the growth of worker collective action in digitally mediated on--demand labor.
% Our results shed light on the potential for digital piecework cooperativism,
% as well as the policies necessary for it to succeed. % ~108 words
  \end{abstract}


\section*{Motivating Work: Historical Analysis of Piecework}
\topic{Piecework has historically disempowered workers and empowered employers, and
many the same dynamics of disenfranchisement and frustration are poised to return today.}
In our most recent work~\cite{pieceworkCrowdworkGigwork}, we found that a number of aspects of piecework
--- a form of work payment that began more than a century ago ---
bears striking resemblance to the contemporary labor we call ``gig work''.
By subdividing work and paying for each task, employees are driven to work faster and longer,
affording them little control over working conditions.
And for most of its history, piecework yielded no benefits, occupational health and safety, or social insurance.
When piecework combined with job routinization, workers became interchangeable and deindividuated.
This combination of piecework and routinization produced the dominant payment system during the industrial era,
starting with agriculture and home production but quickly moving into factories.
However, by the turn of the \nth{20} century in the United States,
campaigns for workers' rights yielded regulations on conditions, and by the late \nth{20} century,
piecework was associated primarily with migrant labor and sweatshops.
This contracted piecework economy has remained relatively stable since.

\topic{Today, with the growth of on--demand labor, piecework is reemerging in new forms.}
Networked computational infrastructure is now mediating each unit of work and worker.
Uber's taxi drivers are paid per ride and assigned jobs by an algorithm~\cite{uberAlgorithm,hall2015analysis}.
Information workers, such as those on Amazon Mechanical Turk and Upwork, are paid per task,
performing vast quantities of data and administrative work~\cite{martin2014being}.
Increasingly numerous workers make money through one--off tasks like housecleaning and food delivery, all mediated by online platforms.
Much of this work has been limited in complexity to easily verifiable work, stymieing gig workers' career opportunities
in many of the same ways that historically limited information workers' careers~\cite{grier2013computers}.

\begin{comment}
Current legal and policy frameworks classify these workers as independent contractors, precluding them from accessing employee benefits and job security associated with employment.
Complicating matters, today's workers often despise unfair employers but prize the flexibility and autonomy afforded by their platforms~\cite{martin2014being}.
Many view themselves as masters of their own fate, free to enter and leave the workforce at will and without repercussion.
Despite this, they still seek collective representation, as Uber drivers have done repeatedly to combat falling rates.
% that bit's a little/a lot too much about collective representation. not really the right tact, right?
\end{comment}


\begin{comment}
% this feels a little collective action--y;
% should i cut it and reframe this to be more about individuals making their work more complex/having a better trajectory?
\topic{How do workers react to, and create collective counterbalances to, the digital piecework economy?}
On the one hand, studies of online communities predict that workers online will face greater challenges, because
trust is more difficult to build in digitally mediated environments~\cite{successfulOnlineCommunities,kollock2005managing,cook2005cooperation}.
Without factory floors, water coolers, and shared neighborhoods~\cite{waterCooler},
how do workers coordinate and build solidarity?
On the other hand,
online forums may make it easier for workers to share information and strategies,
and successful collaborative efforts such as Wikipedia suggest that the online environment may support forms of cooperation that could not have succeeded as readily offline.
\end{comment}

\begin{comment}
\topic{As yet,
Will their fates follow those of historical offline pieceworkers,
or will the affordances of the online world result in different outcomes?
In our most recent work~\cite{pieceworkCrowdworkGigwork}, we argue that
the trajectory of historical pieceworkers can substantively inform the trajectory of gig work today,
but that we must take an active role in the design and development of these platforms
--- that is, taking advantage of the potential to influence the entire architectures as \citeauthor{lessig2006code} points out~\cite{lessig2006code}.
\end{comment}

\topic{In our examination of crowd work and gig work as a reinstantiation of piecework~\cite{pieceworkCrowdworkGigwork}, we identify a number of parallels that relate to this question.}
First, that the first ``human computers'' were geographically distributed and only retained for several years to stifle career development opportunities~\cite{grier2013computers};
second, that effective uses of historical piecework were generally limited to easily measured work due to a general mistrust~\cite{american1921problem,richards1904anything};
third, this mistrust led to such a starkly adversarial relationship between workers and managers that
it may have precipitated the wave of labor advocacy which defined the first half of the \nth{20} century~\cite{hart2013rise,hart2016rise}.
What remains then are serious questions about the future of gig work as informed by piecework's history.
We have limited knowledge of how gig workers today are adapting to digitally mediated, constantly changing labor markets.
Similarly, we know very little about gig work's potential to support complex work, let alone how to turn that potential into reality.

\begin{comment}

We propose paired grounded ethnographic fieldwork and system development to 1) understand the forces shaping informal and formal worker reactions to the online piecework economy,
and 2) create a technical and design foundation for collective action,
ownership and governance in online piecework.


First, we will analyze worker behavior in the current sociotechnical ecosystem through a year of fieldwork on major Mechanical Turk
(data entry)
platforms and forums,
and on gig economy platforms such as Uber
(driving)
and Handy
(domestic work).
Through this fieldwork, we will investigate the techniques that workers use to manage their work environments and job opportunities.
We will seek to understand the challenges they face when balancing
an identity of being self--employed with an identity of being a pieceworker for a privately--owned labor platform, and
the techniques that workers and employers use to circumvent each other's restrictions, barriers, and behaviors.

Second, we will create and launch a cooperatively owned ``gig" marketplace for domestic workers in collaboration with the National Domestic Workers Alliance
(NDWA).
Our own prior work established a collective action platform for crowd workers~\cite{dynamo}.
We now seek to augment this exploration of union--style collective by creating a cooperative that can scale up and set its own standards in the marketplace.
The NDWA's Fair Care Labs are providing access to interested domestic workers
(house cleaners), and we are designing and implementing the platform with them as a pilot group.
The goal is to understand how to design for a set of distributed workers to achieve consensus on issues such as management, service fees, and arbitration.

Our team is uniquely qualified to pursue this opportunity.
Levi has years of experience studying labor, collectives, and unions.
She is pairing with Bernstein, a computer scientist who studies online labor platforms and collective action.
Bernstein, alongside PhD students Salehi and Alkhatib, created the first platform for collective action efforts amongst digital pieceworkers, focused on the Amazon Mechanical Turk platform.
This effort has galvanized hundreds of workers and resulted in outcomes including public media campaigns and the creation of ethical work guidelines for Mechanical Turk employers.
\end{comment}

\topic{This work carries policy implications for the emerging, and as yet largely unregulated, gig economy.}
Will gig work represent empowered, independent professionals who freely control their destinies, or 
will gig work become a geographically distributed ``digital sweatshop''~\cite{dawnDigitalSweatshopCushing}?
Whereas researchers had relatively little control over the markets that emerged throughout the \nth{20} century,
researchers of digital sites of work have the ability not only to influence, but to architect these settings~\cite{lessig2006code}.
It can be argued that with a few lines of code, we can effect outsize change on the future of gig work
--- a form of work that continues to grow by the year~\cite{pewSharing,pewSharing24percent}.

% What affordances and protections do workers need in order to make their voices heard and to have collective power?
% What protections should platforms be required to provide to their workers?
% What support is necessary for people who work on multiple platforms, frequently moving between different employers?
% Are unions the answer, or some other form of worker organization?

\section*{Research plan}
\topic{We intend to build a system that tracks gig workers' professional histories across myriad work platforms,
aggregating \& consolidating that data, and
allowing workers to curate their professional identities, instantiated by a ``digital r\'{e}sum\'{e}''.}
A number of open questions stand to be informed by this endeavor:
first, do tools and systems like these help various stakeholders (such as workers, or requesters) in measurable ways?
And second, what can we learn about collective governance from a system that is designed and operated collectively with workers themselves?

\topic{At a high level, we can study the usefulness of this system using standard experimental approaches.}
First, through quantitative methods (for example, comparing workers' earnings over time); and
second, through qualitative  methods (for example, through interviews to determine whether workers find and exploit specialty niches as a result of this tool).
We can also explore employers perspectives in tangible ways;
for example, given concerns about the quality of work~\cite{Ipeirotis:2010:QMA:1837885.1837906,le2010ensuring,Law:2017:CTR:2998181.2998197},
we will investigate whether a system that better communicates worker expertise allows employers to rely more confidently on crowdwork.

\topic{More broadly, we can think of this as an opportunity to engage in participatory design and democracy, and
to learn best practices through experimentation.}
We can take this opportunity to study collective governance in online settings
--- extending, as \citeauthor{russell1982collective} describes it,
from the ``on--shot'' cases of collective action that we instantiated in our earlier work~\cite{dynamo} to ``ongoing'' collective action~\cite{russell1982collective}.
While considerable effort has gone toward collective governance
(e.g.~\cite{ostrom1990governing,mccallum2013global,ahlquist2013interest,olsonlogic,polletta2002freedom}),
many of these mechanisms rely on collocation to some extent or another;
the best practices of collective governance for communities that never meet
(e.g moderators and administrators on Wikipedia)
remains unclear.



\begin{comment}
\topic{We're pursuing the design of a system that offers a very narrow slice of workers' needs based on
what we've learned so far from gig workers, and from the historical timeline that we saw in piecework.}
Informed by our observations from fieldwork while organizing and motivating collective action~\cite{dynamo}
and from other research~\cite{martin2014being},
we've found that gig workers are highly resistant to traditional labor advocacy and empowerment tools, such as labor unions.
In part, this may be because the conventional systems of affiliations and labor unions are a poor match;
to say nothing of responding to the varied needs of people who participate in the gig economy to differing degrees,
what we consider conventional labor advocacy structures is anathema to gig workers:
``If you mean a `labor union', I would not feel comfortable taking part.
It runs against my grain because I am an individualist.
[\dots]~I consider myself self--employed\dots~not working for anyone in particular''.
Looking through piecework's history, we found that similar sentiments existed in the early years of piecework~\cite{pieceworkCrowdworkGigwork}.
While this sentiment shifted over the following decades, it's worth considering that this shift occurred over a number of years, and
only as people became increasingly familiar with the conclusion that forms of gig 
\end{comment}

\begin{comment}
We intend to create a platform that can become a template for online work that better represents workers' needs.
However, solutions that exist for traditional, non--gig workers
--- especially labor unions ---
face resistance from workers in the gig economy~\cite{martin2014being,dynamo}.
This distaste may be due to a general decline in public opinion of unions.
However, the traditional mechanics of a labor union are also a poor match: challenging issues include managing the benefits and proportional representation of workers whose contracts may last only minutes or hours, enforcing collective decisions on a decentralized network of workers, and establishing trust between workers who may never meet face--to--face.
In our early fieldwork, for example, a worker on Amazon Mechanical Turk stated:
``If by `union' you mean a `labor union', I would not feel comfortable taking part.
It runs against my grain because I am an individualist.
[\dots] I consider myself self--employed\dots\  not working for anyone in particular.''

Given these constraints, what alternative model would give workers more power in an online gig--economy platform while still enabling a market to emerge and function? To explore one such alternative, we will collaborate with the National Domestic Workers Alliance's innovation arm, the Fair Care Labs, to create tools for online pieceworkers to collectively manage and govern their own labor markets.
While the literature in social computing design and crowdsourcing already features many tools for successful decentralized collaborations (e.g., Wikipedia), and collective action movements (e.g., change.org), templates and guidelines for sustained collective action and governance online in high--stakes situations that affect workers' livelihoods are rare.
Such efforts tend to swell quickly, but soon find themselves embroiled as they struggle to satisfy myriad stakeholders .
Our goal is to introduce design mechanisms to enable collectives of gig workers to form, debate policy, and implement their decisions over a long period of time.

Together with the Fair Care Labs, we are designing and launching a cooperative gig labor market for domestic workers (house cleaners).
This cooperative labor market, Alia, is designed for domestic workers in the California Bay Area.
However, our interest (and Fair Care Labs') is to generalize its insights to other workers and industries.
While the driving domain is already crowded with services such as Lyft and Uber, domestic work is a more ideal area to launch a research prototype.
In addition, the NDWA Fair Care Labs' support and network will help us gain rapid traction. The web and mobile application, Alia, enables the basic affordances of ``gig'' work for domestic work
--- rapid scheduling of any available worker in the network (Figure 1).
\end{comment}


\section*{Interdisciplinarity}
\topic{This research will call on methods and skills originally from and generally found in disparate fields.}
As we showed with our paper examining the relationship between piecework and gig work~\cite{pieceworkCrowdworkGigwork},
the lens of a social scientist can inform conversations about digitally mediated work in ways that many computer scientists are not otherwise able to offer.
At the same time, the perspectives of computer scientists
--- and the challenges digitally mediated spaces bring with them ---
offer to reinvigorate deeply theoretically grounded discussion and offer field sites with which to experiment and test these theories.

\begin{blah}
In order to design a system that empowers workers,
a researchers must first attempt to understand a wide range of workers' circumstances:
The needs of workers;
the contexts in which they work;
their relationships with one another, with other groups, \& with institutions such as governments;
and more generally the paradigmatic views of gig workers.
Only then can one reasonably hope to design a system consistent with the views and broader culture of gig work.
\end{blah}

\begin{blah}
\citeauthor{uberAlgorithm} and \citeauthor{crowdcollab} have identified a number of ways that workers subvert and circumvent the intents of system--designers,
both in digital workplaces and where work is simply mediated digitally~\cite{uberAlgorithm,crowdcollab}.
These patterns of behavior elude algorithmic tracking and measurement because they deliberately avoid the \textit{a priori} assumptions made by the designers of systems,
who attempt to structure these sites of work in ways to create incentives for preferred behavior.
Identifying and understanding the details of this behavior thus begins with a qualitative,
ethnographic endeavor.
\end{blah}

\section*{Anticipated Contributions \& Implications}
This research is of significant importance given the recent trends in the ``gig economy''.
With more and more people joining the workforce in this capacity~\cite{pewSharing,pewSharing24percent},
concerns about ``the future of work''~\cite{waterCooler,crowdworkFuture} continue to grow.
Without compelling career trajectories in this space,
the optimistic prediction seems to be that we will have little say in the outcome of this shift in work.
Our hope is that, concretely, the artifact produced by this research raises the limit on the complexity of work that can reliably be crowdsourced.
More broadly, we expect to offer two major contributions to the Human--Computer Interaction community:
first, an existence proof that a gig work market for complex, specialized workers is possible; and
second, insights into how best to scaffold and foster ongoing collective governance.
% and lacking sufficient work toward digitally mediated forms of collective action and governance,
% disenfranchised and marginalized laborers in particular stand to lose significantly.
% Indeed, advances made by labor advocacy groups in the past century have already eroded
% due to the classification of many of these workers as ``contractors'' rather than ``employees'',
% signaling the larger trend of the reversal of benefits and protections for which workers have fought.

This project's will speak to the critique that the internet,
far from the democratizing force that Barlow predicted it would be,
has become an infrastructure facilitating the corporate centralization of power~\cite{barlow2009declaration,jones2011does,EboCybertopia}.
The success of this project will therefore not only
bring a more nuanced understanding of the shifting climate and culture of labor,
but may
--- in a small way ---
deliver on the promise Barlow made that
cyberspace and the Internet would be a tool for democratizing access to information, and with that,
empowering more than the tech elite who engineered it.
% but will add to the increasingly minority claim that the Internet should be a democratizing force,
% rather than a centralizing one;
% \cite{barlow2009declaration}.

% To ensure that system--designers have reasonable access to our findings,
% our research targets will be top--tier venues in Human--Computer Interaction
% which support open access publication paradigms
% (e.g., ACM CHI, ACM CSCW).
% A second set of outlets will be through the NDWA's relationships with labor advocates and the press.

\pagebreak
\printbibliography{}
% \begin{document}
% \bibliographystyle{acm}
% \bibliography{references.bib}
\end{document}