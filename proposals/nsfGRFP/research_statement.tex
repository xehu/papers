\documentclass[12pt]{article}
% Load basic packages
\usepackage{balance,graphics,times,parskip,setspace}
\usepackage[pdftex]{hyperref}
\usepackage[margin=1in]{geometry}
\usepackage[utf8]{inputenc}
\usepackage[inline]{enumitem}
\singlespacing
\usepackage[tiny,compact]{titlesec}
\setlength{\parskip}{.4em}

\newcommand{\sectitle}[1]{\textbf{\MakeUppercase{#1}}}

% \setlist[enumerate,1]{%
%   label=\arabic*.,
% }

\newlist{inlinelist}{enumerate*}{1}
\setlist*[inlinelist,1]{%
  label=\arabic*),
}

\begin{document}
  \begin{center}
  \large{\bf Worker--Managed Crowdsource Labor Markets}
  % Ali Alkhatib
  \end{center}


\sectitle{introduction}\\
In the past several years, researchers have watched new forms of labor
--- exemplified by labor markets like Amazon Mechanical Turk (AMT), Uber, and TaskRabbit ---
emerge and balloon into billion--dollar industries.
%
While these markets allow people to engage in the workforce in novel ways,
this ``gig economy'' has been criticized for commoditizing
and marginalizing workers
\cite{uberAlgorithm}.

On Turkopticon \cite{turkopticon} and in my prior work on Dynamo \cite{dynamo},
we attempted to make these markets more rewarding for workers,
but no labor markets to date place workers first;
existing markets trade one manager for another, maintaining an adversarial role between laborers and their managers.

Workers have faced issues like these for more than a century,
but they bear relevance to the HCI community broadly for two major reasons:
\begin{inlinelist}
  \item the features of digital systems allow researchers to have an outsize impacts on workers,
  whereas historically markets have been fractured \& untenable,
  and
  % --- or emulate their characteristics ---
  \item as researchers of technical systems and social interactions,
  researchers have a prerogative to attempt to steer systems toward more beneficial outcomes,
  especially given the growth of this form of labor and the potential implications for the future of work.
\end{inlinelist}

\sectitle{hypothesis}\\
I propose to investigate the viability of a technologically--mediated,
worker--run market as an alternative to existing systems.
I hypothesize that
such a market can compete effectively,
and that workers and consumers would benefit more, % than under the status quo,
for a number of reasons.
% as a result of community--determined norms and rules.
% and that it would benefit workers \& consumers due to the equitable relationships that it inherently fosters.
For workers,
\begin{inlinelist} % \itemsep 0pt \parskip 0pt
  \item communally determined rules and norms are perceived as, and perhaps are, fairer, and
  \item a protocol for labor markets would allow markets to communicate things like worker reputation.
\end{inlinelist}
Customers, for their parts, would be able
to leverage market competition, minimizing predatory pricing models.


\sectitle{background}\\
The most immediate question is why technologically--mediated
worker--run markets are so rare.
One may have predicted that worker--run markets would emerge inevitably
as online systems parallel (and sometimes model) offline settings.
Furthermore, the structure of the Internet
--- a ``rhizome'', as Miller and numerous others describe \cite{miller2011understanding} ---
suggests that communities should be able to form naturally,
unencumbered by barriers like geography.
Futurists predicted that the Internet would democratize communication,
enabling grassroots organization at unprecedented scales.

The empirical research suggests that the reality of this promise is more qualified
\cite{dynamo,toyama2015geek}.
Indeed, the factors which stymie community--run digital markets are not technical.
Researchers have built on existing markets to experiment with the organization of work;
others have implemented labor markets entirely from scratch
\cite{Alek2011,foundry}.
The components of gig labor market
--- such as scheduling, dispatch, \& payments ---
are sufficiently explored that we should consider them ``solved problems''.

No; we're finding that the challenges to worker--centric markets are social.
Research has shown that crowds can be directed
and can make tactical decisions on an ad hoc basis,
but collective governance and policy--making,
especially in competitive economic settings, remains an open question.
% The emergence, communication, and enforcement of norms
The coalescence of culture
adds a dimension of complexity to the individual cases of collective action represented by the study of individual events.
Engineering instances of collective action is a qualitatively different challenge from
fostering a community of ongoing collective action
\cite{russell1982collective}.
% (e.g. \cite{catalyst,dynamo}).

\sectitle{Methodology}\\
I intend to develop a worker--run labor market,
which the National Domestic Workers Alliance (NDWA)
will operate through their innovation arm, the Fair Care Labs.
I hope to demonstrate that cooperatively operated
digital labor markets can be successful, and that technologically--mediated
cooperative labor markets are viable alternatives to markets operated as firms.


My immediate task is to create a labor market which illustrates the characteristics described
--- technical features like scheduling, dispatch, \& payments, as well as
the necessary affordances for collective governance:
% The most immediate goal is to implement a system consisting of
% the technical requirements of gig labor markets described before
% (factors such as scheduling, dispatch, \& payments).
% I will develop a mobile application,
% similar in appearance to existing labor platforms,
% but importantly differentiated by affordances for community governance:
% \begin{inlinelist}
  % \item
  forums for discourse,
  % \item
  voting systems for referendums, and
  % \item
  communally written bylaws. %  written and approved by the community.
% \end{inlinelist}
These tools will be the starting point for a worker--run crowd labor market,
upon which more affordances will be designed and implemented as needed.

The Fair Care Labs plans to continue the relationship that began over the summer while I was affiliated with Microsoft Research,
deploying this system in the San Francisco Bay Area and the Seattle area
in order to allow me to provide high--level guidance
and to make any necessary course--corrections.
This should ameliorate concerns both about funding the system
and sourcing workers;
my involvement is the extent to which I will be able to contribute as a researcher.

% Past the instantiation of the marketplace and the recruitment of workers,
% the questions that remain relate to
% the ongoing relationships that form,
% the norms that emerge, and the methods and practices Computer Scientists might employ to foster productive relationships,
% informed in large parts by insights made through Anthropological inquiry.

% As early as 1982 it had been argued that
Using Hardin's perspective of collective action,
distinguishing between ``one--shot'' and ``ongoing'' campaigns,
we realize it's necessary to consider how groups sustain collective action over time.
% \cite{russell1982collective}.
Thus, designing ongoing collective action requires skills in 
system--building,
as well as a ``nearly anthropological investigation'' to engage in the sociocultural questions herein
\cite{russell1982collective}.
This describes the type of evaluation involved:
a reflection on the ways in which people work and relate with one another,
and the techniques we can use as engineers of social systems to guide relationships.


% In this area, I am relatively uniquely qualified.
The ethnographic and design work I've done with various labor advocacy groups,
my prior work on Dynamo
\cite{dynamo},
and my background and training as an Anthropologist
make me ideally skilled to pursue this line of research.
I have demonstrated that I can build systems,
and crucially I have proven experience in the qualitative,
sometimes nebulous fieldwork necessary for this research.

% With the NDWA funding and operating this market,
% Continuing the work I began with the NDWA while working with MSR,
% and having secured their commitment to fund and operate this market,
% I'm confident that a functionally viable labor market will be given a real chance to succeed.
% While the NDWA are committed to more worker--centric labor markets,
% my ability to contribute depends in part on my ability to continue my career.

% \dots
% boom microphone drop give money please

\renewcommand\refname{\sectitle{references}}
% % REFERENCES FORMAT
% % References must be the same font size as other body text.
% \bibliographystyle{SIGCHI-Reference-Format}
\bibliographystyle{acm}
{\footnotesize
\bibliography{references}
}

\end{document}

%%% Local Variables:
%%% mode: latex
%%% TeX-master: t
%%% End:
