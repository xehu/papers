% % !TEX program = xelatex
\documentclass[11pt,titlepage]{article}
\usepackage{balance,graphics,setspace,parskip,times}
% \usepackage{fontspec}
% \usepackage{xltxtra}
% \usepackage{xunicode}
\usepackage{hyperref}
\usepackage[margin=1in]{geometry}
\usepackage[inline]{enumitem}

\doublespacing
\usepackage[tiny,compact]{titlesec}

% \usepackage{xesearch}
% \newcounter{words}
% \newenvironment{counted}{%
%   \setcounter{words}{0}
%   \SearchList!{wordcount}{\stepcounter{words}}
%     {a?,b?,c?,d?,e?,f?,g?,h?,i?,j?,k?,l?,m?,
%     n?,o?,p?,q?,r?,s?,t?,u?,v?,w?,x?,y?,z?}
%   \UndoBoundary{'}
%   \SearchOrder{p;}}{%
%   \StopSearching}

\setlength{\parskip}{.4em}

\newlist{inlinelist}{enumerate*}{1}
\setlist*[inlinelist,1]{
  label=\arabic*),
}
\def\labelitemi{\ }
\newcommand*\elide{\textup{[\,\dots]}}
\newcommand{\sectitle}[1]{\textbf{#1}\\}

\author{
  Ali Alkhatib\\
  ali.alkhatib@cs.stanford.edu\\
  \textit{Couldn't find an envelope; will coordinate to meet next quarter (if acceptable)}
}

\title{Final Paper\\
\normalsize{
  \textit{Digital media are an egalitarian force in American society}\\
  Communication 220
  }
}

\begin{document}
\maketitle

\sectitle{Introduction}
Since --- and indeed before ---
the 90s, digital media have promised to unburden participants of the baggage of conventional cultural mores
\cite{barlow2009declaration};
Barlow in particular describes a very crystallized notion of
``cyberspace'' and its ``independence of the tyrannies \dots impose[d] on'' its denizens,
independence from conflicts inherent to embodied existence,
and finally a space to form a culture of universal human equality ---
to foster an egalitarian culture that transcends geographic space,
as well as the holdovers of culture from which he and others like him wished to break.

But the track records of digital media have been mixed, at best;
even in cases where we see what appears,
at first glance,
to be the emergent characteristics of egalitarianism,
closer inspection reveals sexist, racist, and variously exclusive
communities that are ultimately prohibitively difficult for outsiders to penetrate.
More troubling,
deeper consideration reveals worrying parallels with the counterculture movement of the 60s,
namely the trend of experimenting of with nominally egalitarian communes,
which ultimately proved to be sites of intense sexually \& racially charged practices of exclusion.
% with some facets of the broader counterculture movement 
% and generally informed the sensibilities and milieu of the Internet.

At another level, we see that digitally--mediated sites of emergent culture
are being substantively designed and shaped by certain groups of people and,
importantly,
occupied and used by other groups of people, leading to a worrying stark hierarchy ---
% a homogeneous group of people
% --- whereas these sites are largely occupied by others ---
% leading to hierarchical structure almost reminiscent of pre--industrialism ---
that is, people largely divided according to whether
they control digital media or are controlled by them.

I will attempt to argue that
digital media and the sites that emerge therein are
at least potentially
similar to the communes of the counterculture era,
and that the evidence suggests that
many of these sites have similarly paralleled the egalitarian communities from decades past.
I will go further, however, to offer a more optimistic outlook:
that the sites of interaction mediated by digital media can more easily be informed by one another,
and that the failures of one community to empower its members equally
need not necessarily repeat themselves across communities,
as so many communes often did.




To make this argument,
we need first to make the case that
digital media have been egalitarian forces in at least some facets of society,
which we will attempt to illustrate through
\begin{inlinelist}
  % \item the WELL, a community whose values precipitated the broader ethos of the Web, which later emerged,
  \item peer production work
      such as the Free Open Source Software (FOSS) movement \& Wikis, and
  \item efforts toward collective action and solidarity in online communities
\end{inlinelist}.
These cases will make a very simple argument at face value
--- that digital media are monotonically egalitarian forces ---
but we will explore these cases in depth and identify problems these systems have faced,
and sometimes caused, for participants.


We will also explore the much broader case of
systems themselves as tools used to separate users from builders.
Namely, we will look to
\begin{inlinelist}
  \item the oppressive nature of ``crowdwork'' and the broader ``gig labor''
  --- novel names for a well--established practice of
  speculative work which started at least as early as the turn of the 20th century
  (but which has found new purchase in the last 10 years), and
  \item the use of ``big data'' to track and analyze users, and
  the power system designers wield over their users by tracking people as data
  % \item the cultural, gendered, and other factors which emerge even among developers and engineers
\end{inlinelist}.





\sectitle{Peer Production: FOSS and Wikipedia}
Numerous examples come to mind in making the case that
digital media generate egalitarian communities;
perhaps the most compelling examples are those of peer production and
in particular
the decentralized organization that typifies many Free Open Source (FOSS) projects
as well as the ethos of Wikipedia and the many ``Wikis'' which have emerged since.

Suffice it to say that many have pointed to Linux and other open source projects,
in Raymond's case
identifying these as illustrative of an alternative to the ``cathedral'' organizational paradigm by
offering access to a sometimes hectic, ad--hoc ``bazaar'' instead
\cite{raymond1999cathedral,weber2004success,crowston2006core}.
The influence of open source software has been well--documented,
and certainly it represents a compelling counter--argument to
the proprietary ownership paradigm which has otherwise dominated software
\cite{kelty2008two,weber2004success}.

It seems, however, that open source projects often have complicated levels of involvement
--- if not explicitly named, then certainly implied by varying levels of engagement and interaction
\cite{crowston2006core,mockus2002two}.
Kelty points out that, in the community of developers he studied,
most developers were male, into heavy metal, and culturally relatively homogeneous
--- an issue that will resurface later in our analysis of big data,
but is worth pointing to here in our consideration of whether and to what extent FOSS is in fact egalitarian
\cite{kelty2008two}.
Another issue, that many of these open--source projects are so sparsely engaged that
it may be difficult to assess whether these \textit{would} be sites of egalitarianism,
is difficult to speak to, but worth noting
\cite{barcellini2006users}.

Wikipedia, by contrast, deliberately affords participants varying levels of commitment to take on,
and importantly almost any person with access to the Internet and an ability and willingness to write
can participate in the community.
This peer production model,
described by Howe and others as similar but subtly different from crowdsourcing,
theoretically allows all to come,
make edits and suggestions,
and debate edits on their merits rather than the status of the proponent
\cite{howe2008crowdsourcing,atlee2008collective,benkler2013peer}.
Here then we see at least a somewhat viable argument that
digital media facilitate the formation of egalitarian communities.

Shirky makes a similar point toward crowdsourcing in his discussion on the observation that
``everyone is a media outlet'' in this new setting
\cite{shirky2008here}.
His outlook is less optimistic
--- that the \textit{professions} involved in journalism and media have changed
in their mixing with ``amateurs'' ---
but if nothing else this contributes to a broader argument that
digital media have democratized the ability to publish content%
\footnote{%
Mackinnon (2012) discusses this effect to some extent through her analysis of WordPress as well,
but her discussion is largely set in China; while her findings are arguably still relevant
to American society, Shirky's general discussion above suffices
\cite{mackinnon2012consent}.}.

Unfortunately, again we find that reality differs from the promise;
researchers have discovered over the years that significant,
often implicit barriers to participation exist
without sufficient explanation for potential new editors
\cite{halfaker2012rise,zittrain2008future}.
Keegan \& Gergle in particular highlight the hypocrisy of
Wikipedia's stated ``egalitarianism'' in contrast to the reality that new editors experience
\cite{keegan2010egalitarians}.
This backlash that users often face has been documented and,
used to conclude that ``Wikipedians are \textbf{born}, not made'' (emphasis added)
\cite{wikipediansBornNotMade}.








\sectitle{Collective action}
Given the problematic nature of FOSS projects
and Wikipedia's enduring challenges to encourage new users to participate as editors,
one might conclude that
digital media unequivocally results in these problematized instances of egalitarianism.
Instead, I suggest we look further for evidence that
digital media represent forces toward egalitarianism, if not egalitarianism itself.

Collective action coordinated by digital media has,
Shirky and others argue,
at least the potential for democratized organization and planning
\cite{shirky2008here,costanza2014out}.
Here then, in some limited (and even stretched) sense,
digital media as democratized tools to facilitate collective action become
egalitarian influences in their own rights.
Miller in particular illustrates this notion through the use of the ``rhizome''
as a way of describing networked societies
\cite{miller2011understanding}.

Further, the tools needed to expose mistreatment of pieceworkers at the turn of the 20th century
--- namely, the camera Jacob Riis brought into the homes of workers ---
were made more accessible through the propagation of digital media and the tools needed to make them,
to say nothing of distributing that media,
made dramatically easier by blogging such as WordPress and other systems like it
\cite{riisOtherSideLives,facesOfMechanicalTurk}.
Further,
communities like Turkopticon \& Dynamo afford members some sense of equality among one another,
structured by guidelines on behavior to ensure equal opportunity to present views and
% some sense of anonymity and
separation from risk of controversial or contentious views affecting them
outside of the context of that forum
\cite{turkopticon,dynamo}.















\sectitle{Crowdwork, gig labor, and ties to piecework}
Having briefly mentioned communities of crowdworkers,
% (``Turkopticon'' itself is a play on the words ``panopticon'' and ``[Amazon Mechanical] Turk''),
we should acknowledge the unique role that crowdwork and gig labor play in this argument,
and in particular use this to segue from the notion of digitally--situated communities of egalitarianism
(or, as I have argued, \textit{not}) toward a much more general critique of
these communities as the boundary itself between the user and the builder.

Nominally, the gigwork enabled by Amazon's Mechanical Turk platform
(and other systems such as Upwork, Elance, etc\dots) make it possible for workers
--- who otherwise could not participate in the economy ---
to engage meaningfully, even earning a living.
In practice, however, these workers struggle to make ends meet
\cite{Ross}.
These markets for ``information work'' are, it turns out, far from unique;
markets for drivers for hire, cleaners, movers, and other various tasks
prove stifling and challenging for workers to benefit from.

Cynically,
one might argue that
the case for egalitarianism is still being made within the context of this workforce;
that is, everyone is struggling roughly equally compared to one another\footnote{This,
it turns out, is not true;
Irani et al. found that payment stratification benefits more senior Turkers,
a finding that has been corroborated incidentally by others
\cite{turkopticon}.
Still, we could cede this argument entirely in favor of the larger one to be made,
that egalitarianism must take into account the steep inequalities between various stakeholders.}.
This argument is specious, but it might be pernicious enough to merit addressing;
simply, the notion of a community achieving egalitarian ideals must be met entirely.
Amazon Mechanical Turk
--- and indeed \textit{all} of the marketplaces which promote piecework ---
relies on gross information asymmetry between the worker and the employer.
\cite{turkopticon,professionalCrowdworkEthics}.
That information asymmetry prevents participants from interacting as equals,
fundamentally preventing this and any market like it from being a site of egalitarianism.
Indeed, even communities \textit{connected to} these markets potentially suffer the hazards of
preferential statuses in these markets, especially illustrated by heterarchical relationships,
such as seeking support in one context given high status in another
\cite{dynamo}.


Further, the challenges these workers face sometimes come from the system itself,
but crucially from the finite designs of engineers,
causing its own form of oppression that differs from the kinds of coercive forces that offline markets
--- often the results of innumerable influencing factors ---
in that we can trace and even design responses to these issues
\cite{uberAlgorithm,turkopticon}.












\sectitle{Big Data}
In the previous section, we considered the effect that human--made systems such as
Uber, Lyft, etc\dots
have on workers, and the potential to track the design choices of system--engineers compared to the
confluence of broader marketplaces.
In this section,
we'll explore another aspect of human--engineered systems that separates and devalues
some classes of people from others --- the use of ``big data'' as a lens to consider the implications of
how imbalanced the power is between users of a digitally mediated system and the designers of that system.

Two issues come to light in this discussion prompted by big data.
The first is the general issue that system--designers have control over
the architecture of a digitally--mediated environment in only vaguely similar ways to
a conventional architect's control over a physical, embodied space.
Lessig (2006) describes a useful mental model %for conceptualizing the ways various forces influence a body
wherein four forces act on a body:
\begin{inlinelist}
  \item architecture,
  \item market,
  \item norms, and finally
  \item laws
\end{inlinelist}
\cite{lessig2006code}.
As we will soon come to realize,
these forces are largely consolidated into the agendas of system--builders.

It's important to acknowledge that
developers of digital systems have enormous control over the \textbf{architecture} of that system
(e.g. designing a user's interaction with a space
in such a way that it encourages or discourages certain behavior,
going as far as influencing ``emotional contagion'' \cite{kramer2014experimental}).
Additionally, markets such as Uber, Lyft, Amazon, and others
demonstrably have almost --- and perhaps effectively ---
unilateral, complete control over the \textbf{markets} which they expose through tools such as
data lock--in and threats of banishment for violating their own terms.
Norms, it would seem, are more difficult for a system--designer to influence;
deeper consideration reveals, however, that system--designers take insights from myriad fields such as
psychology, anthropology, and sociology, to inform the architecture of their system and 
encourage the emergence of \textbf{norms} preferred by \textit{them}
\cite{anderson2013steering,successfulOnlineCommunities}.


We return to Kelty's observation that developers have been a relatively homogeneous community of people;
notable exceptions aside, most developers in contemporary Silicon Valley have been,
and continue to be, male and relatively culturally homogeneous
\cite{ullman2012close,kelty2008two}.
It would be challenging even for culturally--minded people to appreciate the differing world--views
expressed around the world; for people whose fields of study range engineering and computer science
--- and especially those with minimal to no training in the study of, or exposure to various cultures ---
deep,
\textit{emic} understanding of the cultures of the people who use their systems
may be impossible for the engineers who build these major sites of social interaction.

Given that cold reality,
to say nothing of the interpersonal relationships that take place on digital media,
it's perhaps unreasonable to claim that the designers of systems and the users of systems are ever
part of a single egalitarian society.
Those that build systems are fundamentally more empowered than those that use them,
and with little to no understanding, and limited interaction between these two groups,
it seems unlikely that the characteristics of egalitarianism
--- for instance, mutual respect and understanding ---
will emerge.
I will continue to argue that this power imbalance is further skewed
in favor of those that design, implement, and control systems,
due to their necessary access to (and \textit{de facto} ownership of) users' data.

The second issue that emerges is the core topic of ``big data''
--- that users' collections data often become hostage to system--designers,
and that this fundamentally disrupts any endeavor toward an egalitarian society,
especially in the case to which Barlow attempts to speak (that of cyberspace)
\cite{barlow2009declaration}.
While Bush (1945) predicts a future where
the transmission of information might be reduced to a triviality in cost,
he seemed not to consider the circumstances in which we find ourselves ---
that the information itself might be worth too much to trivialize its free transmission
\cite{bush1979we}.

Indeed,
information has become a cornerstone of many of the companies which drive digital media spaces
\cite{sennett2007culture,mayer2013big}.
It should not, then, be surprising that companies as diverse as
driver--for--hire companies like Uber \& Lyft,
social networking sites such as Facebook \& Twitter,
and providers of services throughout (Google, Apple, \& Microsoft)
all generally maintain as secretly as possible both
the nature and content of what they track, even from the subjects of said tracking.
Lanier (2014) argues interestingly that perhaps
companies should be obligated to compensate people for the value extracted from data collected on them,
but this proposition
--- while interesting ---
is both economically infeasible and technically challenging
\cite{lanier2014owns}.






% \sectitle{Gender, race, \& developers}











\sectitle{Potential to be different}
Thus far, I've described an almost dystopian outlook on
the prospects of digital media being egalitarian forces in society at all,
let alone American society%
  \footnote{I've avoided discussing this issue until now,
  except here to say that efforts toward global egalitarianism are qualitatively
  --- not quantitatively ---
  more complicated.
  That is, the myriad political regimes attempting to moderate and legislate the use of technology
  (discussed to varying degrees by Goldsmith \& Wu, King et al., and others)
  are in themselves challenges similar in complexity to those we discuss here
  \cite{goldsmith2006controls,king2013censorship,king2014reverse}.
  Finding ways to relate these political and cultural contexts holistically,
  emerging with a comprehensive world--view on digital media's ability to affect egalitarian influence
  on various cultures and global culture as a whole, therefore,
  would fall far outside the scope of this paper.}.
%
There is however, reason to be hopeful, if not optimistic.
To say nothing of the efforts described earlier to enable discussion among people as equals
\cite{turkopticon,dynamo},
the broader field of Human--Computer Interaction benefits from insights both in design
--- leading realizations of the benefits of ``participatory design'',
wherein users and builders of systems work collaboratively ---
as well various fields of social and cultural inquiry
\cite{bodker1991cooperative,successfulOnlineCommunities}.

Where Turner points out how the counterculture revolution
and communes in particular
invariably became sites of implicitly defined prejudices,
I argue that our ability to learn from other communities rapidly and freely makes it possible
to avoid sliding into the same causes of failure which almost invariably struck communes
\cite{turner2005counterculture}.
Dynamo learned from Turkopticon and Wikipedia
\cite{dynamo,turkopticon};
bodies of literature are largely informed by existing communities,
the results of this literature often becoming the basis for new endeavors
\cite{bernstein20114chan,horowitz2010anatomy,Nov:2007:MW:1297797.1297798,wikipediansBornNotMade}.

All of this is to say that,
when we
--- in particular, researchers, but also system--designers and \textit{people} in general ---
recognize subtle, pervasive barriers discouraging people from participating in editing Wikipedia,
we can leverage mountains of research into encouraging participation,
leading to design interventions that precipitate and facilitate more inclusive environments.
Indeed, we can point to research on various fronts which corroborates this potential,
the most salient of which being the efforts to change Wikipedia's administrative practices
--- as well as the design of the system itself ---
to ``steer'' more constructive moderation of edits
\cite{ciampaglia2015moodbar,kraut2010dealing,pal2011early}.

\sectitle{Closing thoughts}
I've attempted to articulate a complicated perspective on,
and in fact relationship with,
digital media through their ability to be an egalitarian influence on American society.
Ultimately, my arguments attempted to illustrate that
numerous examples ostensibly settling this debate exist,
but that each of those examples is problematized by deeper,
underlying complications to the relationships between stakeholders.

I chose to end, however, with a call toward optimism to suggest that
it is not built \textit{into} digital media that egalitarianism is,
at best,
a false promise; that instead,
it may be possible to achieve the egalitarianism Barlow wishes to see in his declaration,
but that it will
--- as always ---
take our best efforts to avoid making the same mistakes we've made in the past,
marginalizing people and imposing our own world--views in an attempt to
homogenize cyber--culture as it forms.
I believe, with care and consideration of the past
--- a skill admittedly all too rare among those who need it most ---
system--designers can help construct a cyberspace that welcomes all as Barlow wrote in 1996.

% \end{counted}
% \footnotesize{word count: \thewords}

\bibliographystyle{acm} % this will cause an error if there are no citations!
\bibliography{../../../references.bib}
\end{document}
