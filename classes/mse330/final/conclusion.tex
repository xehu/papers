%!TEX root = final.tex
Bail--setting remains an ill--explored area of research;
this is particularly the case given its critical role in the criminal justice system.
Moreover, the nature of research in this space makes it challenging to
run analyses on substantial quantities of data, as we were able to in this case.
By adopting a number of modeling approaches and methods of analyses,
we were able to verify the effect that pretrial incarceration has on
a defendant's willingness to plead guilty.
Perhaps unsurprisingly,
we found evidence that offers credibility to the fear that bail--setting carries
an undue impact on the process.

Our first attempts to quantify and measure
a dimension that was previously under--explored
suggests that people are experiencing worse outcomes than others
on the basis of their inability to post bail.
While these findings suggest that the courts are not providing the fairest possible
experience to all those who pass through the justice system,
we see these findings as a silver lining.
Indeed, this research begins the discussion of
quantifying and
measuring
the extent to which pretrial incarceration is affecting defendants' plea rates.
This tentative step of measuring, and perhaps tracking, the effects of
pretrial incarceration may yield insights to ensure that defendants are receiving
just and due process.
