%!TEX root = final.tex
In order to analyze our two research questions,
we first quantify bail burden and preprocess the data,
and subsequently perform statistical analyses on the data.
The methods used in these steps are detailed below.


\subsection{Bail Burden}
We conceive of the possible burden placed on a defendant by bail as fitting into one of
\nextitemizecount{} categories:
\begin{inlinelist}
  \item When a defendant is Released on his or her Own Recognizance (ROR),
  the burden is negligible.
  Defendants are required neither to put up money nor to spend time incarcerated; 
  \item when bail is set and a defendant is able to pay it immediately,
  the defendant is lightly burdened; 
  \item when bail is set and a defendant is able to pay it but with a delay,
    the defendant is significantly burdened both
    financially and by
    needing to spend time incarcerated prior to meeting bail; 
  \item when a defendant either has no option to be released on bail,
  or is unable to meet bail,
    he or she is \textit{severely} burdened,
    having to spend the entire time until trial incarcerated.
\end{inlinelist}

\subsection{Preprocessing}
Several preprocessing steps were taken to ensure the validity of our analyses.
Due to some inconsistencies in volume of cases prior to 2013,
only cases from after (and including) 2013 were analyzed.
Additionally,
cases involving violent offenses were removed since
there are extra considerations and complexities in bail--setting for violent offenders,
who may be considered safety risks to the community and evaluated differently for bail.
We also only consider cases that made it past arraignment,
since cases where charges were dropped at arraignment are ineligible for the analyses we perform.
Finally,
for most analyses we consider Quality of Life (QoL) and non--Quality of Life (non--QoL) offenses separately.
Quality of life offenses make up a category of illegal activity which is connected with the broken windows theory of policing
--- these are misdemeanor offenses such as
  disorderly conduct,
  vagrancy,
  or
  loitering.
QoL and non--QoL offenses may differ substantially in the populations being charged and in their handling in the justice system,
so we perform the same analyses in parallel across these two types of data.



\subsection{Analyzing the Effect of Bail on the Probability to Plea}
We analyze the effect of bail burden on the probability of pleading through
two main techniques and
using both random forest models and GLM.
First,
we train models to predict the probability of pleading using all case characteristics through arraignment,
including the burden level set.
Then we create test sets composed of different burden levels
(through two different techniques,
described next),
and analyze the test set probability of pleading conditional on each burden level.
We present the results both as a CDF of the test sets' probability of pleading and as means of the distribution.

To verify the effectiveness of our models,
we produce calibration plots and include them in the appendix.
For each row in the original test set,
we predict the probability of pleading and ROR,
respectively,
for our two models.
We bin these probabilities and observe the true probability of observing the dependent variable for each of the bins.


\subsubsection{``All Else Equal''}
For our preliminary analysis,
we create the test sets for each burden level through an ``all else equal'' assumption
--- that we can change the burden level of a given case while holding all other case characteristics the same.
We duplicate the test set into four sets.
In each data set,
we set the burden level to
  ``ROR'',
  ``Trivial'',
  ``Trivial with Delay'',
  and
  ``Jail'',
respectively.
We then send these test sets to the model and view the resulting predicted probability of pleading.
We note that the ``all else equal'' assumption is not true in general.
In a murder case,
for example,
it would not make sense to change the burden level from ``Jail'' to ``ROR''.
However,
because we analyze only non--violent cases,
the assumption holds approximately.



\subsubsection{Propensity Score Matching}
To further support our analysis,
we relax the assumption through initial propensity score matching.
For this method,
we split the burden into only two levels: ROR and Non--ROR.
We train a model to predict the probability of ROR.
Two test sets are then matched on the probability of ROR through \texttt{matchit}
\citep{matchit},
and the burden bits are set to ROR and Non--ROR,
respectively.
They are then sent to the probability of pleading models.
To do true propensity score matching and make sure our assumption holds,
we should have filtered out cases with either very low or very high probability of ROR.
Otherwise,
we are still using cases where it would not make sense to change the ROR value.


\subsection{Analyzing the Effect of Judge on Bail}
\subsubsection{Analysis --- Random Forest}
For an initial analysis to observe judge effects,
we predict the Probability of ROR being granted for each case,
without using \texttt{JudgeID} in our prediction model.
This analysis uses the same Random Forest ROR model trained in the Propensity Matching analysis.
This model gives a baseline for judge independent granting of ROR.
For each judge with over 100 cases in the test set,
we determine the judge's mean predicted probability of ROR using the model.
This value is the percentage of a given judge's cases that the ``average'' judge would ROR.
We compare this value to the true percentage of cases in which each judge actually granted ROR.

\subsubsection{Analysis --- Generalized Linear Model}
An ``all else equal'' approach was taken in the GLM analysis
--- the general strategy was to create a copy of the data with a test variable set to a given value,
input these modified datasets into the model for each value the test variable can assume,
and then obtain summary statistics for each possible value
to then compare with the summary statistics associated with the other values;
by doing this,
any predictive power this model might have can be used
to check for the relative influences of certain factors on the outcome that this model predicts.

The ROR model was used to check if certain arraignment judges differed substantially in how they treated ROR.
The plea model was used to check if certain levels of burden had a higher predicted rate reaching a plea disposition.
As a cursory additional check,
matching edge cases were identified using a software package
(i.e. cases that can be considered similar on model inputs but differed on observed outcomes),
and in each case,
the same analysis was performed on a reduced dataset of these edge cases to similar results.

Within this reduced set of edge cases,
the assumption of ``all else equal'' has greater validity,
as the matching software is trusted to compute a quantitative measure of this
``equality'' between cases and identify precisely those which are ``sufficiently equal''.
The effects of judge on ROR are summarized in
figures~\ref{fig:ROR_QoL_GLM} and
        \ref{fig:ROR_non--QoL_GLM}.
For non--QoL cases,
judges seem to have more similar ROR rates compared to QoL cases
(one of the most prolific judges is also an outlier in the QoL dataset).
The effects of burden on the rate of plea dispositions are summarized in
figures~\ref{fig:QoL_plead_GLM} and
        \ref{fig:non--QoL_plead_GLM}.

For both QoL and non--QoL cases,
the lower levels of burden had close distributions,
but the distribution for ``nontrivial'',
the highest level of burden had its mass shifted noticeably toward higher plea rates.
Additionally,
the predicted plea distributions on the QoL dataset had mass shifted more to the right compared to the non--QoL distributions.
Analogous plots for the same analysis performed with the ROR model on the matched non--QoL dataset are found in
figures~\ref{fig:mean_ROR_non--QoL_GLM} and
        \ref{fig:mean_ROR_QoL_GLM};
the other combinations of QoL or non--QoL and ROR or plea model yielded matched datasets that were too small
(i.e. on the order of tens of cases) to use.







