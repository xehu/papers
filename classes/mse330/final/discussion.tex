%!TEX root = final.tex
As described in the results above,
we find that our first hypothesis,
that a more burdensome bail will increase likelihood for a defendant to plead out,
is confirmed.
Our secondary hypothesis is also confirmed,
though less strongly.

More specifically,
we observe that in non--QoL cases,
there is a substantive distinction between the first two burden levels,
which do not involve jail time,
and the latter two burden levels,
which do.
Jailing a defendant
increases the predicted probability of pleading
compared to burdens not involving jail time.
We observe a similar,
though slightly smaller effect in QoL cases.
With regards to our second hypothesis,
we find that despite differences in the cases received by judges
(which are unfortunately not randomized),
some variance in the handling of cases by judges
can be attributed to the specific judge handling the case.


These two findings together underscore the importance of an objective,
randomized,
and well--informed justice system in order to achieve fair outcomes.
The degree to which a judge is informed or able to estimate
a defendant's means and ability to pay bail can
make a significant difference in outcomes for that defendant.
This suggests that work aimed at estimating
accurate and
straightforward predictive heuristics for judges to use in setting bail is
a promising strategy towards facilitating fairness for all defendants.
Likewise,
individual differences in setting bail provide the opportunity for
further statistical analyses to discern the effect of
different bail--setting norms on defendant outcomes,
and again suggest that helping judges adjudicate consistently over time and
in comparison to each other can
help make the justice system more fair towards defendants.