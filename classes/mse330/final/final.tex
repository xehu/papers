% !TEX program = xelatex
\input{../../../starter.tex}
\setmainfont[
Ligatures=TeX,
Extension=.otf,
UprightFont= *-regular,
BoldFont=*-bold,
ItalicFont=*-italic,
BoldItalicFont=*-bolditalic]{texgyretermes}



\title{The Effect of Bail on Guilty Pleas\\
\large{\textit{MS\&E 330}}}
\author{
Ali Alkhatib, Nikhil Garg, Martin Lindsey, Danaë Metaxa--Kakavouli
\\
\{\href{mailto:al2@stanford.edu}{al2},
  \href{mailto:nkgarg@stanford.edu}{nkgarg},
  \href{mailto:mlindsey@stanford.edu}{mlindsey},
  \href{mailto:metaxa@stanford.edu}{metaxa}\}@stanford.edu}


\bibliographystyle{apalike}


\begin{document}

\begin{titlingpage}
  \maketitle
\end{titlingpage}

\begin{abstract}
Research on the subject of
bail,
the burden it places on defendants, and
the persuasive and perhaps coercive influence it bears on the justice system
has been explored to some extent ethnographically, but thus far
researchers have not had the data necessary to conduct rigorous quantitative studies on
bail--setting and
the factors which influence pleading behavior. % and the influencing factors therein have not had sufficient data
With a unique insight into the demographics and characteristics of defendants and their outcomes,
we explore two questions:
\begin{inlinelist}
\item Does a greater burden, as measured by bail, cause people to plead guilty? And
\item Are judges handling cases uniquely, or uniformly?
\end{inlinelist}
Our findings reveal that heightened burdens do seem to cause defendants to plead guilty at a higher rate,
and that there is some evidence that variance in bail--setting is unique to judges.
\end{abstract}



\section{Introduction}
%!TEX root = proceedings.tex
\section{Introduction}


% Karl Polyani's \textit{The Great Transformation} describes the processes that led Western Europe from Feudalism to what he calls ``market society"
% \cite{GreatTransformation}.
% His insights gave birth to ``Substantivism", whose subscribers think of economies broadly as ways of navigating the world
% --- socially and otherwise ---
% rather than more formally as ways of maximizing the utilization of scarce resources.
% In the Substantivist school of thought the negotiations found in the ``sharing" or ``peer economy" between and among workers, customers, and the market operators themselves collectively represent a broader economy than the money-for-(rides, housing, etc\dots) exchange that formalists would explore.

In the past several years, a new type of work has emerged where customers can hire someone to complete an individual task, or ``gig''.
This task might involve delivering a package, or driving someone downtown.
Such a model principally requires the sharing of capital-intensive goods, like access to a car or home,
which led to the popularization of its familiar name, the ``sharing economy''.
People could ``share'' their homes (Airbnb, Couchsurfing \cite{airbnbOfficial,couchsurfingOfficial}), cars (Uber, Lyft, and others \cite{uberOfficial,lyftOfficial}), and increasingly one's own time (TaskRabbit, Zaarly, and many others \cite{taskrabbitOfficial,zaarlyOfficial}).

Workers' demographics have changed dramatically in as little as half a decade largely discreetly.
Where workers in the sharing economy were once car and home-owners who had free time to spare,
many now think of these companies and the markets they expose as their primary source of income.
In the last several years, workers for Uber, Homejoy,
and other market operators have initiated
--- and in some cases won \cite{homejoySuit,uberSuit} ---
suits describing mistreatment and misclassification of these workers as
``independent contractors" rather than ``employees" (protected and regulated by labor laws in the United States) \cite{fedsUber}.

In the sharing economy's nascent years, companies enticed workers to join for the potential to do work ``on the side": offering rides to others when they had free time, or renting out their apartment when they were out of town for a weekend.
For various reasons, the culture has since changed, and the notion of doing work in one's free time has largely disappeared.
Instead, drivers report primarily working as drivers, and that their primary sources of income consist of the aggregated sum of passengers they pick up through ride-sharing markets.
Even in the hotelier industry, providers have purchased apartments or re-purposed their own homes primarily to serve guest occupants, rather than to rent out incidentally when they have a spare room or are out of town.

In these ways, workers are neither ``peers", nor are they ``sharing" resources that would otherwise go underutilized.
But they're not conventional workers, either.
They acquire capital
--- sometimes co-signing on leases with the companies that run these markets ---
under their own names, run their businesses relatively independently, and make the majority or even totality of their income based on each individual job, or ``gig", cumulatively summed up.
With their careers described as a series of individual jobs, each self-contained and relatively independent of the others, people have renamed it the ``gig economy", more fairly referencing the differentiating nature of this work.

The widespread nature of these changes suggests that this is part of a larger trend in what may become the future of work;
far from the hopeful but cautious predictions offered of information workers and ``crowd work'' \cite{crowdworkFuture}.
Workers, increasingly find themselves objectified, marginalized, and frustrated by oppressive systems.

We considered, then, how one might design a worker-centric peer market;
how can system designers create technologically enabled markets as successful as existing markets like Uber, Lyft, and others, while also:
\begin{itemize} \itemsep0pt \parskip0pt
  \item giving workers a sense of locus
  \item benefiting workers as well as consumers
  \item facilitating worker organization and communication
  \item enabling collective decision \& action among workers
\end{itemize}

To answer these questions, we engaged in extensive fieldwork alongside workers and labor organizations in a number of industries.
Building in part from backgrounds in the social sciences and as trained computer scientists,
we learned about workers and the industries in which they work from workers themselves.
We report on the processes of making contact with variously formally organized worker advocacy groups,
illustrate some of the ways that we can learn from informants most effectively given our own skills,
and finally describe some of the findings we made as a result of our own use of these methods.

Informed by the input of dozens of workers from numerous industries ranging from highly regulated to informal,
we identify a number of aspects of on-demand work which system-designers should consider in the creation of a worker-centric labor market.
We offer guidance on these design considerations,
and in some cases illustrate the suggested approaches we generated in tandem with these partner organizations and workers.

Specifically, we offer contributions to the following questions
\begin{itemize}\itemsep0pt \parskip0pt
  \item What components of existing markets are inextricable from the features which make these markets successful?
  \item What can be disentangled and abstracted away?
  \item How would these groups be operated?
\end{itemize}

\section{Literature Review}
%!TEX root = final.tex
Broadly, we identify \nextitemizecount{} areas of research which provide some insight into this topic:
\begin{inlinelist}
  \item The use of
    \textit{\lnameref{subsec:bail}},
    and its stated and empirical purposes,
  \item the study of the effect that bail has on
    \textit{\lnameref{subsec:burden}},
    and
  \item the role of
    \textit{\lnameref{subsec:plea}},
    both for the courts and defendants.
\end{inlinelist}
We will discuss the intersection of these bodies of work as they relate to our research question.


\subsection{\nmu Bail}\label{subsec:bail}
In this section, we will explore the murky waters of the use of bail,
but ultimately settle on the conclusion that the courts have made:
that bail is fundamentally a necessity in a judicial system,
despite the philosophical questions and issues it inherently raises.
Despite the need for bail as a robust safeguard against undue imprisonment,
we will illustrate that the practice of bail--setting is
--- perhaps decidedly --- difficult to discuss in broad terms,
often leaving researchers primarily with
    qualitative,
    ethnographic, and sometimes
    anecdotal
evidence upon which to work.

Bail represents an attempt to resolve a challenging cognitive dissonance.
As \citet{foote1959bail} points out, the imprisonment of a defendant who is nominally considered
``innocent until found guilty''
arouses challenges of principle;
if the courts are to assume that a defendant is innocent (and thus treat them as such),
then holding them in custody until the trial begins is patently wrong, as it deprives a
\textit{presumably} innocent person of freedom.

Nevertheless, the United States Constitution's 8th amendment
--- as well as a series of laws attempting to reform bail
(one in 1966, another in 1984) ---
ostensibly protect defendants from unreasonable and excessive bail amounts,
otherwise to be set at the discretion of the presiding judge
\citep{berg1985bail}.
As \citeauthor{berg1985bail} notes, however,
the most recent revisions to the Bail Reform Act
--- those passed in 1984 ---
make it explicit that judges may set or even deny bail on the basis of
the judge's assessment of the defendant's danger to the community.
In \citet{1984schall},
the court ruled that
``there was nothing inherently unconstitutional in preventive detention''
and that
``the restraint on [the defendant]'s liberty
did not amount to punishment because
that was not the express purpose of the pretrial detention''.

The precedents established in the cases of
\citet{1984schall} and
\citet{1987united} broadly supports the judge's prerogative to set bail
--- and importantly, to \textit{not} set it
(in other words, to remand the defendant with no option to pay bail),
if the judge determines that safety to the community
(or the defendant's appearance upon a later court date)
could not be assured by any bail amount.
This latitude becomes an important component that we attempt to explore
in the hopes of determining whether if
--- at least within a single jurisdiction ---
judge bail--setting variance is high or not.
The implication of this question may be that
some judges are more merciful,
and some decidedly harsher,
than their peers in their assessments of defendants.




\subsection{\nmu Burden}\label{subsec:burden}
The 1984 Bail Reform Act thus
afforded the courts a great deal of freedom in setting bail,
bringing us naturally to the question of what effect this has on defendants awaiting trial.
In this area, we find myriad researchers have contributed;
we will attempt to highlight some of the work that is
especially illuminating and
representative of the body of work.

\citeauthor{kellough2002remand} identify qualitatively that the
``moral assessments'' of defendants appear to influence the burden judges assign in the form of bail--setting.
This, they hypothesize, may explain
(at least to some extent)
the disparity in race between
those that are let free in the interim before their court date, and
those that are held in custody on remand
\citep{kellough2002remand}.

\citeauthor{zacharias1997justice} further discusses the burden that bail may impose on defendants,
in this case highlighting the leverage that prosecutors may have over defendants,
for instance in the form of the unfamiliar and unwelcome environment
that temporary incarceration represents, and the promise to expedite one's interaction with the justice system
\citep{zacharias1997justice}.
This, finally, leads us to the outcome variable which we will ultimately investigate.


\subsection{\nmu Plea \nmu Bargaining}\label{subsec:plea}
Most tangibly, the persuasive
(perhaps \textit{coercive})
force that prosecutors can exercise in tandem with judges is
the pressure to encourage defendants to plead guilty in the pretrial phase.
This aspect of the justice system is especially unique, both because
plea--bargaining defies the adversarial model of the court and trial process and because
these dynamics are so ill--understood at broad levels
\citep{zacharias1997justice}.

The research on the \lnameref{subsec:burden} that bail represents on defendants
--- especially defendants with limited financial means and similarly limited access to cash ---
has \textit{qualitatively} found that
trial outcomes may be affected by the burden of pretrial incarceration.
To illustrate this, we turn to
\citet{grossman1983plea} and \citet{alschuler1981changing};
\citeauthor{grossman1983plea} estimates that as many as
``90 percent of convictions \dots result from a negotiated plea of guilty''
\citep[see also][with similar findings]{mccoy1980plea,kaplan1977american,newman1966conviction}.

Most of the existing research in plea bargaining,
especially \citeauthor{grossman1983plea}'s work,
focuses on the economic role of the plea bargain as a
``screening device'';
\citeauthor{alschuler1981changing}
contributes to the argument that pretrial bargaining may serve the defendant's best interests,
observing that neither the courts nor the public defenders who often represent defendants
have sufficient time to see all cases through open court
\citep{alschuler1981changing}.
\citeauthor{alschuler1981changing}
makes the noteworthy observation that we lack sufficient information to determine whether 
  the pressure of incarceration,
  the prosecutor's persuasive offer to settle, and
  the defending attorney's desire to expedite the closure of outstanding cases
is in fact causing defendants to settle more frequently,
primarily due to a dearth of broad--based data, and certainly
--- as \citeauthor{kellough2002remand} describe it ---
a simple ``paucity of research [in plea bargaining]''
\citep{kellough2002remand}.





\section{Method}
%!TEX root = final.tex
In order to analyze our two research questions,
we first quantify bail burden and preprocess the data,
and subsequently perform statistical analyses on the data.
The methods used in these steps are detailed below.


\subsection{Bail Burden}
We conceive of the possible burden placed on a defendant by bail as fitting into one of
\nextitemizecount{} categories:
\begin{inlinelist}
  \item When a defendant is Released on his or her Own Recognizance (ROR),
  the burden is negligible.
  Defendants are required neither to put up money nor to spend time incarcerated; 
  \item when bail is set and a defendant is able to pay it immediately,
  the defendant is lightly burdened; 
  \item when bail is set and a defendant is able to pay it but with a delay,
    the defendant is significantly burdened both
    financially and by
    needing to spend time incarcerated prior to meeting bail; 
  \item when a defendant either has no option to be released on bail,
  or is unable to meet bail,
    he or she is \textit{severely} burdened,
    having to spend the entire time until trial incarcerated.
\end{inlinelist}

\subsection{Preprocessing}
Several preprocessing steps were taken to ensure the validity of our analyses.
Due to some inconsistencies in volume of cases prior to 2013,
only cases from after (and including) 2013 were analyzed.
Additionally,
cases involving violent offenses were removed since
there are extra considerations and complexities in bail--setting for violent offenders,
who may be considered safety risks to the community and evaluated differently for bail.
We also only consider cases that made it past arraignment,
since cases where charges were dropped at arraignment are ineligible for the analyses we perform.
Finally,
for most analyses we consider Quality of Life (QoL) and non--Quality of Life (non--QoL) offenses separately.
Quality of life offenses make up a category of illegal activity which is connected with the broken windows theory of policing
--- these are misdemeanor offenses such as
  disorderly conduct,
  vagrancy,
  or
  loitering.
QoL and non--QoL offenses may differ substantially in the populations being charged and in their handling in the justice system,
so we perform the same analyses in parallel across these two types of data.



\subsection{Analyzing the Effect of Bail on the Probability to Plea}
We analyze the effect of bail burden on the probability of pleading through
two main techniques and
using both random forest models and GLM.
First,
we train models to predict the probability of pleading using all case characteristics through arraignment,
including the burden level set.
Then we create test sets composed of different burden levels
(through two different techniques,
described next),
and analyze the test set probability of pleading conditional on each burden level.
We present the results both as a CDF of the test sets' probability of pleading and as means of the distribution.

To verify the effectiveness of our models,
we produce calibration plots and include them in the appendix.
For each row in the original test set,
we predict the probability of pleading and ROR,
respectively,
for our two models.
We bin these probabilities and observe the true probability of observing the dependent variable for each of the bins.


\subsubsection{``All Else Equal''}
For our preliminary analysis,
we create the test sets for each burden level through an ``all else equal'' assumption
--- that we can change the burden level of a given case while holding all other case characteristics the same.
We duplicate the test set into four sets.
In each data set,
we set the burden level to
  ``ROR'',
  ``Trivial'',
  ``Trivial with Delay'',
  and
  ``Jail'',
respectively.
We then send these test sets to the model and view the resulting predicted probability of pleading.
We note that the ``all else equal'' assumption is not true in general.
In a murder case,
for example,
it would not make sense to change the burden level from ``Jail'' to ``ROR''.
However,
because we analyze only non--violent cases,
the assumption holds approximately.



\subsubsection{Propensity Score Matching}
To further support our analysis,
we relax the assumption through initial propensity score matching.
For this method,
we split the burden into only two levels: ROR and Non--ROR.
We train a model to predict the probability of ROR.
Two test sets are then matched on the probability of ROR through \texttt{matchit}
\citep{matchit},
and the burden bits are set to ROR and Non--ROR,
respectively.
They are then sent to the probability of pleading models.
To do true propensity score matching and make sure our assumption holds,
we should have filtered out cases with either very low or very high probability of ROR.
Otherwise,
we are still using cases where it would not make sense to change the ROR value.


\subsection{Analyzing the Effect of Judge on Bail}
\subsubsection{Analysis --- Random Forest}
For an initial analysis to observe judge effects,
we predict the Probability of ROR being granted for each case,
without using \texttt{JudgeID} in our prediction model.
This analysis uses the same Random Forest ROR model trained in the Propensity Matching analysis.
This model gives a baseline for judge independent granting of ROR.
For each judge with over 100 cases in the test set,
we determine the judge's mean predicted probability of ROR using the model.
This value is the percentage of a given judge's cases that the ``average'' judge would ROR.
We compare this value to the true percentage of cases in which each judge actually granted ROR.

\subsubsection{Analysis --- Generalized Linear Model}
An ``all else equal'' approach was taken in the GLM analysis
--- the general strategy was to create a copy of the data with a test variable set to a given value,
input these modified datasets into the model for each value the test variable can assume,
and then obtain summary statistics for each possible value
to then compare with the summary statistics associated with the other values;
by doing this,
any predictive power this model might have can be used
to check for the relative influences of certain factors on the outcome that this model predicts.

The ROR model was used to check if certain arraignment judges differed substantially in how they treated ROR.
The plea model was used to check if certain levels of burden had a higher predicted rate reaching a plea disposition.
As a cursory additional check,
matching edge cases were identified using a software package
(i.e. cases that can be considered similar on model inputs but differed on observed outcomes),
and in each case,
the same analysis was performed on a reduced dataset of these edge cases to similar results.

Within this reduced set of edge cases,
the assumption of ``all else equal'' has greater validity,
as the matching software is trusted to compute a quantitative measure of this
``equality'' between cases and identify precisely those which are ``sufficiently equal''.
The effects of judge on ROR are summarized in
figures~\ref{fig:ROR_QoL_GLM} and
        \ref{fig:ROR_non--QoL_GLM}.
For non--QoL cases,
judges seem to have more similar ROR rates compared to QoL cases
(one of the most prolific judges is also an outlier in the QoL dataset).
The effects of burden on the rate of plea dispositions are summarized in
figures~\ref{fig:QoL_plead_GLM} and
        \ref{fig:non--QoL_plead_GLM}.

For both QoL and non--QoL cases,
the lower levels of burden had close distributions,
but the distribution for ``nontrivial'',
the highest level of burden had its mass shifted noticeably toward higher plea rates.
Additionally,
the predicted plea distributions on the QoL dataset had mass shifted more to the right compared to the non--QoL distributions.
Analogous plots for the same analysis performed with the ROR model on the matched non--QoL dataset are found in
figures~\ref{fig:mean_ROR_non--QoL_GLM} and
        \ref{fig:mean_ROR_QoL_GLM};
the other combinations of QoL or non--QoL and ROR or plea model yielded matched datasets that were too small
(i.e. on the order of tens of cases) to use.









\section{Results}
%!TEX root = final.tex
\subsection{Analyzing the Effect of Bail on the Probability to Plea}
\subsubsection{All Else Equal --- Random Forest}
We first carry out the All Else Equal analysis through a Random Forest model.
Appendix \ref{appendixA} shows the calibration information for the Random Forest models for both
Non--Quality of Life and
Quality of Life cases.
We observe that the models are accurate and serviceable,
though the Non--QoL model consistently over--predicts the probability of pleading in the unaltered test set.

Tables~\ref{table:non--QoL_plead} and
       \ref{table:QoL_plead},
respectively,
include the mean predicted probability of pleading at different burden levels for
Non--Quality of Life and
Quality of Life cases, respectively.
Figures~\ref{fig:non--QoL_plead} and
        \ref{fig:QoL_plead}, show the CDFs of same values determined through the model.
Applying the GLM to the dataset reveals roughly similar predictions, as illustrated by
Figures~\ref{fig:non--QoL_plead_GLM} and
        \ref{fig:QoL_plead_GLM}.


We observe that,
especially for Non--Quality of Life cases,
jailing someone substantively increases the predicted probability of pleading when compared other burden levels.
This effect can be observed through both the higher means and the right--shift on the CDF plots.
The effect is similar, though much less pronounced, for Quality of Life cases.

\begin{table}
\centering
\begin{tabular}{|p{0.3\textwidth}|p{0.6\textwidth}|}
  \hline
\textbf{Burden Level} & \textbf{Mean Predicted Probability of Pleading} \\ \hline
    ROR & 0.3631537 \\ \hline
    Trivial & 0.3834242 \\ \hline
    Delay & 0.3744825 \\ \hline
    Jail & 698843  \\ \hline
  \end{tabular}
  \caption{Mean Probability of Pleading for Non--Quality of Life offenses through All Else Equal Analysis using Random Forest model}
  \label{table:non--QoL_plead}
\end{table}

\begin{table}
\centering
\begin{tabular}{|p{0.3\textwidth}|p{0.6\textwidth}|}
  \hline
\textbf{Burden Level} & \textbf{Mean Predicted Probability of Pleading} \\ \hline
    ROR & 0.6072936 \\ \hline
    Trivial & 0.6102411 \\ \hline
    Delay & 0.5999771 \\ \hline
    Jail & 0.648984 \\ \hline
  \end{tabular}
  \caption{Mean Probability of Pleading for Quality of Life offenses through All Else Equal Analysis using Random Forest model}
  \label{table:QoL_plead}
\end{table}


\begin{figure}[t!]
  \centering
  \begin{subfigure}[b]{0.49\textwidth}
    \caption{Non--Quality of Life cases}
    \includegraphics[width=\textwidth]{figures/figx.png}
    \label{fig:non--QoL_plead}
  \end{subfigure}
  ~
  \begin{subfigure}[b]{0.49\textwidth}
    \caption{Quality of Life cases}
    \includegraphics[width=\textwidth]{figures/figy.png}
    \label{fig:QoL_plead}
  \end{subfigure}
  \caption{CDFs of predicted probability of pleading guilty at different burden levels
           determined through the All Else Equal Random Forest model}
  \begin{subfigure}[b]{0.49\textwidth}
    \caption{Non--Quality of Life cases}
    \includegraphics[width=\textwidth]{figures/glmplots/plea_cdf.png}
    \label{fig:non--QoL_plead_GLM}
  \end{subfigure}
  ~
  \begin{subfigure}[b]{0.49\textwidth}
    \caption{Quality of Life cases}
    \includegraphics[width=\textwidth]{figures/glmplots/pleaq_cdf.png}
    \label{fig:QoL_plead_GLM}
  \end{subfigure}
  \caption{CDFs of predicted probability of pleading guilty at different burden levels
           determined through the Generalized Linear Model}
\end{figure}


\subsubsection{Propensity Score Matching --- Random Forest}
Finally, we carry out the Propensity Score analysis through a Random Forest model.
Appendix \ref{appendixA}
also includes the calibration information for the Random Forest models to predict ROR for each type of case.
We observe that the ROR prediction models are accurate and unbiased.

Tables \ref{table:ROR_non--QoL_propensity_score} and
       \ref{table:ROR_QoL_propensity_score},
respectively, include the mean predicted probability of pleading for
  ROR and Non--ROR, for
  Non--Quality of Life and Quality of Life cases,
  respectively.
Figures~\ref{fig:ROR_non--QoL_propensity_score} and
        \ref{fig:ROR_QoL_propensity_score} show
the CDFs of same values determined through the model
(also see findings through GLM approach in
Figures~\ref{fig:ROR_non--QoL_GLM} and
        \ref{fig:ROR_QoL_GLM}).


Our results match the All Else Equal analysis.
Especially for Non--Quality of Life cases, a Non--ROR burden level increases the predicted probability of pleading.
Future work should repeat this analysis using
Jail and
Non--Jail burden levels,
as those levels are where the substantive difference occurs in the All Else Equal analysis.


\begin{table}
\centering
  \begin{tabular}{|p{0.3\textwidth}|p{0.6\textwidth}|}
    \hline
    \textbf{Burden Level} & \textbf{Mean Predicted Probability of Pleading} \\ \hline
    ROR & 0.3706164 \\ \hline
    Non--ROR & 0.412566 \\ \hline
  \end{tabular}
  \caption{Mean Probability of Pleading for Non--Quality of Life offenses through Random Forest Modeling}% Propensity Score Matching}
  \label{table:ROR_non--QoL_propensity_score}
\end{table}


\begin{table}
\centering
  \begin{tabular}{|p{0.3\textwidth}|p{0.6\textwidth}|}
    \hline
    \textbf{Burden Level} & \textbf{Mean Predicted Probability of Pleading} \\ \hline
    ROR & 0.6127946 \\ \hline
    Non--ROR & 0.6363191 \\ \hline
  \end{tabular}
  \caption{Mean Probability of Pleading for Quality of Life offenses through Random Forest Modeling}% Propensity Score Matching}
  \label{table:ROR_QoL_propensity_score}
\end{table}


\begin{figure}
  \centering
  \begin{subfigure}[b]{0.49\textwidth}
    \caption{Non--Quality of Life cases}
    \includegraphics[width=\textwidth]{figures/figxx.png}
    \label{fig:ROR_non--QoL_propensity_score}
  \end{subfigure}
  ~
  \begin{subfigure}[b]{0.49\textwidth}
    \caption{Quality of Life cases}
    \includegraphics[width=\textwidth]{figures/figyy.png}
    \label{fig:ROR_QoL_propensity_score}
  \end{subfigure}
  \caption{CDFs for the mean predicted probability of pleading for ROR and non--ROR through Random Forest Modeling}% Propensity Score Matching}
  \begin{subfigure}[b]{0.49\textwidth}
    \caption{Non--Quality of Life cases}
    \includegraphics[width=\textwidth]{figures/glmplots/ror_cdf.png}
    \label{fig:ROR_non--QoL_GLM}
  \end{subfigure}
  ~
  \begin{subfigure}[b]{0.49\textwidth}
    \caption{Quality of Life cases}
    \includegraphics[width=\textwidth]{figures/glmplots/rorq_cdf.png}
    \label{fig:ROR_QoL_GLM}
  \end{subfigure}
  \caption{CDFs for the mean predicted probability of pleading for ROR and non--ROR through Generalized Linear Model}
\end{figure}


\begin{figure}
  \centering
  \begin{subfigure}{0.49\textwidth}
    \caption{Non--Quality of Life cases}
    \includegraphics[width=\textwidth]{figures/figaa.png}
    \label{fig:mean_ROR_non--QoL}
  \end{subfigure}
  ~
  \begin{subfigure}{0.49\textwidth}
    \caption{Quality of Life cases}
    \includegraphics[width=\textwidth]{figures/figbb.png}
    \label{fig:mean_ROR_QoL}
  \end{subfigure}
  \caption{mean probability of ROR for each judge, vs.
  actual percentage of ROR modeled by Random Forest}

  \begin{subfigure}{0.49\textwidth}
    \caption{Non--Quality of Life cases}
    \includegraphics[width=\textwidth]{figures/glmplots/ror_scatter.png}
    \label{fig:mean_ROR_non--QoL_GLM}
  \end{subfigure}
  ~
  \begin{subfigure}{0.49\textwidth}
    \caption{Quality of Life cases}
    \includegraphics[width=\textwidth]{figures/glmplots/rorq_scatter.png}
    \label{fig:mean_ROR_QoL_GLM}
  \end{subfigure}
  \caption{mean probability of ROR for each judge, vs.
  actual percentage of ROR modeled by GLM}
\end{figure}


\subsection{Analyzing the Effect of Judge on Bail}
Figures~\ref{fig:mean_ROR_non--QoL} and
        \ref{fig:mean_ROR_QoL} show
        the mean predicted probability of ROR
        for each judge versus the judge's actual percentage of ROR,
        for Non--Quality of Life and Quality of Life cases,
respectively.
The plots show that some of the variation in
the judges' percentage of cases that are granted ROR is due to
non--randomness in the cases that they adjudicate.
However,
we observe that for a given mean predicted probability of ROR (on the Y axis),
there is significant judge variance on the true percentage of cases that are granted ROR.
Thus,
this initial analysis shows that different judges do utilize ROR differently for the same types of cases.
As before,
figures~\ref{fig:mean_ROR_non--QoL_GLM} and
        \ref{fig:mean_ROR_QoL_GLM} illustrate similar patterns using a different modeling approach (GLM).

\section{Discussion}
\documentclass[trackingWork]{subfiles}
\makeatletter
\def\blx@maxline{77}
\makeatother
\begin{document}
\section{Discussion}
\topic{Having taken a comprehensive look toward crowd work through the piecework lens,
we can't help but take a step back to consider a number of meta--issues that arose in our analysis.}


\subsection{The Hazards of Predicting the Future}\label{sec:perilousProblemsPredicting}

\subsection{Polarizing Tendencies}\label{sec:polarizationOfCrowdWork}

\subsection{Taking Stock of our Research Agenda}\label{sec:whatShouldBeTheFuture}

\begin{enumerate}
  \item \ali{the past is no guarantee of the future; what could be unforeseen?
  \item it's easy to fall into the (dys|u)--topian camp of what piecework/crowd work will look like
  \item what should we be focusing on? the future of crowd work}
\end{enumerate}

\end{document}

\section{Conclusion}
\documentclass[trackingWork]{subfiles}
\makeatletter
\def\blx@maxline{77}
\makeatother

\onlyinsubfile{
\usepackage{xr-hyper}
\usepackage{hyperref}
\externaldocument{complexity}
\externaldocument{decomposition}
\externaldocument{relationships}
}

\begin{document}
\section{Conclusion}
In this paper we considered three overarching questions in crowd work
--- \begin{inlinelist}
  \item ``\namerefl{sec:complexity}?'';
  \item ``\namerefl{sec:decomposition}?'';
        and
  \item ``\namerefl{sec:relationships}?''
\end{inlinelist} ---
by using the scholarship on piecework to inform our predictions and
to contextualize the answers that crowd work researchers have already uncovered.
We subsequently take this historical framing to ask broader questions of crowd work,
including among others what the future of crowd work holds for workers.
\end{document}

%!TEX root = final.tex
\newpage
\appendix
\section{Appendix} \label{appendixA}
Below we provide further details including confusion matrices and
reliability plots on some of the models used.

\begin{figure}[h]
  \centering
    \begin{subfigure}[p!]{\textwidth}
            \caption{Non--Quality of Life Cases}
            \begin{subfigure}[p!]{0.49\textwidth}
              \includegraphics[width=\textwidth]{figures/appa.png}
              \label{fig:AppA}
            \end{subfigure}
            ~
            \begin{subfigure}[p!]{0.49\textwidth}
              \includegraphics[width=\textwidth]{figures/appb.png}
              \label{fig:AppB}
            \end{subfigure}
    \end{subfigure}

    \begin{subfigure}[p!]{\textwidth}
                \caption{Quality of Life Cases}
                \begin{subfigure}[p!]{0.49\textwidth}

                  \includegraphics[width=\textwidth]{figures/appc.png}
                  \label{fig:AppC}
                \end{subfigure}
                ~
                \begin{subfigure}[p!]{0.49\textwidth}

                  \includegraphics[width=\textwidth]{figures/appd.png}
                  \label{fig:AppD}
                \end{subfigure}
      \end{subfigure}
      \caption{Random Forest Model}
\end{figure}
\newpage

\begin{figure}[h]
  \centering
    \begin{subfigure}[p!]{\textwidth}
    \caption{ROR Modeling using GLM}
        \begin{subfigure}[p1]{0.49\textwidth}
            \includegraphics[width=\textwidth]{figures/glmplots/ror_fc.png}
        \end{subfigure}
        ~
        \begin{subfigure}[p1]{0.49\textwidth}
            \includegraphics[width=\textwidth]{figures/glmplots/rorq_fc.png}
        \end{subfigure}
    \end{subfigure}


    \begin{subfigure}[p!]{\textwidth}
    \caption{Plea Rates Modeled using GLM}
        \begin{subfigure}[p1]{0.49\textwidth}
            \includegraphics[width=\textwidth]{figures/glmplots/plea_fc.png}

        \end{subfigure}
        ~
        \begin{subfigure}[p1]{0.49\textwidth}
            \includegraphics[width=\textwidth]{figures/glmplots/pleaq_fc.png}

        \end{subfigure}
    \end{subfigure}

\end{figure}

\subsection{Validating GLM}
With a half and half test/train split,
the ROR model had acceptable performance,
while the plea model's performance was rather unremarkable.
Though their accuracy rates are not abysmal,
it is important to keep in mind that
both outcomes being predicted have relatively high frequency in the data
(ROR rate is close to 65 percent and plea rate is around 50 percent).




\newpage
\bibliography{../../../references.bib}

\end{document}
