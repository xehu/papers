%!TEX root = final.tex
Broadly, we identify \nextitemizecount{} areas of research which provide some insight into this topic:
\begin{inlinelist}
  \item The use of
    \textit{\lnameref{subsec:bail}},
    and its stated and empirical purposes,
  \item the study of the effect that bail has on
    \textit{\lnameref{subsec:burden}},
    and
  \item the role of
    \textit{\lnameref{subsec:plea}},
    both for the courts and defendants.
\end{inlinelist}
We will discuss the intersection of these bodies of work as they relate to our research question.


\subsection{\nmu Bail}\label{subsec:bail}
In this section, we will explore the murky waters of the use of bail,
but ultimately settle on the conclusion that the courts have made:
that bail is fundamentally a necessity in a judicial system,
despite the philosophical questions and issues it inherently raises.
Despite the need for bail as a robust safeguard against undue imprisonment,
we will illustrate that the practice of bail--setting is
--- perhaps decidedly --- difficult to discuss in broad terms,
often leaving researchers primarily with
    qualitative,
    ethnographic, and sometimes
    anecdotal
evidence upon which to work.

Bail represents an attempt to resolve a challenging cognitive dissonance.
As \citet{foote1959bail} points out, the imprisonment of a defendant who is nominally considered
``innocent until found guilty''
arouses challenges of principle;
if the courts are to assume that a defendant is innocent (and thus treat them as such),
then holding them in custody until the trial begins is patently wrong, as it deprives a
\textit{presumably} innocent person of freedom.

Nevertheless, the United States Constitution's 8th amendment
--- as well as a series of laws attempting to reform bail
(one in 1966, another in 1984) ---
ostensibly protect defendants from unreasonable and excessive bail amounts,
otherwise to be set at the discretion of the presiding judge
\citep{berg1985bail}.
As \citeauthor{berg1985bail} notes, however,
the most recent revisions to the Bail Reform Act
--- those passed in 1984 ---
make it explicit that judges may set or even deny bail on the basis of
the judge's assessment of the defendant's danger to the community.
In \citet{1984schall},
the court ruled that
``there was nothing inherently unconstitutional in preventive detention''
and that
``the restraint on [the defendant]'s liberty
did not amount to punishment because
that was not the express purpose of the pretrial detention''.

The precedents established in the cases of
\citet{1984schall} and
\citet{1987united} broadly supports the judge's prerogative to set bail
--- and importantly, to \textit{not} set it
(in other words, to remand the defendant with no option to pay bail),
if the judge determines that safety to the community
(or the defendant's appearance upon a later court date)
could not be assured by any bail amount.
This latitude becomes an important component that we attempt to explore
in the hopes of determining whether if
--- at least within a single jurisdiction ---
judge bail--setting variance is high or not.
The implication of this question may be that
some judges are more merciful,
and some decidedly harsher,
than their peers in their assessments of defendants.




\subsection{\nmu Burden}\label{subsec:burden}
The 1984 Bail Reform Act thus
afforded the courts a great deal of freedom in setting bail,
bringing us naturally to the question of what effect this has on defendants awaiting trial.
In this area, we find myriad researchers have contributed;
we will attempt to highlight some of the work that is
especially illuminating and
representative of the body of work.

\citeauthor{kellough2002remand} identify qualitatively that the
``moral assessments'' of defendants appear to influence the burden judges assign in the form of bail--setting.
This, they hypothesize, may explain
(at least to some extent)
the disparity in race between
those that are let free in the interim before their court date, and
those that are held in custody on remand
\citep{kellough2002remand}.

\citeauthor{zacharias1997justice} further discusses the burden that bail may impose on defendants,
in this case highlighting the leverage that prosecutors may have over defendants,
for instance in the form of the unfamiliar and unwelcome environment
that temporary incarceration represents, and the promise to expedite one's interaction with the justice system
\citep{zacharias1997justice}.
This, finally, leads us to the outcome variable which we will ultimately investigate.


\subsection{\nmu Plea \nmu Bargaining}\label{subsec:plea}
Most tangibly, the persuasive
(perhaps \textit{coercive})
force that prosecutors can exercise in tandem with judges is
the pressure to encourage defendants to plead guilty in the pretrial phase.
This aspect of the justice system is especially unique, both because
plea--bargaining defies the adversarial model of the court and trial process and because
these dynamics are so ill--understood at broad levels
\citep{zacharias1997justice}.

The research on the \lnameref{subsec:burden} that bail represents on defendants
--- especially defendants with limited financial means and similarly limited access to cash ---
has \textit{qualitatively} found that
trial outcomes may be affected by the burden of pretrial incarceration.
To illustrate this, we turn to
\citet{grossman1983plea} and \citet{alschuler1981changing};
\citeauthor{grossman1983plea} estimates that as many as
``90 percent of convictions \dots result from a negotiated plea of guilty''
\citep[see also][with similar findings]{mccoy1980plea,kaplan1977american,newman1966conviction}.

Most of the existing research in plea bargaining,
especially \citeauthor{grossman1983plea}'s work,
focuses on the economic role of the plea bargain as a
``screening device'';
\citeauthor{alschuler1981changing}
contributes to the argument that pretrial bargaining may serve the defendant's best interests,
observing that neither the courts nor the public defenders who often represent defendants
have sufficient time to see all cases through open court
\citep{alschuler1981changing}.
\citeauthor{alschuler1981changing}
makes the noteworthy observation that we lack sufficient information to determine whether 
  the pressure of incarceration,
  the prosecutor's persuasive offer to settle, and
  the defending attorney's desire to expedite the closure of outstanding cases
is in fact causing defendants to settle more frequently,
primarily due to a dearth of broad--based data, and certainly
--- as \citeauthor{kellough2002remand} describe it ---
a simple ``paucity of research [in plea bargaining]''
\citep{kellough2002remand}.



