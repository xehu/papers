%!TEX root = final.tex
Maintaining an efficient justice system is in the interest of both the professionals managing and the citizens interacting with it.
Computational methods including large--scale data analysis and machine learning excel at quantitatively identifying points of inefficiency.
We propose to use these techniques to build models predicting points of inefficiency in the pretrial processes,
in order to provide recommendations for improving efficiency to judges and other court officials.

Notably,
a large majority of cases are never sent to trial; rather,
charges are dropped or settled out of court.
Pretrial procedures,
then,
such as the setting of bail or decision to pursue a case,
are especially fruitful sites for consideration with regards to efficiency.
A more insightful understanding of how to approach these pretrial procedures will benefit court officials in the time and money spent processing cases needlessly.
For defendants,
these insights could lead them to spend less time in prison awaiting trial if they are unable to pay bail,
and fewer resources towards a case that will never be brought to trial.


With this potential for intervention in mind,
we propose to examine the Gotham dataset,
using case features available to the judge,
and in particular bail,
which is set by the judge,
to predict whether a defendant will plead or not later in court proceedings.
The goal of this procedure is to analyze one attribute of a defendant's process through the justice system from the perspective of a judge,
including information the judge knows about the case and,
most importantly,
the decision a judge can make: setting bail.


Specifically,
we conceive of bail setting as imposing a type of (intentional) burden upon a defendant.
This burden can occur at several different levels,
from release on a defendant's own recognizance (ROR) --- a nonexistent burden,
to refusal to set bail at all in order to keep a defendant in custody --- a very high burden.
In between these two levels are bails set which the defendant is able to pay (or find a bondsman to put up bond),
and bails which the defendant is not able to pay.
These two categories are irrespective of the bail amount,
and instead are specific to the means of the defendant who is either able or not to pay bail.
Similarly,
it is worth considering whether different judges set bail in substantially different ways given similar cases,
as a starting point for different analyses.


We make \nextitemizecount{} hypotheses:
\begin{enumerate}
\item Effect of burden on pleading:
      that placing a larger bail burden on a defendant will correlate with a
      higher likelihood to plead later in the case.
      We posit a causal psychological mechanism by which
      defendants who have been required to put up substantial financial assets or,
      even further,
      had to spend time in jail awaiting trial are placed in a vulnerable position
      by the justice system and
      therefore are less likely to prolong their case defend themselves forcefully against charges
      --- in other words, more likely to plead guilty.
\item Our secondary hypothesis predicts that
      different judges will set substantially different bail on similar cases,
      since we anticipate that there is significant room for human error or
      difference this aspect of the justice system.
\end{enumerate}

The underlying goal of this work is
to provide predictions which aid judges in setting effective but
not excessive bail by
better understanding one way in which the process of setting bail influences
defendant experiences with and attitudes toward the justice system.