\documentclass[11pt]{article}
\usepackage{balance,graphics,setspace}
\usepackage{hyperref}
\usepackage[margin=1in]{geometry}
\usepackage[inline]{enumitem}
\usepackage[tiny,compact]{titlesec}

\setlength{\parindent}{0em}
\setlength{\parskip}{0em}
\renewcommand{\baselinestretch}{0}

\newlist{inlinelist}{enumerate*}{1}
\setlist*[inlinelist,1]{%
  label=\arabic*),
}
\begin{document}
\begin{itemize}
\item
Everything is so great. 5/5
\item
Nominate for best paper. much wow. such research.
\end{itemize}

\section*{Introduction}
\begin{itemize}
\item
``Gig work'' is not a new thing; piecework has been around since the 19th century, and a lot of that research and work can inform this.
\item why has the notion of ``on the side'' work changed? This seems too important to dismiss out of hand.
\item The questions the paper claims to answer are the following:
\begin{enumerate}
\item
What components are inextricable from successful markets?
\item
what can be disentangled?
\item
how would worker--run groups operate themselves?
\end{enumerate}
Let's see if this gets answered...
\end{itemize}

\section*{Background}
\begin{itemize}
\item \textbf{Collective Action} sounds unrelated to the topic introduced earlier; worker--centric labor markets are different from worker--run labor markets, and it sounds like this is conflating the two; did I miss that connection? The only thing I see that really makes a strong connection in the introduction is the vague reference to ``giving workers a sense of locus'', but that's not entirely the same as worker--run markets\ldots
\end{itemize}

\section*{Background}
Seems fine, not sure how these really 

\end{document}