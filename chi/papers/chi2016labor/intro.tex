%!TEX root = proceedings.tex
\section{Introduction}


% Karl Polyani's \textit{The Great Transformation} describes the processes that led Western Europe from Feudalism to what he calls ``market society"
% \cite{GreatTransformation}.
% His insights gave birth to ``Substantivism", whose subscribers think of economies broadly as ways of navigating the world
% --- socially and otherwise ---
% rather than more formally as ways of maximizing the utilization of scarce resources.
% In the Substantivist school of thought the negotiations found in the ``sharing" or ``peer economy" between and among workers, customers, and the market operators themselves collectively represent a broader economy than the money-for-(rides, housing, etc\dots) exchange that formalists would explore.

In the past several years, a new type of work has emerged where customers can hire someone to complete an individual task, or ``gig''.
This task might involve delivering a package, or driving someone downtown.
Such a model principally requires the sharing of capital-intensive goods, like access to a car or home,
which led to the popularization of its familiar name, the ``sharing economy''.
People could ``share'' their homes (Airbnb, Couchsurfing \cite{airbnbOfficial,couchsurfingOfficial}), cars (Uber, Lyft, and others \cite{uberOfficial,lyftOfficial}), and increasingly one's own time (TaskRabbit, Zaarly, and many others \cite{taskrabbitOfficial,zaarlyOfficial}).

Workers' demographics have changed dramatically in as little as half a decade largely discreetly.
Where workers in the sharing economy were once car and home-owners who had free time to spare,
many now think of these companies and the markets they expose as their primary source of income.
In the last several years, workers for Uber, Homejoy,
and other market operators have initiated
--- and in some cases won \cite{homejoySuit,uberSuit} ---
suits describing mistreatment and misclassification of these workers as
``independent contractors" rather than ``employees" (protected and regulated by labor laws in the United States) \cite{fedsUber}.

In the sharing economy's nascent years, companies enticed workers to join for the potential to do work ``on the side": offering rides to others when they had free time, or renting out their apartment when they were out of town for a weekend.
For various reasons, the culture has since changed, and the notion of doing work in one's free time has largely disappeared.
Instead, drivers report primarily working as drivers, and that their primary sources of income consist of the aggregated sum of passengers they pick up through ride-sharing markets.
Even in the hotelier industry, providers have purchased apartments or re-purposed their own homes primarily to serve guest occupants, rather than to rent out incidentally when they have a spare room or are out of town.

In these ways, workers are neither ``peers", nor are they ``sharing" resources that would otherwise go underutilized.
But they're not conventional workers, either.
They acquire capital
--- sometimes co-signing on leases with the companies that run these markets ---
under their own names, run their businesses relatively independently, and make the majority or even totality of their income based on each individual job, or ``gig", cumulatively summed up.
With their careers described as a series of individual jobs, each self-contained and relatively independent of the others, people have renamed it the ``gig economy", more fairly referencing the differentiating nature of this work.

The widespread nature of these changes suggests that this is part of a larger trend in what may become the future of work;
far from the hopeful but cautious predictions offered of information workers and ``crowd work'' \cite{crowdworkFuture}.
Workers, increasingly find themselves objectified, marginalized, and frustrated by oppressive systems.

We considered, then, how one might design a worker-centric peer market;
how can system designers create technologically enabled markets as successful as existing markets like Uber, Lyft, and others, while also:
\begin{itemize} \itemsep0pt \parskip0pt
  \item giving workers a sense of locus
  \item benefiting workers as well as consumers
  \item facilitating worker organization and communication
  \item enabling collective decision \& action among workers
\end{itemize}

To answer these questions, we engaged in extensive fieldwork alongside workers and labor organizations in a number of industries.
Building in part from backgrounds in the social sciences and as trained computer scientists,
we learned about workers and the industries in which they work from workers themselves.
We report on the processes of making contact with variously formally organized worker advocacy groups,
illustrate some of the ways that we can learn from informants most effectively given our own skills,
and finally describe some of the findings we made as a result of our own use of these methods.

Informed by the input of dozens of workers from numerous industries ranging from highly regulated to informal,
we identify a number of aspects of on-demand work which system-designers should consider in the creation of a worker-centric labor market.
We offer guidance on these design considerations,
and in some cases illustrate the suggested approaches we generated in tandem with these partner organizations and workers.

Specifically, we offer contributions to the following questions
\begin{itemize}\itemsep0pt \parskip0pt
  \item What components of existing markets are inextricable from the features which make these markets successful?
  \item What can be disentangled and abstracted away?
  \item How would these groups be operated?
\end{itemize}