% !TEX program = lualatex
%%%%%%%%%%%%%%%%%%%%%%%%%%%%%%%%%%%%%%%%%
% Beamer Presentation
% LaTeX Template
% Version 1.0 (10/11/12)
%
% This template has been downloaded from:
% http://www.LaTeXTemplates.com
%
% License:
% CC BY-NC-SA 3.0 (http://creativecommons.org/licenses/by-nc-sa/3.0/)
%
%%%%%%%%%%%%%%%%%%%%%%%%%%%%%%%%%%%%%%%%%

\documentclass{beamer}

\usepackage{tikz,lmodern,textpos,hyperref,graphicx,booktabs,appendixnumberbeamer,nth,xcolor}
\usepackage[citestyle=numeric,backend=bibtex]{biblatex}
\usepackage[export]{adjustbox}
\usepackage[en-US]{datetime2}
\bibliography{references}
\usetheme{metropolis}
\beamertemplatenavigationsymbolsempty
\renewcommand\textbullet{\ensuremath{\bullet}}
\definecolor{stanfordRed}{HTML}{8C1515}
% \mode<presentation>{
  % \usetheme{metropolis}
  \usecolortheme{seagull}


\definecolor{Orange}{HTML}{E98125}
\definecolor{Blue}{HTML}{0645AD}
\hypersetup{colorlinks,linkcolor=,urlcolor=Blue}

\definecolor{PineGreen}{HTML}{008800}
\definecolor{Red}{HTML}{FF0000}
\definecolor{Blue}{HTML}{0000FF}
\definecolor{JokeGreen}{HTML}{00C953}
\newcommand{\msb}[1]{{\color{PineGreen}[MSB: #1]}}
\newcommand{\ali}[1]{{\color{Red}[al2: #1]}}


% Theme colors are derived from these two elements
\setbeamercolor{alerted text}{fg=Orange}
\setsansfont[BoldFont={Source Sans Pro Semibold},
              Numbers={OldStyle}]{Source Sans Pro}
\setmonofont{Source Code Pro}
% \setbeamercolor{frametitle}{bg=stanfordRed}

  % \setbeamercolor*{palette tertiary}{use=structure,fg=stanfordRed,bg=stanfordRed}
% }
% \metroset{set}
\setbeamercolor{institute in head/foot}{fg=stanfordRed}
% \setbeamercolor{progress bar}{fg=stanfordRed}
% \setbeamercolor{progress bar}{bg=gr}
% \setsansfont{Source Sans Pro}

\title{Examining Crowd Work and Gig Work Through The Historical Lens of Piecework}
\metroset{titleformat=smallcaps}

\author{\textbf{Ali Alkhatib},
                Michael Bernstein,
                Margaret Levi\\
\texttt{ \scriptsize{\href{mailto:ali.alkhatib@cs.stanford.edu}{ali.alkhatib@cs.stanford.edu} ||
         \href{http://twitter.com/_alialkhatib}{@\_alialkhatib}} }}

\institute[Stanford]{Stanford University}
\date{\today}

\begin{document}

\begin{frame}
\titlepage
\end{frame}



% ## Intro [`1 minute`] [`1`]
% - Open problems in crowdsourcing: complexity, decomposition, relationships [`?`]
% - Piecework! [`?`]
% ## Wait what's piecework again [`1 minute`] [`2`]
% ## Some case studies: [`3 minutes`] [`5`]
% - airy's computers [`1`]
% - domestic/farmhand labor [`0.5`]
% - matchstick girls [`1`]
% - industrial workers [`0.5`]
% ## Complexity limits [`3 minutes`] [`8`]
% - what did crowdsourcing say? [`0.5`]
% - what did piecework say? [`1`]
% - comparisons? [`1`]
% - implications? [`0.5`]
% ## Decomposition limits [`3 minutes`] [`11`]
% - what did crowdsourcing say? [`0.5`]
% - what did piecework say? [`1`]
% - comparisons? [`1`]
% - implications? [`0.5`]
% ## Worker relationships [`3 minutes`] [`14`]
% - what did crowdsourcing say? [`0.5`]
% - what did piecework say? [`1`]
% - comparisons? [`1`]
% - implications? [`0.5`]
% ## Discussion [`1 minute`] [`15`]
% - this is a very tricky thing to do, predicting the future [`?`]
% - anyway let's predict the future: utopian/dystopian visions [`?`]
% - Research agendas! go go gadget contextual analysis [`?`]


\section{Introduction}

\begin{frame}{Open problems in crowdsourcing
                                                                                [\texttt{1 minute}]
}
\begin{itemize}[<+- | alert@+>]
  \item Complexity~\cite{suzukiAtelier,KimStoria,yuanAlmost,
                         YuEncouragingOutside,
                         Nebeling:2016:WCW:2858036.2858169,
                         Hahn:2016:KAB:2858036.2858364}
  \item Decomposition~\cite{sensitiveTasks,LykourentzouPersonalityMatters,
                            Law:2016:CKC:2858036.2858144,
                            Chang:2016:ACC:2858036.2858411,
                            Newell:2016:OMA:2858036.2858490}
  \item Relationships~\cite{turkopticon,storiesIraniSilberman,crowdcollab,
                            takingAHITMcInnis}
\end{itemize}
\end{frame}


\section{Piecework Primer}
\begin{frame}{Piecework Review
                                                                                [\texttt{1 minute}]
}
% What is piecework?
% \begin{figure}

  \begin{columns} % [<+- | alert@+>]
    \begin{column}{0.5\textwidth}
      \includegraphics[width=1\textwidth]{figures/svgs/timecard.png}
    \end{column}
    \begin{column}{0.5\textwidth}
      \includegraphics[width=1\textwidth]{figures/svgs/scale.png}
    \end{column}
  \end{columns}
  
% \end{figure}
\end{frame}



\section{Case Studies}
\begin{frame}{Airy's Computers
                                                                                [\texttt{1 minute}]
    }
\begin{itemize}
  \item images of hand--drawn tables
  \item maybe some illustration of the mathematical work that was being done
  \item draw to attention that you could break this into pieces
        (an example of breaking a formula down?)
\end{itemize}
\end{frame}

\begin{frame}{Domestic \& Farmhand Labor
                                                                                [\texttt{0.5 minutes}]
}
\end{frame}


\begin{frame}{The Match Girls
                                                                                [\texttt{1 minute}]
}
\end{frame}


\begin{frame}{Industrial Workers
                                                                                [\texttt{0.5 minutes}]
}
\end{frame}



\section{Research Threads}
\begin{frame}{Complexity Limits
                                                                                [\texttt{3 minutes}]
}
\begin{itemize}[<+- | alert@+>]
  \item Crowdwork's perspective
  \item Piecework's perspective
  \item Comparisons
  \item Implications
\end{itemize}
\end{frame}

\begin{frame}{Decomposition Limits
                                                                                [\texttt{3 minutes}]
}
\begin{itemize}[<+- | alert@+>]
  \item Crowdwork's perspective
  \item Piecework's perspective
  \item Comparisons
  \item Implications
\end{itemize}
\end{frame}

\begin{frame}{Worker Relationships
                                                                                [\texttt{3 minutes}]
}
\begin{itemize}[<+- | alert@+>]
  \item Crowdwork's perspective
  \item Piecework's perspective
  \item Comparisons
  \item Implications
\end{itemize}
\end{frame}

\section{Discussion}
\begin{frame}{Discussion
                                                                                [\texttt{1 minute}]
}
\end{frame}



\begin{frame}{Contact}

    name: {Ali Alkhatib} \\
    email: \href{mailto:ali.alkhatib@cs.stanford.edu}{ali.alkhatib@cs.stanford.edu} \\
    twitter: \href{https://twitter.com/_alialkhatib}{@\_alialkhatib} \\
\end{frame}


% \bibliographystyle{SIGCHI-Reference-Format}
\printbibliography{}
\end{document} 