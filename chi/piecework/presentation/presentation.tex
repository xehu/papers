% !TEX program = lualatex
%%%%%%%%%%%%%%%%%%%%%%%%%%%%%%%%%%%%%%%%%
% Beamer Presentation
% LaTeX Template
% Version 1.0 (10/11/12)
%
% This template has been downloaded from:
% http://www.LaTeXTemplates.com
%
% License:
% CC BY-NC-SA 3.0 (http://creativecommons.org/licenses/by-nc-sa/3.0/)
%
%%%%%%%%%%%%%%%%%%%%%%%%%%%%%%%%%%%%%%%%%

\documentclass{beamer}

\usepackage{tikz,lmodern,textpos,hyperref,graphicx,booktabs,appendixnumberbeamer,nth,xcolor,svg,subfiles}
\usepackage[citestyle=numeric,backend=bibtex]{biblatex}
\usepackage[export]{adjustbox}
\usepackage[en-US]{datetime2}
\bibliography{references}
\usetheme{metropolis}
\beamertemplatenavigationsymbolsempty
\renewcommand\textbullet{\ensuremath{\bullet}}
\definecolor{stanfordRed}{HTML}{8C1515}
% \mode<presentation>{
  % \usetheme{metropolis}
  \usecolortheme{seagull}


\definecolor{Orange}{HTML}{E98125}
\definecolor{Blue}{HTML}{0645AD}
\hypersetup{colorlinks,linkcolor=,urlcolor=Blue}

\definecolor{PineGreen}{HTML}{008800}
\definecolor{Red}{HTML}{FF0000}
\definecolor{Blue}{HTML}{0000FF}
\definecolor{JokeGreen}{HTML}{00C953}
\newcommand{\msb}[1]{{\color{PineGreen}[MSB: #1]}}
\newcommand{\ali}[1]{{\color{Red}[al2: #1]}}

\newenvironment{mystepwiseitemize}{\begin{itemize}[<+-| alert@+>]}{\end{itemize}}
% Theme colors are derived from these two elements
\setbeamercolor{alerted text}{fg=Orange}
\setsansfont[BoldFont={Source Sans Pro Semibold},
              Numbers={OldStyle}]{Source Sans Pro}
\setmonofont{Source Code Pro}
% \setbeamercolor{frametitle}{bg=stanfordRed}

  % \setbeamercolor*{palette tertiary}{use=structure,fg=stanfordRed,bg=stanfordRed}
% }
% \metroset{set}
\setbeamercolor{institute in head/foot}{fg=stanfordRed}
\setbeamercovered{transparent}
\setbeamercovered{again covered={\opaqueness<1->{15}}}
% \setbeamercolor{progress bar}{fg=stanfordRed}
% \setbeamercolor{progress bar}{bg=gr}
% \setsansfont{Source Sans Pro}

\title{Examining Crowd Work and Gig Work Through The Historical Lens of Piecework}
\metroset{titleformat=smallcaps}

\author{\textbf{Ali Alkhatib},
                Michael Bernstein,
                Margaret Levi\\
\texttt{ \scriptsize{\href{mailto:ali.alkhatib@cs.stanford.edu}{ali.alkhatib@cs.stanford.edu} ||
         \href{http://twitter.com/_alialkhatib}{@\_alialkhatib}} }}

\institute[Stanford]{Stanford University}
\date{\today}
\setbeamertemplate{itemize items}{--}

\hypersetup{
  colorlinks = true,
  % linkcolor = blue,
  citecolor = blue
}

% \def\labelitemi{--}


% \let\oldcite=\textcite
% \renewcommand{\textcite}[1]{\textcolor[rgb]{.2,.2,.9}{[\oldcite{#1}]}}



\newcommand{\onlyinsubfile}[1]{#1}
\begin{document}
\renewcommand{\onlyinsubfile}[1]{}
\begin{frame}
\titlepage
\end{frame}
% \setbeamercovered{transparent}


\subfile{intro.tex}

\begin{frame}[standout]
    What is the future of work?
\end{frame}

\begin{frame}{What is the future of work?}
    
\end{frame}

\begin{frame}{Introduction}
  We hope to provide:
      \begin{itemize}
        \item A useful ontological lens for making sense of crowdsourcing and gig work (which we collectively call ``\textit{on--demand work}'') as a resurgence of \textit{piecework}.
        \item A method for making sense of contemporary phenomena through \textit{historical analysis}.
      \end{itemize}
\end{frame}

\begin{frame}{A case for comparative historical analysis}
\begin{itemize}
  \item Historical analysis isn't new
  \begin{itemize}
    \item In general (see~\textcite{rosenberg1994exploring,rosenberg1982inside})
    \item In HCI (see~\textcite{Wyche2006,bodker1993historical})
  \end{itemize}
  % \item Still can be useful.
\end{itemize}
\end{frame}

\begin{frame}{A brief glossary}
    \begin{itemize}
      \item Crowd work: digitally mediated \textbf{information work}
      --- for example, work done on Amazon Mechanical Turk~\cite{crowdworkFuture}
      \item Gig work: digitally mediated --- but often \textbf{physically embodied} --- one--off jobs,
      such as
      \textit{driving},
      \textit{courier services},
      and \textit{administrative support}~\cite{friedman2014workers,Parigi:2016:GE:3026779.3013496}
    \end{itemize}
\end{frame}


\subfile{complexity}
% \subfile{decomposition}
% \subfile{relationships}
% \subfile{contact}


% \bibliographystyle{SIGCHI-Reference-Format}
\printbibliography{}
\end{document} 