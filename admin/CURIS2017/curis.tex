\documentclass[12pt]{article}
\usepackage{balance,times,graphics,setspace,microtype}
\usepackage[margin=1in]{geometry}
\usepackage[inline]{enumitem}
\usepackage[tiny,compact]{titlesec}
\usepackage{hyperref}

\setlength{\parskip}{.4em}
\setlength{\parindent}{0cm}

\setlist[itemize]{itemsep=0pt,topsep=0pt,parsep=0pt}
\setenumerate{itemsep=0pt,topsep=0pt,parsep=0pt}
\pagenumbering{gobble}

\begin{document}
\section{Project Name}

Making a ``R\'{e}sum\'{e} for Gig Workers''

\section{Description}

Gig workers are increasingly
piecing together earnings from various work platforms to make ends meet.
But insight into their work performance and history
are virtually nonexistent.
Requesters
--- the people that hire workers ---
increasingly struggle with existing qualifications and credentials services,
which lack sufficient detail about worker quality, driving them to undervalue workers.
Requesters don't get the information they need to make confident, informed decisions, and
as a result they're reluctant to trust gig workers to pay for creative, complex work.

This summer we'll be building a \textit{gig worker r\'{e}sum\'{e}} service,
an online system that aggregates
workers' histories, provides analytics on their performance over time, and helps requesters find high--quality workers.
You'll play a major part in designing and implementing this system,
involving web scraping, some statistical analysis, data visualization, and potentially more.
You'll also have the opportunity to study how people use the system,
using a mix of qualitative and quantitative analysis,
to inform our next steps.


\section{Recommended Background}\label{sec:RecommendedBackground}

Ideally, you have experience in
Human--Computer Interaction
(think CS 147, 247).
Much of the day--to--day work will involve programming for the web
(think CS 142),
so regardless of your research goals you should be able to code
(and be reasonably competent with existing libraries, frameworks, and APIs).

It might be worth pointing out that
I don't expect you to be an expert in these things right now.
You should be competent enough to start working on day 1, but
this is a chance to hone your skills and maybe learn something completely new.

If your background doesn't fit with anything above,
please tell us the most ambitious thing you've undertaken,
academic or not!
(Hard work can overcome lack of preparation!)



\section{Number of students allowed}

2

\section{Prerequisite/Preparation}

If you feel unprepared in any of the areas listed under the
``\nameref{sec:RecommendedBackground}'' section,
then your preparation between now and the start of the program will be to
try to fill in gaps.
If you want, we can discuss this further and start to prepare for your time through spring quarter.



\bibliographystyle{acm}
\bibliography{../references.bib}
\end{document}