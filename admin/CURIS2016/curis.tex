\documentclass[12pt]{article}
\usepackage{balance,times,graphics,setspace}
\usepackage[margin=1in]{geometry}
\usepackage[inline]{enumitem}
\usepackage[tiny,compact]{titlesec}

\setlength{\parskip}{.4em}
\setlength{\parindent}{0cm}

\setlist[itemize]{itemsep=0pt,topsep=0pt,parsep=0pt}
\setenumerate{itemsep=0pt,topsep=0pt,parsep=0pt}


\begin{document}
\section*{Project Name}

Alia: Building a worker--owned gig market

\section*{Description}

Working with the National Domestic Workers Alliance (NDWA),
we're building a cooperative labor market for house cleaners;
This labor market will allow workers to
manage and run both
their individual businesses \&
the platform on which they work.
We expect to expand significantly over the summer,
with communities coming ``online'' in
the SF Bay Area,
New York City,
and Seattle
(we're open to substantial growth,
but the purpose of this project isn't to scale).

This research is about exploring how a digitally--mediated cooperative labor market
would operate; in other words, what would a worker--owned version of
Handy,
or Uber,
or Amazon's Mechanical Turk,
look like?
The broader question at stake is how to enable and facilitate
collective governance,
something we see to varying extents in online communities like
Reddit, Wikipedia, etc\dots,
but which we don't seem to understand well enough to make routine,
or to replicate consistently.

More tangibly,
over the summer we're going to be working with
groups of workers that are on (more or less) the same platform.
We'll be building features to facilitate work
(one example:
communicating availability and negotiating booking a time)
as well as tools to facilitate decision--making
(one possible example:
voting mechanisms to hold a referendum).

\section*{Recommended Background}

Ideally, you have experience in
Human--Computer Interaction
(think CS 147, 247).
If your background is especially unorthodox,
you might have some experience in ethnographic methods
(think Anthropology, Sociology),
and that would be interesting.
Much of the day--to--day work will involve programming for the web
(think CS 142),
so regardless of your research goals you should be able to code
(and be reasonably competent with existing libraries, frameworks, and APIs).

It might be worth pointing out that
I don't expect you to be an expert in these things right now.
You should be competent enough to start working on day 1, but
this is a chance to hone your skills and maybe learn something completely new.

If your background doesn't fit with anything above,
please tell us the most ambitious thing you've undertaken,
academic or not!
(Hard work can overcome lack of preparation!)

\section*{Number of students allowed}

1--2

\section*{Prerequisite/Preparation}

Your preparation would be to try and backfill some of
the stuff in the ``Recommended Background'' section
that you don't already have.
That'll be unique to your circumstances and background,
so make this judgment call on your own;
if we end up working together,
I can put together a more actionable list of things to do between now and this summer.


\bibliographystyle{acm}
\bibliography{../references.bib}
\end{document}